
\documentclass[10pt,twocolumn]{article}

\let\inodot\i
\renewcommand{\i}[1]{\textit{#1}}

\newcommand{\mdash}{---}

\usepackage[letterpaper, left=.45in,right=.45in,top=1in,bottom=1in]{geometry}

\usepackage{graphicx}
\newcommand{\aDXTH}{%
\resizebox{!}{7.5pt}{H}%
\raisebox{2pt}{\rotatebox{-10}{\resizebox{!}{5pt}{\ensuremath{\times}}}}\hspace{-4pt}%
\raisebox{1pt}{\rotatebox{10}{\resizebox{!}{5pt}{\ensuremath{\times}}}}%
}

\newcommand{\DXTH}{%
\resizebox{!}{7.5pt}{H}\hspace{-2pt}%
\raisebox{2pt}{\resizebox{!}{9pt}{\ddag}}%
}


\newcommand{\xDXTH}{%
\resizebox{!}{7.5pt}{H}\hspace{-2pt}%
\raisebox{2pt}{\resizebox{!}{5pt}{\ensuremath{\times}}}\hspace{-3pt}%
\raisebox{2pt}{\resizebox{!}{5pt}{\ensuremath{\times}}}%
}


\newcommand{\eeeDXTH}{%
\resizebox{!}{7pt}{\ensuremath{\varsigma}}%
\resizebox{!}{7pt}{\ensuremath{\xi}}%
\resizebox{!}{7.5pt}{TX}%
\resizebox{!}{7pt}{H}%
}


\usepackage{ tipa }
\newcommand{\eeDXTH}{%
\resizebox{!}{7pt}{\ensuremath{\varsigma}}%
\resizebox{!}{7.5pt}{\ensuremath{\xi}}%
\resizebox{!}{7.5pt}{\textthorn}%
}

\newcommand{\eDXTH}{%
\resizebox{!}{8pt}{\ensuremath{\varsigma}}%
\resizebox{!}{7.5pt}{\ensuremath{\xi}}%
\resizebox{!}{7.5pt}{T}%
\resizebox{!}{6.5pt}{H}%
}


\colorlet{codegr}{black!80!blue}

\setlength{\columnsep}{7mm}



\usepackage{tcolorbox}
\newenvironment{frquote}{%
\begin{tcolorbox}[
	colback=orange!6,
	colframe=yellow!20!gray,
	width=1.95\linewidth,
	arc=3mm, auto outer arc,enhanced jigsaw,
	]
\begin{scriptsize}
\begin{minipage}{57em}
\begin{flushright}
\begin{minipage}{58em}}{%
\end{minipage}
\end{flushright}
\end{minipage}
\end{scriptsize}
\end{tcolorbox}
}
	



\AddToShipoutPicture{%
  \AtPageLowerLeft{%
    \hspace*{3pt}
 \rotatebox{90}{%
        \begin{minipage}{\paperheight}
   \centering
   {\color{codegr!65}\textcopyright ~\today{} Nathaniel Christen}
        \end{minipage} %
      }
    } %
  }%



\begin{document}

\title{From \cq{Naturalizing Phenomenology} to 
Formalizing Cognitive Linguistics (I): Cognitive Transform Grammar}
\author{Nathaniel Christen}

\newsavebox{\qboxi}
\newsavebox{\qboxii}

\begin{lrbox}{\qboxi}
\begin{frquote}
On conna\^{\inodot}t la c\'{e}l\`{e}bre 
affirmation de Claude L\'{e}vi-Strauss:
\q{les sciences humaines seront des sciences naturelles ou ne seront pas}.  Evidemment, sauf 
\`{a} en revenir \`{a} un r\'{e}ductionnisme dogmatique, une telle
affirmation n'est soutenable que si l'on peut suffisamment 
g\'{e}n\'{e}raliser le concept
classique de \q{naturalit\'{e}}, 
le g\'{e}n\'{e}raliser jusqu'\`{a} pouvoir y faire droit, 
comme \`{a} des ph\'{e}nom\'{e}nes naturels, aux 
ph\'{e}nom\'{e}nes d'organisation structurale. 
\\ \longdash{} Jean Petitot \cite[p. 1]{PetitotSyntaxe}
\end{frquote}
\end{lrbox}

\begin{lrbox}{\qboxii}
\begin{frquote}
The nature of any entity, I propose, divides into three aspects 
or facets, which we may call its form, appearance, and substrate.  
In an act of consciousness, accordingly, we must distinguish three 
fundamentally different aspects: its form or intentional structure, 
its appearance or subjective \q{feel}, and its substrate or 
origin.  In terms of this three-facet distinction, we can define 
the place of consciousness in the world.
\\ \longdash{} David Woodruff Smith, \cite[p. 11]{DavidWoodruffSmith}
\end{frquote}
\end{lrbox}

\twocolumn[\begin{@twocolumnfalse}

\maketitle{}
\begin{abstract}
In this paper I try to identify certain \q{paleostructures}
in language which transcend (or are somehow more fundamental 
than) the metatheoretical paradigm-opposition between 
faith in language being a \q{computable}, logico-mathematical 
system, or, conversely, a faculty bound up with overall 
human cognition, emotion, and embodiment such that nonsentient 
computers can never truly \q{understand} language.  I will, 
in fact, analyze some examples suggesting a context-sensitivity 
and a level of \q{epistemic} engagement which I feel argues 
for a register of linguistic nuance that is probably 
computationally intractable.  However, I also claim 
that \i{some} facets of language are more logically 
straightforward, and \i{some} sentences (or other 
linguistic artifacts) evince that logical register 
primarily or exclusively.  Juxtaposing these 
cases, I claim that language is not \i{intrinsically}
either tractable or non-tractable to computers or to 
any other system whose nature is essentially to carry 
out logicomathematical operations.  Instead, \q{tractability} in 
this sense is a parameter which is more or less evident 
in different artifacts.  A comprehensive theory 
must therefore address samples of language which 
are indeed strongly tractable {\mdash} so that the machinery 
of computational linguistics becomes explanatorily relevant 
{\mdash} \i{and also} examples which (via some interpretive 
or contextual complexity) are effectively \q{intractable}.  
Theories which make sense of the latter cases may be important 
to linguistic analysis even if they are not conducive 
to computational treatment or formulation.  In sum, 
different theories, with different levels of formalizability 
and of adherence to reductionist logico-mathematical 
paradigms, are appropriate for different acts of 
language.  An overall philosophy of language, then, 
rather than judging the merits or demerits of 
logical, mathematical, or computational formalization as a 
criteria of theoretical rigor, should instead seek 
to integrate diverse analyses which paradigmatically 
differ on this issue into a aggregated, multi-paradigm 
theoretical framework.
\end{abstract}

\begin{flushright}

\usebox{\qboxi}
\usebox{\qboxii}
\end{flushright}

\decoline{}
\vspace{3em}
\end{@twocolumnfalse}
]

\addcontentsline{toc}{section}{From \dq{Naturalizing Phenomenology} ...}

%
\p{The question of whether computers can be programmed to understand 
language may be philosophical, but it overlaps with 
broad methodological bifurcations: after all, linguists
\i{are} programming computers to \q{understand} language, at 
least to some approximation.  Given that computational 
linguistics is now a well-established practice, we can 
consider how this program for investigating the nature of 
language orients into linguistics as a whole: to what degree 
are the computers really \q{understanding} their linguistic 
input?  How much does \i{behavior} consistent with language-understanding 
suggest actual understanding?  Is linguistic competence mostly a 
behavioral phenomenon, or something more holistic and (inter-) subjective?  
Are the imperfections of automated Natural Language Processing 
inevitable, and if so, does that foreclose the possibility of
\NLP{} engines being considered truly linguistic?  That is, 
should we treat flawed and oversimplistic (but practically useful)
\NLP{} software {\mdash} or \q{personas} driven by this software, like
\q{digital assistants} {\mdash} as bonafide (if rather primitive) 
participants in the world of human language?  Or are they merely 
machines that simulate linguistic behavior without manifesting 
real linguistic behavior, as a computer simulation of a 
celestial galaxy is not a real galaxy?}

\p{These are methodological as well as thematic questions.  There is a 
wide swath of formal and computational linguistics, for instance, for 
which the measure of a theory is its chance of being operationalized 
on \NLP{} terms and within \NLP{} tools, yielding automated systems whose accuracy and/or 
computational efficiency competes favorably with other systems.  
Faithfulness to how \i{humans} process language is at most a secondary 
concern.  Conversely, there is a broad literature in Cognitive Linguistics 
and the Philosophy of Language for which uncovering the cognitive and 
interpretive registers through which \i{we} understand, produce, and 
are affected by language is the main goal.  For scholars chasing 
that telos, failure to encode theoretical models in mathematical 
or software systems is not \i{prima facie} an explanatory limitation 
{\mdash} conversely, we might take this as evidence that cognitive models 
are addressing the deep, subtle realities of language that are 
opaque to computer simulation.}

\p{Then there is hybrid work, like attempts to formalize 
Cognitive Grammar (Matt Selway \cite{MattSelway}, 
Kenneth Holmqvist \cite{KennethHolmqvist}, \cite{HolmqvistDiss}), or
other branches of Cognitive Linguistics 
(cf. Terry Regier's influential \cite{TerryRegier}), 
or Conceptual Space Theory as initiated by 
Peter \Gardenfors{} (which has seen several attempts at 
mathematical-computational formalization, such as 
Frank Zenker, Martin Raubal, and Benjamin Adams's 
metascientific perspectives \cite{RaubalAdams},
\cite{RaubalAdamsCSML}, \cite{Zenker}, and more recent 
Category-Theoretic structures linked to mathematicians 
such as Bob Coecke and David Spivak \cite{InteractingConceptualSpaces}).  
To this list we could add research that extends beyond 
language alone to broader cognitive-perceptual and 
conceptual themes, like formal descriptions rooted in 
Husserlian analyses by phenomenologists whose methods 
encompass some computer-scientific techniques, like 
Barry Smith (as in \cite{BittnerSmithDonnelly}) and 
Jean Petitot (see \cite{JeanPetitot}); we can see 
these accounts as generalizing cognitive-linguistic 
theories by noting the phenomenological basis 
of linguistic phenomena, as articulated by (say) Olav 
K. Wiegand (\cite{OlavKWiegand}, \cite{WiegandGestalts}) 
and Jordan Zlatev (\cite{JordanZlatev}).  
In each of the works just cited (prior anyhow to the 
last three)  
we can find formal/computational models whose 
rationale is, in large part, to 
shed light on human cognitive processes 
(albeit not necessarily translating to practical
\NLP{} components in any straightforward way).}

\p{This kind of \q{intermediate} research is perhaps 
under-appreciated, because it neither accepts the dismissive 
attitude that formal models are a distraction from 
the analysis of \q{real} language, nor the reductionistic 
faith that language is \i{intrinsically} computational {\mdash} 
that progress toward ideal \NLP{} avatars is just a matter 
of time.  To be sure, layering formal systems on a 
cognitive/phenomenological foundation adds a complexity of 
theoretical structure, which could prompt questions about the 
efficacy of the theoretical dilation: is extra complexity 
desirable as an end in itself, if the new formalizations 
have only limited explanatory or practical pay-offs?  
If there is a human kernel in language that is intrinsically 
non-computable and non-formalizable, does analysis of 
language truly benefit from complex but only partly 
applicable structural overlays?  On the other hand, if 
language \i{is} computationally tractable, shouldn't
\NLP{} implementation be a factor in assessing which 
formal models are worthy of attention?}

\p{Perhaps for these kinds of considerations, linguistics seems to 
bifurcate between a camp that essentially ignores 
computational methodology and resources and a camp 
that centers its whole attention on building better automated
\NLP{} tools.  Left out of this division is research that 
invokes formal models as explanatory vehicles while not 
enmeshing them in an ecosystem oriented toward automation {\mdash} 
the difference between deploying formal repesentations to 
model (some aspects) of language processing, syntax, and 
semantics, and trying to program software to \i{automate} the 
construction, translation, and pipeline between and among
formal models.  When situating research relative to 
computational linguistics, we should bear in mind the 
metatheoretical point that \i{incorporating} formal 
schema into linguictis models does not \i{necessarily}
mean committing ourselves to a task of programming 
computers to discover the target representations 
on their own, given raw linguistic input.}

\p{But this perspective is not only metatheoretical: I believe  
that the nature of language is \i{intrinsically} \q{hybrid}
in a manner that warrants neither blind faith in 
automation nor \i{a priori} disengagement with formalizing 
projects.  This is first of all because language is neither 
wholly isolated from other cognitive phenomena nor 
without some structural autonomy.  It is reasonable to
suppose that there are distinct intellectual faculties 
internal to our understanding of language, while other 
reasonings intrinsic to parsing the form and intentions 
of a linguistic unit are drawn from the wider 
inventory of situational, conceptual, social, and 
practical/enactive cognition.  Sentences can vary 
in terms of their context-sensitivity and the degree to 
which extralinguistic rationality is implicit in 
grasping intended meanings.  So neither a theory which 
ignores extralinguistic cognition, nor one which 
treats \i{all} linguistic processing as inseparable
from the totality of our cognitive processes from moment to moment, 
are complete.  A fully-forged theory needs to place 
sentences on a spectrum, where extralinguistic and (I'll say)
\q{intra-linguistic} theoretical machinery is 
available to analyze different sentences as their 
form and context demands.}

\newsavebox{\mxbox}
\begin{lrbox}{\mxbox}
\begin{tikzpicture}[baseline=(current bounding box.center)]
\matrix (m) [matrix of math nodes,nodes in empty cells,right delimiter={.},left delimiter={.}]{
	\text{Logicomorphic}  & \text{\rotatebox{90}{$\parallel$}} & \text{Extra-Linguistic} \\
	\| &  &   \| \\
	\text{Intra-Linguistic}  & \text{\rotatebox{90}{$\parallel$}}  &  \text{Interpretive} \\
};
\draw (m-1-1)-- (m-3-3);
\draw (m-3-1)-- (m-1-3);
\end{tikzpicture}
\end{lrbox}


\p{Extending this point, I believe a comparable spectrum 
matches the duality of language seen as intrinsically 
formalizable and computable or as too subtle, social/cultural, 
embodied, and context-dependent to be tractable to any 
computer or any idealized logicomathematical abstraction.  
Some sentence are more logically straightforward; others 
are more elusive, requiring holistic and context-sensitive 
interpretation on conversants' parts to be understood.  
Combined with my claims last paragraph, I argue accordingly 
that we can (at least as a suggestive picture) view 
linguistic artifacts (canonically, sentences) 
along a two-axis spectrum defined both by extralinguistic 
integration (or lack thereof) and by formal tractability 
(or lack thereof), like so: (this is intended as an intuitive 
sketch, not a formal model).
\begin{center}

\usebox{\mxbox}
\end{center}
I will elucidate the terms on that picture later.  Summarily, though, 
I claim that \i{some} sentences evince logically straightforward 
compositional patterns that can be analyzed \i{either} at the
language-specific (syntactic or semantic) level \i{or} within 
cognitive registers outside of language proper (e.g., situational 
schemata); conversely, some sentences have nuances that
call for interpretive judgments which appear to transcend 
formal simulacra outside the full range of human 
intelligence, emotion, and embodiment, \i{either} in 
terms of parsing complex syntactic or semantic structures
\i{or} in grounding linguistic phenomena in ambient contexts.  
My overall point is then that sentences take a spectrum of 
models spanned by these axes; no one paradigm is 
self-contained as a metalinguistic commitment.}

\p{In effect, the choice between paradigms wherein language is or 
is not formally/computationally tractable, and between paradigms 
wherein language is or is not intellectually autonomous \visavis{}
our total cognitive faculties, should not be seen as a 
metaphysical alternative anterior to language as a totality.  
Instead, these spectra are threaded into language internally, 
competing polarities which rise or fall from sentence to 
sentence.  Language is not \i{intrinsically} either formal 
or non-formal; autonomous or non-autonomous.}

\p{But at the same time, sentences are clearly phenomena of the 
same ilk; the distinctions I have made are not so sharp as to 
disrupt the ontological similitude among sentences, so that 
two sentences (however much they differ on the spectra of 
my diagram) are still manifestations of the same ontological 
place; they are still roughly the same \i{sort} of existents.  
Accordingly, we should conclude philosophically that there are 
certain aspects of sentences that lie beneath the formal/interpretive  
and intralinguistic/extralinguistic dualities.  There are, in short,
paleostructures in language that manifest \i{either} with formal 
specificity or with contextual nuance; \i{either}
internal to syntax or semantics or external to intrinsically 
linguistic cognition; varying from sentence to sentence.  
This paper will present a theory of one such paleostructure, 
drawing inspiration both from formal theories (Dependency Grammar, 
Type-Theoretic Semantics, Generative Lexicon) and from 
more philosophical approaches (Cognitive Grammar, Semiotics, Phenomenology).}

\p{The central element in my analysis is the \i{conceptual modification}
implied or effected by one word in the presence of another word.  
More precisely, some words' linguistic roles can be analyzed as 
adding cognitive detail (conceptual and/or perceptual and/or pragmatic) 
to the ideas or referents signified via other words.  The underlying scheme, 
at this level where the model is undeveloped and thus fairly simple, 
is close to Dependency Grammar: in lieu of a head/dependent relation 
we can treat one word (or similar lexical unit) as a \i{modifier} and the  
second as a \i{ground}.  I believe this two-pronged picture is not 
completely vacuous, but is general and underspecified enough to
spread over both syntax and semantics, and over competing 
paradigms.  I will call the modifier's effect on its ground 
a \i{transform}, and the two words together (taking \q{word} as a 
convenient designation for lexemes in general) as a
\i{transform pair}.}

\p{I will argue that the underlying transform-pair concept is 
amenable to both more formal and more philophical/interpretive 
development.  One the one hand, transform-pairs can sometimes 
be mapped explicitly to dependency or link-grammar pairs, 
so the theory can be treated as a philosophical preliminary 
or motivation for Dependency or Link Grammar.\footnote{Dependency and Link Grammar both build syntactic structures
from inter-word pairs (rather than larger phrasal units), 
so the latter is at least in part a species of the former.  
In lieu of head/dependent relations, however, Link Grammar 
identifies each linguistically meaningful connection between 
words (and lexical or constituent sentence units in general) 
as motivated by a specific connection rule, which depends on 
compatible connecting potentiality present on both words in a 
connection, which are part of their lexical profile.  Formally then 
Link Grammar removes one theoretical structuring posit 
(the \q{ranking} of pair-elements as head or dependent) 
but adds a different posit related to the two \q{potentialities}
which get fused into a single link; linkage capacities become rooted 
in the lexicon arguably more so than in (conventional) Dependency Grammar
\cite{Schneider}, \cite{SleatorTamperley},
\cite{GoertzelPLN}, \cite{ErwanMoreau}.} Relatedly,  
transforms themselves can be integrated into type systems 
(e.g., the kinds of transforms associated with adjectives 
depend on their ground being typed as a noun), so 
a theory of transform-pairs can motivate elaboration in a 
Type-Theoretic context.  On the other hand, we can focus 
on the interpretive and situational nuance often evident in 
cognitive transformations to find evidence for linguistic 
phenomena which do not fit neatly into Dependency Grammar
or Type-Theoretic (or any other) formalization.  
These various continuations, which I will present further 
over the next several sections, try not to foreclose either 
formal or philosophical/interpretive paradigms.  The goal is 
to trace both formal models of language and alternative
models {\mdash} for which excessive formalization is reductionistic 
and depends on ad-hoc avoidance of many real-world 
significations {\mdash} to a common structural kernel from which 
both perspectives can be deployed on a case-by-case 
(e.g., sentence-by-sentence) basis.}

\p{In keeping with the perspective that formal models have \i{some}
merit, I have tried to orient the presentation around certain 
computational-linguistic techniques.  Some of my examples are 
drawn from popular annotated corpora, and I provide models 
of other examples as processed by representative \NLP{} technologies 
(e.g., Malt Parser trained against the most recent Universal 
Dependency training data at the time of writing).  
I also select multiple examples from the Blackwell \i{Handbook of Pragmatics}
\cite{BlackwellPragmatics}; the data set includes code I used 
to compile all examples from that volume into a collection.  
I have packaged the examples and supporting code into an 
open-source data set for purposes of demonstration.  
I do not dwell on the accuracy of the \NLP{} components, 
in part because I am motivated in this paper to examine 
how computational methods can be employed as explanatory 
tools separate and apart from their feasibility in 
automated pipelines.  That is, the computational 
resources I present here are designed more as technological 
supplements to philosophical, interpretive, and 
speculative examinations of interesting linguistic 
examples, rather than as algorithms ultimately targeted 
at fully automated \NLP{} frameworks.  I incorporate 
code and data as a supplement to my argumentation in the 
hope that this can serve as an example of computational 
linguistics adopted outside the priorities of \NLP{}
automation and Artificial Intelligence in its more reductive, 
science-fictional sense.  I am not aspiring to 
develop code or theoretical claims that could advance the 
hypothetical project of implementing artificial agents that 
can mimic and understand human language and behavior in 
all its subjectivity and complexity.  However, I \i{do}
want to leverage certain computational techniques as  
offering their own explanatory perspectives on structures 
in human language.}

\p{The remainder of this paper will draw and expand on the outline 
of terms and structures sketched thus far, specifically the 
contrast of intra/extra-linguistic and \q{logicomorphic}/interpretive 
aspects from my \q{diagram} of sentences' paradigmatic spectrum, 
and the basic modifier/ground transform-pair account.  I 
use the phrase Cognitive Transform Grammar for the core notion of
\q{transform-pairs} as, at core, cognitive phenomena, which nonetheless 
allow for further exposition via different methods.  I will explain 
some of this variation in the first section.}

%
\section{Cognitive Transform Grammar and Transform-Pairs}
\label{s1}

\p{The idea that inter-word pairs are a foundational linguistic 
unit {\mdash} from which larger aggregates can be built up recursively 
{\mdash} is an central tenet of Dependency Grammar.  Here I will 
generalize this perspective outside (but not excluding) 
grammar, to overall semantic, pragmatic, and even 
extralinguistic relations indicated via interword relations.}

\p{In some cases word-relations can still be theorized mostly via 
syntax.  Consider hypothetical, example sentences like

\begin{sentenceList}

\sentenceItem{} \swl{itm:having}{His having lied in the past damages his credibilty in the present.}{syn}
\sentenceItem{} \swl{itm:whether}{Voters question whether he is truthful this time around.}{syn}
\end{sentenceList}

In (\ref{itm:having}), \i{having} is necessary to syntactically transform its ground
\i{lied} from a verb-form to a noun (something which can be inserted into a 
possessive clause).  Analogously, in (\ref{itm:whether}) \i{whether} modifies
\i{is} (since this is the head of a subordinate clause), wrapping a propositional 
clause into a noun so that it furnishes a direct object to the verb
\i{question}.  The essential transformation in these cases is 
motivated by grammatic considerations, particularly part-of-speech: 
a verb and a subordinate, propositionally complete clause (in (\ref{itm:having}) and 
(\ref{itm:whether}), respectively) need (for syntactic propriety) to be 
modified so as to play a role in a site where a noun is expected 
(in effect, they need to be bundled into a noun-phrase).}

\p{The relevant transforms here {\mdash} signified by \i{having} and \i{whether} {\mdash} have a 
semantic dimension also, and we can speculate that the syntactic 
rules (requiring a verb or propositional-clause to be transformed into a 
noun) are actually driven by semantic considerations.  Conceptually, 
for example, \i{his having lied} packages a verb into a possessive 
context because the sentence is not foregrounding a specific lying-event 
but rather the fact of the existence of such occasions.  We cannot perhaps
\q{possess} an event, but we possess (as part of our nature or history) 
the fact of past occurrences, viz., events in the form of things we
have done.  In this sense \i{his having lied} marks a conceptual transformation, 
from events qua occurrents to events (as factical givens) qua states or
possessions, and the grammatical norm {\mdash} how we cannot just say
\q{his lied} {\mdash} is epiphenomenal to the conceptual logic here; 
the erroneous \q{his lied} sounds flawed because it does not 
match a coherent conceptual pattern in how events and states fit together.  
But, still, the syntactic requirement {\mdash} the expectation 
that a noun or noun-phrase serve as the ground of a possessive 
adjective, or the direct object of a verb {\mdash} manifests these 
underlying conceptual patterns in the order of everyday language.  
Syntactic patterns become entrenched \i{because} they are comfortable 
translations of conceptual schema, but \i{as} entrenched we hear 
these patterns as grammatically correct, not just as 
conceptually well-formed.  Likewise, we hear errata like
\q{his lied} as \i{ungrammatical}, not as conceptually incongruous.}

\p{I contend, therefore, that many conceptually-motivated word-pairing 
patterns become syntactically entrenched and thus engender a class 
of transform-pairs where the crucial, surface-level transformation
is syntactic, often in the form of translations between parts of speech, 
or between morphological classes (singular/plural, object/location, etc.).  
Consider locative constructions like

\begin{sentenceList}

\sentenceItem{} \swl{itm:grandma}{Let's go to Grandma.}{syn}
\sentenceItem{} \swl{itm:lawyers}{Let's go to the lawyers.}{syn}
\sentenceItem{} \swl{itm:press}{Let's go to the press.}{syn}
\end{sentenceList}

Here nouns like \i{Grandma}, \i{the lawyers}, and \i{the press} are 
used at sites in the surrounding sentence-forms that call for a designation 
of place {\mdash} this compels us to read the nouns as describing a place, 
even while they are not intrinsically spatial or geographical 
(e.g., Grandma is associated with the place where she lives).}

\p{In (\ref{itm:press}) and perhaps (\ref{itm:lawyers}), this locative 
figuring may be metaphorical: going \i{to the press} does not necessarily 
mean going to the newspaper's offices.  Indeed, each of these 
usages are to some degree conventionalized: going \i{to Grandma}
is subtler than going \i{to Grandmas house}, because the former 
construction implies that you are going to a \i{place}
(\q{Grandma} is proxy for her house, say), but also that Grandma is actually 
there, and that seeing her is the purpose of your visit.  In other words, 
the specific \i{go to Grandma} formation carries a supply of 
situational expectations.  There are analogous implications in 
(\ref{itm:lawyers}) and (\ref{itm:press}) {\mdash} going \i{to the press}
means trying to get some news story or information published.  
But the underlying manipulation of concepts, which structures the 
canonical situations implicated in (\ref{itm:grandma})-(\ref{itm:press}), 
is organized around the locative grammatical form as binding noun-concepts to a 
locative interpretation.  However metaphorical or imbued with additional 
situational implications, a person-to-location or institution-to-location 
mapping is the kernel conceptual operation around which the further 
expectations are organized.  Accordingly, the locative case qua 
grammatic phenomenon signals the operation of these situational 
conventions, and the syntactic norms in turn are manifest via 
word-pairings, such as \i{to Grandma}.}

\p{In short, a transform-pair like \i{to Grandma} can be analyzed in 
several registers; we can see it as the straightforward 
syntactic rendering of a locative construction (via inter-word morphology, 
insofar as English has no locative case-markers) or explore further 
situational implications.  In these examples, though, there is an 
obvious grammatic account of pairs' transformations, notwithstanding 
that there are also more semantic and conceptual accounts.  
Part-of-speech transforms (like \i{having lied}) and 
case transforms (like \i{to Grandma}) are mandated by 
syntactic norms and therefore can be absorbed into conventional 
grammatic models, such as Dependency Grammar: the head/dependent 
pairings in \i{having lied} and \i{to Grandma} are each 
covered by specific relations within the theory's inventory 
of possible inter-word connections.  So a subset of 
transform-pairs overlaps with (or can be associated with) 
corresponding Dependency Grammar pairings.}

\p{Another potential embedding of transform-pairs into 
formal models can be motivated by Type Theory.  
Such analysis may proceed on several levels, but 
in general terms we can assume that parts of speech 
form a functional type system (as elucidated, say, in 
Combinatory Categorial Grammar; e.g.,
\cite{BiskriDescles}).
For instance, we can recognize 
nouns and propositions (sentences or sentence-parts forming 
logically complete clauses) as primitive types, and 
treat other parts of speech as akin to \q{functions} between 
other types.  A verb, say, combines with a noun to 
form a proposition, or complete idea: \i{go} acts on
\i{We} to yield the proposition \i{We go}.  Schematically, 
then, verbs are akin to functions that map nouns to propositions.  
Similarly, adjectives map nouns to nouns, and 
adverbs map verbs to other verbs (here I use \q{noun} or \q{verb} to 
mean a linguistic unit which functions 
(conceptually and/or \visavis{} syntactic propriety) as a noun, or verb; 
in this sense a noun-phrase is a kind of noun {\mdash} i.e., 
a linguistic unit whose \i{type} is nominal).  
This provides a type-theoretic architecture through 
which transform-pairs can be analyzed.  
An adverb modifies a verb; so an adverb in a transform 
pair must have a verb as a ground.  Moreover, the \q{product}
of that transform is also a verb, in the sense that the
adverb-verb pair, parsed as a phrase, can only be
situated in grammatic contexts where a verb is expected.}

\p{In effect, we can apply type-theoretic models to both 
parts of a transform-pair and to the pair as a whole, 
producing structural requirements on how words link up 
into transform-pairs.  We can then see an entire sentence 
as built up from a chain of such pairs, with the rules 
of this construction expressed type-theoretically.  
Given, say, \i{his having lied flagrantly} we can identify a 
chain of pairs \i{flagrantly}-\i{lied},
\i{having}-\i{flagrantly}, and \i{his}-\i{having}, where 
the \q{outcome} of one transform becomes subject to a 
subsequent transform.  So \i{flagrantly} modifies \i{lied}
by expressing measure and emphasis, adding conceptual detail; 
grammatically the outcome is still a verb.  Then \i{having}, 
as I argued earlier, applies a transform that maps this
verb-outcome to a noun, which is then transformed by
the possessive \i{his}.  Each step in the chain is 
governed by type-related requirements: the output of one 
transform must be type-compatible with the modifier for 
the next transform.  This induces a notion of \i{type-checking}
transform-chains, which is analogous to how type-checking 
works in formal settings like computer programming languages, 
Typed Lambda Calculus, and Dimensional Analysis.}

\p{This gloss actually understates the explanatory power of 
type-theoretic models for linguistics, since I have 
mentioned only very coarse-grained type classifications 
(noun, proposition, verb, adjective, adverb); more 
complex type-theoretic constructions come into play 
when this framework is refined to consider plural/singular, 
classes of nouns, and so forth, establishing a basis for 
more sophisticated structures adapted to language from 
formal type theory, like type-coercions and
dependent types (I will revisit these theories 
in a later section).  Here, though, I will just point out 
that Dependency Grammar and Type-Theoretic Semantics can
often overlap in their analysis of word-pairs (inter-word 
relations is not centralized in type-oriented methodology as much 
as in Dependency Grammar, but type concepts can certainly
be marshaled toward word-pair analysis).}

\p{Even though Dependency and Type-Theoretic analyses will often reinforce 
one another, they can offer distinct perspectives on 
how pairs aggregate to form complete phrases and sentences.  
In the transform-pair \i{having lied}, \i{lied} is clearly the 
more significant word semantically.  This is reflected in
\i{having} being annotated (at least according to the Universal Dependency 
framework) as auxiliary, and the dependent element of the pair, while
\i{lied} is the \q{head}.  Then \i{lied} is also connected to
\i{his}, establishing a verb-subject relation.  So \i{lied} becomes 
the nexus around which other, supporting sentence elements are 
connected.  This is a typical pattern in Dependency Grammar parses, where
the most semantically significant sentence elements also tend to be 
the most densely connected (if we treat the parse-diagram as a 
graph, these nodes tend to have the highest \q{degree}, a measure of 
nodes' importance at least as this is reflected in how many 
other nodes connect to it).  Indeed, by counting word connections we 
can get a rough estimation of semantic importance, distinguishing
\q{central} and \q{peripheral} elements.  These are not 
standard terms, but they suggest a norm in Dependency Grammar that the
structure of parse-graphs generally reflects semantic priority: 
the central \q{spine} of a graph, so to speak, captures the 
primary signifying intentions of the original sentence, while the 
more peripheral areas capture finer details or syntactic auxiliaries 
whose role is for grammatical propriety more than meaningful content.}

\p{Conversely, a type-theoretic analysis might incline us to question this
sense of semantic core versus periphery: in the case of
\i{his having lied}, the transform \i{having} supplies the outcome 
which is content for the possessive \i{has}.  If we see the sentence 
as a cognitive unfolding, a series of mental adjustments toward an 
ever-more-precise reading of speaker intent, then each step in the 
transformation contributes consequential details to the final
understanding.  Moreover, \i{lied} is only present in the transformation 
signified by \i{his} insofar as it has in turn been transformed by
\i{having}: each modifier in a transform-pair has a degree of 
temporal priority because \i{its} effects are directly present
in the context of the following transformation.  This motivates a 
flavor of Dependency Grammar where the head/dependent ordering is 
inverted: a seemingly auxiliary component (like the function-word 
to a content-word) can be notated as the head because its 
output serves as \q{input} to a subsequent transform.  In the analogy 
to Lambda Calculus, \i{his having lied} would be graphed with
\i{having} being the head for \i{lied}, and \i{his} the head for
\i{having}, reflecting the relation of functions to their arguments.  
In lisp-like code, this could be written functionally as 
(his (having lied)), showing \i{having} as one function, and
\i{his} as a second one, the former's output being the latter's input.  
(Later I will include diagrams contrasting these different 
styles of parse-representation.)}

\p{Implicitly, then, Type-Theoretic Semantics and Dependency Grammar 
can connote different perspectives on semantic importance and
the unfolding of linguistic understanding.  I will explore 
this distinction further below, with explicit juxtaposition of 
parse graphs using the two methods.  I contend, however, that the 
distinction reflects a manifest duality in linguistic meaning: 
we can treat a linguistic artifact as an unfolding process or 
as a static signification with more central and more peripheral 
parts.  Both of these aspects coexist: on the one hand, we 
understanding sentences via an unfolding cognitive process;
on the other hand, this cognition includes forming a mental 
review of the essential points of the sentence, a collation of 
key ideas such as (for \ref{itm:having})
\i{his}, \i{lied}, \i{damages}, and \i{credibility}.
Given this two-toned cognitive status {\mdash} part dynamic process, 
part static outline {\mdash} it is perhaps understandable that 
different methodologies for deconstructing a sentence into 
word-pair aggregates would converge on different structural 
norms for how the pairs are interrelated, internally and to one another.}

\p{This analysis, which I will extend later, has considered transform-pairs 
from a syntactic angle {\mdash} in the sense that I have highlighted pairs 
which obviously come to the fore via grammatic principles.  As I indicated, 
I believe the notion of transform-pairs cuts across both syntax 
and semantics, so I will pivot to some analyses which attend more 
to the semantic dimension.}

\spsubsectiontwoline{Semantic Analyses of Transform-Pairs}
\p{In the simplest cases, a transform-pair represents a modifier 
adding conceptual detail to a ground, like \i{black dogs}
from \i{dogs}.  But the nature of this added detail 
{\mdash} and its evident relation to surface language {\mdash} can be highly 
varied, even among similar sentences at the surface level.  
Compare between examples like:

\begin{sentenceList}

\sentenceItem{} \swl{itm:black}{I saw my neighbor's two black dogs.}{sem}
\sentenceItem{} \swl{itm:rescued}{I saw my neighbor's two rescued dogs.}{sem}
\sentenceItem{} \swl{itm:latest}{I saw my neighbor's two latest dogs.}{sem}
\end{sentenceList}

Whereas (\ref{itm:black}) presents a fairly straightforward conceptual 
transformation, the detailing in (\ref{itm:rescued}) is a lot subtler;
mentioning \i{rescued} dogs makes no reference to perceptual qualities, 
but rather implies intricate situational background.  The term
\i{rescued dogs} strongly suggests that the dogs were adopted by their 
current owner, probably after an animal-welfare organization 
found them abandoned, or removed them from a prior abusive owner.  
This kind of backstory is packaged up, as a kind of 
situational prototype, in the conventionalized phrase
\i{rescued dogs}, implying a level of specificity more 
precise than the ajective \i{rescued} alone implies.  
Correspondingly, the verb \i{to rescue} when applied to 
dogs suggest more information than in more 
generic contexts.}

\p{The phrase \i{latest dogs} carries implications in its own 
right; we assume the neighbor had owned other dogs before.  
Of course \q{latest} implies some temporal order, but the 
understood time-scale depends on context.  
If we hear talk about a \i{vet}'s two latest dogs, we would presumably 
interpret this in terms of patients the vet has seen over the course of 
a day:

\begin{sentenceList}

\sentenceItem{} \swl{itm:vet}{We have to wait until after the vet's two latest dogs.}{sem}
\sentenceItem{} \swl{itm:organization}{I'm concerned for the rescue organization's
two latest dogs.}{sem}
\end{sentenceList}

Understanding the relevant time-frame depends on understanding the relation
between the dogs and the possessive antecedent.  In (\ref{itm:latest})
the neighbor (in a typical case) actually owns the dogs, so the 
situational context grounding the modifier \i{latest} would be understood 
against the normal time-scale for dog ownership (at least several years).  
In (\ref{itm:vet}), the vet only \q{possesses} the dogs in the sense 
of endeavoring to examine them, a process of minutes or hours.  
In (\ref{itm:organization}), the implication of the \i{organization's}
possessive \visavis{} rescued dogs is that the group endeavors 
to rehabilitate and find permanent homes for the rescuees.  So in each
case \i{latest} implies a succession of dogs, leading over time to 
two most recent ones, but the implied time-frame for our conceptualizing 
this sequence can be minutes-to-hours, or days-to-months, or years.}

\p{We should also observe that the implied time-frames and backstories in 
(\ref{itm:rescued})-(\ref{itm:organization}) are not directly signified via 
morphosyntax or lexical resources alone.  The word \i{rescued} only 
carries the \i{rescued dog} backstory when used in a context 
involving the dogs' eventual owners; in some context the more 
generic meaning of \i{rescue} could supersede:

\begin{sentenceList}

\sentenceItem{} \swl{itm:boatmen}{Boatmen rescued dogs from the flooded streets.}{sem}
\sentenceItem{} \swl{itm:firemen}{Firemen rescued dogs from the burning building.}{sem}
\end{sentenceList}

Neither (\ref{itm:boatmen}) nor (\ref{itm:firemen}) imply that the dogs were abandoned, or 
will have new owners, or be sent to a shelter, or that their rescuers are 
members of an animal-welfare organization {\mdash} in short, no element of 
the conventionalized backstory usually invoked by \i{rescued dogs} is present.  
Analogously, there is no lexical subdivision for \i{latest} which regulates 
the variance in time-frames among (\ref{itm:latest})-(\ref{itm:organization}).  
It is only by inferring a likely situational background that 
conversants will make time-scale assumptions based on one situation 
involving dog ownership, another involving veterinary exams, and a 
third involving animal-welfare rehabilitation.}

\p{That is to say, the time-scale inference I have analyzed is essentially
\i{extralinguistic}: there is no specific \i{linguistic} knowledge 
(lexical or grammatical, or even pragmatic inferences in the sense of 
deictic or anaphora resolution) which warrants the situational classification 
of (\ref{itm:latest})-(\ref{itm:organization}) into different time scales.  
Instead, the inference is driven by (to some degree socially or culturally 
specific) background-knowledge about phenomena like veterinary clinics
or animal rescue groups.  Whether or not the nuances in \i{rescued dogs}
are similarly extra-linguistic is an interesting question {\mdash} we can 
argue that the phrase is now entrenched as a \i{de facto} lexical 
entrant in its own right, so the role of \i{rescued} is not only to 
lend adjectival detail but to construct a recurring phrase with a 
distinct meaning, like \i{red card} (in football) or \i{stolen base}
(in baseball).  Lexical entrenchment is, I would argue, an 
intra-linguistic phenomenon, in the sense that understanding entrenched 
phrases is akin to familiarity with specific word-senses, which is a 
properly linguistic kind of knowledge.  But even in that case, entrenchment 
is only possible because the phrase has a signifying precision more 
rigorous than its purely linguistic composition would imply.  
There are, in short, extralinguistic considerations governing \i{when}
phrases are candidates for entrenchment, and a language-user's 
ability to learn the conventionalized meaning (which I believe 
is an intra-linguistic cognitive development) depends on their 
having the relevant (extra-linguistic) background knowledge.}

\p{If we consider then the contrast between transform-pairs like
\i{black dogs}, \i{rescued dogs}, and \i{latest dogs}, 
the similar grammatic constructions {\mdash} indeed similar semantic 
constructions, in that each pair has an adjective modifying a 
straightforward plural noun (\i{dogs} designates a similar 
concept in each case; this is not a case of surface grammar 
hiding semantic diversity, like \i{strong wine} vs.
\i{strong opinion} vs. \i{strong leash} or
\i{long afternoon} vs. \i{long history} vs. \i{long leash}) 
{\mdash} package transforms whose cognitive resolution spans a 
range of linguistic and extralinguistic considerations.  
Straightforward adjectival modification in \i{black dogs} gives 
way to lexical entrenchment in \i{rescued dogs} which, as 
I argued, carries significant extra-linguistic background knowledge 
even though possession of this knowledge is packaged into basic 
linguistic familiarity with \i{rescued dogs} as a signifying unit; 
and in the case of \i{latest dogs} the morphosyntactic evocation of 
temporal precedence and two different multiplicities (the latest
dogs and earlier ones) is fleshed out by 
extra-linguistic estimations of time scale.  
The same surface-level linguistic structures, in short, can 
(or so such examples argue for) lead conversants on a 
cognitive trajectory in which linguistic and extra-linguistic factors 
interoperate in many different ways.}

\p{This diversity should call into question the ability of conventional 
syntactic and semantic analysis to elucidate sentence-meanings with 
any precision or granularity.  Lexical and morphosyntactic 
observations may certainly reflect details which \i{contribute} to 
sentence-meanings, but the overall understanding of each sentence in
context depends on holistic, interpretive acts by competent 
language users in light of extra-linguistic, socially mediated 
background knowledge and situational understanding.  Contextuality 
applies here not only in the pragmatic sense that pronoun resolution, 
say, depends on discursive context (who is \i{her} in \i{her dogs}); 
more broadly, transcending even pragmatics, context describes 
presumptive familiarity with conceptual structures like veterinary 
clinics, animal shelters, and any other real-world domain which 
provides an overall system wherein particular lexical significations 
can be standardized.  Without the requisite conceptual background it 
is hard to analyze how speakers can make sense even of 
well-established variations in word-sense, like \i{treat} as in 
a veterinarian treating a dog, a doctor treating a wound, a carpenter 
treating a piece of wood, or how an actor treats a part.  
These senses have lexical specificity only in the domain-specific 
contexts of medicine, carpentry, theater, and so forth.}

\p{The problem of holistic cognitive interpretation (as requisite for 
sentence-meanings) can be seen even more baldly in examples where 
semantic readings bifurcate in ways wholly dependent on 
extra-linguitic conceptualization. Consider for instance:

\begin{sentenceList}

\sentenceItem{} \swl{itm:boroughs}{All New Yorkers live in one of five boroughs.}{sem}
\sentenceItem{} \swl{itm:commute}{All New Yorkers complain about how long it 
takes to commute to New York City.}{sem}
\sentenceItem{} \swl{itm:Cambridge}{The south side of Cambridge voted Conservative.}{ref}
\sentenceItem{} \swl{itm:lower}{Lower Manhattan voted Republican.}{ref}
\sentenceItem{} \swl{itm:si}{Staten Island voted Republican.}{ref}
\end{sentenceList}

Sentence (\ref{itm:Cambridge}) is taken from the \i{Handbook of Pragmantics}
(example 40, page 379, chapter 15) which borrows in turn from Ann Copestake and 
Ted Briscoe.  In the \i{Handbook} analysis (chapter by Geoffrey Nunberg), 
(\ref{itm:Cambridge}) is seen as ambiguous between 
reading \q{The south side of Cambridge} as an oblique description of 
the \i{voters} in that territory or as topicalizing the territory 
itself as a civic entity:

\begin{quote}
On the face of things, we might analyze (40) in either of two ways: either the
description within the subject NP has a transferred meaning that describes a
group of people, or the VP has a transferred meaning in which it conveys the
property that jurisdictions acquire in virtue of the voting behavior of their
residents.
\end{quote}

The duality is significant because the designation in \q{south side of Cambridge}
would be more informal in the prior reading, both geographically 
and in the how the implied collective of people is figured.  The prior 
reading accommodates a hearing wherein the speaker construes
\q{south side} not as a precise electoral district (or districts) but as a 
vaguely defined part of the city.  That imprecision also allows the claim
\q{voted Conservative} to be only loosely committal, implying that 
some majority of voters appeared to vote Conservative but not that 
this tendency is directly manifest in election results.  In short, 
we can interpret (\ref{itm:Cambridge}) as making epistemically more 
rigorous or more noncommittal claims depending on how we read 
the geographical reference \q{south side of Cambridge} (as crisp 
or fuzzy), the group of people selected via that reference 
(mapping the region to its inhabitants, a kind of \q{type coercion}
since places do not vote), and the assertive force of the 
speech-act: how precisely the speaker intends her claim to be 
understood.  Each of these \q{axes} contribute to the sentence's 
meaning insofar as they constrain what would be dialogically 
appropriate responses.}

\p{Meanwhile, in (\ref{itm:boroughs}), \i{New Yorkers} refers specifically to everyone who lives 
in the City of New York, since the five boroughs collectively span the 
whole of city.  In (\ref{itm:commute}), by contrast, we should understand
\i{New Yorkers} as referring to residents of the metropolitan area \i{outside}
the city itself (who commute \i{to} the city); and moreover \i{All} should 
be read less then literally: we do not hear the speaker in (\ref{itm:commute}) 
committing to the proposition that \i{every single} New Yorker complains.  
So both \i{All} and \i{New Yorkers} have noticeably different meanings in the 
two sentences.}

\p{And yet, I cannot find any purely linguistic mechanism 
(lexical, semantic, syntactic, morphological) which would account for 
these difference as linguistic signifieds \i{per se}: the actual differences 
depend on conversants knowing some details about New York 
(or, respectively, Cambridge) geography, and 
also general cultural background.  It does not make too much sense to 
commute to a place where you already live, so our conventional picture of 
the word \i{commute} constrains our interpretation of (\ref{itm:commute}) {\mdash} 
but this depends on \i{commute} having a specific meaning, of traveling 
in to a city, usually from a suburban home, on a regular basis; a meaning 
in turn indebted to the norms of the modern urban lifestyle (it would be 
hard derive an analogous word-sense in the language spoken by a nomadic 
tribe, or a pre-industrial agrarian community).  Likewise, 
reading \i{All} in (\ref{itm:boroughs}) as \i{literally} \q{all}
depends on knowing that the five boroughs are in fact the whole of 
the city's territory.  I am from New York, not Cambridge; 
perhaps residents of the latter city would clearly read
\q{south side} as referencing a fixed civic/electoral area 
(like \i{Staten Island}) or as only vaguely defined (like
\i{Lower Manhattan}).  For New Yorkers, 
(\ref{itm:lower}) would be read as fuzzy and 
(\ref{itm:si}) as fixed: the latter sentence has a clearly 
prescribed fact-check (since Staten Island is a distinct 
electoral district) which the former lacks.}

\p{Given that in everyday speech quantifiers like \i{all} or \i{every}
are often only approximate {\mdash} and that designations like
\i{New Yorker} are often used imprecisely, with not-identical 
alternative meanings intended on a case-by-case basis {\mdash} 
these kind of examples point to signifying ambiguities that 
can easily arise as a consequence.  Often extra-linguistic 
considerations resolve the ambiguity by rejecting one or another 
(otherwise linguistically plausible) reading as non-sensical.  
Consider:

\begin{sentenceList}

\sentenceItem{} \swl{itm:beat}{The Leafs failed to beat the Habs for the first time this year.}{amb}
\sentenceItem{} \swl{itm:consecutive}{The Leafs failed to win two consecutive games for the 
first time this year.}{amb}
\sentenceItem{} \swl{itm:goal}{The Leafs failed to score a goal for the 
first time this year.}{amb}
\end{sentenceList}

Sentence (\ref{itm:beat}) has two competing readings: either the Toronto Maple Leafs 
won \i{all} or \i{none} of their previous games, in the relevant year, against the 
Montreal Canadiens.  The difference is whether \i{for the first time this year}
attaches to \i{beat} or to \i{fail}.  In (\ref{itm:consecutive}), 
on the other hand, the only sensible interpretation is that the Leafs had 
not yet won two games: while it is logically accurate to describe a 
team on a long winning streak as repeatedly winning two consecutive games, it would 
be very unexpected for (\ref{itm:consecutive}) to be used in a case where the Leafs 
lost for the first time, after a three-plus-game winning streak.  
And in (\ref{itm:goal}) any hockey fan would hear that the Leafs had scored 
at least one goal in all prior games; even though there is no linguistic rule 
foreclosing the reading such that the Leafs have not scored a goal in \i{any} game.}

\p{These variations {\mdash} the degree to which superficial ambiguity is actually 
perceived by competent language-users as presenting competing plausible 
meanings {\mdash} depend on background factors; the contingencies of 
hockey fix how potential ambiguities resolve out because one or another 
alternative is extralinguistically incoherent.  But these cases point to 
how linguistic criteria alone, no matter how broadly understood, 
cannot necessarily predict in what sense linguistic structurations 
have empirically plausible meanings {\mdash} or whether they have 
sensible meanings at all.}

\p{Notice however that all these examples have alternate versions which are less 
subtle or ambiguous, which shows that the complications are not 
localized in the communicated ideas themselves, but in their 
typical linguistic encoding:

\begin{sentenceList}

\sentenceItem{} \swl{itm:allresidents}{All residents of the city of New York live in one of five boroughs.}{log}
\sentenceItem{} \swl{itm:manyresidents}{Many residents of the New York metropolitan area 
complain about how long it takes to commute to New York City.}{log}
\sentenceItem{} \swl{itm:SouthCambridge}{All districts on the south side of Cambridge voted Conservative.}{log}
\sentenceItem{} \swl{itm:leafshabs}{For the first time this year, the Leafs failed to beat 
the Habs.}{log}
\end{sentenceList}

These versions are more logically transparent, in that their propositional 
content is more directly modeled by the structure of the sentences. 
Indeed, hearers unfamiliar with New York (respectively Cambridge) 
or with hockey might find these 
versions easier to understand; more context-neutral and journalistic.  
But perhaps for this reason the \q{journalistic} versions actually 
sound stilted or non-idiomatic for everyday discourse.}

\p{Geoffrey Nunberg's chapter (15) in the
\i{Handbook} also has interesting examples that point to extralinguistic 
factors determining proxies and substitutions in referential contexts:

\begin{sentenceList}

\sentenceItem{} \swl{itm:parked}{I am parked out back.}[(example 3, page 371)]{ref}
\sentenceItem{} \swl{itm:waiting}{I am parked out back and have been waiting for 15 minutes.}[(example 3)]{ref}
\sentenceItem{} \swl{itm:start}{I am parked out back and may not start.}[(example 6, page 371)]{ref}
\sentenceItem{} \swl{itm:Whitney}{I'm in the Whitney Museum.}[(page 375)]{ref}
\sentenceItem{} \swl{itm:beers}{I drank two beers.}[(page 375)]{ref}
\sentenceItem{} \swl{itm:Michelobs}{I drank two Michelobs.}[(page 375)]{ref}
\sentenceItem{} \swl{itm:Sauternes}{I drank two Sauternes last night.}[(page 375)]{ref}
\end{sentenceList}

Someone says \q{I am parked} as proxy for their car, or \q{in the Whitney} as 
proxy for their art work.  Nunberg questions the acceptability of 
(\ref{itm:start}) on the premise that once we establish talking of 
people as an oblique reference to their cars, it sounds awkward to switch 
to content that can \i{only} apply to the car, like \i{may not start}.  
He also argues that we interpret (\ref{itm:beers}) and (\ref{itm:Michelobs}) 
as references, quite possibly, two glasses or bottles of \i{the same}
beverage, while (\ref{itm:Sauternes}) implies two samples of
\i{different} wines.  The analysis turns on conventions for 
creating referring expressions; in particular, he argues, 
extralinguistic factors (albeit not using this term) 
regulate when referential substitutions as exemplified in 
(\ref{itm:parked})-(\ref{itm:Sauternes}) are comfortable.
We accept an artist proxying her work in a museum, but 
(\ref{itm:Whitney}) arguably does not generalize to a case like 
(\ref{itm:library}), and (\ref{itm:waiting}) does not generalize to 
(\ref{itm:autoshop})

\begin{sentenceList}

\sentenceItem{} \swl{itm:library}{I'm in the library.}[(said by an author whose books are there)]{ref}
\sentenceItem{} \swl{itm:autoshop}{I've been in the auto shop for the 
last 15 days.}{ref}
\end{sentenceList}

Nunberg attributes these discrepancies to cultural conventions, like the
difference in prestige between having a painting in a museum compared to 
having one copy of one's book in a library; or chugging a cheap been 
versus savoring a celebrated wine.  Here too there are 
logically transparent alternatives:

\begin{sentenceList}

\sentenceItem{} \swl{}{My car is parked out back and have been waiting for 15 minutes.}{log}
\sentenceItem{} \swl{}{I drank two pints of beer last night.}{log}
\sentenceItem{} \swl{}{I tried two different Sauternes last night.}{log}
\end{sentenceList}

Again, though, the seemingly simpler versions {\mdash} whose forms 
generalize more readily to variant cases {\mdash} seem \i{less} natural 
or fluent, as if their very generalizability makes them sound 
awkward compared to logically more opaque, but idiomatically 
more sociable, renderings.  Impersonal, journalistic 
language can sound rather cold or unfriendly in contexts 
where speakers expect a dialect register characteristic 
of conversations among peers.}

\p{In short, even if sentences have a basically transparent 
logical content, \i{how} sentences holistically signify this 
content does not always emerge straightforwardly from semantic 
or syntactic structures on their own.  I think this weakens the 
case for semantic paradigms that concentrate on logically-structured 
content which appears to be signified through sentences 
{\mdash} even if we grant that this propositional ground of meaning 
is real, it does not follow that propositional contents are 
designated by purely linguistic means, rather than by 
a cohort of cognitive processes many of which are extra-linguistic.  
This is the basis of my proposing \q{logicomorphic} qualities 
as one axis for evaluating sentences, which I will now discuss further.}

\spsubsectiontwoline{Gaps in Logical Phrase-Models}
\p{Assume we have a baseline lambda-calculus-like 
functional summary of sentences and derived types.  That is, 
any sentence can be rewritten as if a sequence of \q{function calls}, 
assuming an underlying representational vocabulary of a typed 
Lambda Calculus, with sentences having overall \q{proposition} types 
(I will present this model in detail next section).}

\p{My overall goal is to embrace a hybrid methodology {\mdash} 
accepting formal analyses when they shed light on linguistic 
processes, but not going so far as to treat logical, mathematical, 
or computational models as full explanations for linguistic 
rationality qua scientific phenomenon.  
Cognitive Grammar, in particular, challenges our assumption that 
grammar and semantics are methodologically separate.  Received 
wisdom suggests that grammar concerns the \q{form} of sentences 
whereas semantics considers the meaning of words {\mdash} implicitly 
assuming that \i{word combinations} produce new meanings, and 
that the \i{order} by which words are combines determines how 
new meanings are produced.  This notion, in turn, is allied 
with the essentially logical or propositional picture of 
signifying via doxa: the idea that inter-word relations cue up 
different logically salient transformations of an underlying 
predicate model.  Thus {\mdash} to initiate a case-study I will 
return to several times {\mdash} \i{many students} as a phrase is 
more significatorily precise than \i{students} as a word, because 
the phrase (with intimations of quantitative comparison) has 
more logical detail.  Similarly \i{many students complained}
is more logically complete because, provisioning both a verb-idea 
and a noun-idea, it represents a whole proposition.}

\p{In general, then, phrases are more complete than words because 
they pack together more elements which have some logical role, 
establishing individuals, sets, spatiotemporal setting, and 
predicates which collectively establish sufficiently 
completed propositional attitudes.  On this account the 
key role of phrase-structure is to establish phrases as 
signifying units on a logical level analogous to 
how lexemes are signifying units on a referential or conceptual 
level.  Moreover, phrases' internal structure are understood 
to be governed by rule defining \i{how} word-combinations 
draw in extra logical detail.  A link between words is not a 
random synthesis of concepts, but rather implies a certain 
logical connective which acts as a de facto \q{third party} in 
a double-word link, proscribing with orientation to predicate 
structures \i{how} the words' semantic concepts are to be 
joined.  In \i{many students} the implied connector is 
the propositional act of conceiving a certain quantitative 
scale to a conceptualized set; in \i{students complained} the 
implied connector is a subject-plus-verb-equals-proposition 
assertiveness.  Phrases acquire logical specificity by 
building up word-to-word connections into 
more complex aggregates.}

\p{One implication of this model is that phrases are semantically 
substitutable with individual lexemes that carry similar meanings, 
having been entrenched by convention to capture a multipart concept 
which would otherwise be conveyed with the aggregation of a 
phrase: consider \q{MP} for \i{member of parliament}, or
\q{primaried} for \i{subject without your own party to a primary 
challenge}.  Conversely, phrases can be repeatedly used in a 
specific context until they function as quasi lexical units in 
their own right.  These patterns of entrenchment imply that we 
hear language in term of phrases bearing semantic content; and 
insofar as we are comfortable with how we parse a sentence, each 
word sited in its specific phrasal hierarchy, we do not tend to 
consider individual words semantically outside of their 
constituent phrases.}

\p{This theory of the syntax-to-semantic relationship is a paradigmatic 
partner, at the grammatic level, to the belief that meanings are 
fixed by \q{truth makers}: that is a sentence asserts a true fact, 
this fact is its first and foremost meaning, and if not, the meaning 
is somehow the description of a possible world where it \i{is} true.  
One of my projects here is to criticize (to some extent) these
\q{truth-theoretic} paradigms.  At present I want to point out 
how this philosophical paradigm can influence our 
appraisal of phrase-structure.  Simplistic \q{logic-based} conceptions of 
phrases can be based on two concerns: first, the idea that lexemes retain some 
syntactic and semantic autonomy even within clearly defined phrases 
where they are included; and, second, that the shape of phrases 
insofar as they are perceived as holistic signifying units 
is often driven by figurative or \q{gestalt} principles rather than 
neat logical structuration.  I'll call the former the issue of
\q{phrasal isolation} (or lack thereof): syntactic and semantic effect 
often cross phrasal boundaries, even outside the overarching hierarchy 
whose apex is the whole sentence.  Both of these lines of reasoning 
{\mdash} arguably especially the second {\mdash} are developed in 
Cognitive Grammar literature.}

\p{In Ronald Langacker's \i{Foundations of Cognitive Grammar}, the sentence

\begin{sentenceList}

\sentenceItem{} \swl{itm:threetimes}{Three times, students asked an interesting question.}{sco}
\end{sentenceList}

is used to demonstrate how
grammatical principles follow from cognitive \q{construals} of the relevant situations,
those which language seeks to describe or takes as presupposed context.\footnote{For example, \cite[pp. 119 and 128]{LangackerFoundations},
discussed by \cite[p. 189]{LineBrandt}, and \cite[p. 9]{EstherPascual}.}
In particular, Langacker argues that \q{students} and \q{question} can both be either singular or
plural: syntax is open-ended here, with neither form more evidently correct.  Langacker uses this
example to make the Cognitive-Linguistic point that
we assess syntactic propriety relative to cognitive frames and conversational context.  In this
specific case, we are actually working with two different cognitive frames which are interlinked
{\mdash} on the one hand, we recognize distinct events consisting of a student asking a question, but
the speaker calls attention, too, to their recurrence, so the events can also be understood
as part of a single, larger pattern.  There are therefore two different cognitive foci, at two
different scales of time and attention, a \q{split focus} which makes both singular and plural
invocations of \q{student} and \q{question} acceptable.}

\p{Supplementing this analysis, however, we can additionally focus attention directly on
grammatical relations.  The words \i{student} and \i{question} are clearly linked as the subject and
object of the verb \i{asked}; yet, contrary to any simple presentation of rules,
no agreement of singular or plural is required between them (they can be singular and/or plural in
any combination).  Moreover, this anomaly is only in force due to the context established
by an initial phrase like \i{three times}; absent some such framing, the singular/plural
relation would be more rigid.  For example, \q{A student asked interesting questions} would
(in isolation) strongly imply \i{one} student asking \i{several} questions.  So the initial
\q{Three times} phrase alters how the subsequent phrase-structure is understood while remaining
structurally isolated from the rest of the sentence.  Semantically, it suggests a
\q{space builder} in the manner of Gilles Fauconnier or Per Aage Brandt
\cite{Fauconnier}; \cite{PerAageBrandt}, but we
need to supplement Mental Space analysis with a theory of how these spaces
influence syntactic acceptability, which would seem to be logically prior to the stage where Mental Spaces
would come in play.}

\p{The mapping of (\ref{itm:threetimes}) to a logical
substratum would be more transparent with a case like:

\begin{sentenceList}

\sentenceItem{} \swl{itm:threes}{Three students asked interesting questions.}{log}
\end{sentenceList}

(\ref{itm:threes}) is a more direct translation of the facts
which the original sentence conveys.  But this \q{more logical} example has different
connotations than the sentence Langacker cites; (\ref{itm:threetimes}) places the emphasis
elsewhere, calling attention more to the idea of something temporally drawn-out,
of a recurrence of events and a sense of time-scale.  The \q{more logical} sentence
lacks this direct invocation of time scale and temporal progression.}

\p{We can say that the \q{Three students} version is a more direct statement of fact, whereas
Langacker's version is more speaker-relative, in the sense that it elaborates more
on the speaker's own acknowledgment of belief.  The speaker retraces the steps of her
coming to appreciate the fact {\mdash} of coming to realize that the \q{interesting questions}
were a recurrent phenomenon and therefore worthy of mention.  By situating expressions
relative to cognitive processes rather than to the facts themselves, the sentence
takes on a structure which models the cognition rather than the states of affairs.
But this shift of semantic grounding from the factual to the cognitive also apparently
breaks down the logical orderliness of the phrase structure. \q{Three times}, compared
to \q{three students}, leads to a morphosyntactic choice-space which is
\q{underdetermined} and leaves room for speakers' shades of emphasis.}

\p{This is not an isolated example.  Many sentences can be provided with similar
phrase-structure complications, particularly with respect to singular/plural agreement.

\begin{sentenceList}

\sentenceItem{} \swl{}{Time after time, tourists (a tourist) walk(s) by this building
with no idea of its history.}[]{sco}[Time after time, tourists walk by this building
with no idea of its history.]

\sentenceItem{} \swl{}{The streets around here are confusing; often people (someone)
will ask me for directions.}[]{sco}[The streets around here are confusing; often someone
will ask me for directions.]

\sentenceItem{} \swl{itm:students}{Student after student came with their (his/her)
paper to complain about my grade(s).}[]{sco}[Student after student came with their
paper to complain about my grades.]

\sentenceItem{} \swl{itm:parents}{Student after student {\mdash} and their (his/her) parents
{\mdash} complained about the tuition increase.}[]{sco}[Student after student {\mdash} and their parents
{\mdash} complained about the tuition increase.]
]

On a straightforward phrase-structure reading, \i{student after student} reduces to an
elegant equivalent of \i{many students}, with the rhetorical flourish abstracted away
to a logical form.  But our willingness to accept both singular and plural agreements
(his/her/their parents, grades, papers) shows that clearly we don't simply substitute
\i{many students}; we recognize the plural as a logical gloss on the situation but
engage the sentence in a more cognitively complex way, recognizing connotations of temporal
unfolding and juxtapositions of cognitive frames.  The singular/plural underdeterminism
is actually a signification in its own right, a signal to the listener that the
sentence in question demands a layered cognitive attitude.  Here again, syntactic
structure (morphosyntactic, in that syntactic allowances are linked with
variations in the morphology of individual words, such as singular or plural form)
serves to corroborate conversants' cognitive frames rather than to model logical
form.]

\p{The contrast between the phrases \i{Student after student} and \i{Many students}
cannot be based on \q{abstract} semantics alone {\mdash} how the evident temporal implications of the
first form, for example, are concretely understood, depends on conversants' mutual recognition
of a relevant time frame.  The dialog may concern a single day, a school year, many
years.  We assume that the speakers share a similar choice of time \q{scale}
(or can converge on one through subsequent conversation). \i{Some} time-frame
is therefore presupposed in the discursive context, and the first phrase invokes
this presumed but unstated framing.  The semantics of the phrase are therefore somewhat open-ended:
the phrase \q{hooks into} shared understanding of a temporal cognitive framing without referring
to it directly.  By contrast, the second phrase is less open-ended: it is consistent with both
a more and less temporally protracted understanding of \i{many}, but leaves such details (whatever
they may be) unsignified.  The factual circumstance is designated with a level of abstraction that sets
temporal considerations outside the focus of concern.  The second (\q{\i{Many students}}) phrase 
is therefore both
less open-ended and also less expressive: it carries less detail but accordingly also relies
less on speaker's contextual understanding to fill in detail.\footnote{The examples I have used so far may also imply that a choice of phrase structure is
always driven by semantic connotations of one structure or another;
but seemingly the reverse can happen as
well {\mdash} speakers choose a semantic variant because its grammatic realization lends a useful
organization to the larger expression.  There are many ways to say \q{many},
for example: \i{a lot of},
\i{quite a few}, not to mention \q{time after time} style constructions.  Whatever their
subtle semantic variations, these phrases also have different syntactic properties:
\i{Quite a few} is legitimate as standalone (like an answer to a question);
\i{A lot of} is not, and \i{A lot} on its own is awkward.  On the other hand the \q{of} in
\i{A lot of} can \q{float} to be replicated further on: \q{A lot of students, of citizens,
believe education must be our top priority} sounds more decorous than the equivalent sentence with the
second \q{of} replaced by \q{and}.  If the cadence of that sentence appeals to the speaker, then
such stylistic preference will influence taking \q{A lot of} as the \q{many} variant of choice.
So speakers have leeway in choosing grammatic forms that highlight one or another aspect of
situations; but they also have leeway in choosing rhetorical and stylistic pitch.  Both cognitive
framings and stylistic performance can be factored when reconstructing what compels the
choice of one sentence over alternatives.}
}

\p{One consequence of these analyses is that grammar
needs to be approached holistically: the grammatic structure of phrases cannot,
except when deliberate oversimplification is warranted, be isolated from
surrounded sentences and still larger discourse units.  Semantic roles
of phrases have some effect on their syntax, but phrases are nonetheless chosen
from sets of options, whose variations reflect subtle semantic and syntactic
maneuvers manifest at super-phrasal scales.  The constituent words of phrases retain some
autonomy, and can enter into inter-word and phrasal structures with other words outside their
immediate phrase-context.  We can still apply formal models to phrase
structure {\mdash} for example, Applicative and Cognitive Grammar (\ACG{}) considers phrases
as \q{applications} of (something like) linguistic or cognitive functions,
with (say) an adjective modeled as a function applied to a noun,
to yield a different noun (viz., something playing a noun's conceptual role)
\cite{Descles2010} {\mdash} but we should not read these transformations
too hastily as a purely semantic correlation within a space of denotable concepts
\i{such that} the new concept wholly replaces the
contained parts, which then cease to have further linguistic role and effect.
Instead, applicative structures represent shifts or evolutions in
mental construal, which proceed in stages as conversants form
cognitive models of each others' discourse.  Even if phrase structure
sets landmarks in this unfolding, phrases do not wholly subsume their
constituents; the parts within phrases do not \q{vanish} on the higher scale,
but remain latent and may be \q{hooked} by other, overlapping phrases.}

\p{Consider the effect of \q{Many students complained}.  Propositionally, this appears
to say essentially that \i{students} complained; but, on hermeneutic charity, the
speaker had \i{some} reason to say \q{many}.  The familiar analysis is that
\q{many} suggests relative size; but this
is only half the story.  If the speaker chose merely \i{students complained}, we would hear an assertion
that more than one student did, but we would also understand that there were several
occasions when complaints happened.  Adding \q{many} does not just
imply \q{more} students, but suggests a mental shift away from the particular episodes.
In the other direction, saying \i{a student complained} is not just
asserting how at least one student did so, but
apparently reports one specific occasion (which perhaps the speaker wishes to
elaborate on).  In other words, we cannot really capture the singular/plural semantics,
or different varieties of plural, just by looking at the relative size of implied
sets; we need to track how representations of singleness or multitude imply
temporal and event-situational details.}

\p{Against this backdrop, \i{Student after student complained} captures both dimensions,
implying both a widespread unrest among the student body and also
temporal recurrence of complainings.
The \BlankAfterBlank{} idiom is a special case \visavis{} \q{after}, where 
the \q{argument} to
\i{after} is repeated in both positions, suggesting an unusual degree of repetition,
something frustratingly recurrent: \i{He went on and on}; \i{Car after car passed
us by}; \i{Time after time I got turned down}.
Although I have no problem
treating these constructions as idiomatic plurals, I also contend (on the
premise of phrase-overlap) that the dependent constituents in the \BlankAfterBlank{}
construction can be hooked to other phrases as well (which is why
\q{and [their/his/her] parents} can also be singular, in this case).  I dwell on
this example because it shows how type-theoretic accounts of phrase structure, 
which I will explore next section, 
can be useful even if we treat phrases more as frames which overlay linguistic
structure, not as rigid compositional isolates.  Each \q{students} variation uses
morphology to nudge cognitive attention in one direction or another, toward events or the
degree to which events are representative of some global property (here of
a student body), or both.  The pluralizing transformation {\mdash} from 
one student to many students {\mdash} is not \i{the}
morphosyntactic meaning, but instead the skeleton on which the full meaning
(via cognitive schema) is designed.}

\p{If this analysis has merit, it suggests that a Combinatory Grammar or type-logical
approach to phrases like \i{many students} or \i{student after student}
(singular-to-plural or plural-to-plural mappings) should be understood not just
as functions among Part of Speech (\POS{}) types but as adding cognitive shading, foregrounding
or backgrounding cognitive elements like events or typicality in some context.
In other words, \i{many students} is type-theoretically a pluralizing action, 
but, in more detail, it adds a kind of cognitive rider attached to the mapping which focuses
cognition in the subsequent discourse onto events (their recurrence and temporal distribution);
similarly \q{student after student} has a \q{rider} suggesting more of a temporal
unfolding.  The second form implies not only that many students complained, but that
the events of these complainings were spread out over some stretch of time.
Each such functional application (mappings between \POS{} understood as linguistic types)
produces not only a resulting \POS{} \q{type}, but also a reconfiguration of cognitive
attitudes toward the relevant situation and context.
Language users have many ways to craft a sentence with similar meanings, and arguably one
task for linguistic analysis is to model the space of choices which are available in a
given situation and represent what specific ideas and effects are invoked by one
choice over others.}

\whdecoline{}
\p{Logical semantic models build up propositional representations from the 
word level to phrase and sentence levels.  Even when we find that a logical
\i{telos} is in effect, however, at each of these levels we can 
find interpretive and situational details which may complicate, flesh out, 
or cognitively supersede a more transparent logical meaning.
So at the word-pair level, there is more going on in 
a transform like \i{rescued dog} than simply applying the 
predicate \q{rescue} to the ground \q{dog}.  The construction 
drafts a complex backstory that influences our cognizing both 
component words.  Scaling up, as I argued in the last several 
paragraphs, a phrase like \i{student after student} carries 
conceptual effects beyond just effectuating a logical 
operation of pluralization.  I will argue that similar 
effects can be identified at the sentence level {\mdash} so 
there are certain lacunae in narrowly logically-based 
reconstructions of language components at the 
distinct levels of word (transform) pairs, phrases, 
and whole sentences.  I'll sketch my arguments 
here and give a more thorough analysis later in 
Section \ref{sec:Gaps}.}

\spsubsectiontwoline{Logical Structure versus Sentence Structure}
\p{Let us grant in general that particular sentences can be 
mapped to distinct, relatively transparent propositional 
contents.  In some cases sentences expresses propositional 
attitudes to such content (requests, commands, questioning) 
rather than unadorned locutionary assertions.  To 
properly respond to speech-acts, however (even ones with 
illocutionary force) conversants need to derive the 
content which is logically conjured via the discourse,
either as the speaker's primary intent or as a condition 
for that intent.  In effect, a proposition like \i{the window 
is closed} furnishes logical content to assertions like
\i{The window is closed now} but also statements of 
belief (\i{I think the window is closed}) or 
requests or opinions (\i{The window should be closed};
\i{Please close the window}).}

\p{Philosophical \q{truth-theoretic} paradigms imply that such 
propositional contents are the \i{essential} meanings within 
language; that analyzing semantic forms via logical 
structure is the core of a rigorous theory of semantics.  
It is certainly true that many elements of language 
can be translated, or deemed as conventionalized encodings 
for, structures in predicate logic {\mdash} invocations of 
multiplicity and quantification; logical connectives between 
propositions; negations, modalities, and possibilia.  
This provides an analytic matrix wherein \i{some}
sentences' structures may be analyzed.  I will argue, 
however, that in typical cases logical forms are 
invoked only indirectly {\mdash} which calls into question 
the applicability of logical analysis as explanatory 
vehicles for \i{linguistic} analysis in itself 
(as opposed to more general cognitive/extralinguistic 
processing).}

\p{There are several cognitive operations requisite for grasping 
sentence-meanings as a logical gestalt: figuring individuals 
or multiplicities as conceptual foci (verb subjects or objects); 
establishing relationships between individuals and multiplicities 
or among multiplicities (member/part of, larger/smaller, 
overlap/disjoint); predicating properties to individuals 
or multiplicities; quantification; logical conjunction or disjunction, 
between predicates (also negation).  In some cases we can find 
these operations fairly directly encoded in explicit language form 
{\mdash} sentences which are precise in figuring multiplicities numerically, 
or through unambiguous use of determiners like \i{all} and \i{every}; 
which are structured to avoid scope ambiguities; which use transparent
semantic resources to describe verb subjects and objects; and 
so forth.  In the most recent Universal Dependencies Shared Task corpus 
we can find examples like:

\begin{sentenceList}

\sentenceItem{} \swl{itm:tumour}{It is the most common tumour found in babies, occurring in one of every 35,000 births.}{log}
\udref{en_pud}{n01051007}
\sentenceItem{} \swl{itm:Dengue}{Dengue fever is a leading cause of illness and death in the tropics and subtropics, with as many as 100 million people infected each year.}{log}
\udref{en_partut-ud-train}{en_partut-ud-1498}
\sentenceItem{} \swl{itm:TalibanKarzai}{Many Taliban living in Afghanistan voted for President Karzai.}{log}
\udref{en_ewt-ud-train}{w03002048}
\sentenceItem{} \swl{itm:Ashraf}{Most of the girls I was meeting had grown up in Mujahedeen schools in Ashraf, where they lived separated from their parents.}{log}
\udref{en_ewt-ud-train}{weblog-blogspot.com_rigorousintuition_20060511134300_ENG_20060511_134300-0112}
\sentenceItem{} \swl{itm:ChinaSpace}{Most experts believe China intends to develop a small space station of its own over the next several years.}{log}
\udref{en_ewt-ud-train}{newsgroup-groups.google.com_FOOLED_7554c5ce34a5a49e_ENG_20051012_144800-0020}
\sentenceItem{} \swl{itm:tastings}{Check out their wine tastings every Friday night!}{log}
\udref{en_ewt-ud-train}{reviews-322225-0005}
\sentenceItem{} \swl{itm:tag}{For each start tag , there is a corresponding end tag.}{log}
\udref{en_lines-ud-train}{166}
\sentenceItem{} \swl{itm:Warhol}{Each collection donated by the Andy Warhol Photographic Legacy Program holds Polaroids of well-known celebrities.}{log}
\udref{en_gum-ud-train}{GUM_news_warhol-35}
\end{sentenceList}

These sentences have straightforward logical structure, in terms of 
how they establish topical foci (\i{one of every 35,000 births};
\i{as many as 100 million people};
\i{Many Taliban}; \i{every Friday night}; \i{For each start tag}), 
and how predicates or references are bound together to create more 
precise significations (\i{the tropics and subtropics};
\i{in Mujahedeen schools in Ashraf}; \i{a corresponding end tag}).
Properties ascribed to subject foci are neatly drawn, both in 
conveying the property intended and its bearer, according to 
the sentence's terms: \i{the most common tumour found in babies};
\i{a leading cause of illness and death};
\i{China intends to develop a small space station};
\i{holds Polaroids of well-known celebrities}.  With 
aggregate foci and/or quantification, there is an unambiguous 
framing of predication and quantifier scope {\mdash} Each collection 
has its set of Polaroids; the set of Karzai voters, Dengue infections, 
birth tumours, etc., are crisply figured.}

\p{For many philosophers of language, identifying similar 
logical structuration is an intrinsic aspect of 
coming to terms with human language in general.  This paradigm 
also reinforces the goal of Artificial Intelligent Natural Language Processing, 
because computers can certainly engage in the kind of 
symbolic-logical reasoning outlining signified meanings in cases 
where language reciprocates propositional morphology very clearly.  
The problem is that language artifacts very often cloak their 
logical core, such that examples like (\ref{itm:tumour})-(\ref{itm:Warhol}) 
are not representative of language as a whole.  Logical patterns
may certainly be present, but they are not necessarily 
structurally reproduced in surface-level formations; rather a 
sentences' propositional content may depend on a subtle 
interpretive trajectory.  I will present examples throughout this 
paper, but a few further corpus items are reasonable 
case-studies:

\begin{sentenceList}

\sentenceItem{} \swl{itm:ants}{A furry black band of ants led up a cupboard door to some scrap that had flicked from a plate.}{sem}
\udref{en_lines-ud-train}{2157}
\sentenceItem{} \swl{itm:current-waiting-period}{The current waiting period is eight weeks.}{sem}
\udref{en_pud}{n01050009}
\sentenceItem{} \swl{itm:immersed}{I think that's why they immersed themselves in pattern and colour.}{sem}
\udref{en_pud}{n01087018}
\sentenceItem{} \swl{itm:princess}{With her appearance finalized, Jasmine became Disney's first non-white princess as opposed to being of European heritage.}{sem}
\udref{en_pud}{w01119076}
\end{sentenceList}

It requires a certain cognitive flexibility to understand a band of ants as \q{flurry}, or 
to parse the disjoint timeframes in \i{current waiting period}.  In (\ref{itm:immersed}), 
the presumed sense of \q{immerse} transcends any immediate, perceptual immersion, 
instead involving scholarship or engagement with artistic form; and (\ref{itm:princess}) 
depends on us understanding the meaning of temporality in Jasmine's appearance being
\i{finalized}, and also her \i{becoming} non-white.  As a fictional character, 
discourse about Jasmine can be evaluated in the time-frame of her artistic 
creation, distinct from the fictional time of her narrated world.}

\p{I think the intended propositional content in (\ref{itm:ants})-(\ref{itm:princess}) 
is no less evident than in (\ref{itm:tumour})-(\ref{itm:Warhol}); however, interpreting 
the topical foci and predicate attributions constituting such 
propositional content requires a holistic reading whose compositional 
structure is not recapitulated in the sentence-forms themselves.  
In the latter examples, then, merely notating propositional 
content in logical fashion does not yield a very informative
\i{linguistic} analysis, since it does not address the key question 
of \i{how} the sentences signify those propositions.}

\p{I propose to use the term \q{logicomorphic} for sentence in the former vein; 
in such cases, pointing out propositional content is linguistically 
useful because we can treat that content as a prototype for 
sentence organization.  That is, propositional content is not only
\i{holistically} signified but, in its logical structure, sheds 
light on pattern in the language.  The purpose of phraseology 
like \i{most common tumour}, \i{Many Taliban living in Afghanistan},
\i{Most of the girls I was meeting}, etc., is to circumscribe 
a focus or a property suitable for predication, and we can 
logically model the tools used to do so: logical superlative (\i{most common}), 
assertions of magnitude (\i{Many}, \i{Most of}), refining an multiplicity 
with some further criteria (\i{the girls I was meeting},
\i{Taliban living in Afghanistan}), and so forth.  These are
\q{logicomorphic} constructions in that we can read the logical 
structure of signified propositional content as a direct cause 
of the given phrasal morpholgy.}

\p{On the other hand, I call examples like (\ref{itm:ants})-(\ref{itm:princess})
\q{interpretive} because the sentences' propositional content, with 
its logical structure, does not explain the compositional 
rationale for the explicit linguistic form: we cannot read any 
pattern in the logic as a direct motivation for how the 
sentence is pieced together.  The spectrum between
\i{logicomorphic} and \i{interpretive} represents different 
strategies by which language is composed in anticipation of 
its cognitive reception, with the eventual goal of 
establishing a signified propositional content, but in different ways.  
On the logicomorphic side, logical form informs language directly; 
on the interpretive side, the actual rationale for 
compositional structures transcends exact predicative structure 
{\mdash} a more perceptual or indirect figuring of topical 
focus, for instance, or a more elliptical construal of 
predicate attributes, leaving the hearer to piece together the 
final propositional via some pragmatic or extralinguistic calculation.}

\p{Accordingly, the \i{logicomorphic}/\i{interpretive} distinction {\mdash} along with the 
overall contrast between linguistic and extralinguistic 
aspects of meaning {\mdash} are contrasts between sentences that become manifest 
in the compositional maxims evident at subsentence (phrasal and 
inter-word) scales.  We can apply all four criteria to 
estimate the cognitive as well as syntactic and semantic 
paradigms in effect for given inter-word pairs and phrasal structure; 
identifying sentences as logicomorphic or interpretive propagates down 
to how phrasal and interword patterns should be analyzed.  
With this in mind, having presented certain claims as to the 
holistic nature of sentences \visavis{} propositional content, 
I will now switch attention to the composition of 
sentences from the interword level upward.}

\p{}

\end{sentenceList}
}

%
\section{Functional Type Theory in Cognitive and Dependency Grammars}
\label{s2}
\p{My discussion so far has focused on 
characterizing sentences' holistic meaning.  On the face of 
it, such holistic analysis is more semantic than syntactic.  
However, syntactic paradigms can be grounded in theories 
of how language elements aggregate \i{toward} holistic meaning.}

\p{Here I propose a notion of \q{cognitive transforms}
{\mdash} that holistic meanings emerge from a series of 
interpretive and situational modeling modifications 
which progressively refine our understanding of a 
speaker's construal of our environing context and her 
propositional attitudes.  While elucidation of these 
transforms as cognitive phenomena may belong to semantics, 
syntactic structure dictates the \i{sequence} of 
transforms.  Many transforms are expressed by 
individual word-pairs.  Taking the temporal or logical 
order of transforms into consideration, we can derive a 
syntactic model of sentences by introducing an order among 
word-pairs {\mdash} a methodology akin to using Dependency Grammar 
parse-graphs as an intermediate stage, then ordering the 
graph-edges around an estimation of cognitive aggregation.  
One transform is a successor to a predecessor if the 
modifications induced by the predecessor are 
consequential for the cognitive reorientation pertinent to 
the successor, and/or to the morphosyntactic features which 
trigger it.}

\p{In this spirit I talk of Cognitive Transform \i{Grammar}, because 
while in the general case transforms are semantic and interpretive 
{\mdash} not the purview of grammar per se {\mdash} we can theorize 
grammar as governing the \i{order of precedence} among transforms.  
More precisely, there is a particular order of precedence germane 
to sentence meaning; sentences have their precise syntax 
in order to compel recipients' reception of the linguistic 
performance according to that same ordering.}

\p{From this perspective, an essential aspect of grammar theory is that 
whatever units are understood as syntactic constituents {\mdash} like 
phrase structure or word-pairs {\mdash} an order of precedence should
\q{fall out} of grammatic reconstructions.  We should be able 
to supplement parse-representations with a listing of salient 
syntactic features in order, retracing the \i{cognitive} steps 
by which localized sense-alterations synthesize into holistic 
meaning.  The details of this precedence-establishment 
will vary across grammatic paradigms, so one way to assess 
grammar theories is to consider how the engender corresponding 
cognitive-transform models.}

\p{Type theory can be adopted in this context because 
most versions of type theory include a notion of
\q{function-like} types: types whose instances modify 
instances of some other type (or types).  This 
establishes an order of precedence: anything modified 
by a function is in some sense logically prior to 
that function.  In formal (e.g., programming) languages, 
the procedure whose output becomes input to a different 
procedure must be evaluated before the latter procedure 
begins (or at least before the output-to-input value is 
used by that latter procedure).}

\p{A common paradigm is to consider natural-language types as generated 
by just two bases {\mdash} a noun type \N{} and a proposition type
\Prop{}, the type of sentences and of sentence-parts which are 
complete ideas {\mdash} having in themselves a propositional content 
(see e.g. \cite{BarkerShanTG} or \cite{KubotaLevine}).    
Different models derive new types on this basis in different ways.
One approach, inspired by mathematical \q{pregroups}, establishes 
derivative types in terms of word pairs {\mdash} an adjective followed 
by a noun yields another noun (a noun-phrase, but \N{} is the phrase's
\i{type}) {\mdash} e.g., \i{rescued dogs}, like \i{dogs}, is conceptually 
a noun.  Adjectives themselves then have the type of words which form 
nouns when paired with a following noun, often written as
\NOverN{}.  Pregroup grammars distinguish left-hand and right-hand 
adjacency {\mdash} \i{bark loudly}, say, demonstrates an adverb \i{after} a
verb, yielding a verb phrase: so \q{loudly} here has the type of a 
word producing a verb in combination with a verb to its \i{left}
(sometimes written \VUnderV{}); by contrast adjectives combine with 
nouns to their \i{right}.}

\p{A related formalization, whose formal 
analogs lie in Typed Lambda Calculus, abstracts from left-or-right 
word order to models derived types as equivalent (at least 
for purposes of type attribution) to \q{functions}.  
This engenders a fundamental notion of functional \i{application}
and operator/operand distinctions:

\begin{quote}

\i{Categorial Grammars} make the connection between the 
first and the second level of the ACG.  These models are 
typed applicative systems that verify if a linguistic 
expression is syntactically well-formed, and then construct 
its functional semantic interpretation.  They use the
\i{fundamental operation of application} of an operator to an 
operand.  The result is a new operand or a new operator. 
This operation can be repeated as many times as necessary 
to construct the linguistic expressions' functional semantic 
representation.
\cite[p. 1]{Rossi}
\end{quote}

So, e.g., an adverb becomes a function which takes a verb and produces another verb; 
an adjective takes a noun and produces another noun; 
and a verb takes a noun and produces a proposition.  By \q{function} we can consider 
some kind of conceptual transformation: \i{loudly} transforms the
concept \i{bark} into the concept \i{loud bark}.  
Assuming all lexemes are assigned a Part of 
Speech drawn from such a type system, the definition of 
functional types directly yields a precedence order: 
instances of functional types are functionally dependent 
on their inputs, which are therefore precedent to them.  
On this basis, any well-typed functional expression has a 
unique precedence ordering on its terminal elements 
(i.e., its \q{leaves} when the expression is viewed as a 
tree, or its nodes when viewed as a graph), which 
can be uncovered via a straightforward algorithm 
(one implementation is part of this paper's data set; 
see the \q{parse\_sxp} method in file ntxh-udp-sentence.cpp).}

\p{Functional parts of speech 
that can be formally modeled with one \q{argument}
(the most common case),  
having a single input and output type, 
conveniently lend themselves 
to cognitive transforms defined through word 
pairs {\mdash} an adjective modifies a noun to another 
noun, an adverb maps a verb to a verb, an auxiliary 
like \i{that} or \i{having} can map verbs or propositions 
to nouns, and so forth.  The only main complication 
to this picture is that verbs, which typically have subjects 
as well as objects, can take two or three \q{inputs}
instead of just one.  Instead of a transform \i{pair} we 
can then consider a three- or four-part transform structure 
(verb, subject, direct object, indirect object).  
We can still assign a precedence ordering to verb-headed phrases, 
however, perhaps by stipulating that the subject takes 
precedence before the direct object, and the direct object
before the indirect.  This ordering seems cognitively 
motivated: our construal of the significance of a 
direct object appears to intellectually depend on the verb's 
subject; likewise the indirect object depends on the direct 
object to the degree that it is rationally consequential.}

\p{A secondary complication involves copulae like \i{and}, which 
can connect more than two words or clauses.  Here, though, a 
natural ordering seems to derive from linear position in the 
sentence: given \i{x, y, and z} we can treat \i{x} as 
precedent to \i{y}, and \i{y} to \i{z}, respectively.}

\p{In total, sentences as a whole can thus be seen as structurally 
akin to nested expressions in lambda calculi
(and notated via \q{S-Expressions}, like code in the 
Lisp programming language).  S-Expressions are occasionally 
recommended as representations for some level of 
linguistic analysis (cf. \cite{SivaReddy}, \cite{KiyoshiSudo},
\cite{ChoeCharniak}), and if this form by itself may add little 
extra data, it does offer a succinct way to capture the 
functional sequencing attributed to a sentence during analysis.  
Given, say,

\begin{sentenceList}

\sentenceItem{} \swl{itm:ambience}{The city's ambience is colonial and the climate is tropical.}{sem}
\udref{en_gum-ud-train}{GUM_voyage_merida-20}
\end{sentenceList}

the gloss \gl{(and (is ((The ('s city)) ambience) colonial) (is (the climate) tropical))}
summarizes analytic commitments with regard to the root structure of the sentence 
(in my treatment the copula is the overall root word) and to precedence between words 
(which words are seen as modifiers and which are their ground, for instance).  
So even without extra annotations (without, say, the kind of tagging data included 
by treebanks using S-Expression serializations), rewriting sentences as 
nested expressions captures primitive but significant syntactic details.}

\p{Nested-expression models also give rise directly to two other representations: 
a precedence ordering among lexemes automatically follows by taking 
function inputs as precedent to function (words) themselves\footnote{But 
note that using \q{function-words} as terminology here generalizes 
this term beyond its conventional meaning in grammar.}; 
moreover, S-Expression formats can be rewritten as sets of 
word-pairs, borrowing the representational paradigms (if not 
identical structures) of Dependency Graphs.  This allows Dependency Graphs 
and S-Expressions to be juxtaposed, which I will discuss in the remainder of this section.}

\subsection{Double de Bruijn Indices}
\p{Assume then that all non-trivial sentences are nested expressions, and that 
all lexemes other than nouns are notionally \i{functions}, which take typed
\q{inputs} and produce typed \q{outcomes}.  Expression \q{nesting} means 
that function inputs are often outcomes from other functions (which establishes a 
precedence order among functions).  Since there is an obvious notion of
\q{parent} {\mdash} instances of functional types are parents of the words or 
phrase-heads which are their inputs {\mdash} nestable expressions are formally 
trees.  Via tree-to-graph embedding, they can also be treated as graphs, 
with edges linking parents to children; since parse-graphs are canonical 
in some grammar theories (like Link and Dependency grammar), it is 
useful to consider the graph-style representation as 
the intrinsic structure of linguistic glosses based on S-Expressions.  
That is, we want to define a Category of labeled graphs 
each of whose objects is isomorphic to an S-Expression (using this 
terminology in the sense of mathematical Category Theory); equivalently, a 
bijective encoding of S-Expressions within labeled graphs given 
a suitable class of edge-labels.}

\p{Indeed, labels comprised of 
two numbers suffice, generalizing the lambda-calculus
\q{de Bruijn Indices}.  The de Bruijn notation is an alternative 
presentation of lambda terms using numeric indices in lieu of 
lambda-abstracted symbol (cf., for a linguistic
or discourse-representation context, \cite{JanVanEijck}
and \cite{RikVanNoord}).  The \i{double} indices
accommodate the fact that, in the general case, the 
functional component of an expression may be itself a nested 
expression, meaning that \q{evaluation} has to proceed in several 
stages: a function (potentially with one or more inputs) is 
evaluated, yielding another function, which is then applied 
to inputs, perhaps again yielding a function applied to still 
more inputs, and so forth.  I use the term \q{evaluate} which 
is proper to the computer-science context, but in linguistics 
we can take this as a suggestive metaphor.  More correctly, 
we can say that a function/input structure represents a
cognitive transform which produces a new function 
(i.e., a phrase with a function-like part of speech), that is
then the modifier to a new transform, and so forth.  
In general, the result of a transform can either be 
the \i{ground} of a subsequent transform, which is akin to 
passing a function-result to another function; or 
it can be the \i{modifier} of a subsequent transform, which is akin to 
evaluating a nested expression to produce a new function, then 
applied to other inputs in turn.}

\p{For a concrete example, consider

\begin{sentenceList}

\sentenceItem{} \swl{itm:actually}{The most popular lodging is actually camping on the beaches.}{sem}
\udref{en_gum-ud-train}{GUM_voyage_socotra-16}
\end{sentenceList}

with gloss, as I take it, \gl{((actually is) ((The (most popular)) lodging) ((on (the beaches)) camping))}.  
Here the adverb \i{actually} is taken as a modifier to \i{is}, so we imagine that 
interpreting the sentence involves first refining \i{is} into \i{is actually}, 
yielding a new verb (or \q{verb-idea}) that then participates in a verb-subject-object 
pattern.  Hence, the parse opens with the evaluation \gl{(actually is)} in the 
head-position of the sentence's top-level expression.  Similarly, I read \i{on the beaches} as 
functionally a kind of adverb, like \q{outside} in \i{camping outside}.  
In the generic pattern, a verb can be paired with a designation of location 
to construct the idea of the verb happening at such location; the designation-of-location 
is then a modifier to the verb's ground.  When this designation is a locative 
construction, the whole expression becomes a modifier, while it also has its own 
internal structure.  In \i{on the beaches}, \i{on} serves as a modifier which 
maps or reinterprets \i{the beaches} to a designation of place.}

\p{So here is the 
unfolding of the phrase: in \i{the beaches} the determinant (\i{the}) is a modifier 
to the ground \i{beaches}, signifying that \i{beaches} are to be circumscribed 
as an aggregate focus.  Then \i{on} modifies the outcome of that first transform, 
re-inscribing the focus as a place-designation.  Then \i{that} transform's output 
becomes a modifier for \i{camping}, wherein the locative construction becomes 
a de-facto adverb, adding detail to the verb \i{camping} (camping on the 
beach as a kind of camping, in effect).\footnote{If it seems better to read camping as a \i{noun} {\mdash} the act or phenomenon 
of camping {\mdash} then we could treat the locative 
as an \i{adjective}, with the stipulation that the operation 
converting verbs to nouns (from \i{X} to the phenomenon, act, or 
state of \i{X}-ing) 
propagates to any modifiers on the verb: modifying constructions that 
refine \i{X} as verb are implicitly mapped to be adjectives likewise 
modifying \i{X} in the nominal sense of \q{the phenomenon of \i{X}-ing}.}
}

\p{Notice in this review that \i{the} as modifier in \i{the beaches} yield 
a pair whose outcome is the \i{ground} for \i{on}.  If we take the 
modifier as representative for a modifier-ground pair, \i{the} is the
\i{modifier} in its own transform pair but then the \i{ground} in the 
subsequent pair; the pattern is modifier-then-ground.  However,
\i{on} is the modifier \visavis{} \i{the} and then \i{also} modifier
\visavis{} \i{camping}; the pattern is modifier-then-modifier.  
This latter case is the scenario where a lexeme will be a modifier 
on two or more different \q{levels}, giving rise to the \q{doubling}
of de Bruijn indices.  The first index, that is, represents 
the \q{level} tieing a modifier to a ground, while the second 
index is the \i{normal} notation of lambda-position.  
In \i{camping on the beaches}, the indices for the pair
\i{on}/\i{the} would be \gl{1,1} (meaning \i{the} is the first 
argument to \i{on} on the first transform level); the 
indices for \i{on}\i{camping} would be \gl{2,1} (\i{camping}
is the first argument to \i{on} on the \i{second} transform level).}


\begin{figure*}
\caption{Parse Graph with Dependency Relations as well as Double (Functional-Type) Indices}
\label{fig:actuallydbl}

\begin{dependency}[theme=brazil]
\begin{deptext}[column sep=14pt]
The \& most \& popular \& lodging \& is \& actually \& camping \& on \& the \& beaches \& . \\
DET \& ADV \& ADJ \& NOUN \& AUX \& ADV \& VERB \& ADP \& DET \& NOUN \& PUNCT \\
DT \& RBS \& JJ \& NN \& VBZ \& RB \& VBG \& IN \& DT \& NNS \& . \\
\end{deptext}

\depedge{1}{4}{det}
\depedge{2}{3}{advmod}
\depedge{3}{4}{amod}
\depedge{4}{7}{nsubj}
\depedge{5}{7}{aux}
\depedge{6}{7}{advmod}
\deproot{7}{root}
\depedge{8}{10}{case}
\depedge{9}{10}{det}
\depedge{10}{7}{obl}
\depedge[edge unit distance=2ex]{11}{7}{punct}

\deproot[edge below, edge theme=iron, label theme=grassy, edge unit distance=2.5ex]{6}{root} 
\depedge[edge below, edge theme=iron, label theme=grassy]{6}{5}{11} 
\depedge[edge below, edge theme=iron, label theme=grassy, edge unit distance=2ex]{6}{2}{21}
\depedge[edge below, edge theme=iron, label theme=grassy]{2}{3}{11} 
\depedge[edge below, edge theme=iron, label theme=grassy]{2}{1}{21} 
\depedge[edge below, edge theme=iron, label theme=grassy]{2}{4}{31} 
\depedge[edge below, edge theme=iron, label theme=grassy]{6}{8}{22} 
\depedge[edge below, edge theme=iron, label theme=grassy]{8}{9}{11} 
\depedge[edge below, edge theme=iron, label theme=grassy]{9}{10}{11} 
\depedge[edge below, edge theme=iron, label theme=grassy]{7}{6}{21}


\end{dependency}

((actually is) (((most popular) The) lodging) ((on (the beaches)) camping))

\begin{dependency}[theme=night,edge theme=copper,label theme=iron]
\begin{deptext}[column sep=14pt]
actually \& is \& most \& popular \& The \& lodging \& on \& the \& beaches \& camping \\
ADV \& AUX \& ADV \& ADJ \& DET \& NOUN \& ADP \& DET \& NOUN \& VERB \\
RB \& VBZ \& RBS \& JJ \& DT \& NN \& IN \& DT \& NNS \& VBG \\
\end{deptext}

\deproot[edge below, edge unit distance=3.5ex]{1}{root} 
\depedge[edge below]{1}{2}{11} 
\depedge[edge below]{1}{3}{21} 
\depedge[edge below]{3}{4}{11} 
\depedge[edge below]{3}{5}{21} 
\depedge[edge below]{3}{6}{31} 
\depedge[edge below, edge unit distance=2ex]{1}{7}{22} 
\depedge[edge below]{7}{8}{11} 
\depedge[edge below]{8}{9}{11} 
\depedge[edge below]{7}{10}{21} 


\end{dependency}

\end{figure*}


\p{By combining an index for \q{transform levels} {\mdash} capturing cases 
where a modifier produces an outcome which is a modifier again, not a
ground {\mdash} with an index for lambda position (e.g. the direct object 
has index 2 relative to the verb, and the indirect object has 
index 3), we can transform any expression-tree into a labeled graph. 
Parse graphs can then be annotated with these double-indices 
via the same presentations employed for Link or Dependency Grammar 
labels.  Sentence (\ref{itm:actually}) could be visualized as in 
Figure~\ref{fig:actuallydbl}, with the double-indices juxtaposed 
alongside conventional Dependency labels (the indices below the sentence, 
and relation labels above it); the upper parse is drawn from the 
Universal Dependency corpus (other annotated examples are included 
in this paper's downloadable data set).
%\input{figactually}
%\input{figambience}
}

\p{As a natural corollary to this notation,  
parts of speech can have \q{type signatures}
notionally similar to the signatures of function types in programming languages: a verb
needing a direct object, for example, \q{transforms} two nouns (Subject and Object)
to a proposition, which could be notated with something like \NNtoProp{}.\footnote{A note on notation: I adopt the Haskell convention (referring to the Haskell
programming language and other functional languages) of using arrows both between
parameters and before output notation, but for visual cue I add one dot above the
arrow in the former case, and two dots in the latter: \argsToReturn{}.
I will use \N{} and \Prop{} for the broadest designation of nouns and 
propositions/sentences (the broadest
noun type, respectively type of sentences, 
assuming we are using type-theoretic principles).  I will 
use some extra markings (in diagrams below) for more specific versions of 
nouns.}
The notation is consistent so long as each constituent of a verb 
phrase has a fixed index number.\footnote{The subject at position one, for instance; 
direct object at position two; and indirect object at position three.}
Transforms (potentially with two or more arguments) then combine lexemes 
{\mdash} having the right signatures {\mdash} with one or more words or phrases. 
Type analysis recognizes criteria on these combinations, insofar 
as the phrases or lexemes at a given index have a type consistent 
with (potentially a subtype of) the corresponding position in the 
head-word's signature.  Ideally, this representation can 
explain \i{ungrammatical} constructions via \i{failure} for 
these types to align properly.  If the combination \q{type-checks}, 
then we can assign the phrase as a whole the type indicated 
in the signature's output type {\mdash} \Prop{} in \NNtoProp{}, say; 
in this case the signature points out how \Prop{} derives from the 
phrase with two \N{}s as its components (along with the verb), 
a derivation we could in turn notate like \NNtoPropYieldsProp{}.   
A resulting phrase can then be included, like a nested expression, 
in other phrases; for instance a \Prop{} joined to
\i{that} so as to create a noun-phrase (recall my 
analysis of \i{question whether} last section); notationally
\PropToNYieldsN{}.\footnote{Numeric indices take the place of left and right adjacency
{\mdash} \q{looking forward} or backward {\mdash} in Combinatory Categorial 
Grammar; the type-theoretic perspective abstracts from word order.  
The theory of result types \q{falling out} from a 
type-checked phrase structure, however, carries over to this 
more abstract analysis.}
}

\p{Type \q{signatures} like \NNtoProp{} may 
seem little more than notational variants of conventional linguistic
wisdom, such as sentences' requiring a noun (-phrase) and a verb (\SeqNPplVP{}).
Even at this level, however, type-theoretic intuitions
offer techniques for making sense of more complex, layered sentences,
where integrating Dependency Graphs and phrase structures can be complex.
One complication is 
the problem of applying Dependency Grammar where phrases do not seem
to have an obviously \q{most significant} word for linkage with other phrases.}

\p{Often phrases are refinements of one 
single crucial component {\mdash} a phrase
like \i{many students} becomes in some sense collapsible to its semantic
core, \i{students}.  In real-world examples, however, lexemes tend 
to be neither wholly subsumed by their surrounding phrase nor wholly 
autonomous:

\begin{sentenceList}

\sentenceItem{} \swl{students}{Many students and their parents 
came to complain about the tuition hikes.}{sco}
\sentenceItem{} \swl{office}{Many students came by my office to complain about their grades.}{sco}
\sentenceItem{} \swl{after}{Student after student complained about the tuition hikes.}{sco}
\sentenceItem{} \swl{parents}{Student after student came with their parents to complain 
about the tuition hikes.}{sco}
\end{sentenceList}

In (\ref{students}) and (\ref{office}), we read \i{Many students} as topicalizing a multitude, 
but we recognize that each student has their own parents, grade, and we assume they came to 
the office at different times (rather than all at once).  So \i{students} links 
conceptually with other sentence elements, in a way that pulls it partly outside the
\i{Many students} phrase; the phrase itself is a space-builder which leaves open 
the possibility of multiple derived spaces.  This kind of space-building duality is 
reflected in how the singular/plural alternative is underdetermined in a multi-space 
context; consider Langacker's example:
\newsavebox{\Langackerboxi}
\begin{lrbox}{\Langackerboxi}
(repeating \ref{itm:threetimes})
\end{lrbox}
\begin{sentenceList}

\sentenceItem{} \swl{}{Three times, students asked an interesting question.}[\usebox{\Langackerboxi}]{sco}
\sentenceItem{} \swl{}{Three times, a student asked an interesting question.}{sco}
\end{sentenceList}

Meanwhile, in (\ref{after}) and (\ref{parents}) the phrase
\i{Student after student} invokes a multiplicity akin to \i{Many students}, 
but the former phrase has distinct syntactic properties; in particular
we can replace \i{their parents} (which is ambiguous between a plural and 
a gender-neutral singular reading) with, say, \i{his parents} (at a 
boy's school), a valid substitution in (\ref{after})-(\ref{parents})
but not (\ref{students})-(\ref{office}).}

\p{Cases like \i{Student after student} (or consider \i{time after time},
\i{year after year}, and so forth; this is a common idiomatic pattern 
in English) present a further difficulty for Dependency Grammar, 
since it is hard to identify which word of the three is the more 
significant, or the \q{head}.  Arguably Constituency Grammar is 
more intuitive here because then phrases as a whole can get linked 
to other phrases, without needing to nominate one word to proxy 
the enclosing phrase.  As I feel the \q{students} examples illustrated, however, 
it is too simplistic to treat phrases as full-scale replacements for 
semantic units, as if any phrase is an ad-hoc single lexeme 
(of course some phrases \i{do} get entrenched as de facto lexemes, 
like \i{Member of Parliament}, or my earlier examples
\i{red card} and \i{stolen base}).  
In the general case though component words retain some syntactic and 
semantic autonomy (entrenchment diminishes but does not entirely 
eliminate such autonomy).  There is, then, a potential dilemma:
phrases link to other phrases (sometimes via subsumption and 
sometimes more indirectly, as in anaphora resolution), but phrases 
are not undifferentiated units; lexemes, which on this sort of 
analysis \i{are} units, can be designated as proxies for their 
phrase; but then we can have controversy over which word in a 
phrase is the most useful stand-in for the whole 
(\cite{OsborneMaxwell} has an interesting review of a similar 
controversy in Dependency Grammar).}

\begin{figure*}[!h]
\caption{Dependency-style graph with type annotations}	
\label{fig:Iknow}
\vspace{1em}
\hspace{0.15\textwidth}	
\begin{minipage}{0.7\textwidth}	
\begin{tikzpicture}

%\draw

%\node [s1] at (0,0) {Student};

\node (I) at (1,1) {\textbf{I}};
\node (know) [right=9mm of I] {\textbf{know}};
\node (that) [right=9mm of know] {\textbf{that}};
\node (he) [right=12mm of that] {\textbf{he}};
\node (is) [right=10mm of he] {\textbf{is}};
\node (at) [right=10mm of is] {\textbf{at}};
\node (school) [right=5mm of at] {\textbf{school}};


\node (IRep) [double,draw=black,shape=circle,thick,fill=gray!50,inner sep=.5em,below=2cm of I] {};
\node (knowRep) [double,draw=black,shape=circle,thick,fill=gray!50,inner sep=.5em,below=1.5cm of know] {};
\node (thatRep) [double,draw=black,shape=circle,thick,fill=gray!50,inner sep=.5em,below=2cm of that] {};
\node (heRep) [double,draw=black,shape=circle,thick,fill=gray!50,inner sep=.5em,below=2.6cm of he] {};
\node (isRep) [double,draw=black,shape=circle,thick,fill=gray!50,inner sep=.5em,below=2.1cm of is] {};
\node (atRep) [double,draw=black,shape=circle,thick,fill=gray!50,inner sep=.5em,below=2.6cm of at] {};
\node (schoolRep) [double,draw=black,shape=circle,thick,fill=gray!50,inner sep=.5em,below=3cm of school] {};

\node (knowRepType) [below right = .2cm and -1.2cm of knowRep] 
 {\colorbox{yellow!20!red!30}{\scalebox{.7}{\NNtoProp}}}; 

\node (thatType) [below right = .2cm and -.8cm of thatRep] 
{\colorbox{yellow!20!red!30}{\scalebox{.7}{\PropToN}}}; 

\node (isRepType) [below right = .1cm and -1.4cm of isRep] 
{\colorbox{yellow!20!red!30}{\scalebox{.7}{\NNtoProp}}}; 

\node (atRepType) [below right = .15cm and -.9cm of atRep] 
{\colorbox{yellow!20!red!30}{\scalebox{.7}{\NtoN}}
}; 

\node (atRepTypeNote) [below right = .5cm and .1cm of atRep] {
	\footnotesize{(location)}
}; 


\draw [ |-,-|, <->, line width = .8mm, draw=gray!70, 
 dashed, double equal sign distance, >= stealth, shorten <= .25cm, shorten >= .25cm ]
 (I) to (IRep);

\draw [ |-,-|, <->, line width = .8mm, draw=gray!70, 
 dashed, double equal sign distance, >= stealth, shorten <= .25cm, shorten >= .25cm ]
(know) to (knowRep);
 
\draw [ |-,-|, <->, line width = .8mm, draw=gray!70,  
 dashed, double equal sign distance, >= stealth, shorten <= .25cm, shorten >= .25cm ]
(that) to (thatRep);

\draw [ |-,-|, <->, line width = .8mm, draw=gray!70, 
 dashed, double equal sign distance, >= stealth, shorten <= .25cm, shorten >= .25cm ]
(he) to (heRep);
 
\draw [ |-,-|, <->, line width = .8mm, draw=gray!70, 
dashed, double equal sign distance, >= stealth, shorten <= .25cm, shorten >= .25cm ]
(is) to (isRep);

\draw [ |-,-|, <->, line width = .8mm, draw=gray!70,  
dashed, double equal sign distance, >= stealth, shorten <= .25cm, shorten >= .25cm ]
(at) to (atRep);

\draw [ |-,-|, <->, line width = .8mm, draw=gray!70, 
dashed, double equal sign distance, >= stealth, shorten <= .25cm, shorten >= .25cm ]
(school) to (schoolRep);
 
 
\draw [shorten <= .25cm, shorten >= .25cm ] 
(knowRep) to node [draw=black,shape = star,star points=4,thick,inner sep = 0mm, above] {1} (IRep);

\draw [shorten <= .25cm, shorten >= .25cm ] 
(knowRep) to node [draw=black,shape = star,star points=4,thick,inner sep = 0mm,above] {2} (thatRep);

\draw [shorten <= .05cm, shorten >= .05cm, bend left=60] 
(thatRep) to node [draw=black,shape = star,star points=4,thick,inner sep = 0mm,above, 
near start] {3} (isRep);


\draw [shorten <= .15cm, shorten >= .15cm ] 
(isRep) to node [draw=black,shape = star,star points=4,thick,inner sep = 0mm,
 above, pos=0.4] {1} (heRep);

\draw [shorten <= .15cm, shorten >= .15cm ] 
(isRep) to node [draw=black,shape = star,star points=4,thick,inner sep = 0mm,above, 
 pos=0.4] {2} (atRep);

\draw [shorten <= .15cm, shorten >= .15cm ] 
(atRep) to node [draw=black,shape = star,star points=4,thick,inner sep = 0mm,above] {4} (schoolRep);


%\draw [shorten <= .25cm, shorten >= .25cm ] 
%(s1Rep) to node [draw=black,shape = star,star points=4,thick,inner sep = 0mm, below] {2} (s2Rep);

%\draw [shorten <= .5cm, shorten >= .5cm ] 
%(afterRep) edge [bend left=20,looseness=1] node [draw=black,shape = star,star points=4,thick,inner sep = 0mm, 
% above, near end ] {3} (complainedRep);

%\node (frameTopLeft) [below left = 1.5cm and -.65 cm of s1] {};
%\node (frameBottomLeft) [below = 2.5cm of frameTopLeft] {};
%\node (frameBottomRight) [right = 3.95cm of frameBottomLeft] {};
%\node (frameTopRight) [above = 2.5cm of frameBottomRight] {};

%\draw [shorten <= 0.15cm, shorten >= 0.15cm ] 
%(frameTopLeft) edge [bend right=30,looseness=1] (frameBottomLeft);

%\draw [shorten <= 0.15cm, shorten >= 0.15cm ] 
%(frameBottomLeft) edge [bend right=30,looseness=1] (frameBottomRight);

%\draw [shorten <= 0.15cm, shorten >= 0.15cm ] 
%(frameBottomRight) edge [bend right=30,looseness=1] (frameTopRight);

%\draw [shorten <= 0.15cm, shorten >= 0.15cm ] 
%(frameTopRight) edge [bend right=30,looseness=1]
%node [draw=black,shape = regular polygon,regular polygon sides=3,thick,inner sep = .2mm, 
%above, near start, shape border rotate = 180] {4} (frameTopLeft);


%node [draw=black,shape = star,star points=4,thick,inner sep = 0mm, above, 
%bend left=100,looseness=3] {3}

%;


\end{tikzpicture}
\end{minipage}

\hspace{0.1\textwidth}
\begin{minipage}{0.8\textwidth}
		\renewcommand{\labelitemi}{$\blacklozenge$}
	
\begin{itemize}\setlength\itemsep{-.3em}
\item 1 \hspace{12pt}  Verb's subject argument
\item 2 \hspace{12pt}  Verb's direct object argument
\item 3 \hspace{12pt}  Propositional \q{packaging} (\q{typed} as {}\PropToN{})
\item 4 \hspace{12pt}  Locative auxiliary link \\
(may be typed as converting nouns to place-designations) 
\end{itemize}
\end{minipage}
\end{figure*}

\p{Incorporating 
type theory, we can skirt these issues by modeling phrases through the perspective of
type signatures: given Part of Speech annotations for phrasal units and then for
some of their parts, the signatures of other parts, like verbs or adjectives
linked to nouns, or adverbs linked to verbs, tend to follow automatically.  
A successful analysis yields a formal tree, where if (in an act of semantic
abstraction) words are replaced by their types, the \q{root} type is something like
\Prop{} and the rest of a tree is formally a reducible structure in
Typed Lambda Calculus: \NNtoProp{} \q{collapses} to \Prop{}, \ProptoN{} collapses
to \N{}, and so forth, with the tree \q{folding inward} like a
fan until only the root remains {\mdash} though a more subtle analysis would
replace the single \Prop{} type with variants that recognize different
forms of speech acts, like questions and commands.
In Figure ~\ref{fig:Iknow},
this can be seen via the type annotations: from right to left \NtoN{} yields the
\N{} as second argument for \i{is}, which in turn yields a \Prop{} that is mapped
(by \i{that}) to \N{}, finally becoming the second argument to \i{know}.  This calculation
only considers the most coarse-grained classification (noun, verb, proposition) {\mdash} as I
have emphasized, a purely formal reduction can introduce finer-grained grammatical or
lexico-semantic classes (like \i{at} needing an \q{argument} which is somehow an expression
of place {\mdash} or time, as in \i{at noon}).  Just as useful, however, may be analyses
which leave the formal type scaffolding at a very basic level and introduce
finer type or type-instance qualifications at a separate stage.}

\p{In either case, Parts of Speech are modeled as (somehow analogous to) 
functions, but the important
analogy is that they have \i{type signatures} which formally resemble functions'.
Words with function-like types proxy their corresponding phrase, not because 
they are necessarily more important or are Dependency \q{heads}, but 
because they supply the pivot in the type resolutions which, 
collectively/sequentially, progress to a propositional culmination.  
This epistemological telos induces a sequencing on the 
type resolutions {\mdash} there is a fixed way that trees collapse {\mdash} 
which motivates the selection of function-words to proxy phrases; 
they are not semantically more consequential, necessarily, but 
are landmarks in a dynamic figured syntactically as \q{folding inward}
and semantically as a progressive signifying refinement.   
Phrases are modeled via a \q{function-like} Parts of Speech along with one or more
additional words whose own types match its signature; the type calculations
\q{collapsing} these phrases can mimic semantic simplifications
like \i{many students} to \i{students}, but here the theory is explicit
that the simplification is grammatic and not semantic: the collapse
is acknowledged at the level of \i{types}, not \i{meanings}.  In addition,
tree structures can be modeled purely in terms of inter-word relations 
{\mdash} as I have proposed here with double-indices {\mdash} 
so a type-summary of a sentence's phrase structure can be notated and
analyzed without leaving the Link or Dependency Grammar paradigm.}

\p{In sum, then, function-like words can always be represented as 
the \q{head} of corresponding phrases, but this implies 
neither greater semantic importance nor that phrases are 
conceptual units that fully subsume their parts.  Instead, 
a \q{head-function} notation captures the idea that 
sentence-synthesis is bounded by considerations of 
conceptual integrity that we can model, to some coarse 
approximation, via type theory.  Designating \i{after} as 
the \q{head} in \i{student after student} means that we have a 
type machinery that models (coarsely but formally) how successive cognitive refinement 
converges onto something with the conceptual profile of a 
proposition, and that we can leverage this formality by 
identifying certain words as functional pivots in the 
synthesis-toward-proposition: these words are not necessarily 
pivotal in term of meaning, but they are the core skeleton of
the schematic type-based model which we attach to a 
sentence while modeling the constraints on its synthesizing 
process.  (There is a further dimension in the \i{student after student}
phrase {\mdash} the repetition of \q{student} {\mdash} which I will discuss later.)}

\p{Or at least, this is one area of analysis where type theory 
is relevant for linguistics.  I will argue that there are 
several different methodologies where linguists have 
turned to type theory, and moreover that they can 
be integrated into a unified picture organized around 
the granularity of types themselves, according to the 
relevant theories.}

\spsubsectiontwoline{Three tiers of linguistic type theory}

\p{When explaining grammaticality as type-checking {\mdash} the 
concordance between function-like signatures and 
word or phrase \q{arguments} {\mdash} types are 
essentially structural artifacts; their 
significance lies in the compositional patterns 
guiding phrases to merge into larger phrases in a 
well-ordered way {\mdash} specifically, that the \q{outermost}
expression, canonically the whole sentence, is 
type-theoretically a proposition.  I proposed earlier that 
sentence-understanding be read as an accretion of detail 
culminating in a complete idea; type-checking then imposes 
regulatory guidelines on this accretion, with each 
constituent phrase being an intermediate stage.  Assigning 
types to phrases presents a formal means of checking that 
the accretion stays on track to an epistemological 
telos {\mdash} that the accumulated detail will eventually 
cohere into a propositional whole, a trajectory formally 
captured by the progressive folding-inward of phrase types 
to a propositional root.}

\p{Types themselves are therefore partly structural fiats 
{\mdash} they are marks on intermediate processing stages 
embodying the paradigm that type-checking \i{captures}
the orderliness of how successive cognitive transforms 
accrue detail toward a propositionally free-standing end-point.  
At the same time, types also have semantic interpretations; 
the \N{}/\Prop{} distinction, for example, is motivated by 
the cognitive difference between nominals and states of affairs 
as units of reason.  Type-theoretic semantics allows the 
structural paradigm of type-checked resolutions, the 
tree \q{folding inward} onto its root, to be merged with 
a more semantic or conceptual analysis of types qua 
categories or classifications of meanings (or of 
units comprising meanings).  I have described this merger 
at a coarse level of classification, taking broad 
parts of speech as individual types, but similar 
methods apply to more fine-grained analysis.
By three \q{tiers} of linguistic organization, I am thinking of
different levels of granularity, distinguished by relative scales of
resolution, amongst the semantic implications of
putative type representations for linguistic phenomena.}

\p{From one perspective, grammar is just a
most top-level semantics, the primordial Ontological division of language into designations of
things or substances (nouns), events or processes (verbs), qualities and attributes (adjectives),
and so forth.  Further distinctions like count, mass, and plural nouns add
semantic precision but arguably remain in the orbit of grammar (singular/plural
agreement rules, for example); the question is whether semantic detail gets
increasingly fine-grained and somewhere therein lies a \q{boundary} between syntax and
semantics.  The mass/count distinction is perhaps a topic in grammar more so than
semantics, because its primary manifestation in language is via agreement
(\i{some} wine in a glass; \i{a} wine that won a prize; \i{many} wines
from Bordeaux).  But what about distinctions between natural and constructed objects,
or animate and inanimate kinds, or social institutions and natural
systems, matters more of grammar or of lexicon?}

\p{For example, the template \i{I believed
X} generally requires that \i{X} be a noun
(\qmarkdubious{}\i{I believed run}), but more narrowly a
certain \i{type} of noun, something that can be interpreted
as an idea or proposition of some kind (\qmarkdubious{}\i{I believed flower}).
Asher and Pustejovsky point out the anomaly in a sentence
like \q{Bob's idea weighs five pounds}
\cite[example 2, p. 5]{AsherPustejovsky}, which
possesses a flavor of unacceptability that feels akin to
Part of Speech errors but are not in fact syntactic
errors.  The object of \i{weigh} is \q{five pounds} and
its subject is \q{Bob's idea}, which is admissible
\i{syntactically} but fails to honor our semantic convention
that the verb \q{to weigh} should be applied to things
with physical mass (at least if the direct object denotes a quantity;
contrast with \i{Let's all weigh Bob's idea}, where the
\i{idea} is object rather than subject).  These conventions are
analogous to Part of Speech rules but more fine-grained:
there is a meaning of \i{weigh} which has (like any transitive
verb) to be paired with a subject and object noun, but beyond
just being nouns the subject must be a physical body
(in effect a sub-type of nouns) and the object a quantitative
expression (another sub-type of nouns).  Potentially, type
restrictions on a coarse scale (e.g. that the subject of a verb
must be a noun) and those on a finer scale (as in this
sense of \i{to weigh}) can be unified into an overarching theory,
which spans both grammar and semantics {\mdash} for instance,
both Part of Speech rules and usage conventions of the
kind often subtly or cleverly subverted in metaphor and
idioms (see \i{flowers want sunshine}, \i{my computer died},
\i{neutrinos are sneaky}, as rather elegantly compactified
by assigning sentient states to inert things).  This is one way of
reading the type-theoretic semantic project.}

\p{Certainly \q{Ontological} qualities of signifieds engender
agreements and propriety which appear similar to
grammatic rules. \i{The tree wants to run away from the dog} sounds wrong {\mdash} because
the verb \i{want}, suggestive of propositional attitudes, seems incompatible with
the nonsentient \i{tree}.  Structurally, the problem with this sentence seems analogous
to the flawed \i{The trees wants to run away}: the latter has incorrect singular/plural linkage,
the former has incorrect sentient/nonsentient linkage, so to speak.  But does this
structural resemblance imply that singular/plural is as much part of semantics as grammar, or
sentient/nonsentient as much part of grammar as semantics?  It is true that there are no
morphological markers for \q{sentience} or its absence, at least in English {\mdash} except
perhaps for \q{it} vs. \q{him/her} {\mdash} but is this an accident of English or revealing
something deeper?}

\p{In effect, assessments of propriety seem to operate on several levels.  First 
(not to imply an actual temporal priority, though) 
we may consider fine-grained word-sense: which lexical entrant for 
such-and-such word is plausible in the current context?  Then 
we can consider larger-scale Ontological criteria: is the subject 
of this sentence figured as material or immaterial, 
sentient or nonsentient, natural or sociocultural, spatially and/or 
temporally extended or pointwise; and so forth?  And finally, 
what Part of Speech is consistent for various words given syntactic
principles and morphological cues {\mdash} distinguishing noun/verb/adjective, 
and etc., along with singular/plural, mass/count, verb tense and case, 
and other criteria of morphosyntactic fit?}

\p{So type-related observations can be grouped (not necessarily
exclusively or exhaustively) into those I will call
\i{macrotypes} {\mdash} relating mostly to Parts of Speech and the functional treatment
of phrases as applicative structures; \i{mesotypes} {\mdash} engaged with
existential/experiential qualities and \q{Ontological} classifications
like sentient/nonsentient, rigid/nonrigid, and
others I have discussed; and \i{microtypes} {\mdash} related to lexemes and word-senses.
This lexical level can include \q{microclassification}, or
gathering nouns and verbs by the auxiliary prepositions they allow and
constructions they participate in (such as, different cases), and
especially how through this they compel various spatial and
force-dynamic readings; their morphosyntactic resources for describing states
of affairs; and, within semantics, when we look toward even more fine-grained classifications
of particular word-senses, to reason through contrasts in usage.\footnote{So, conceiving microclasses similar in spirit to Steven Pinker in
Chapter 2 of \cite{Pinker}, though I'm not committing to using the
term only in the way Pinker uses it.  Cf. also \cite{AnneVilnat}, which
combines a microclass theory I find reminiscent of \i{The Stuff of Thought} with
formal strategies like Unification Grammar.} Microclasses can point out similarities
in mental \q{pictures} that explain words' similar behaviors, or
study why different senses of one word succeed or fail to be acceptable in particular phrases.
There are \i{stains all over the tablecloth} and \i{paint splattered all over the tablecloth},
but not (or not as readily) \i{dishes all over the tablecloth}.  While \q{stains} is count-plural and
\q{paint} is mass-aggregate, they work in similar phrase-structures because both
imply extended but not rigid spatial presence; whereas \q{dishes} can work for
this schema only by mentally adjusting to that perspective, spatial construal shifting
from visual/perceptual to practical/operational (we might think of dishes \q{all over} the
tablecloth if we have the chore of clearing them).  Such observations support
microclassification of nouns (and verbs, etc.) via Ontological and
spatial/dynamic/configuration criteria.}

\p{Type-theoretic semantics can also apply Ontological tropes to unpack the overlapping mesh of word-senses,
like \i{material object} or \i{place} or \i{institution}.
This mode of analysis is especially well illustrated when competing senses
collide in the same sentence.  Slightly modifying two examples:\footnote{

\cite[p. 40]{ChatzikyriakidisLuo} (former) and
\cite[p. 4]{MeryMootRetore} (latter).}

\begin{sentenceList}

\sentenceItem{} \swl{}{The newspaper you are reading is being sued.}{lex}
\sentenceItem{} \swl{itm:Liverpool}{Liverpool, an important harbor, built new docks.}{lex}
\end{sentenceList}

Both have a mid-sentence shift between senses, which is analyzed
in terms of \q{type coercions} (see also \cite{ZhaohuiLuo}
and \cite{ZhaohuiLuoSignatures}).
The interesting detail of this treatment
is how it correctly predicts that such coercions are not guaranteed to
be accepted:

\begin{sentenceList}

\sentenceItem{} \swl{}{The newspaper fired a reporter and fell off
the table.}[(?)]{lex}
\sentenceItem{} \swl{}{Liverpool beat Tottenham and built new docks.}[(?)]{lex}
\end{sentenceList}

(again, slightly modifying the counter-examples).  Type coercions are
\i{possible} but not \i{inevitable}.  Some word-senses \q{block} certain coercions
{\mdash} that is, certain sense combinations, or juxtapositions, are disallowed.
These preliminary, motivating analyses carry to more
complex and higher-scale types, like plurals (the plural of a type-coercion
works as a type-coercion of the plural, so to speak).
As it becomes structurally established that type rules at the
simpler levels have correspondents at more complex levels, the use of
type notions \i{per se} (rather than just \q{word senses} or other
classifications) becomes more well-motivated.}

\p{Clearly, for example,
only certain kinds of agents may have beliefs or desires, so
attributing mental states forces us to conceive of their referents
in those terms:

\begin{sentenceList}

\sentenceItem{} \swl{}{Liverpool wants to sign a left-footed striker.}{ont}
\sentenceItem{} \swl{}{That newspaper plans to fire its editorial staff.}{ont}
\end{sentenceList}

This \i{can} be analyzed as \q{type coercions}; but the type-theoretic machinery should contribute
more than just obliquely stating linguistic wisdom, such as
maintaining consistent conceptual frames or joining only suitably
related word senses.  The sense of \i{sign} as in \q{employ to play on
a sports team} can only be linked to a sense of Liverpool as the
Football Club; or \i{fire} as in
\q{relieve from duty} is only compatible with newspapers as
institutions.  These dicta can be expressed in multiple ways.
But the propagation of classifications
(like \q{inanimate objects} compared to
\q{mental agents}) through complex type structures lends credence to the
notion that type-theoretic perspectives are more than just an expository tool;
they provide an analytic framework which integrates grammar and semantics, and
various scales of linguistic structuration.
For instance, we are prepared to accept some examples of dual-framing
or frame-switching, like thinking of a newspaper as a physical object and a city government
(but we reject other cases, like \i{Liverpool voted in a new city government and signed a
new striker} {\mdash} purporting to switch from the city to the Football Club).  The rules for
such juxtapositions appear to reveal a system of types with some parallels to
those in formal settings, like computer languages.}

\p{In short, \q{Ontological} types like \i{institution} or \i{place} serve in some
examples to partition senses of one multi-faceted word.  Here they reveal
similar cognitive dynamics to reframing-examples like \i{to the press}, where
Ontological criteria (like reading something as a place) are triggered by
phrase-scale structure.  But there are also interesting contrasts:
the \i{newspaper} and \i{Liverpool} examples
imply that some words have multiple framings which are well-conventionalized;
newspaper-as-institution feels less idiomatic and metaphorical than
press-as-place.  So these examples suggest two \q{axes} of variation.
First, whether the proper Ontological framing follows from other word-choices
(like \q{fire} in \i{the newspaper fired the reporter}, which has
its own semantic needs), or from morphosyntax
(like the locative in \i{to the press}); and, second, whether triggered framings work
by selecting from established word senses or by something more metaphorical.
Metaphors like \i{to the press} do have an element of standardization;
but apparently not so much so to be distinct senses: note how \i{the press} as metaphorical place
does not work in general: \qmarkdubious{}\i{at the press}, \qmarkdubious{}\i{near the press}
(but \i{at the newspaper}, \i{near the newspaper}
{\mdash} imagine two journalists meeting outside the paper's offices {\mdash} sound quite reasonable).}

\p{The \q{type coercion} analysis works for mid-sentence frame-shifts; but other
examples suggest a more gradual conceptual \q{blending}.  For example, the
place/institution dynamic is particularly significant for \i{restaurant}
(whose spatial location is, more so, an intrinsic part of its
identity).  Being a \i{place} implies both location and extension; most places are not single
points but have an inside where particular kinds of things happen.  I am not convinced
that restaurant as place and as institution are separate word senses; perhaps, instead,
conversations can emphasize one aspect or another, non-exclusively.  
We need not incorporate all framing effects via \q{subtypes} (restaurant as either
subtype of hypothetical \q{types of all} places or institutions, respectively).  But
\q{placehood}, the Ontological quality of being a place {\mdash} or analogously being
a social institution {\mdash} identify associations that factor into cognitive frames; types
can then be augmented with criteria of tolerating or requiring one association or another.
So if \q{restaurant} is a type, one of its properties is an institutionality that \i{may}
be associated with its instances.  In conversation,
a restaurant may be talked about as a business or community, foregrounding this
dimension.  Or (like in asking for directions) its spatial dimension may be foregrounded.
The availability of these foregroundings is a feature of a hypothetical restaurant type,
whether or not these phenomena are modeled by subtyping or something more sophisticated.
The \q{newspaper} examples suggest how Ontological considerations
clearly partition distinct senses marked by properties like objecthood or
institutionality (respectively).  For \q{newspaper} the dimensions are less available for
foregrounding from a blended construal, than \q{unblended} by conventional usage; that
is why reframings evince a type \i{coercion} and not a gentler shift of emphasis.
The example of \i{restaurant}, in contrast, shows that competing routes for
cognitive framing need not solidify into competing senses, though they trace
various paths which dialogs may follow.
But both kinds of examples put into evidence an underlying
cognitive-Ontological dynamic which has potential type-oriented models.}

\p{At the most general level {\mdash} what I called \i{macrotype} modeling {\mdash} a type
system recognizes initially only the grammatical backbone of expressions, and
then further type nuances can be seen as shadings and interpretations which add substance
to the syntactic form.  So in type-theoretical analysis at this more grammatic level,
we can still keep the more fine-grained theory in mind:
the relation of syntax to semantics is like the relation of a spine to its flesh,
which is a somewhat different paradigm than treating syntax as a logical or temporal
stage of processing.  Instead of a step-by-step algorithm where grammatical parsing
is followed by semantic interpretation, the syntax/semantics interface can be seen
as more analogous to stimulus-and-response: observation that a certain grammatic
configuration appears to hold, in the present language artifact, triggers a marshaling
of conceptual and cognitive resources so that the syntactic backbone can be filled in.
Perhaps a useful metaphor is grammar as gravitation, or the structure of a gravitational
field, and semantics is like the accretion of matter through the interplay of multiple
gravitational centers and orbits.  For this analogy, imagine typed lambda
reductions like \PropToNYieldsN{} taking the place of gravitational equations;
and sentences' grammatic spine taking the place of curvature pulling mass into a planetary center.}

\p{As I have argued, sentences' progression toward complete ideas can be 
assessed more semantically {\mdash} accretion of conceptual detail {\mdash} 
or more syntactically, in terms of regulated type resolutions pulling in 
from a tree's leaves to its root.  The latter model is a kind of 
schematic outline of the former, marking signposts in the accretion 
process rather like a meetings' agenda.  Type theory allows points in 
conceptual accretion to be selected {\mdash} corresponding to nested phrases 
{\mdash} where type-checking signals that the accretion is progressing 
in an orderly fashion.  Or, more precisely, type-checking acts 
as a window on a cognitive process; phrasal units are like 
periodic gaps in a construction wall allowing us to reconstruct interpretive 
processes, and the possibility of certain linguistic elements being 
assigned types marks the points where such windows are possible.  
So type theory can impose a formal paradigm on our assessment 
of sentence structure, but at the cost of sampling only 
discrete steps of an unfolding completion toward understanding.  
In practice, this discrete analysis should be supplemented 
with a more holistic and interpretive paradigm, which explores 
{\mdash} perhaps speculatively, without demanding thorough 
formalization {\mdash} the gaps between the formalizable windows.  
I will transition toward this style of analysis in the next section.}

\thindecoline{}
\p{At the same time, I feel that the foundations of \q{cognitive}
linguistics deserve a little more attention.  I have used
\i{cognitive} rather informally, depending on the intuitive 
picture of \q{cognitive} linguistics, grammar, or indeed
\q{cognitive phenomenology}, which emerges from the 
speculative project we associate with linguists/philosophers 
like George Lakoff, Mark Johnson, Leonard Talmy, 
Ronald Langacker, and Peter \Gardenfors{} {\mdash} along with, as
\Gardenfors{} points out, phenomenologists like Jean Petitot.\footnote{ \Gardenfors{} mentions Lakoff, Langacker, Talmy, Fauconnier, 
and others alongside \q{a French semiotic tradition, 
exemplified by [Jean-Pierre] Descl\'{e}s ... and 
[Jean] Petitot-Cocorda ... which shares many features with the 
American (mainly Californian) group} \cite[p. 4]{Gardenfors}.} At the same time, Petitot also links with a 
tradition that combines elements of phenomenology and 
Analytic Philosophy, represented by philosophers like 
Barry Smith and David Woodruff Smith and by \q{Analytic Phenomenology}
or \q{Naturalizing Phenomenology} projects, the latter also being 
a large volumes Petitot co-edited.\footnote{Petitot's and Barry Smith's formalizing projects were parallel
and collaborative to some extent.  Maxwell James Ramstead
in a 2015 master's thesis reviews the history elegantly:

\begin{quote}
Now, the \q{science of salience}
proposed by Petitot and Smith (1997) illustrates the
kind of formalized analysis made possible through the direct
mathematization of phenomenological descriptions.
Its aim is to account for the invariant descriptive
structures of lived experience (what Husserl called \q{essences})
through formalization, providing a descriptive geometry of
macroscopic phenomena, a \q{morphological eidetics} of the
disclosure of objects in conscious experience (in Husserl's
words, the \q{constitution} of objects).
Petitot employs differential geometry and morphodynamics
to model phenomenal experience, and Smith uses formal structures from
mereotopology (the theory of parts, wholes, and their boundaries)
to a similar effect. \cite[p. 38]{Ramstead}
\end{quote}
}
\q{Cognitive} in these contexts 
tends to imply desire to ground analyses on holistic
human experience, in its experiential, embodied, pragmatically-oriented, 
first-personal, and intersubjective dimensions.\footnote{Note that in this sense
\q{Cognitive} connotes a very different perspective than this same 
term in \AI{} research, for example; on the one side we have a 
philosophical commitment to the irreducibility of
human reason to computable \q{symbol processing}, whereas 
on the other there is a paradigm where, in effect 
(perhaps simply because mental activity is presumably 
reducible to low-level biological processes), there 
exists \i{some} computable core of cognition, which scientists 
can unlock to build powerful \q{Artificial General Intelligence}.}
}

\p{Cognitive linguistics can be called \q{speculative} because its 
methodology generally relies on linguists' assessments of 
acceptability rather than empirical data from surveys, 
computational or statistical analyses of copora, or 
psycholinguistic studies of language processing or acquisition.  
Analytic Phenomenology is speculative in similar ways; 
although in some cases phenomenological structures are 
related to formal/mathematical theories like 
Mereotopology or Differential Geometry (cf. Barry Smith and 
Petitot, respectively, or Kit Fine's \i{Part-whole}
\cite{KitFine}), phenomenologists' assessments of 
common perceptual patterns (and how they are 
situationally or conceptually interpreted and engaged with) 
is principally introspective.  Both phenomenologists 
and cognitive linguists, in short, introspect on 
conscious and linguistic experience to identify patterns 
which they believe are not eccentric to their own cognition,
but have some public disputability and theoretical merit.  
Of course, the overall dialog wherein philosophers debate 
and compare their own introspective reports allows 
this speculative method to have some rigor, because descriptions 
of cognitive processes {\mdash} in their first-personal 
facticity {\mdash} which seem both subjectively faithful and
structurally revealing will emerge as analyses of 
general pattern so long as multiple philosophical treatments 
agree on their fidelity to experience.  So 
analyses get theoretically favored if they meet three 
different criteria of structural productivity {\mdash} in the 
sense that produce new insight onto cognitive processes, 
rather than just describing cognition as an experienced
givenness {\mdash} plus both faithfulness to each person's 
conscious experience and also generality to many 
people's experience.\footnote{So for instance, Husserl's examination of 
protention and retention in sensing spatial form during 
perceptual episodes focused on discrete, extended objects 
(e.g. in \i{Thing and Space}), or his investigation of 
how intersubjectivity contributed to consolidating our 
conceptual integration of perceptual givens 
(e.g. in the \i{Cartesian Meditations}), can be 
considered classic phenomenological analyses because they 
have been deemed both experientially accurate and 
theoretically insightful by subsequent generations' worth of 
public review.}
}

\p{At their best, then, both phenomenology and cognitive linguistics 
combine introspective analysis and public disputation to 
develop theories of cognitive-experiential structures {\mdash} of 
how the immediate structuraion of perceptual experience as 
primordial conscious content unfolds into the
schematic and rational models of our surrounding environment
and situations, for purpose of goal-directed activity and 
inter-personal, collective reasonableness.  The underlying assumption 
is that raw structures, below the threshold of conscious 
deliberation, are enmeshed in immediate experience, and that from 
there we can identify ambient and situational patterns; 
mental representations of 
the structural and material properties of surrounding 
objects and places and the social/pragmatic rules 
governing interpersonal situations.  We can then 
posit situational prototypes and morphological principles 
apparently operating in these representations, which become 
the basis of systematic theorizing of cognitive activity 
in general, including language.}

\p{In this paradigm, situational and organizational prototypes 
and patterns lend their structure to language, so 
{\mdash} in many cases, i.e., with respect to many 
language artifacts {\mdash} the scenarios influencing 
linguistic structure are these extra- or pre-linguistic 
gestalts rather than semantic to syntactic rules
\i{per se}.  But for this belief about the origin of 
(at least some) surface-level linguistic form to be 
leveraged as a diligent semantic or syntactic method, 
we need a systematic account of how phenomenological 
pattern evolve into (by grounding) linguistic 
structures.  A thorough treatment of this problem 
is far beyond the scope of one paper, but I will 
offer a few ideas in the remainder of this section.}

\subsection{Types and Phenomenology}
\p{In their proper context, understanding linguistic expressions 
requires binding language to objects or \q{phenomena} in 
speakers' collective perceptual (or conceptual) horizon.  
Referents are not always material things; they can even 
be the \i{absence} of objects, or of substance:

\begin{sentenceList}

\sentenceItem{} \swl{itm:footprints}{There are footprints on the beach.}{ref}
\sentenceItem{} \swl{itm:hole}{There's a hole in the bucket.}{ref}
\sentenceItem{} \swl{itm:footprintsleading}{There are footprints leading up the hill.}{ref}
\sentenceItem{} \swl{itm:traintracks}{There are train tracks leading up the hill.}{ref}
\end{sentenceList}

In (\ref{itm:footprints}) our attention is direct not to any \i{object}, but 
to a certain perceptual pattern which has some factual significance, enough 
to warrant a distinct concept.  Insofar as \i{one} footprint is a 
focus of attention, we notice some pattern of discontinuity which allows 
a foreground to emerge from a background {\mdash} but there is perceptual 
blend of discontinuity and continuity, since a footprint is literally 
situated in a material expanse with its surroundings.  To focus on the 
footprint {\mdash} a cognitive act which is both perceptual and conceptual 
{\mdash} we have to retain awareness both of the footprint materially 
continuous with the surrounding sand (say) and also distinct from it 
via a somewhat different color or composition (e.g., the sand in the footprint 
may be darker or more compact than the sand around it).  When talking 
about \i{footprints}, plural, we have to direct attention to a perceptual 
totality, something extending over our visual field and possessing inner 
parts, but also suggesting a conceptual totality, a worthiness of 
being cognized in aggregate.}

\p{In (\ref{itm:footprintsleading}), moreover, we conceive the totality in conjunction 
with a sense of direction, and protention.  The implication is that we 
do not see \i{all} of the footprints, but we can discern from their 
pattern a direction which, we anticipate, will reveal more footprints.  
Meanwhile the footprints are presumably disjoint, unlike train tracks.
So the perceptual foreground in (\ref{itm:footprintsleading}) is phenomenologically 
complex, including a totality we perceive that implies a greater 
totality, part of which is experienced anticipatorily rather than 
explicitly, along with a fragmentation which nonetheless permits a 
perceptual unification into a coherent whole.  These kinds of 
perceptual/conceptual complexities in arrangements 
{\mdash} blending continuity and discontinuity, wholeness and 
fragmentation, sensation and protention {\mdash} are canonical 
to the phenomenology of attentional foci in any cognition 
engaged with ambient situations, which certainly 
includes language.  We are not robots experiencing the 
world as a tableau of simply discrete, integral object-\q{things}.}

\p{Language serves as a guide to negotiating the complexities of 
perceptual foci in inter-personal environments.  
We therefore have to understand how language
\q{hooks} into conversants' phenomenological faculties. 
So we have, at a basic level, a 
contrast between situationally grounded or conceptually 
generalized references:

\begin{sentenceList}

\sentenceItem{} \swl{itm:Salesmen}{Salesmen are intelligent.}{ont}
\sentenceItem{} \swl{itm:knocking}{Salesmen are knocking on the door.}{ont}
\sentenceItem{} \swl{itm:Should}{Should I let them in?}{ont}
\end{sentenceList}

The effects of (\ref{itm:knocking}) 
(which with (\ref{itm:Salesmen}) are from the \i{Handbook}, example 44, page 169, 
chapter 6) are manifest in several 
changes to conversants' collective understanding: (\ref{itm:knocking}) 
establishes both a domain of potential reference (the salesmen 
could be subsequently identified as \i{the salesmen} or
\i{those salesmen} or \i{them}), and the fact of their being 
at the door established as a basis for further dialog (e.g., 
(\ref{itm:Should})).  The more generic (\ref{itm:Salesmen}) 
does not have any comparable situational effects, though 
it permits further dialog on a more generic plane.}

\p{Sometimes, of course, people dialog about things that are 
perceptually evident around them collectively.  Probably 
more common, though, is that the structures of 
presentational perception translated to general 
cognitive patterns that are signified through language.  
That is to say, the rules of perceptual gestalts 
{\mdash} the partial continuity and partial discontinuity 
between background and foreground; and the mixture of 
singular integrity and divisibility characterizing attentional  
foregrounds {\mdash} are taken for granted as patterns 
typifying perception in general, whether that be 
perceptions presently occurring in the situational 
context or those conceptualized indirectly, abstractly, 
or hypothetically.  How foci of attention figure into 
perceptual continua become part of their extralinguistic 
background, reflected in lexical conventions:

\begin{sentenceList}

\sentenceItem{} \swl{itm:ink}{There are ink stains on the pages of this book.}{ref}
\sentenceItem{} \swl{itm:sunset}{There is a pretty sunset over the river.}{ref}
\sentenceItem{} \swl{itm:cicada}{In some neighborhoods, cicada insects make loud sounds from sunset to sunrise.}{ref}
\udref{en_gum-ud-train}{GUM_voyage_phoenix-30}
\sentenceItem{} \swl{itm:applause}{The audience rose to give thundering applause.}{ref}
\end{sentenceList}

The unifying theme in these examples is that their central concepts are 
experienced in terms of a matrix of phenomenological criteria that can 
be harnessed into a general theory: how an ink stain is (in part) 
materially continuous with the stained paper that extends around it; 
how \i{a pretty sunset} is an imprecisely bounded but conceptually 
integral atmospheric event; how (\ref{itm:applause}) figures the audience as 
discrete individuals who, at that moment, act in such a way as to present a 
conceptual and perceptual totality; how (\ref{itm:cicada}) likewise proposes 
a totality but in a more complex fashion, because the speaker is 
describing a typical \i{kind} of experience rather than the specific 
sounds of cicadas on one specific occasion.}

\p{The matrix of continuity/discontinuity, and individual coherence balanced
(within intentional \q{noemata}) 
against internal structuration and diversity, 
are \i{conceptually} intrinsic to notions like \i{ink stains} (or \i{footprints}),
\i{applause}, \i{sunsets}, and even \i{cicadas} and \i{noises}.  
They become part of a language's lexical machinery and 
therefore embed, in lexical conventions, phenomenological 
prototypes that imply a situational background, called 
forth by the lexical potency of the specific words.  
The point is not that the hearer of (\ref{itm:sunset}), say, 
sees the sunset also, and thus \i{explicitly} undergoes the 
disclosure-experience of a sunset (in all its sensory 
vividness and individuating vagueness).  Instead, the 
patterns of sensation and (dynamic, imprecise, un-fixed) 
individuation are abstracted to prototypes poised within 
the lexicon itself to be perceived as applicable to the 
talked-about situation.  We have some idea of the
\i{kind} of experience is involved in a sunset, not just 
perceptually but in the nexus of conceptual and 
rational processes which leads us to identify the sunset 
as such and estimate its phenomenal specificity 
{\mdash} its spatial form, its temporality {\mdash} so that it 
has some determinate epistemic content, something 
that can be discussed with others (\i{Did you see the 
sunset}; \i{Is it too late to see the sunset}; etc.).}

\p{Phenomenological patterns in construing conceptual and 
referential foci, solicited by lexical and idiomatic 
conventions, give language flexibility and rhetorical 
flair, often bounding expressive possibilities on 
phenomenological rather than narrowly semantic grounds.  
I would dispute some of Nunberg's analyses I mentioned 
in the last section, for example, on the 
dubiousness of (\ref{itm:start}) or the exceptionality 
of (\ref{itm:Sauternes}).  I do not find
(\ref{itm:start}) exceptionally jarring, considering the 
situational construal common to (\ref{itm:start}) along 
with (\ref{itm:parked}) and (\ref{itm:waiting}): 
the formation \i{I am parked} is not \i{just} a matter of 
a person proxy-referring their car.  Usually a person 
saying \i{I am parked} is \i{in} their car, so they are 
describing a larger situation {\mdash} the conceptual foreground 
is the totality of the car and themselves in it.  Were the 
addressee to find them, their attention would be directed 
to the car \i{and} the speaker, as a complex but 
integral (in that episode of time) whole.  
This then carries over to cases where the speaker 
is \i{not} in their car.  Saying \i{I am parked} (along with 
a place-description) outlines the steps needed to return to 
the car; it is a way of framing the car's location 
operationally.  Using the first person for where the car is 
parked {\mdash} even when the speaker is not at that spot 
{\mdash} indicates that the speaker is conceiving the car's 
location in first-personal terms; that is, in terms 
of the actions the speaker is taking or anticipates 
taking.}

\p{This analysis then implies that (\ref{itm:start}) is 
acceptable on similar operational grounds; first-personalizing 
the cars location, so to speak, foregrounds that location 
only in the context of an operational totality involving 
(presumably) going to the car and then driving it.  
If the speaker is concerned about the car not starting, 
this is relevant to how the overall situation is 
cognized, and therefore the transition from the speaker-centered
\i{I am parked ...} to the discordant \i{and may not start}
is understandable.  I would similarly argue that the 
apparent difference between (\ref{itm:Sauternes}) and
(\ref{itm:beers})-(\ref{itm:Michelobs}) depends on how 
referential foci are established.  The phrase
\i{a beer} can designate a specific glass or bottle of 
beer {\mdash} something perceptually and enactively/kinaesthetically 
individuated {\mdash} or also a specific preparation of beer,
individuated by the unique flavor of the beverage rather than 
the material identity of the liquid.  These 
differences have phenomenological overtones: if I say
\i{I drank two beers} I could mean two glasses, pints, or 
bottles; I could also mean two \i{kinds} of beer 
(maybe multiple pints of each).  The phrasing itself 
is ambiguous, but the ambiguity is extra-linguistic: 
it relates to how linguistic content binds to empirical 
givens with their rules of phenomenological 
individuation.  The \q{two pints} reading implies 
one kind of phenomenological architecture governing
\q{word-to-world fit}, where elements in language 
link to discrete, perceptually/operationally integral
objects like glasses (though of course a glass of 
beer is also an integral complex where the beer and 
the glass are semi-autonomous parts).  The \q{two kinds}
reading is more subtle, resting conceptual focus on the 
belief that a distinct kind of beer has a distinct 
flavor that is both unique \visavis{} other beers and 
also consistent across bottles (or kegs), so it has
\i{a} taste that people can jointly experience and 
talk about.  Nunberg makes the reasonable claim 
that we are more likely the hear the 
former interpretation when it comes to beers, and 
the latter one for Sauternes, but I can imagine plausible
conversations where these conventions would be reversed.}

\p{To return to type theory, then, this style of 
formalization is a potential window onto how the 
lexicalized concepts negotiate the phenomenological 
options in \q{binding} words to phenomena.  
The matrix of continuity/discontinuity and 
individuation (singularity)/complexity (internal 
structure or diversity) forms a tableau which 
different word senses and, in explicit phrase contexts, 
different usages \q{hook into} in different ways.  
So beer \i{qua} liquid has one conventional 
pattern in \q{word-to-world fit}, while beer
\i{qua} consumer product, or a brewer's creative 
endeavor, evinces a different phenomenological 
pattern.  This is modeled, to some approximation, 
by the \q{type} (or as I put it \q{mesotype}) 
distinction of a liquid (more generally a substance) 
against a consumer product or social good 
(more generally a socio-cultural artifact).  We 
can refer to beers in both senses, and I have 
mentioned theories of type \q{coercions} or
juxtapositions (cf. Pustejovsky's \q{dot product}
theories in \cite{JamesPustejovsky} or
\cite{AsherPustejovsky}) that explore when 
the different \q{Ontological} construals or sentences can be 
combined or alternated.  At this point I would add that 
these Ontological details are not only relevant to 
semantics; they also govern how phenomenological 
patterns can be reified and typified in lexical and 
idiomatic conventions.}

\thindecoline{}

\p{For successful conversation, participants need to converge 
for each sentence on a common conceptual focus; not surprisingly, 
this process often reciprocates the phenomenology of 
perceptual focus, or enactive/kinaesthetic attention.  
Drawing form from phenomenological structure, referential 
signification often takes shapes that can seem opaque 
on purely logical considerations.  In analysis of 
referring expressions, then {\mdash} analogously to 
semantic analyses as earlier in this section {\mdash} we need 
to be sensitive to the possibilities of language form 
molding to perceptual and situational gestalts rather 
than predicate structure.}

\p{Nunberg, for instance, investigates how the proxying effect in
\i{I am in the Whitney} can be generalized to other cases.  
I'll point out that this fits a not-uncommon rhetorical 
pattern, as in:

\begin{sentenceList}

\sentenceItem{} \swl{itm:hall}{I am in the Hall of Fame.}{ref}
\sentenceItem{} \swl{itm:ontv}{I am on TV.}{ref}
\end{sentenceList}

Nunberg argues that reference, in these kinds of cases, will
\q{transfer} between conceptually
linked designata, so I can refer to my car's location as my location, 
or to my painting as myself.  If we analyze this logically, 
the rule appears to be that I can substitute first-person 
reference for reference to something associated with 
myself (that I own or have created, etc.).  
He then finds that the pattern does not generalize to a 
scenario such as a painter {\mdash} referring to the location of her
work in transit {\mdash} saying something like (his example 12):

\begin{sentenceList}

\sentenceItem{} \swl{itm:crate}{I'm in the second crate on the right.}{ref}
\end{sentenceList}

In other words, Nunberg's analysis turns on theorizing various
\q{I am in...} constructions as a kind of referential transfer 
or (as I would say) proxying, and then seeking the logical rules 
behind how and when such proxying works in a first-person
(morphosyntactic) context.\footnote{Obviously \q{first person} in this setting concerns
verb tense and other linguistic cues linking to
the speaker/enunciator of a piece of language; the
same term is also encountered (including elsewhere
in this paper) in the phenomenological
(and Philosophy-of-Mind) sense of
conscious, intentional experience (in contrast
to experience, or thoughts and feelings, we
attribute to others based on their behavior).}
}

\p{On the other hand, I would argue that the pattern in 
(\ref{itm:hall})-(\ref{itm:ontv}) is not reducible to 
referential proxying as a logical maxim.  We can speculate that
\i{I} obliquely references, say, the bust and info about an 
athlete who has been elected to a Hall of Fame, or to the 
image of a TV personality which viewers see on screen.  
So analyzing this construction as a case where reference 
is transferred from the \q{subject of enunciation} to 
some subordinate vehicle {\mdash} her bust, image, and so forth 
{\mdash} is a plausible gloss on the pattern.  But looking for a 
single maxim to accommodate these different cases overlooks 
the contextual particulars, in particular the backstory 
which is lexically embedded in expressions like
\i{Hall of Fame} and \i{on TV}.  There is a long process 
leading to retired athletes being recognized as worthy of 
a Hall; there is also a long process whereby personalities 
get a chance to appear on television.  Actually, (\ref{itm:ontv}) 
has two interpretations {\mdash} someone could be on TV just momentarily, 
e.g. as a bystander during on-location news coverage {\mdash} or 
as someone like a policy expert who is interviewed on occasion.  
But the more interesting reading of (\ref{itm:ontv}) involves 
a person being described as \q{on TV} not on the occasion of 
their appearing on-screen in that moment, but as a recurring 
event.  In this case \i{being on TV} encapsulates a backstory 
reflecting a measure of personal success and esteem, not unlike 
an athlete in a Hall of Fame or an artist in a prestigious museum.}

\p{The relevant backstories govern interpretations for first-person 
reference: \i{I am in the Hall} or \i{on TV} indirectly reports 
that the speaker has undergone some process which the addressee, 
assumed competent with the relevant lexicon, at least
minimally understands.
So a pattern like (\ref{itm:hall}) packages up that backstory
\i{in the guise} of a referring expression; it is not so much the 
athlete's bust as a target of referential proxying, but more of the 
act of referring to that bust (i.e., to the referentiable 
object manifesting the speaker's being elected) entering
the Hall of Fame backstory into the discursive ledger.  This 
carries over to the Whitney case: calling oneself
\q{in} a museum does not just proxy the self for one's art, 
but uses patterns of reference to the art work as a 
vehicle for introducing the backstory of one's 
rise through the art world (or whatever is the relevant 
autobiographical context) into the conversation.}

\p{I would argue, in short, that referential proxying is not a 
primarily logical operation, but relies on backstory context to 
regulate how referring patterns are received.  To establish conceptual 
foci, addressees have to negotiate the language given to them, in 
search of the focal element or foreground, often an explicit or 
hypothetical perceptual nexus.  So a painter \i{in the Whitney} means 
that a visitor can, in the right room and orientation, perceptually 
encounter her work; a personality \i{on TV} means that the 
addressee can, at times, see her on-screen.  Those potential perceptual 
givens ground the semantics of expressions like (\ref{itm:Whitney}) or 
(\ref{itm:ontv}).  This does not mean that the sentences are only 
meaningful to an addressee who wants to go find the speaker at the 
museum or on the television.  But it does present the perceptual 
ground as a foundation for the conceptual implications which 
lead from that perceptual situation: from the perceptual presence 
of an art work in a museum to the backstory of its provenance 
and implications for the artist's career or place in history; 
or from the on-screen image to the context of how TV shows 
are produced and the careers and reputations of the figures that 
show up.}

\p{We approach referring expressions in these contexts at two levels: 
we identify the canonical perceptually-oriented references which 
supply a conceptual ground (the painting as perceptual 
object, the on-screen image as perceptual simulacrum); but then 
we recognize the backstory and situational context which 
determines how the perceptual ground should be understood.  
Usually the discourse is \q{about} that backstory, not about 
the precise situations where explicit perceptual grounds would 
be relevant {\mdash} an artist probably would not say 
(\ref{itm:Whitney}) only to someone looking to visit a museum to 
see her art work.  The \q{I am in...} pattern 
turns on how these two different levels are played off 
in interpretation.
In particular, the addressee needs to ascertain why the
speaker's information is being proferred: is boasting
of their art being hung in an elite museum mostly a
way of conveying the speaker's status, so the main theme is
essentially autobiographical; or is the actual location
of the painting directly relevant to the present conversation?}

\p{A statement like (\ref{itm:crate}) (the speaker being
\q{in the crate on the right}) does not have the more
\q{autobiographical} reading, since (absent some bizarry
imaginary scene) there is no status or reputation attached
to which crate is carrying your paintings.  But the second kind of reading,
where specific location is operationally relevant,
is plausible in some contexts; perhaps several different
artists' works are in several different crates and
the speaker wants to direct focus onto the one holding
her own works.  So, in the proper context,
I have no objection to
(\ref{itm:crate}) or \i{I am in that crate} (forms which Nunberg rejects).
The gist here, as in (\ref{itm:Whitney}), is to establish conceptual
focus onto a painting or art work, and to do so via a
first-person referential lead-in.  So the situation determines 
first \i{why} the focus is thus singled out and second
\i{how} the first-person reference can proxy that focus.  
In (\ref{itm:Whitney}), we single out a painting insofar as it 
is a product of the artist's creativity, so appreciation of the 
work is manifestly appreciation of the artist.  The artist is 
then referentially linked to the work because the backstory of 
how museums acquire art works overlaps with the story of 
artists' careers and reputations.}

\p{In (\ref{itm:crate}) the situation is different: presumably 
the focus is on one or several art works because of some concern 
about transporting or accessing them, so we're attending to the 
works in their guise as (fragile) physical objects.  Moreover 
the link between the artist and the work, which guides the 
referential proxying, is presumably that the artist is 
concerned about accessing and protecting those objects.  
The situational elements are different, but in both these 
cases {\mdash} whether paintings are in the crate or in the 
Whitney {\mdash} there is some structural resonance in how
situational, backstory, and hypothetical-perceptual 
gestalts are all integrated into patterns of grounding 
referential interpretation in ambient conceptual contexts.}

\p{I think that linguists sometimes underestimate the 
multiplicity of layers {\mdash} situational, contextual, 
phenomenological (explicit and hypothetical), 
referential, contextual {\mdash} that all converge on 
meaning and reference.  Perhaps this tendency can
be empirically examined by testing judgments of
acceptability: the more that we incorporate
multiple layers of posssible context, the wider becomes
the circle of sentences that feel reasonable
(cf. the reasonableness, in my estimation, of
(\ref{itm:crate}) and (\ref{itm:Sauternes})).\footnote{Granted that, perhaps, with a lot of imagination
almost any construction could be deemed
plausible in \i{some} context.  Accordingly,
one could argue that it is analytically reasonable
to distinguish sentences that exemplify proper
language in a wide range of contexts, as compared
to sentences which could only be meaningful
in very select circumstances.  Even on this perspective,
however, we have to unpack the distinction between
\q{generic} from \q{select} circumstances {\mdash} what
qualities of these situations, together with
relevant word-meanings, make these examples
of broad usage patterns as against unexpected
usages that are nonsensical without a very
specific background?  Rather than being a neutral
arbiter of acceptability, such genericity is a
phenomenon that needs to be explained.  I would
argue that assessments of unacceptability will in
many cases overrate the distinction between
\q{normal} and \q{exceptional} circumstances.
For example, I dispute that the beer/wine
contrast, or painting in a museum or in a crate
contrast, are such a divergence between normal and
unusual contexts that (\ref{itm:crate}) has
essentially different plausibility than
(\ref{itm:Whitney}), or (\ref{itm:Sauternes})
compared to (\ref{itm:beers}) and (\ref{itm:Michelobs}).} Failure to identify these
layers' workings can lead analysis toward searches 
for reproducible logical rules governing which constructions  
are accepted by a language-community as recurring patterns, 
and logical explanations for the limits on that 
generalizability.  This is one example of
overestimating the logicality of language in 
general, an issue I have approached in this section 
from both semantic and reference-theoretic angles.  
I will focus on this issue of logicality {\mdash} both its applicability 
and its limits {\mdash} in the next section.}

%\section{Gaps in Truth-Theoretic Semantics}
\label{s3}
\label{sec:Gaps}
\p{I take as a given that typical sentences have a propositional core, 
against which they take a performative stance 
(which can be outright assertion, or else asserting speakers' 
more complex propositional attitudes).  I would further 
say that \i{truth-theoretic} semantics, in particular, 
is organized around this propositional content as the 
core target of linguistic analysis.  I am thinking of 
truth-theoretic semantics in a broad sense, perhaps the 
most influential paradigm in the Philosophy of Language 
and by extension philosophy and linguistics in general 
(not to mention Computer Science and Artificial Intelligence 
research).  The most notable counter-paradigm is Cognitive Linguistics; 
consider George Lakoff and Mark Johnson's extended critique 
of truth-theoretic paradigms in \i{Philosophy in the Flesh}.  
Adherents of the latter perspective need not 
dispute the logical substance of language artifacts' propositional 
content, but tend to direct theoretical attention not to 
the nature of propositional content itself, but to the cognitive 
processes through which this content is understood.}

\p{As I argued earlier, many sentences do not simplistically 
reproduce the logical structure of their propositional 
content, so models of that structure are only tangentially 
relevant to analysis on the language side.  This is why we 
need distinct analyses, beyond a mere logical gloss, 
covering the interpretive steps leading to holistic 
sentence-understanding.   This section will consider several 
cognitive and pragmatic themes moving toward a general theory 
of this phenomenon.}

\subsection{Enaction and Illocutionary Force}

\p{I will start by reviewing illocutionary pragmatics, to 
identify some of the contextual and interpretive 
transformations that pertain to mapping surface language 
to propositional contents.  My point is to establish 
what should be a common theory of logicality that can be
shared by both critics and defenders of \q{truth-theoretic}
paradigms, on which basis their legitimate disputes 
can be investigated.}

\p{Many linguists (on both sides, I would say, of my 
central truth-theoretic pro/con), 
seem to analyze hedges like \q{could you please}
as merely dressing over crude commands: we don't
want to come across as giving people orders, but
sometimes we do intend to ask people to do specific
things.  As a result, we feel obliged to couch the
request in conversational gestures that signal
our awareness of how bald commands may lie outside
the conversational norms.  These ritualistic
\q{could you please}-like gestures may have
metalinguistic content, but {\mdash} so the theory
goes {\mdash} they do not \i{semantically} alter
the speech-act's directive nature.}

\p{The problem with this analysis is that sometimes
directive and \q{inquisitive} dimensions can
overlap:

\begin{sentenceList}

\sentenceItem{} \swl{itm:almond}{Do you have almond milk?}{pra}
\sentenceItem{} \swl{}{Can you get MsNBC on your TV?}{pra}
\sentenceItem{} \swl{itm:needcorkscrew}{This isn't a screw-cap bottle: I need a corkscrew.}{pra}
\end{sentenceList}

These \i{can} be read as bare directives, and would
be interpreted as such if the hearer believed the
speaker already knew that yes, he has almond milk, and yes,
he gets MsNBC.  In (\ref{itm:needcorkscrew}), if both parties
know there's one corkscrew in the house,
the statement implies a directive to fetch \i{that} corkscrew.
But, equally, (\ref{itm:almond})-(\ref{itm:needcorkscrew}) can \i{also} be read as bare
questions with no implicature: say, as fans of
almond milk and MsNBC endorsing those selections,
or pointing out that opening the bottle
will need \i{some} corkscrew.
And, meanwhile, (\ref{itm:almond})-(\ref{itm:needcorkscrew})
can \i{also} be read as a mixture of the
two, as if people expressed themselves like this:

\begin{sentenceList}

\sentenceItem{} \swl{}{I think the window is open, can you close it?}{pra}
\sentenceItem{} \swl{}{I see you have almond milk, can I have some?}{pra}
\sentenceItem{} \swl{}{If you get MsNBC, can you turn on Rachel Maddow?}{pra}
\sentenceItem{} \swl{}{If there is a corkscrew in the house, can you get it?}{pra}
\end{sentenceList}
}

\p{I think the mixed case is the most prototypical, and pure
directives or inquiries should be treated as degenerate
structures where either directive or inquisitive content
has dropped out.  After all, even a dictatorial
command includes the implicit assumption that the order
both makes sense and is not impossible.  On
the other hand, we don't ask questions for no
reason: \q{do you have almond milk} may be a
suggestion rather than a request, but it still
carries an implicature (e.g., that the addressee
\i{should} get almond milk).}

\p{Ordinary requests carry the assumption that addressees
can follow through without undue inconvenience,
which includes a package of assumptions about both
what is currently the case and what is possible.
\q{Close the window} only has literal force if the
window is open.  So, when making a request, speakers
have to signal that they recognize the request involves
certain assumptions and are rational enough to
accept modifications of these assumptions in
lieu of literal compliance.  This is why
interrogative forms like \q{can you} or
\q{could you} are both semantically nontrivial
and metadiscursively polite: they leave open the
possibility of subsequent discourse framing the original
request just as a belief-assertion.  Developments
like \i{can you open the window} {\mdash} \i{no, it's closed}
are not ruled out.  At the same time, interrogative forms
connote that the speaker assumes the addressees can
fulfill the request without great effort: an implicit
assumption is that they \i{can} and also \i{are
willing to} satisfy the directive.  This is an
assumption, not a presumption: the speaker
would seem like a bully if he acted as if he
gave no thought to his demands being too much
of an imposition {\mdash} as if he were taking
the answer to \q{can you} questions for granted.
This is another reason why requests
should be framed as questions.  So, in short,
\q{commands} are framed as questions because the speaker
literally does not know for sure whether the command is
possible; given this uncertainty a command \i{is} a question,
and the interrogative form is not just a non-semantic
exercise in politesse.}

\p{Sometimes the link between directives and
belief assertions is made explicit.  A common
pattern is to use \i{I believe} or \i{I believe that} as an
implicature analogous to interrogatives:

\begin{sentenceList}

\sentenceItem{} \swl{}{I believe you have a reservation for Jones?}{pra}
\sentenceItem{} \swl{}{I believe this is the customer service desk?}{pra}
\sentenceItem{} \swl{}{I believe we ordered a second basket of garlic bread?}{pra}
\sentenceItem{} \swl{}{I believe you can help me find computer
accessories in this section?}{pra}
\end{sentenceList}

These speakers are indirectly signaling what they want
someone to do by openly stating the requisite
assumptions {\mdash} \i{I believe you can} in place
of \i{can you?}.  The implication is that
such assumptions translate clearly to a
subsequent course of action {\mdash} the guest who
\i{does} have that reservation should be checked in;
the cashier who \i{can} help a customer find
accessories should do so.  But underlying these
performances is recognition that
illocutionary force is tied to background
assumptions, and conversants are reacting to
the propositional content of those assumptions
as well as the force itself.  If I \i{do} close the
window I am not only fulfilling
the request but also confirming that the window
\i{could} be closed (a piece of information
that may become relevant in the future).}

\p{In sum, when we engage pragmatically with other
language-users, we tend to do so cooperatively,
sensitive to what they wish to achieve with
language as well as to the propositional
details of their discourse.  But this often means
that I have to interpret propositional
content in light of contexts and implicatures.
Note that both of these are possible:

\begin{sentenceList}

\sentenceItem{} \swl{}{Do you have any milk?}{pra}
\sentenceItem{} \swl{}{Yes, we have almond milk.}{pra}
\sentenceItem{} \swl{}{No, we have almond milk.}{pra}
\end{sentenceList}

A request for milk in a vegan restaurant could plausibly be
interpreted as a request for a vegan milk-substitute.
So the concept \i{milk} in that context may actually be
interpreted as the concept \i{vegan milk}.  
Responding to the force of speech-acts
compels me to treat them as not \i{wholly}
illocutionary {\mdash} they are in part statements of
belief (like ordinary assertions).  One reason I need
to adopt an epistemic (and not just obligatory)
attitude to illocutionary acts is that I need to
clarify what meanings the speaker intends, which
depends on what roles she is assigning to
constituent concepts.}

\p{Suppose my friend says this, before and after:

\begin{sentenceList}

\sentenceItem{} \swl{itm:put}{Can you put some almond milk in my coffee?}{pra}
\sentenceItem{} \swl{itm:after}{Is there milk in this coffee?}{pra}
\end{sentenceList}

Between (\ref{itm:put}) and (\ref{itm:after}) I do put almond milk
in his coffee and affirm \q{yes} to (\ref{itm:after}).  I feel it
proper to read (\ref{itm:after})'s \q{milk} as really meaning
\q{almond milk}, in light of (\ref{itm:put}).  Actually
I should be \i{less} inclined to say \q{yes}
if (maybe as a prank) someone had instead
put real (cow) milk in the coffee.  In responding
to his question I mentally substitute what
he almost certainly \i{meant} for how
(taken out of context) (\ref{itm:after}) would usually
be interpreted.  In this current
dialog, the \i{milk} concept not only
includes vegan milks, apparently, but
\i{excludes} actual milk.}

\p{It seems {\mdash} on the evidence of cases like this one {\mdash}
as if when we are dealing with
illocutionary force we are obliged to subject
what we hear to extra interpretation, rather
than resting only within \q{literal} meanings
of sentences, conventionally understood.
This point is worth emphasizing because it complicates
our attempts to link illocution with propositional
content.  Suppose grandma asks us to close the
kitchen window.  Each of these are plausible and
basically polite responses:

\begin{sentenceList}

\sentenceItem{} \swl{}{It's not open, but there's still some
cold air coming through the cracks.}{pra}
\sentenceItem{} \swl{}{It's not open, but I closed the window in
the bedroom.}{pra}
\sentenceItem{} \swl{}{I can't {\mdash} it's stuck.}{pra}
\end{sentenceList}

In each case I have not fulfilled her request \visavis{}
its literal meaning, but I \i{have} acted benevolently
in terms of conversational maxims.  Similarly, 
the \i{Handbook} has this case (example 12, p. 203, chapter 8):

\begin{sentenceList}

\sentenceItem{} \swl{itm:TheWindow}{The window, it's still open.}{pra}
\sentenceItem{} \swl{itm:AWindow}{A window, it's still open.}{pra}
\end{sentenceList}

Chapter authors Jeanette K. Gundel and Thorstein Fretheim 
suggest that (\ref{itm:AWindow}) is dubious, but in context 
it may make perfect sense {\mdash} particularly if it follows 
a discourse where a requester wanted \i{whatever} window 
closed (whichever window was causing a draft); even if that 
wish was \i{expressed} via a \i{close the window}.\footnote{Consider also the case of (\i{Handbook}, example 9, 
page 132, chapter 6):
\begin{sentenceList}

\sentenceItem{} \swl{itm:scribbled}{He scribbled on a living-room wall.}{sem}
\end{sentenceList}

Barbara Abbott (chapter author) finds the indefinite article 
in (\ref{itm:scribbled}) awkward.  Here, like in (\ref{itm:AWindow}), 
I think the acceptability of \i{both} definite and indefinite 
articles points to the flexibility of articles in English: 
we can use \i{close \underline{the} window} even if we are referentially 
ambiguous about which window is open {\mdash} both (\ref{itm:AWindow}) 
and (\ref{itm:TheWindow}) are almost interchangeable.  The reason
is apparently that the very act of talking about an open or 
closed window foregrounds it such that it can be approached 
definitely.  In (\ref{itm:scribbled}) Abbott's intuition is 
probably that we usually speak of \i{the} wall of a room {\mdash} even 
in the normal case of a room with four walls {\mdash} because there is 
no common reason to single out one wall, as with an indefinite 
article (saying \i{a} wall implies that the other walls are 
differentiated from the one referenced).  But I believe we can 
clearly cognize the walls of a room as a collection of discrete 
things, and that the formation \i{the} wall {\mdash} unifying 
them into a single {\mdash} is not so much a matter of discounting 
the perceptual multiplicity of four walls, but 
of conceiving \q{walls} themselves functionally as well as 
perceptually.  In my reading, \i{the wall} refers not only 
to something perceptually individuated but to an element 
of the architectural complex of a building: a wall is something 
that prohibits movement, muffles sounds, provides some privacy, 
protects people inside the room (part of a building's 
structural integrity), and so forth.  We can say \i{the} wall 
because we cognize walls as occupying that phenomenal niche, 
so all four walls of a room are collectively \q{the} wall
\visavis{} such a niche, even while on sensory grounds we 
can switch to treating them as separate (warranting the (\ref{itm:scribbled}) 
version).}
}

\p{Part of reading propositional content is
syncing our conceptual schemas with our fellow
conversants.  The illocutionary
dimension of a request like \i{can I have some milk?}
makes this interpretation especially important,
because the addressee wants to make a good-faith effort
to cooperate with the pragmatic intent of the
spech-act.  But cooperation requires the
cooperating parties' conceptual schemas to
be properly aligned.  I therefore have to
suspend the illocutionary force of a directive
temporarily and treat it as locutionary
statement of belief, interpret its apparent
conceptual underpinnings in that mode, and
then add the illocutionary force back in: if I
brought \i{real} milk to a vegan customer who
asked for \q{milk} I would be \i{un}-cooperative.}

\p{The upshot is that conversational implicatures
help us contextualize the conceptual negotiations
that guarantee our grasping the correct
propositional contents, and vice-versa.  This means
that propositionality is woven throughout both
assertive and all other modes of language, but it
also means that propositional content can be
indecipherable without a detailed picture
of the current context (including illocutionary
content).  The propositional content of,
say, \i{there is milk in this coffee} has to be
judged sensitive to contexts like \i{milk}
meaning \i{vegan milk} {\mdash} and this
propagates from a direct propositional
to any propositional attitudes which may
be directed towards it, including requests like
\i{please put milk in this coffee}.}

\p{Suppose the grandkids close grandma's bedroom window
when she asks them to close the kitchen window.
The propositional content at the core of grandma's
request is that the kitchen window be closed; the
content attached to it is an unstated belief that
this window is open.  Thus, the truth-conditions
satisfying her implicit understanding would be
that the kitchen window went from being open to being
closed.  Suppose, as it happens, that window is already
closed.  So the truth-conditions that would satisfy
grandma's initial belief-state do not obtain {\mdash} her
beliefs are false {\mdash} but the truth conditions satisfying
her desired result \i{do} obtain.  The window
\i{is} closed.  Yet the grandkids should not thereby
assume that her request has been properly responded to;
it is more polite to guess at the motivation behind
the request, e.g., that she felt a draft
of cold air.  In short, they should look outside
the truth conditions of her original
request taken literally, and \i{interpret}
her request, finding different content
with different truth-conditions that are both consistent
with fact and address whatever pragmatic goals
grandma had when making her request.  They might
infer her goal is to prevent an uncomfortable
draft, and so a reasonable \q{substitute content} is
the proposition that \i{some} window is open,
and they should close \i{that} one.}

\p{So the grandkids should reason as if translating
between these two implied meanings:

\begin{sentenceList}

\sentenceItem{} \swl{}{I believe the kitchen window
is open {\mdash} please close it!}{pra}
\sentenceItem{} \swl{}{I believe some window
is open {\mdash} please close it!}{pra}
\end{sentenceList}

They have to revise the simplest reading of
the implicit propositional content of grandma's
\i{actual} request, because the actual request is
inconsistent with pertinent facts.  In short, they
feel obliged to explore propositional alternatives
so as to find an replacement, implicit request whose
propositional content \i{is} consistent with
fact and also meets the original request's illocutionary
force cooperatively.}

\p{In essence, we need to express a requester's desire as
itself, in its totality, a specific propositional content,
thinking to ourselves (or even saying to others) things
like

\begin{sentenceList}

\sentenceItem{} \swl{}{Grandma wants us to close the window.}{pra}
\sentenceItem{} \swl{}{He wants a bottle opener.}{pra}
\end{sentenceList}

But to respond politely we need to modify
the parse of their requests to capture the
\q{essential} content:

\begin{sentenceList}

\sentenceItem{} \swl{}{Grandma wants us to eliminate the cold draft.}{pra}
\sentenceItem{} \swl{}{He wants something to open that bottle.}{pra}
\end{sentenceList}

We have to read outside the literal interpretation
of what they are saying.  This re-reading is something
that may be appropriate to do with respect to
other forms of speech also; 
but our conversational responsibility to infer
some unstated content is especially pronounced
when we are responding to an explicit
request for something.}

\p{Certainly, in many cases, meanings are not literal.
But how then do we understand what people are saying?
Trying to formulate a not-entirely-haphazard
account of this process, we can speculate
that interpreting what someone is \q{really} saying
involves systematically mapping their apparent
concepts and references to some superimposed
inventory designed to mitigate false beliefs or
conceptual misalignments among language users in some
context.  That means, we are looking for mappings
like \i{milk} to \i{almond milk} in (\ref{itm:can}) from a
vegan restaurant, or \i{kitchen window} to
\i{bedroom window} in (\ref{itm:close}) if it is the latter
that is open:

\begin{sentenceList}

\sentenceItem{} \swl{itm:can}{Can I have some milk?}{pra}
\sentenceItem{} \swl{itm:close}{Can you close the kitchen window?}{pra}
\end{sentenceList}

The point of these \q{mappings} is that they
preserve the possibility of
modeling the \i{original} propositional content
by identifying truth conditions
for that content to be satisfied.}

\p{A \i{literal} truth-condition model doesn't work in
cases like (\ref{itm:can}) and (\ref{itm:close}): the diner's request
is \i{not} satisfied if it is the case
that there is now (real) milk in her coffee; and
grandma's request is not necessarily satisfied if it is
the case that the kitchen window is closed.  The
proposition \q{the kitchen window is closed} only bears on
grandma's utterance insofar as she believes that
this window is open and causing a draft.  So if we want
to interpret the underlying locutionary content
of (\ref{itm:can}) and (\ref{itm:close}) truth-theoreticaly, we need to
map the literal concepts appearing
in these sentences to an appropriate translation,
a kind of \q{coordinate transformation} that
can map concepts onto others, like milk/almond milk
and kitchen window/bedroom window.}

\p{In sum, a theory of sentences' logical nexus can only 
be complete with some model of discursive context
\i{structured in such a way} that we can repesent the 
interpretations and concept-transforms internal to 
parsing sentences to their propositional core.  
I will now consider what such a \q{theory of context}
might look like.}

\spsubsectiontwoline{The co-framing system and the
doxa system}

\p{Illucutionary acts expressly signify our desire for 
something to change in our environment (with the 
help of our addressees), but similar implications 
of pragmatic desire are evident even when sentences 
are more directly assertorial, or less directly 
illocutionary.  Compare between:

\begin{sentenceList}

\sentenceItem{} \swl{itm:store}{Remember that wine we tasted on the Niagara 
Peninsula last summer?  Can you find it in our 
local liquor store?}{pra}
\sentenceItem{} \swl{itm:varietal}{Remember that wine we tasted on the Niagara 
Peninsula last summer?  What varietal was that again?}{pra}
\end{sentenceList}

The first sentence in each pair attempts to 
establish a common frame of reference between 
addresser and addressee {\mdash} it does not, in and 
of itself, request any practical (extramental) action.  
The second sentence in (\ref{itm:store}) \i{can} be read as 
requesting that the addressee buy a bottle, though an alternate
interpretation is to learn for \i{future reference}
whether someone \i{could} buy that bottle.  The 
second sentence in (\ref{itm:varietal}) carries no directive 
implicature at all, at least with any directness; 
it asks for more information.}

\p{Despite these variations, it seems reasonable to say that 
language is always performed in an overarching setting 
where concrete (extralinguistic) activity 
will \i{eventually} take place.  If in (\ref{itm:varietal}) I intend
to recommend that grape variety to a friend, I may not be  
making a direct request of him, but I \i{am}
proposing an eventual action that he 
might take.  If in (\ref{itm:store}) I am not issuing a directive, I 
am however establishing (and reserving the future possibility) 
that such a directive would be reasonable.  As a result, some 
extralinguistic state change seems to be lurking 
behind the linguistic content: I want my friend 
to go from having never tasted that varietal 
to having tasted it. 
Or I want to go from not having a bottle of that 
wine to having one.  Or, if I do not 
want these things at the moment, I want 
to confirm intellectually that these wishes are 
plausible.  We seem to use language to 
set up the interpersonal understandings needed 
to \i{eventually} engage in (usually collective) 
practical activity, which means effectuating some 
(extralinguistic) change.}

\p{That is, most expressions are not direct 
requests or suggestions of the \q{close the window}
or \q{let's get some wine} variety, but they are 
stitches in the thread of coordinated human actions.
Often however we use language to \i{prepare},
\i{negotiate}, and \i{decide upon} joint actions.  
We may have a
\i{holistic} sense that meanings orbit around 
extralinguistic and extramental state-change, 
but at the level of particular sentences most 
changes that occur, or are proposed, tend to be 
changes in our conceptualization of situations.  
Accordingly, we can pursue a semantic theory 
based on \i{change of state} if we accept that 
such changes run the gambit from changes
\i{internal} to language {\mdash} to conversants' 
appraisal of dialogic context {\mdash} to 
changes effectuated by human activity inspired by language. 
Dialogs 
themselves change: the first sentences in 
(\ref{itm:store}) and (\ref{itm:varietal}) 
modify the discursive frame so that, for example, 
a particular wine becomes available as the anaphoric
target for \i{that} and \i{that wine} {\mdash} and also, 
metonymically, \i{that varietal}, \i{that grape},
\i{that winery}.  Conceptual frames can change: 
if we are discussing a visit to Ontario and 
I mention a particular winery, one effect is to 
(insofar as the conversation follows my lead) 
refigure our joint framing to something 
narrower and more granular that the prior frame (but 
still contained in it; I am not changing the subject 
entirely).  We can pull a frame out as well as in
{\mdash} e.g., switch from talking about one winery visit to 
the whole trip, or one Leafs game to the entire season.
Moreover, our beliefs can change/evolve: if you tell me 
the wine was Cabernet Franc, I have that piece of 
info in my arsenal that I did not have before.}

\p{So I assume in this paper that
linguistic meanings are grounded in state-changes, 
with the stipulation that the \q{register} where the changes occur can 
vary over several cognitive and extramental options: 
actual change in our environment (the window closed, 
milk in the coffee, the bottle opened); changes to the 
dialog structure (for anaphoric references, pronoun 
resolution, metalinguistic cues like \i{can you say that 
again}, etc.); changes to conceptual framings
(zoom in, zoom out, add detail); changes to beliefs.
Each of these kinds of changes deserve their own analysis, 
but we can imagine the totality of such analyses 
forming an umbrella theory of meanings.}

\p{During the course of a conversation {\mdash} and indeed 
of any structured cognitive activity {\mdash} we 
maintain conceptual frames representing relevant 
information; what other people know or believe;
what are our goals and plans (individually and 
collectively); and so forth.  We update
these frames periodically, and use language to 
compel others to modify their frames in
ways that we can (to some approximation) anticipate 
and encode in linguistic structure.}

\p{In the simplest case, we can effectuate changes in 
others' frames by making assertions they are likely 
to believe to be true (assuming they deem us 
reliable).  In general, it is impossible 
to extricate the explicit content of the relevant 
speech-acts from the relevant cognitive, linguistic, and 
real-world situational contexts:

\begin{sentenceList}

\sentenceItem{} \swl{}{That wine was a Cabernet Franc.}{ref}
\sentenceItem{} \swl{}{Those dogs are my neighbor's.  
They are very sweet.}{ref}
\end{sentenceList}

Although there is a determinate propositional content being 
asserted and although there is no propositional attitude 
other than bald assertion to complicate the pragmatics, 
still the actual words depend on addressees drawing from 
the dialogic context in accord with how I expect them to
(as manifest in open-ended expressions like \i{that wine},
\i{those dogs}, \i{they}).  Moreover, the  
open-ended components can refer outward in different
\q{registers}: in \i{that wine} I may be 
referencing a concept previously established in 
the conversation, while \i{those dogs} may refer to 
pets we saw or heard but had not previously 
talked about.  Of course, the scenarios 
could be reversed: I could introduce \i{that wine}
into the conversation by gesturing to a bottle 
you had not noticed before, and refer via
\i{those dogs} to animals you have never seen or heard but 
had talked about, or heard talk about, in the recent past.
These dialog steps need to be resolved via a mixture of 
linguistic and extra-linguistic cues: 
surface-level language is not always clear as to whether 
referring expressions are to work \q{deictically}
(drawing content from the ambient context, 
signified by gestures, rather than from any 
linguistic meaning proper), \q{discursively}
(referring within chains of dialog, e.g. 
anaphora), or \q{descriptively} (using 
purely semantic means to establish a designation, 
like \q{my next-door neighbor's dogs} or
\q{Inniskillin Cabernet Franc Icewine 2015}).}

\p{Let's agree to call the set of entities sufficiently 
relevant to a discourse or conversation context the \i{ledger}.
By \q{sufficiently relevant} I mean whatever is already 
established in a discourse so it can be referenced with something 
less that full definite description (and without the aid 
of extralinguistic gestures).  I assume that gestures
and/or descriptions are communicative acts which \q{add}
to the ledger.  The purely linguistic case {\mdash} let's say,
\i{descriptive additions} {\mdash} can themselves be distinguished
by their level of grounding in the current context.  
A description can be \q{definitive} in a specific situation without 
being a \i{definite description} in Bertrand Russell's sense
(see \q{that wine we tasted last summer}).}

\p{So, descriptive additions to the ledger are one kind of 
semantic side-effect: we can change the ledger via 
language acts.  I will similarly dub another facet of 
cognitive-linguistic frames as a \i{lens}: the idea
that in conversation we can \q{zoom} attention 
in and out and move it around in time. \q{That wine we 
tasted last summer in Ontario} both modifies 
the \i{ledger} (adding a new referent for convenient
designation) and might alter the \i{lens}:
potentially compelling subsequent 
conversation to focus on that time and/or place.  
Finally, I will identify a class of frame-modifications 
which do directly involve propositional content: 
the capacity for language to promote shared beliefs 
between people whose cognitive frames are in 
the proper resonance, by adding details to conceptual
pictures already established: \i{those dogs are Staffordshires},
\i{that wine is Cabernet Franc}, \i{we have almond milk}, etc.}

\p{For sake of discussion, I will call this latter part of the \q{active}
cognitive frame, for some discussion 
{\mdash} the part concerning shared beliefs or asserted facts {\mdash}
the \i{doxa inventory}.
This \q{database}-like repository stands alongside the
\q{ledger} and \q{lens} to track propositional content 
asserted, collectively established, or already 
considered as background knowledge, \visavis{} some 
discourse.  Manipulations of the lens and ledger allow 
speakers to designate (using referential cues
that could be ambiguous out-of-context) 
propositional contents which they 
wish to add to the \q{doxa inventory}.  I'll also say 
that modifying this inventory \i{can} be done through 
language, but participants in a discourse are entitled 
to assume that everyone formulates certain beliefs 
which are observationally obvious, and can therefore 
be linguistically presupposed rather than reported 
(the likes of that a traffic light 
is red, or a train has pulled into a station, or 
that it's raining).}

\p{So, I will assume that the machinery of frames is cognitive, 
not just linguistic.  We have analogous faculties for
\q{refocusing} attention and adding conceptual details
via interaction with our environment, both alone and with others, 
and both via language and via other means.  Some 
aspects of \i{linguistic} cognitive framing {\mdash} like 
the \q{ledger} of referents previously established in a 
conversation {\mdash} may be of a purely linguistic character, 
but these are the exception rather than the rule.  
In the typical case we have a latent ability to 
direct attention and form beliefs by 
direct observation \i{or} by accepting others' reports as
proxies for direct observation.}

\p{When we are told that two dogs are male, for instance, 
we may not perceptually encounter the dogs but we understand what 
sorts of perceptual disclosures could
serve as motivation for someone believing that idea.  We therefore 
assume that such belief was initially warranted by 
observation and subsequently got passed through a chain 
of language-acts whose warrants are rooted in the perceived 
credibility of the speaker.  Internal to this 
process is our prior knowledge of the parameters for judging 
statements like \i{this dog is male} observationally.}

\p{True, sometimes such observational warrants are less on 
display.  If I had never heard of Staffies (Staffordshire
pit bulls), I would be fuzzier about observational warrants
and could end up in a conversation like:

\begin{sentenceList}

\sentenceItem{} \swl{}{Those dogs are Staffordshires.}{pra}
\sentenceItem{} \swl{}{What's a Staffordshire?}{pra}
\sentenceItem{} \swl{}{It's a breed of dog.}{pra}
\end{sentenceList}

\noindent{}Here I still don't really have a picture of what it is 
like to tell observationally that a dog is a Staffordshire.  
There may not be any visual cues {\mdash} at least none I 
know of {\mdash} which announce to the world that some dog is
a Staffy (compared to those announcing that it is male,
say).  But insofar as I am acquainted 
with the concept \i{dog breed}, I also understand 
the general pattern of these observations.  For instance 
I may know breeds like poodles or huskies and be able to 
identify \i{these} by distinctive visual cues.  I also 
understand that dogs' parentage is often documented, allowing 
informed parties to know their breeds via those of their 
forebearers.  That is, I am familiar with 
how beliefs about breeds are formed based on 
observation rather than just accepting others' 
reports, so I know the extralinguistic epistemology 
anchoring chains of linguistic reports in this area 
to originating observations {\mdash} even if 
I cannot in this case initiate such a chain myself.}

\p{My overall point is that language enables us to formulate beliefs 
based on the beliefs of others, but this is possible because 
we also realize what it is like to formulate \i{our own}
beliefs, and envision that sort of practice at the 
origin of reports that later get circulated via language.  
If we can't sufficiently picture the originating 
observations, we don't feel like we are grasping 
the linguistic simulacrum of those reports with enough 
substance.  If I never learn what Stafforshire is, 
an assertion that some dogs are Staffordshires 
has no real meaning for me {\mdash} even if I trust the asserter 
and do indeed thereby believe that the dogs are Staffordshires.  
Notice that merely knowing Staffordshire is a breed of 
dog does not expand my conceptual repertoire very much
{\mdash} it does not tell me how to recognize a Staffordshire 
or what I can do with the knowledge 
that a dog is one (it cannot, for instance, help 
me anticipate his behavior).  Nevertheless even (only) knowing 
that Staffordshire is a breed of 
dog seems to fundamentally change the status 
of sentences like \i{those dogs are Staffies}
for me: I do not \i{have} the conceptual machinery 
to exploit that knowledge, but I understand
what \i{sort} of machinery is involved.}

\p{In short, the \i{linguistic} meaning of concepts is tightly bound 
to how concepts factor in perceptual observations anterior 
to linguistic articulation.  As a result, 
during any episode wherein conversants use language to 
compel others' beliefs, an intrinsic dimension of the 
unfolding conversation is that people will form 
their own (extralinguistic) beliefs {\mdash} and can also 
imagine themselves in the role of originating the 
reports they hear via language, whether or not they 
can actually test out the reports by their own 
observations.}

\p{This extralinguistic epistemic 
capacity is clearly exploited by the form of language itself.  
If a tasting organizer hands me a glass and says
\q{This is Syrah}, she clearly expects me to infer that I 
should take the glass from her and taste the wine (and 
know that the glass contains wine, etc.).  These conventions
may be \i{mediated} by language {\mdash} we are more likely to 
understand \q{unspoken} norms by asking questions, until we gain 
enough literacy in the relevant practical domain to 
understand unspoken cues and assumptions.  But many situational 
assumptions are extralinguistic because 
they are (by convention) not explicitly stated, 
even if they accompany content that \i{is} explicitly stated.
\i{This is Syrah} accompanied by the gesture of handing
me a glass is an indirect invitation for me to drink it 
(compare to \i{Please hold this for a second?} or
\i{Please hand this to the man behind you?}).}

\p{I bring to every linguistic situation a 
capacity to make extralinguistc observations, and 
to understand every utterance in the context of hypothetical 
extralinguistc observations from which is originates.  
My conversation peers can use language to trigger 
these extralinguistic observations.  Sometimes the
\q{gap} {\mdash} the conceptual slot which 
extralinguistic reasoning is expected to fill 
{\mdash} is directly expressed, as in \i{See the
dog over there?}.  But elsewhere the
\q{extralinguistic implicature} is more indirect, 
as in \i{This is Syrah} and my expected belief that 
I should take and taste from the glass.  But in any 
case the phenomenon of triggering these 
extralinguistic observation is 
one form of linguistic \q{side effect}, initiating a 
change in my overall conceptualization of a situation by 
compelling me to augment beliefs with new observations.}

\p{All told, then, the language which is presented to me has the effect 
of initiating changes in what I believe 
{\mdash} partly via signifying propositional
content that I could take on faith, but partly 
also via directing my attention and my interpretive 
dispositions to guide me towards extralinguistic 
observations.  Here I will argue that side-effects like 
these are not side-effects \i{of} linguistic meaning, 
but are in some sense \i{constitutive} of 
meaning.}

\spsubsectiontwoline{Side-Effects and Logical Incompleteness}
\p{In the most common analysis, I would argue, side-effects and propositional 
content overlap.  The effect of an utterance, all else being equal, is 
that addressees understand the claimed report as believed by the speaker and 
{\mdash} depending on their deeming the speaker credible {\mdash} accept it as 
a provisional doxic given themselves.  There are many other effects, 
such as how one speaker's assertion changes dialogic context in a way 
that alters the current topical focus and the interpretation of 
context-specific expressions.  But these generally build off of 
a statements' coherence and credibility {\mdash} a non-sensical 
comment is less likely to veer a discourse in new directions 
or reinscribe the dialogic \q{ledger}.  For these reasons, we 
may be tempted to focus analysis on the architecture of 
propositional content rather than the itemization of side-effects.  
I would argue, however, that linguistic side-effects are more 
universal than logically pristine propositional signification, and 
that communicating doxic content is a special case of side effects 
more than side-effects are the passive consequence of rational 
conversation.}

\p{I will defend this perspective via cases where 
language users seem to traffic in 
a relative \i{absence} of semantic determinism, with no 
detrimental effects to the \i{telos} of language in context.  
These cases buttress an idea that language is not targeted at 
doxic specificity as a precondition for meaning in general, 
but rather packages doxa along with other contextualizing 
constituents in the service of pragmatic ends.  
Consider:

\begin{sentenceList}

\sentenceItem{} \swl{}{My colleague Ms. O'Shea would like to interview
Mr. Jones, who's an old friend of mine.  Can he take this call?}{pra}
\sentenceItem{} \swl{itm:Jones}{I'm sorry, this is his secretary.  Mr. Jones
is not available at the moment.}{pra}
\end{sentenceList}

It sounds like Ms. O'Shea is trying to use personal
connections to score an interview with Mr. Jones.  Hence
her colleague initiates a process intended to
culminate in Ms. O'Shea getting on the telephone
with Mr. Jones.  But his secretary demurs with a
familiar phrase, deliberately formulated to
foment ambiguity: (\ref{itm:Jones}) could mean that Mr. Jones
is not in the office, or that he is in a meeting, or he is
unwilling to talk, or even missing (like
the ex-governor consummating an affair in Argentina
while his aides thought he was hiking in Virginia).
Or:

\begin{sentenceList}

\sentenceItem{} \swl{}{Mr. Jones, were you present at a meeting where
the governor promised your employer
a contract in exchange for campaign contributions?}{pra}
\sentenceItem{} \swl{}{After consulting with my lawyers, I decline
to answer that question on the grounds that it
may incriminate me.}{pra}
\end{sentenceList}

Here Mr. Jones neither confirms nor denies his
presence at a corrupt meeting.}

\p{As these examples intimate, the processes language
initiates do not always result in a meaningful
logical structure.  But this is not necessarily
a complete breakdown of language:

\begin{sentenceList}

\sentenceItem{} \swl{itm:isJones}{Is Jones there?}{pra}
\sentenceItem{} \swl{itm:not}{He is not available.}{pra}
\end{sentenceList}

The speaker of (\ref{itm:not}) does not
provide any prima facie logical content: it neither affirms
nor denies Jones's presence.  Nonetheless that speaker
is a cooperative conversational partner
(even if they are not being very cooperative in real life):
(\ref{itm:not}) responds to the implicature in
(\ref{itm:isJones}) that what the
first speaker really wants is
to interview Jones.
So the second speaker conducts what I called
a \q{transform} and maps \i{Jones is here} to
\i{Jones is willing to be interviewed}.
Responding to this \q{transformed} question allows
(\ref{itm:not}) to be (at least) linguistically cooperative
while nonetheless avoiding a response at the
\i{logical} level to (\ref{itm:isJones}).  (\ref{itm:not}) obeys
conversational maxims but is still rather obtuse.}

\p{So one problem for theories that read meanings in terms
of logically structured content {\mdash} something like, the
meaning of an (assertorial) sentence is what the world would be
like if the sentence were true {\mdash} is that the actual
logical content supplied by some constructions
(like \i{Jones is not available}) can be pretty
minimal {\mdash} but these are still valid and
conversationally cooperative segments of discourse.  To be sure, this
content does not appear to be \i{completely}
empty: \q{Jones is not available} means the
conjunction of several possibilities (he cannot be found
or does not want to talk or etc.).
So (\ref{itm:not}) does seem to evoke some
disjunctive predicate.  But such does not mean
that this disjunctive predicate is the \i{meaning} of
(\ref{itm:not}).  It does not seem as if (\ref{itm:not})
when uttered by a bodyguard is intended first and foremost to
convey the disjunctive predicate.  Instead, the
bodyguard is responding to the implicature
in the original \i{Is Jones there?} query {\mdash} the
speaker presumably does not merely want
to know Jones's location, but to see Jones.
Here people are acting out social roles, and just happen
to be using linguistic expressions to negotiate
what they are able and allowed to do.}

\p{Performing social roles {\mdash} including through language {\mdash}
often involves incomplete information: possibly
the secretary or bodyguard themselves do not know
where Jones is or why he's not available.
We could argue that there is \i{enough} information to
still ground \i{some} propositional content.  But this
is merely saying that we can extract some propositional content from
what speakers are supposed to say as social acts, which seems
to make the content (in these kinds of cases)
logically derivative on the enactive/performative
meaning of the speech-acts, whereas a truth-theoretic
paradigm would need the derivational dependence
to run the other way.  By saying \i{Jones is unavailable}
the speaker is informing us that our own prior speech
act (asking to see or talk to him) cannot have
our desired effect {\mdash} the process we initiated cannot be
completed, and we are being informed of that.  The
person saying \i{Jones is unavailable} is likewise
initiating a \i{new} process, one that counters our process
and, if we are polite and cooperative, will have its
own effect {\mdash} the effect being that we do not insist on
seeing Jones.  The goal of \q{Jones is unavailable} is to create
that effect, nudging our behavior in that direction.
Any \i{logic} here seems derivative on the practical initiatives.}

\p{And moreover this practicality is explicitly marked by how
the chosen verbiage is deliberately vague.  The declaration
\q{Jones is unavailable} does not \i{need} logical precision to
achieve its effect.  It needs \i{some} logical content, but it exploits a
kind of disconnect between logical and practical/enactive
structure, a disconnect which allows \q{Jones is unavailable} to
be at once logically ambiguous and practically clear {\mdash}
in the implication that we should not try to see Jones.
I think this example has some structural
analogs to the grandma's window case: \i{there} we
play at logical substitutions to respond practically
to grandma's request in spirit rather than \i{de dicto}.
\i{Here} a secretary or bodyguard can engage in logical
substitution to formulate a linguistic performance
designed to be conversationally decisive
while conveying as little information as possible.  The logical
substitution in grandma's context \i{added} logical content by
trying alternatives for the window being closed; here,
the context allows a \i{diminution} in
logical content.  We can strip away logical detail from
our speech without diminishing the potency of
that speech to achieve affects.  And while the remaining residue
of logical content suggests that some basic logicality is still
essential to meaning, the fact that logical content can
be freely subtracted without altering practical effects
suggests that logic's relation to meaning is something
other than fully determinate: effect is partially autonomous from
logic, so a theory of effect would seem to be
partially autonomous from a theory of logic.
I can be logically vague without being
conversationally vague.   This evidently means that conversational
clarity is not identical to logical clarity.}

\p{Let us agree that {\mdash} beneath surface-level
co-framing complexity {\mdash} many language acts have a
transparent content as \q{doxa} that gets conveyed between
people with sufficiently resonant
cognitive frames.  So \i{in the overall course of communication}
we have propositional contents that converge among discourse 
partners, suspended between the various cognitive and pragmatic 
units which contextualize a given, unfolding dialog.  
There is in short a \i{holistic} mapping between units of 
discourse and \q{units} of propositionality, or \q{doxa}.  
This general observation leaves unstated, however,
\i{how} language elements map to corresponding doxic 
particulars.  I have argued that focusing on the \i{logical 
structures} of propositions can lead us astray if we 
seek to find concordant formations on the language side.}

\p{Consider our attempts to close grandma's kitchen window.  
My analysis related to conceptual \q{transforms}
assumed that we can find, substituting for \i{literal}
propositional content, some \i{other}
(representation of a) proposition that fulfills a
speaker's unstated \q{real} meaning.  Sometimes
this makes sense: the proposition \q{that the
\i{bedroom} window is closed} can neatly,
if the facts warrant, play the role of the
proposition that \i{the kitchen window is closed}.
But we can run the example differently: there
may be \i{no} window open, but instead a draft
caused by non-airtight windows (grandma might ask
us to put towels by the cracks).  Maybe there is
no draft at all (if grandma is cold, we can
fetch her a sweater).  Instead of a single
transform, we need a a system of potential transforms
that can adapt to the facts as we discover them.
Pragmatically, the underlying problem is
that \i{grandma is cold}.  We can address this
{\mdash} if we want to faithfully respond to her request,
playing the role of cooperative conversation
partners (and grandkids) {\mdash} via a matrix
of logical possibilities:

\begin{sentenceList}

\sentenceItem{} \swl{}{If the kitchen window is closed,
we can see if other windows are open.}{pra}
\sentenceItem{} \swl{}{If no windows are open,
we can see if there is a draft through the window-cracks.}{pra}
\sentenceItem{} \swl{}{If there is no draft, we can
ask if she wants a sweater.}{pra}
\end{sentenceList}

This is still a logical process: starting from an
acknowledged proposition (grandma is cold) we
entertain various other propositional possibilities,
trying to rationally determine what pragmas we
should enact to alter that case
(viz., to instead make true the proposition that
\i{grandma is warm}).  Here we are not just testing
possibilities against fact, but strategically
acting to modify some facts in our environment.}

\p{But the kind of reasoning involved here is not logical
reasoning per se: abstract logic does not tell us
to check the bedroom window if the kitchen window is closed,
or to check for gaps and cracks if all windows
are closed.  This all solicits practical, domain-specific knowledge
(about windows, air, weather, and houses).
Yet we are still deploying our practical
knowledge in logical ways {\mdash} there is a logical
structure underpinning grandma's request and
our response to it.  In sum: we (the grandkids)
are equipped with some practical knowledge
about houses and a faculty to logically utilize this
knowledge to solve the stated problem, reading
beyond the \i{explicit} form of grandma's discourse.
We use a combination of logic and background
knowledge to reinterpret the discourse as needed.
By making a request, grandma is not expressing
one attitude to one proposition, so much as
\i{initiating a process}.  This is why it would
be impolite to simply do no more
if the kitchen window is closed: our conversational
responsibility is to enact a process trying to
redress grandma's discomfort, not to entertain the
truth of any one proposition.}

\p{For all that, there is still an overarching logical
structure here that language clearly marshals.  We read
past grandma's explicit request to infer what she is
\q{really saying} {\mdash} e.g., \i{that she is cold} {\mdash} but we
still regard her speech act in terms of its (now indirect)
propositional content.  However, notice 
how our ascertaining this content only one step toward 
legitimate understanding of the original speech-act 
(even accounting for its illocutionary dimensions).  
The doxa are \i{factors} in understanding but, 
given these cases, are not straightforward \i{designata}
of linguistic compounds.  This implies a critique of 
truth-theoretic paradigms from a semiotic and 
compositional perspective: language is not \i{composed}
to convey doxa through semantic reference and 
grammatic form internally (without the mediation of 
extralinguistic cognition); propositional content does 
not \i{fall out} of syntax and semantics.  I will expand 
on this critique in the next section.}

\p{To summarize my current arguments, then, I believe that most sentences 
have an accompanying propositional content, and that during conversations 
we interpret this content as a factor in sentence meanings, becoming aware of 
what our partners believe, desire, or inquire to be the case.  We retain this 
awareness in a cumulative model of conversational context {\mdash} a \q{doxa inventory}
{\mdash} alongside other referential and deictic axes establishing each dialogic 
setting.  Essentially, 
I grant that this doxic layer is central to 
linguistic performance in general {\mdash} but given this very centrality I 
will argue that the logical substratum of language cannot be \i{separated}
from the totality of syntactic, semantic, and pragmatic processing such 
that models based on formal logic could be curated in isolation from 
the overarching interconnectedness of language as a cognitive system.}

\subsection{The Illogic of Syntax}

\p{As I understand it, a non-trivial truth-theoretic semantics 
requires more than a holistic association between 
sentences and propositional content: it requires that this 
association be established \i{by linguistic means} and
\i{on linguistic grounds} (syntax, semantics, pragmatics).  
I will present several arguments against this possibility, in 
the general cases {\mdash} that is, against the possibility that 
for \i{typical} sentences we can analyze syntactic form through 
the lens of the logical structure of propositions signified 
via a sentence; or analyze natural-language semantics through a 
logically well-structured semantics of propositions.  
I will emphasize two issues: first, that the architecture of 
linguistic performances \i{does not}, in the general case,
\i{recapitulate propositional structure}; and, second, 
that language-acts work through gaps in logical specificity that 
complicate how we should theorize the triangular relation between 
surface language, propositional content, and side-effect meanings.}

\p{Since it is widely understood that the essence of language
is compositionality, the clearest path to
a truth-theoretic semantics would be via the
\q{syntax of semantics}: a theory of how
language designates propositional content by
emulating or iconifying propositional structure
in its own structure (i.e., in grammar).
This would be a theory of how linguistic
connectives reciprocate logical connectives,
phrase hierarchies reconstruct propositional
compounds, etc.  It would be the kind of theory motivated
by cases like

\begin{sentenceList}

\sentenceItem{} \swl{}{This wine is a young Syrah.}{log}
\sentenceItem{} \swl{}{My cousin adopted one of my neighbor's dog's puppies.}{log}
\end{sentenceList}

where morphosyntactic form {\mdash} possessives, adjective/noun
links {\mdash} seems to transparently recapitulate predicate
relations.  Thus the wine is young \i{and}
Syrah, and the puppy is the offspring of a dog who
is the pet of someone who is the neighbor of the speaker.  These
are well-established logical forms: predicate conjunction, here;
the chaining of predicate
operators to form new operators, there.  Such are embedded in
language lexically as well as grammatically: the conjunction
of husband and \q{former, of a prior time} yields ex-husband;
a parent's sibling's daughter is a cousin.}

\p{The interesting question is to what extent \q{morphosyntax
recapitulates predicate structure} holds in general
cases.  This can be considered by examining the logical
structure of reported assertions and then the structures
via which they are expressed in language.  I'll
carry out this exercise \visavis{} several sentences,
such as these (supplementing my earlier, more preliminary discussion of 
(\ref{itm:ants})-(\ref{itm:princess})):

\begin{sentenceList}

\sentenceItem{} \swl{itm:maj}{The majority of students polled were
opposed to tuition increases.}{log}
\sentenceItem{} \swl{itm:most}{Most of the students expressed disappointment
about tuition increases.}{log}
\sentenceItem{} \swl{itm:many}{Many students have protested the tuition increases.}{log}
\end{sentenceList}
}

\p{There are several logically significant elements here that
seem correspondingly expressed in linguistic
elements {\mdash} that is, to have some model
in both prelinguistic predicate structure
and in, in consort, semantic or syntactic principles.
All three of (\ref{itm:maj})-(\ref{itm:many}) have similar but not
identical meanings, and the differences are
manifest both propositionally and
linguistically (aside from the specific superficial
fact that they are not the same sentence).
I will review the propositional differences first,
then the linguistic ones.}

\p{One obvious predicative contrast is that
(\ref{itm:maj}) and (\ref{itm:most})
ascribes a certain \i{quality} to students (e.g., disappointment),
whereas (\ref{itm:most}) and (\ref{itm:many})
indicate \i{events}.  As such
the different forms capture the contrast between \q{bearing
quality $Q$} and \q{doing or having done action
$A$}: the former a predication and the latter an event-report.
In the case of (\ref{itm:most}), both forms are available because we can
infer from \i{expressing} disappointment
to \i{having} disappointment.
There may be logics that would map one
to the other, but let's assume we can
analyze language with a logic expressive enough
to distinguish events from quality-instantiations.}

\p{Other logical forms evident here involve how the
subject noun-phrases are constructed.
\q{A majority} and \q{many} imply a multiplicity
which is within some second multiplicity, and
numerically significant there.  The sentences differ
in terms of how the multiplicities are circumscribed.
In the case of \i{students polled}, an extra determinant is
provided, to construct the set of students forming the
predicate base: we are not talking about students in
general or (necessarily) students at one school,
but specifically students who participated in a poll.}

\p{Interrelated with these effects are how the
\i{tuition increases} are figured.  Using
the explicit definite article suggests that there is
\i{some specific} tuition hike policy
raising students' ire.  This would also favor a
reading where \q{students} refers collectively to those
at a particular school, who would be directly affected by the
hikes.  The \i{absence} of an article on
\q{tuition increases} in (\ref{itm:most}) leaves open an interpretation
that the students are not opining on some specific policy, but on
the idea of hikes in general.}

\p{Such full details are not explicitly laid out in the sentences,
but it is entirely possible that they are clear in context.
Let's take as given that, in at least some cases where they would
occur, the sentences have a basically pristine
logical structure given the proper contextual framing {\mdash}
context-dependency, in and of itself, does not weaken
our sense of language's logicality.  In particular,
the kind of structures constituting the sentences'
precise content {\mdash} the details that seem context-dependent
{\mdash} have bona fide logical interpretations.  For
example, we can consider whether students are
responding to \i{specific} tuition hikes
or to hikes in general.  We can consider
whether the objectionable hikes have already
happened or instead are proposed for the future.  Context
presumably identifies whether \q{students} are
drawn from one school, one governmental
jurisdiction, or some other aggregating criteria
(like, all those who took a poll).
Context can also determine whether aggregation is
more set- or type-based, more extensional
or intensional.  In (\ref{itm:maj})-(\ref{itm:many}) the implication
is that we should read \q{students} more as a set or
collection, but variants like \i{students
hate tuition hikes} operates more at the level
of students as a \i{type}.  In \q{students polled}
there is a familiar pattern of referencing a set by
marrying a type (students in general) with a descriptive
designation (e.g., those taking a specific
poll).  The wording of (\ref{itm:maj}) does not
mandate that \i{only} students took the poll; it
does however employ a type as a kind of operator on a set:
of those who took the poll, focus on students
in particular.}

\p{These are all essentially logical structures and can
be used to model the propositional
content carried by the sentences {\mdash} their \q{doxa}.
We have operators and distinctions like past/future,
set/type, single/multiple, subset/superset, and
abstract/concrete comparisons like tuition hikes \i{qua} idea
vs. \i{fait accompli}.  A logical system could
certainly model these distinctions and accordingly capture
the semantic differences between (\ref{itm:maj})-(\ref{itm:many}).
So such details are all still consistent
with a truth-theoretic paradigm, although
we have to consider how linguistic form actually
conveys the propositional forms carved out via
these distinctions.}

\p{Ok, then, to the linguistic side.  My first observation is that some
logically salient structures have fairly clear analogs
in the linguistic structure.  For instance, the logical operator for
deriving a set from criteria of \q{student} merged with \q{taking a
poll} is brought forth by the verb-as-adjective
formulation \i{students polled}.  Subset/superset arrangements
are latent as lexical norms in senses like \i{many} and
\i{majority}.  Concrete/abstract and past/future distinctions
are alluded to by the presence or absence of a definite
article.  So \q{ \i{the} tuition increases} connotes that
the hikes have already occurred, or at least been
approved or proposed, in the past relative to the
\q{enunciatory present} (as well as that they are a
concrete policy, not just the idea),
whereas articleless \q{tuition increases} can be read as
referring to future hikes and the idea of hikes
in general: past and concrete tends to
contrast with future and abstract.}

\p{A wider range of
logical structures can be considered by subtly varying
the discourse, like:

\begin{sentenceList}

\sentenceItem{} \swl{}{Most students oppose the tuition increase.}{sem}
\sentenceItem{} \swl{itm:indef}{Most students oppose a tuition increase.}{sem}
\end{sentenceList}

These show the possibility of \i{increase} being singular
(which would tend to imply it refers to a concrete
policy, some \i{specific} increase), although in
(\ref{itm:indef}) the \i{in}definite article \i{may} connote
a discussion about hikes in general.}

\p{But maybe not; cases like these are perfectly plausible:

\begin{sentenceList}

\sentenceItem{} \swl{itm:today}{Today the state university system
announced plans to raise tuition by
at least 10%.  Most students oppose a tuition increase.}{sem}
\sentenceItem{} \swl{itm:colleges}{Colleges all over the country, facing
rising costs, have had to raise tuition, but
most students oppose a tuition increase.}{sem}
\end{sentenceList}

In (\ref{itm:today}) the definite article could also be used, but saying
\q{ \i{a} tuition increase} seems to reinforce the
idea that while plans were announced, the details
are not finalized.  And in (\ref{itm:colleges}) the plural \q{increases}
could be used, but the indefinite singular connotes the
status of tuition hikes as a general phenomenon
apart from individual examples {\mdash} even though the
sentence also makes reference to concrete examples.  In other
words, these morphosyntactic cues are like
levers that can fine-tune the logical designation
more to abstract or concrete, past or future, as
the situation warrants.  Again, context should
clarify the details.  But morphosyntactic forms
{\mdash} e.g., presence or absence of articles (definite
or indefinite), and singular/plural {\mdash} are
vehicles for language, through its own
forms and rules, to denote propositional-content structures
like abstract/concrete and past/future.}

\p{So these are my \q{concession} examples: cases where 
language structures \i{do}, in their compound architectonics, 
signifying propositional contents {\mdash} and moreover the 
lexical and morphosyntactic cues (like singular/plural or the 
choice of articles) drive this language-to-logic mapping in 
an apparently rule-bound and replicable fashion.  These are 
potential case-studies of how a truth theory of language, 
without neglecting contextual and semantic subtelties,
\i{could} work: capturing granular semantic constructs via sufficiently
nuanced logics, and theorizing word-senses and morphology through 
the aegis of a structural reduction between surface language and 
predicate structure.  My tactic for critiquing truth-theoretic 
paradigms is to argue that many sentences \i{fail} to 
display a mapping between lexico/morphosyntactic details and 
predicate structure \i{in this relatively mechanical fashion}.
By pointing out examples where morphosytax \i{does} rather seamlessly 
recapitulate propositional content (e.g. \i{the tuition hike} plural/definite), 
we can appreciate the more circuitous hermeneutics for examples I 
will present wherein the morphosytax-to-logic translation, while 
present, is not \i{sui generis}.}

\p{Varying the current examples yields cases 
{\mdash} in a sense, intermediate between 
logical transparency and oblique constructions I will 
discuss next {\mdash} 
where logical implications are more circuitous.  For instance,
describing students as \i{disappointed} (\ref{itm:most})
implies that the disliked hikes have already occurred,
whereas phraseology like \q{students are gearing for a fight}
would imply, conversely, that they are sill only planned or proposed.  
The mapping from
propositional-content structure to surface language here
is less mechanical than, for instance, merely
using the definite article on \i{the tuition increases}.
Arguably \q{disappointment} {\mdash} rather than just, say,
\q{opposition} {\mdash} implies a specific timeline
and concreteness, an effect analogous to the definite
article.  The semantic register
of \q{disappointment} bearing this implication is a
more speculative path of conceptual resonances, compared
to the brute morphosyntactic \q{the}.  There is subtle
conceptual calculation behind the scenes in the former
case.  Nonetheless, it does seem as if via this
subtlety linguistic resources are expressing
the constituent units of logical forms, like
past/future and abstract/concrete.}

\p{So, I am arguing (and conceding) that there are units of
logical structure that are conveyed by units of
linguistic structure, and this is partly how
language-expressions can indicate propositional
content.  The next question is to explore this
correspondence compositionally {\mdash} is there a
kind of aggregative, hierarchical order in terms
of how \q{logical modeling elements} fit together,
on one side, and linguistic elements fit
together, on the other?
There is evidence of compositional concordance
to a degree, examples of which I have cited.  In
\i{students polled}, the compositional structure of
the phrase mimics the logical construct {\mdash} deriving
a set (as a predicate base) from a type crossed with
some other predicate.  Another example is the
phraseology \i{a/the majority of}, which directly
nominates a subset/superset relation and so
reciprocates a logical quantification (together
with a summary of relative
size; the same logical structure, but with
different ordinal implications, is seen in cases
like \i{a minority of} or \i{only a few}).
Here there is a relatively mechanical translation
between propositional structuring elements
and linguistic structuring elements.}

\p{However, varying the examples {\mdash} for instance,
varying how the subject noun-phrases are conceptualized
{\mdash} points to how the synchrony between
propositional and linguistic composition can break down:

\begin{sentenceList}

\sentenceItem{} \swl{itm:sas}{Student after student came out against the tuition hikes.}{sem}
\sentenceItem{} \swl{itm:substantial}{A substantial number of students
have come out against the tuition hikes.}{sem}
\sentenceItem{} \swl{itm:mass}{The number of students protesting the tuition hikes
may soon reach a critical mass.}{sem}
\sentenceItem{} \swl{itm:tipping}{Protests against the tuition hikes
may have reached a tipping point.}{sem}
\end{sentenceList}

Each of these sentences says something about a large number
of students opposing the hikes.  But in each
case they bring new conceptual details to the fore, and
I will also argue that they do so in a way that
deviates from how propositional structures are composed.}

\p{First, consider \i{student after student} as a way of
designating \i{many students} (I analyzed the very similar 
case (\ref{after}) in terms of \q{space building}).  
There is a little more
rhetorical flourish here than in, say, \i{a majority
of students}, but this is not just a matter of
eloquence (as if the difference were stylistic,
not semantic). \q{Student after student} creates a
certain rhetorical effect, suggesting via how it
invokes its multiplicity a certain recurring or
unfolding phenomenon.  One imagines the speaker, time
and again, hearing or encountering an angry student.
To be sure, there are different kinds of contexts
that are consistent with (\ref{itm:sas}): the events could
unfold over the course of a single hearing
or an entire semester.  Context would foreclose
some interpretations {\mdash} but it would do so in any
case, even with simpler designations like \i{majority
of students}.  What we \i{can} say is that the speaker's
chosen phraseology cognitively highlight a
dimension in the events that carries a certain
subjective content, invoking their temporality and
repetition.  The phrasing carries an effect of
cognitive \q{zooming in}, each distinct event
figured as if temporally inside it; the sense
of being tangibly present in the midst of the event is
stronger here than in less temporalized language,
like \q{many students}.  And then at the same time
the temporalized event is situated in the context of many
such events, collectively suggesting a recurring presence.
The phraseology zooms in and back out again,
in the virtual \q{lens} our our cognitively figuring
the discourse presented to us {\mdash} all in just
three or four words.  Even if \q{student after student}
is said just for rhetorical effect {\mdash} which
is contextually possible {\mdash} \i{how} it
stages this effect still introduces a subjective
coloring to the report.}

\p{Another factor in (\ref{itm:sas}) and (\ref{itm:substantial})
is the various possible meanings of
\q{come out against}.  This could be
read as merely expressing a negative opinion, or as a
more public and visible posturing.  In fact, a similar
dual meaning holds also for \q{protesting}.  Context,
again, would dictate whether \i{protesting} means
actual activism or merely voicing displeasure.  Nonetheless,
the choice of words can shade how we frame situations.
To \i{come out against} connotes expressing disapproval
in a public, performative forum, inviting
the contrast of inside/outside (the famous example
being \i{come out of the closet} to mean publicly
identifying as \LGBTQ{}).  Students may not literally
be standing outside with a microphone, but {\mdash} even
if the actual situation is just students
complaining rather passively {\mdash} using \i{come out against}
paints the situation in an extra rhetorical hue.  The students
are expressing the \i{kind} of anger that can goad
someone to make their sentiments known theatrically and
confrontationally.  Similarly, using \q{protest} in lieu
of, say, \q{criticize} {\mdash} whether or not students are actually
marching on the quad {\mdash} impugns to the students a level
of anger commensurate with politicized confrontation.}

\p{All these sentences are of course \i{also} compatible with
literal rioting in the streets; but for sake of argument
let's imagine (\ref{itm:sas})-(\ref{itm:tipping}) spoken in contexts
where the protesting is more like a few comments to a school
newspaper and hallway small-talk.  The speakers have still
chosen to use words whose span of meanings
includes the more theatrical readings: \q{come out against}
and \q{protest} overlap with \q{complain about} or \q{oppose},
but they imply greater agency, greater intensity.  These lexical
choices establish subtle conceptual variations; for instance,
to \i{protest} connotes a greater shade of anger than to \i{oppose}.}

\p{Such conceptual shading is not itself unlogical; one
can use more facilely propositional terms to evoke
similar shading, like \q{very angry} or \q{extremely angry}.
However, consider \i{how} language like
(\ref{itm:sas})-(\ref{itm:tipping})
conveys the relevant facts of the mater: there is
an observational, in-the-midst-of-things staging at
work in these latter sentences that I find missing
in the earlier examples. \q{The majority of} sounds
statistical, or clinical; it suggests journalistic reportage,
the speaker making an atmospheric effort to sound like someone
reporting facts as established knowledge rather than
observing them close-at-hand.
By contrast, I find (\ref{itm:sas})-(\ref{itm:tipping})
to be more \q{novelistic} than
\q{journalistic}.  The speaker in these cases is reporting
the facts by, in effect, \i{narrating} them.  She is building
linguistic constructions that describe propositional
content through narrative structure {\mdash} or, at least, cognitive structures
that exemplify and come to the fore in narrative understanding.  Saying
\q{a substantial number of students}, for example, rather than
just (e.g.) \i{many} students, employs semantics
redolant of \q{force-dynamics}: the weight of student
anger is described as if a \q{substance}, something with the
potency and efficacy of matter.}

\p{This theme is also explicit in
\q{critical mass}, and even \i{tipping point} has material
connotations.  We can imagine different versions of what
lies one the other side of the tipping point
{\mdash} protests go from complaining to activism?  The school
forced to reverse course?  Or, contrariwise, the
school \q{cracking down} on the students
(another partly imagistic, partly force-dynamic metaphor)?
Whatever the case, language
like \q{critical mass} or \q{tipping point} is language
that carries a structure of story-telling;
it tries tie facts together with a narrative coherence.  The
students' protests grew more and more strident until ...
the protests turned aggressive; or the school dropped its plans;
or they won public sympathy; or attracted media attention, etc.
Whatever the situation's details, describing the facts in
force-dynamic, storylike, spatialized language (e.g. \q{come \i{out}
against}) represents an implicit attempt to report
observations or beliefs with the extra fabric and completeness
of narrative.  It ascribes causal order to how the
situation changes (a critical mass of anger could \i{cause} the
school to change its mind).  It brings a photographic or
cinematic immersion to accounts of events and descriptions:
\i{student after student} and \i{come out} invite us
to grasp the asserted facts by \i{imagining} situations.}

\p{The denoument of my argument is now that these narrative, cinematic,
photographic structures of linguistic reportage {\mdash} signaled
by spatialized, storylike, force-dynamic turns of phrase {\mdash}
represent a fundamentally different way of signifying
propositional content, even while they
\i{do} (with sufficient contextual grounding) carry
propositional content through the folds of the narrative.
I don't dispute that hearers understand logical forms
via (\ref{itm:sas})-(\ref{itm:tipping})
similar to those more \q{journaslistically}
captured in (\ref{itm:maj})-(\ref{itm:many}).  Nor do I
deny that the richer rhetoric of
(\ref{itm:sas})-(\ref{itm:tipping}) play a logical
role, capturing granular shades of meaning.  My
point is rather that the logical picture painted by the latter
sentences is drawn via (I'll say as a kind of suggestive
analogy) \i{narrative structure}.}

\p{I argued earlier that elements of propositional
structure {\mdash} for example, the
set/type selective operator efficacious in \i{students polled} {\mdash} can
have relatively clean morphosyntactic manifestation in
structural elements in language, like the verb-to-adjective
mapping on \i{polled} (here denoted, in English, by unusual
word position rather than morphology, although the
rules would be different in other languages).  Given my
subsequent analysis, however, I now want to claim that
the map between propositional structure and linguistic structure
is often much less direct.  I'm not arguing that
\q{narrative} constructions lack logical structure, or
even that their rhetorical dimension lies outside
of logic writ large: on the contrary, I believe
that they use narrative effects to communicate granular
details which have reasonable logical bases, like
degrees of students' anger, or the causative
interpretation implied in such phrases as \i{critical mass}.
The rhetorical dimension does not prohibit a reading
of (\ref{itm:sas})-(\ref{itm:tipping}) and 
(\ref{itm:students})-(\ref{itm:parents}) 
as expressing propositional content {\mdash} and using
rhetorical flourishes to do so.}

\p{I believe, however, that \i{how} they do so unzips any neat
alignment between linguistic and propositional structure.
Saying that students' protests \q{may have reached a critical mass}
certainly expresses propositional content (e.g., that enough
students may now be protesting to effectuate change),
but it does so not by mechanically asserting its propositional
idea; instead, via a kind of mental imagery which portrays
its idea, in some imaginative sense, iconographically.
\q{Critical mass} compels us to read its meaning imagistically;
in the present context we are led to actually visualize
students protesting \i{en masse}.  Whatever the actual, empirical
nature of their protestation, this language paints a picture
that serves to the actual situation as an interpretive
prototype.  This is not only a conceptual image, but
a visual one.}

\p{Figurative language {\mdash} even if
it is actually metaphorical, like \q{anger boiling over}
{\mdash} has similar effect.  Alongside the analysis
of metaphor as \q{concept blending}, persuasively
articulated by writers like Gilles Fauconnier and
Per Aage Brandt, we should also recognize how metaphor
(and other rhetorical effects) introduces into
discourse language that invites visual imagery.  Sometimes
this works by evoking an ambient spatiality (like
\q{come out against}) and sometimes by figuring
phenomena that fill or occupy space (like \q{students protesting}
{\mdash} one salience of this language is that we imagine
protest as a demonstrative gesture expanding outward, as if
space itself were a theater of conflict:
protesters arrayed to form long lines, fists splayed
upward or forward).  There is a kind of visual patterning
to these evocations, a kind of semiotic grammar:
we can analyze which figurative senses work via connoting
\q{ambient} space or via \q{filling} space,
taking the terms I just used.  But the details of such a
semiotic are tangential to my point here, which is that
the linguistic structures evoking
these visual, imagistic, narrative frames are not
simply reciprocating propositionl structure
{\mdash}- even if the narrative frames, via an \q{iconic}
or prototype-like modeling of the actual situation,
\i{are} effective vehicles for \i{communicating}
propositional structure.}

\p{What breaks down here is not propositionality but \i{compositionality}:
the idea that language signifies propositional content \i{but
also} does so compositionally, where we can
break down larger-scale linguistic elements to smaller parts
\i{and} see logical structures mirrored in the parts'
combinatory maxims.  In the later examples, I have argued that
the language signifies propositional content by creating narrative
mock-ups.  The point of these imagistic frames
is not to recapitulate logical structure, but to have a kind
of theatrical coherence {\mdash} to evoke visual and narrative order,
an evolving storyline {\mdash} from which we then understand
propositional claims by interpreting the imagined scene.
Any propositional signifying in these kinds of
cases works through an intermediate stage of
narrative visualization, whose structure is
holistic more than logically compositional.
It relies on our faculties for imaginative
reconstruction, which are hereby drafted into
our language-processing franchise.}

\p{This kind of language, in short, leverages its
ability to trigger narrative/visual framing as a
cognitive exercise, intermediary to the eventual
extraction of propositional content.  As such
it depends on a cognitive layer of narrative/visual
understanding {\mdash} which, I claim, belongs
to a different cognitive register than building
logical models of propositional content directly.}

\subsection{Triggers vs. Compositionality}
\p{In the absence of a compelling analysis of \i{compositionality}
in the structural correspondence between narrative-framed language
and logically-ordered propositional content, I consequently think we
need a new theory of how the former signifies the latter.  My own
intuition is that language works by trigering \i{several
different} cognitive
subsystems.  Some of these hew closely to predicate
logic; some are more holistic and narrative/visual.
Cognitive processes in the second sense may be informed
and refined by language, but they have an extralinguistic
and prelinguistic core: we can exercise faculties of narrative
imagination without explicit use of
language (however much language orders our imaginations
by entrenching some concepts more than others, via lexical
reinforcement).}

\p{I'm not just talking here about \q{imagination} in the sense
of fairy tales: we use imaginative cognition to make sense
of any scenario described to us from afar.  When presented with
linguistic reports of not-directly-observable situations,
we need to build cognitive frames modeling the context
as it is discussed.  In the terms I suggested earlier, we
build a \q{doxa inventory} tracking beliefs and
assertions.  Sometimes this means internalizing
relatively transparent logical forms.  But sometimes
it means building a narrative/visual account, playing
an imaginary version of the situation in our minds.
Language could not signify in its depth and
nuance without triggering this \i{interpretive-imaginative}
faculty.  Cognitively, then, language is an
\i{intermediary} to this cognitive system.  
Using terms from Olin Vakarelov's \q{Interface Theory 
of Meaning} \cite{OrlinVakarelov}, \cite{VakarelovAgent}
we might say that 
language is an \i{interface to} interpretive-imaginative
cognitive capabilities.}

\p{My argument is then that extra-linguistic cognition {\mdash} e.g., narrative 
construals, or situational understanding {\mdash} can supply crucial 
steps in the emergence of sentences' intended propositional content; 
so relying on logical assessment of propositional content and 
correlated linguistic patterns is, at best, incomplete.  
I have presented numerous sentences which, in my opinion, evince 
patterns whose rational contributions lie outside linguistic 
cognition proper, but which become attached as interpretations 
or refinements of given linguistic content; for example 
the semantic model of \i{critical mass} or \i{tipping point}
encapsulating a complex situational presenting.}

\p{The extra-linguistic source of many signifying contributors can 
be observed directly in reference to more formalized 
(say, Dependency-Grammatic) analyses (or indeed what they 
leave out).  In the (\ref{itm:actually}) case, the modifier
\i{actually} implies that the asserted fact is somehow surprising 
and counterintuitive; it's easy enough to conclude that the 
speaker senses a certain lexical frisson, particularly in 
how \i{camping} is described as \i{lodging}.  So the work 
of (\ref{itm:actually}) is not just to point out that 
people camp on the beach, but also that this arrangement is the 
most popular \q{lodging} even though in its usual sense
\i{lodging} applies more to hotels and inns.  This latter 
pattern circuits especially through the words \i{actually},
\i{lodging}, and \i{camping}, none of which are word-relations 
identified by the Universal Dependency parse (refer back to 
Figure~\ref{fig:actuallydbl}) but which can be seen as a second-order 
network superimposed on the sentence's syntactic core.}

\p{For other examples, in (\ref{itm:ambience}) the \i{colonial ambience} and
\i{tropical climate} disjunction partly shifts the sense of
\i{the city} from its status as something constructed and
historical (the \i{architecture} is colonial) to a geographic location.  We 
of course read \i{the climate} as \i{its} (the city's) climate, which 
establish a link between the two \i{the}.  This is almost anaphoric 
{\mdash} if the phrase were \i{its climate} we would resolve \i{it} backward
to \i{the city} {\mdash} but the pattern is varied, by replacing
\i{its climate} with \i{the climate}, which conceptually foregrounds 
the aspect of climate as an ambient phenomenon (amenable to a 
definite article) rather than just a possession (viz., a property 
of geographic places).  So there are several intersecting patterns 
{\mdash} both referential commonality and interpretive changes (see also 
the \q{Liverpool} examples for city qua architecture and qua locale) 
{\mdash} between the two sentence-halves, which are not described in the 
parse-graph.  And in, say,

\begin{sentenceList}

\sentenceItem{} \swl{itm:pipe}{We understand that the pipe fitters are also planning to picket the Lake Worth, Florida project as well.}{sco}
\udref{en_gum-ud-train}{email-enronsent03_01-0143}
\end{sentenceList}

there is a two-layer scope implication: given \i{also} we read the hearer as knowing that 
they are picketing some \i{other} project.  But we also hear \i{the pipe fitters}
as referring to some specific group of pipefitters, with which the speaker has 
some contractual relation; we do not interpret this as a comment about pipe-fitters in 
general.}

\p{The details of (\ref{itm:pipe}), in short, bound the scope of
\i{the pipe fitters} from both above and below.  The importance of
\i{also} for this interpretive understanding is elided from the 
corpora parse-graph, which just sees \i{also} as a common auxiliary.  
In general, the extra \q{layers} of meaning I have identified for 
(\ref{itm:actually}), (\ref{itm:ambience}), and (\ref{itm:pipe}) 
require some interpretive (and seemingly extralinguistic) 
reasoning, so they are not explicitly traced in purely 
formulaic Intermediate Representations involving core 
linguistic aspects, such as parse-graphs \visavis{}
syntax.}

\p{The extralinguistic dimensions of language are not, of 
course, completely apart from language: extralinguistic 
reasoning is appropriate to fill in the gaps left 
by straightforward syntactic or logico-semantic 
intellection alone, but those gaps exist \i{because}
language leads (via intra-linguistic structures) to 
signifying complexes.  The logic and conceptual detail 
achieved through intra-linguistic processes set forth 
the parameters on extra-linguic cognition that 
supplements them; so analytic coverage of the 
intra-linguistic aspect is still requisite for 
good analysis of the extra-linguistic.  This 
formal necessity, however, should not be mistaken 
for a belief that logically-inflected models 
of syntax and semantics are complete or 
self-sufficient.  Full implementation of a 
rigorous, logical syntactic-semantic system still 
does not get us to \i{language}.}

\thindecoline{}

\p{This then raises the question of what kinds of 
formal systems \i{are} appropriate models for language.  
In a truth-theoretic semantics, the founding analogy is 
that meanings are propositional, so that language builds 
up aggregate content much as logical expressions build 
up complex propositions.  This analogy allows the 
resources of mathematical logic to transfer, with 
some degree of rigor, over to linguistic 
representation {\mdash} not only predicate logic but 
also modal logic, typed lambda calculus, and 
type theory in general.  On the other hand, if 
we accept the picture that linguistic structures 
are often triggers to extralinguistic cognitive 
activity {\mdash} that this is their contribution 
to language comprehension {\mdash} then we may need 
to draw inspiration from a different suite of 
formal theories.}

\p{Probably influenced by early Analytic Philosophy, 
linguistics' formalizing paradigms have gravitated 
to mathematical logic, reflecting the influence of  
early 20th-century symbolic logic on a broad swath of 
Anglo-American science and intellectual life.  
Mathematics {\mdash} and by extension formal logic as a
theory of mathematical foundations {\mdash} came to be 
seen as a template for rigorous analysis to which 
philosophy should aspire; on that hypothesis 
the speculative styles of phenomenology and 
Continental Philosophy were widely seen as
methodologically flawed, compared to the 
systematized argumentation presented by the likes
of Gottlob Frege, Bertrand Russel, Rudolf Carnap, or 
Willard Van Orman Quine.  It is not a stretch to 
see the contemporary split between Cognitive and 
Computational linguistics as retracing, reiterating, 
or in a sense derived from last century's 
divorce between Analytic and Continental Philosophy.}

\p{On the other hand, in the digital age, a 
host of new formalizing projects offer an 
alternative to this mathematics-centric 
ideology.  These probably include \AI{} research 
to some degree, but also many other 
technical disciplines associated with the 
inner workings of computers and software 
{\mdash} compilers and compiler design; 
stack machines and other low-level computing 
frameworks; applied type theory; database implementation; 
process algebras and other theories of 
communicating, interdependent, or concurrent 
processes.  This technical ecosystem is 
arguably a better mine for linguistic 
inspiration precisely because all of these 
fields grapple with the problem of how 
to wrangle rationalistic behavior out of 
stubbornly unintelligent machines.  
When trying to coax a compiler to identify 
the correct functions for function-calls 
notated at each source-code location, for 
example, it does not directly matter whether 
the function-resolution is theoretically 
transparent {\mdash} i.e., that a correct 
and unique resolution can be demonstrated
by formal type analysis.  A compiler 
(or other software) ecosystem does not 
internally manifest type theory; at best
\i{humans} engineer software to abide by 
regulations which guarantee program  
behaviors through type-theoretic 
(and similar) analysis.  But engineering 
computational systems in conformance with 
mathematical, or otherwise formalizable, 
abstractions is itself a significant 
technical project.  The mere presence of 
abstractly provable structural properties 
is of only indirect relevance to 
computer implementations.}

\p{Likewise, it seems, the mere presence of system 
invariants which can be formalized from mental 
activity and given their own eidetic analysis is 
only indirectly relevant to cognitive science
\i{per se}.  Presumably brains have evolved under 
evolutionary pressures to undergird reasonable behavior; 
and perhaps structuralistic laws alongside physical ones 
are inevitable for any intelligent systems 
manifest in physical substance.  So it is likely that 
logical and other formal theories have some explanatory 
value in identifying regulative parameters influencing 
the gestation and physical realization of intelligence.  
But unlike mathematics, where the mere demonstration of 
the necessary truth of a theorem guarantees its existence 
as mathematical fact, in the realms of both cognition 
and computation abstract formal truths are not 
manifested merely by being true.  Organized systems 
must provide a dynamic space whose states are constrained 
by the relevant formal properties, in order for 
these formal certainties to be empirically 
influential.}

\p{Ultimately, computers have a limited collection of kernel 
functions, and software stacks are architectures that 
strategically call these functions in turn {\mdash} 
through several layers of translation and interpretation 
{\mdash} to achieve holistic effects (like displaying a 
textual document on a computer screen, or creating a 
three-dimensional animation from a set of 
geometric instructions).  While some logical 
considerations structure the process by which high-level 
computer code is translated to low-level system functionality, 
the manifest ground of this system is not any logical 
structure but the low-level functions themselves, along with 
their physical side-effects (like rendering colors on-screen).  
This is a useful analogy for natural language: the 
material unfolding of language as an empirical 
phenomenon is not a direct concretization of any 
logical structure, but rather the inventory of 
cognitive processes which language may trigger.  
Extralinguistic cognition is then analogous, in the 
human mind, to a computer's \q{system kernel}.  
While \q{mind as computer} analogies are 
usually reductionistic {\mdash} software has no human 
subtlety to (inter)subjectivity {\mdash} this architectural 
metaphor perhaps has some merit.  At least it can 
point to topics in computer science that could be
placed alongside predicate logic as formalizing 
intuitions for the study of natural language.}

%\section{Cognitive and Computational Processes}
\label{s4}
\p{Any attempt to bridge Computational Linguistics and
Cognitive Grammar or Phenomenology must solicit one or several
\q{founding analogies}, linking phenomena on the
formal/computational side with those on the
cognitive/computational side.  Here, I will start from
the analogy of \i{cognitive} and \i{computational} \i{process},
or generically \q{process} (of either variety).
Processes, per se, I will
leave undefined, although a \q{computational} process
can be considered roughly analogous to a single
procedure implemented in a computer programming language.
The story I want to tell goes something like this: understanding
language involves many cognitive processes, many of
which are subtly determined by each exact language artifact
and the context where it is created.  Properly understanding a
piece of language depends on correctly weaving together
the various processes involved in understanding its
component parts, and the structure of the
multi-process integration is suggested by the grammar of
the artifact.  Grammar, in a nutshell, uses relationships
between words to evoke relationships between
cognitive processes.}

\p{My formal elaboration of this model will be inspired at an
elementary level by process \i{algebra} in the computational
setting, as well as applied \i{type theory}.
Inter-process relations are the core topic of Process
Algebra, including sequentiality (one process followed by
another) and concurrency (one process executing alongside
another).  In practice, detailed research around Process Algebra
seems to focus especially on concurrency, perhaps because
this is the more complex area of application
(designing computer systems which can run multiple threads in
parallel).  It is likewise tempting to
imagine that cognitive-linguistic processes exhibit some
degree of parallelism, so that the various pieces of
understanding \q{fall into place} together as we grasp
the meaning of a sentence (henceforth using \i{sentence} as a
representative example of a mid-size language artifact in general).
Nevertheless, I will focus more on \i{sequential} relations between
processes, suggesting a language model (even if rather idealized)
where cognitive processes unfold in a temporal order.}

\p{On both the cognitive and computational side, temporality is relative
rather than quantitative: the significant detail is not
\q{before} and \q{after} in the sense of measuring time but rather how
one process logically precedes another in effects and prerequisites.
No theoretical importance is attached to \i{how long} it takes
before processes finish, or how much time elapses between
antecedent and subsequent processes (in contrast to subjects like
optimization theory, where such details are often significant).
We can set aside notions of a temporal continuum
where subsequent processes occupy disjoint, extended time-regions;
instead, one process follows another if anything affected by the first
process reflects this effect at the onset of the second process.
Time, in this sense, only exists as manifest in the variations
of any state relevant to processes {\mdash} in the computational
context, in the overall state of the computer (and potentially
other computers on a network) where a computation is
carried out.  Two times are different only insofar as the
overall state at one time differs from the state at the second time.
Time is \i{discrete} because the relevant states are discrete, and
because beneath a certain scale of time delta there is no
possibility of state change.}

\p{Analogously, in language, I suggest that we set aside notions of
an unfolding process reflecting the temporality of expression.
Of course, the fact that parts of a sentence are heard first
biases understanding somewhat; and speakers often exploit
temporality for rhetorical effect, elonging the pronunciation
of words for emphasis, or pausing before words to
signal an especially calculated word choice, for example.
These data are not irrelevant, but, for core semantic and
syntactic analysis, I will nonetheless treat a sentence as
an integrated temporal unit, with no value attributed to
temporal ordering amongst words except insofar as temporal
order establishes word order and word order has grammatical
significance in the relevant natural language/dialect.}

\p{While antecedent/subsequent inter-process relations are among those
formally recognized in Process Algebra, this specific genre
of relation is implicit to other models important
to computer science, such as Type Theory and Lambda Calculus.
If \typeT{} is a type, then any computational process
which produces a value of type \typeT{} has a corresponding
(\q{functional}) type (for sake of discussion, assume a \q{value}
is anything that can be encoded in a finite sequence of numbers
and that \q{types} are classifications for values that introduce
distinctions between functions {\mdash} e.g., the function to add two
integers is different than the function to add two decimals; more
rigorous definitions of primordial notions like \q{type} and
\q{value} are possible but not needed for this paper).
Similarly a process which takes as \i{input} a value of
\typeT{} is its own type.  If two processes have these two
types respectively {\mdash} one outputs \typeT{} and the other
inputs \typeT{} {\mdash} then the two can be put in sequence, where
the output from the antecedent becomes the input to the subsequent.
In this manner inter-process sequential relations become
subsumed into \q{type systems} and can be studied using
type-theoretic machinery rather than Process Algebras or
Process Calculi as such.}

\p{The above comments apply to type theory applied in 
constructed environments like computer programming 
languages; but as I have discussed here 
there exists a robust type-theoretic tradition
in (Natural Language) semantics, which is disjoint from
but not entirely irrelevant to the type systems of
formal and programming languages.  Semantic types are
recognized at several different levels of classification 
{\mdash} I proposed the terms \i{macrotype}, \i{mesotype}, 
and \i{microtype}.  Some of the most 
interesting type-theoretic effects
involve medium-grained semantic criteria that are
more general than lexical entries but more specific than
Parts of Speech; this is the level where linguists have 
seemed to find the most fertile applications of 
sophisticated aspects to formal type theory, like 
dependent types and type-coercion.  This perhaps 
reflects the situation of \q{mesotypes} as 
intermediary between and therefore interconnecting 
types in a mostly grammatic (Part of Speech) and 
a mostly semantic (lexical) sense; and therefore 
a topic or tool for both syntax and semantics.}

\p{Here my argumentation is informed 
particularly by the merged notion of \i{typed}
processes.  If we say that something (sticking with the 
mesotype level) has the \i{type}
of a physical-body noun {\mdash} that \q{Physical Body} is
a type in the overall semantics of language {\mdash} then
I propose a corresponding type of cognitive
(perceptual and conceptual) processes characteristic
of perceiving and reasoning about physical things.  A particular
designatum {\mdash} a bag of rice, say {\mdash} is subsumed under
the semantic type insofar as our perceptual encounters with that
thing {\mdash} and/or our conceptual exercises pertaining to
its properties and proclivities (like being difficult to carry)
{\mdash} are roughly prototyped by a certain generic kind of
cognitive process.  This assumes that there is a similitude among
processes of perceiving and thinking about physical bodies
(at least the mid-sized, quotidian physical things that tend
to enter nonspecialist language) sufficient to subsume them under
a common prototype, which I then argue forms the cognitive
substratum for the semantic type \q{Physical Object}.
Moreover, I contend a similar cognitive substratum
can be found for other mid-scale semantic types that underlie analyses
of semantic acceptability and metaphoricality, like
\q{Living Thing}, \q{Sentient Living Thing} (\i{flowers want
sunshine} is metaphorical because it ascribes propositional
attitudes to something whose type does not literally support them),
and \q{Social Institutions} (\i{The newspaper you're reading
fired its editor} exhibits a \q{type coercion} where \i{newspaper} is read
first as an object and then as a company).  One feature of semantic
types is the lexical superposition of different types to produce what
(in a slightly different context) Fauconnier calls a \q{blend}:
in \i{Liverpool, which is near the ocean, built new docks}
(see \ref{itm:Liverpool}), the
city is treated as both a geographic region and a body politic.}

\p{

\q{Weighs}, too, as a verb, can be given a typed-process
interpretation.  In its least metaphoric sense, \q{to weigh}
connotes a practical action of measuring some object's weight by
using something like a scale; as \i{cognitive} process the
verb embodies are ability to plan, reflect upon, or contemplate
this practice.  So an \q{idea weighing 5 pounds} is anomalous because
it is hard to play out in our minds a form of this practical act
where the thing being weighed is mental.  However, there are plenty
of more figurative uses related to \q{weighing ideas}, \q{heavy ideas},
and so forth, so we are able to isolate the dimension of
\q{judging} and \q{measuring} which is explicit in literal
\q{weighing}, and abstracting from the physical details use
\q{weigh} to mean \q{measure} or \q{assess} in general.
The phrase \i{weigh an idea} therefore connotes its own cognitive
process {\mdash} imagining someone thinking about the idea in an
evaluative way {\mdash} but this figurative \q{script} is closed off by
\q{5 pounds} which forces us to conceive the weighing literally
with a scale, not figuratively as a kind of mental assessment.
Once again, the type anomaly can be seen as a failure to
map the linguistic senses evident in a sentence to an internally
consistent set of cognitive procedures for dilating the semantic
content seeded within each word.}

\p{Notice that I am treating cognitive processes, in themselves,
as semantic more than grammatical phenomena.  Literally,
weighing something is a multi-step act
(lifting it on the sale, reading the measurement), and even in
our mental replay of hypothetical weighing-acts it seems impossible
not to imagine distinct phases (just as it is impossible not
to picture left and right sides of an imaginary cow).
However, I assume that the cognitive script is figured by the
lexeme \q{weighs} as a connotative unit: whatever internal structure
our mental script of \q{weighing something} has,
this structure is not a \i{linguistic} structure that must be
encoded grammatically.  Similarly, the concept
\i{buttered toast} suggests a confluence of
perceptual, physical-operational, and conceptual aspects
{\mdash} we are inclined to regard toast as \i{buttered} if it
looks a certain way and also if we have seen someone apply
butter to it (or have done so ourselves) and also if
we are in a context where we expect to find toast that may be
buttered (we are not disposed to call a piece of bread in a
grocery store \q{buttered toast} even if it has that appearance).
So the adjective \i{buttered} introduces multiple cross-modal
parameters in addition to the underlying concept \i{toast}; but I feel
that the lexeme aggregates these parameters into a
single \i{linguistic} unit.  In Langacker's terms, the various
elements of \q{buttered} do not suggest \i{constructive effort},
as if deliberate \i{linguistic} processing were needed to unpack the
linguistic entity to its constituent parts.  Instead, \q{buttered}
functions \i{semantically} as a unit (and likewise syntactically
as the unit entering relations with other words {\mdash} e.g. buttered-toast
is an adjective/noun pair, not the noun \i{butter} at the root of the
adjective) {\mdash} even if its cognitive process
is multi-faceted.  Indeed, this is precisely the signifying economy
of language: it captures complex cognitive procedures by
iconic, repeatable lexical units.}

\p{On that theory, tieing specific word-senses to stereotyped
cognitive processes is a matter of semantics, not grammar per se.
Grammar, I contend, comes into play when multiple processes need to
be integrated.  The concept \q{buttered toast}, for example, seems
to start from a more generic concept (toast) and then add
extra detail (the buttering, with all that implies conceptually,
pragmatically, and sensorially).  This is suggested by the
substitutability of just \i{toast} for \i{buttered toast}:

\begin{sentenceList}

\sentenceItem{} \swl{}{I snacked on toast and coffee.}{sem}
\sentenceItem{} \swl{}{I snacked on buttered toast and iced coffee.}{sem}
\end{sentenceList}

Because the first sentence is perfectly clear, it seems
that the ideas expressed (at least in this context) by
\i{toast} and \i{coffee} are reasonably complete
in themselves, so the adjectives have the effect of starting
with a complete idea and adding on extra detail.  Procedurally,
then, it seems like we have some process which takes us to
\q{toast} and \q{coffee} and then, subsequent to that
(logically if not temporally) we add the wrinkle of
re-conceiving the toast as buttered and the coffee as iced.
In short, the adjective-noun pairing is compelling us to
run a pair of cognitive processes in sequence, one
establishing the noun-concept as a baseline and one adding
descriptive detail by an \q{adjectival}, a specificational
process.}

\p{Counter to that analysis, someone might judge that
phrases like \q{buttered toast} and \q{iced coffee} are
conventional enough that we don't interpret them through
two meaningfully disjoint processes.  This is entirely possible,
given how erstwhile aggregate expressions become established units.  
Different phrases exhibit different levels of entrenchment:

\begin{sentenceList}

\sentenceItem{} \swl{}{I snacked on toast and instant coffee.}{sem}
\sentenceItem{} \swl{}{I snacked on toast and Eritrean coffee.}{sem}
\end{sentenceList}

Arguably \q{instant coffee} is a de facto lexical unit, partly
because reading it in terms of constituent parts is rather
nonsensical (there's no non-oblique way to understand
\q{coffee} being qualified as \q{instant}).  Surely, however,
\i{Eritrean coffee} is heard as a compound phrase
(at least in 2019 {\mdash} it is unlikely, but not
impossible, that future Eritrean coffee growers will be so
successful that we hear the phrase as a brand name
or culinary term of art, like \q{Hershey's kisses}
or \q{French toast}).  The status of \i{iced coffee} is probably
somewhere between these two.  But to the degree that
a language element (whether word or phrase) is entrenched and
generally processed linguistically as a unit, I maintain,
it tends to be governed by an integrally complete cognitive
process {\mdash} not necessarily one without inner structure, but where the
elements of this structure piece together perceptually and
situationally, rather than seeming to be
\i{linguistically} disjoint conceptualizations that are brought together
by the shape of linguistic phrases.  Conversely, where a cognitive
process has this integral character, discursive pressures nudge
the language toward entrenching some descriptive phrase as a
quasi-lexeme; what starts being heard as a compound designation
evolves to the point where language users don't attend to
constituent parts.}

\p{Obviously, this theory presupposes that there is an available
distinction to be drawn between a \q{procedural} synthesis of
disparate cognitive processes and a perceptual and/or conceptual
synthesis constitutive of individual cognitive episodes.
Phenomonology seems to back this up {\mdash} there are some
conceptual compounds that come across as more episodically fused than
others.  Buttered toast may evoke a temporally
not-quite-instantaneous conceptualization {\mdash} at the core of
the concept is a practical activity that takes a few seconds to
complete {\mdash} but we also can imagine the buttering-act
apprehended in one sole episode.  On the other hand,
\q{Eritrean coffee} ties together concepts of much more scattered
provenance; the perceptual unity of \i{coffee} (in the sense
of a specific liquid in a specific container) along with
the geopolitical \q{background knowledge}
implicit in the adjective \i{Eritrean}.  As a cognitive
synthesis \i{Eritrean coffee} is conceptual rather than
perceptual.  Provisionally we can treat this in the context of
\i{buttered toast} being a partially-entrenched phraseology
while \i{Eritrean coffee} is undeniably a phrasal compound,
something whose constructive form must be parsed linguistically
rather than figuratively.}

\p{This analysis, though, needs many caveats.  After all, many
bonafide \i{phrases} (not \q{quasi-lexemes}) nevertheless
exhibit significant phenomenological unity {\mdash} i.e., they
evoke integral perceptual complexes: \i{big dog};
\i{hot coffee}; \i{speeding car}; \i{red foliage}.
Linguistically these seem like an underlying concept
acquiring perceptual specificity via adjectival
modification: \q{hot} was how the coffee came to my
experience because I experienced it as hot (it wasn't like
I experienced the coffee and then had to contemplate
whether it is hot or cold).  Coffee can't \i{not} be
experienced as hot, cold, or lukewarm; it cannot be experienced
without temperature (assuming I am coming into contact
with it and not just looking at it).  Similarly a car must be
seen as at rest, moving slowly, or speeding along; 
foliage must be
seen as having some color(s).}

\p{I have argued, however, that
unless entrenched as idiomatic phrases adjective-noun pairs
like \i{hot coffee} or \i{buttered toast} should be read
as grammatical complexes and accordingly (in my theory) as junctures
between distinct cognitive processes.  On the other hand, I argued
that \q{instant coffee} was effectively entrenched \i{because}
there is no simplistic conceptual unity between \i{instant} and
\i{coffee}, which makes it harder to hear the phrase as descriptive.
Instead, the semantics of that particular adjective-noun
connection are circuitous and a little hyperbolic: \q{instant}
coffee is coffee as a substance (not a drink, in that state)
from which coffee the drink can be quickly (but not
instantaneously) prepared.  There is a lot going on the seemingly
simple \q{instant coffee}: the shift from coffee-as-substance
to coffee-as-drink; the \q{instant} exaggeration.  In this
case, the adjectival modification has \i{so many} moving
parts that, I'm inclined to argue, it is hard to cover the
whole scenario with a descriptive phrase; which in turn
creates selective pressures for some pseudo-lexical unit
to emerge (which turned out to be \i{instant coffee})
as a mnemonic for the whole conceptual multiplex.  Indeed
conceptually intricate wholes tend to quickly acquire
pithy conventional nominalizations simply
for rhetorical convenience (\q{Brexit};
\q{Quantum Gravity}; \q{International Transfer
Window}; \q{#metoo}).}

\p{Notwithstanding these variations, I still find a certain logic
in the relation between phenomenological unity and semantic
entrenchment.  Perceptually integrated wholes may correlate
with linguistically aggregate constructions insofar as there
is a transparent logical destructuring in the perceptual
unity: in the case of substance-attribute pairs (like
\i{hot coffee}) {\mdash} deferring in the phenomenological
context to Husserl's account of \q{dependent moments} {\mdash} there
is a basically unsubtle distinction between an underlying
concept (like coffee) and the qualities which are its
mode of appearance as well as conceptual predicates (like
hot, cold,  black, or light, describing sensory properties
innate to the experience of a coffee-token as well as
state-reports that can be propositionally attributed to it).
Although the minimal sensate intention of the coffee and
the predicative disposition toward ascriptions like \i{black}
and \i{hot} are consciously intertwined, surely I am aware of
a logicality in experience that gives the sensate and predicative
dimensions different epistemic status.  I don't think of
my experience of the coffee's being hot as just a hot sensation
qua medium of my sensorily apprehending the coffee, but rather as
the sensate mechanism by which I observe the apparent fact that
the coffee is hot, as a state of affairs and not just as
subjective impression.  We are constantly extrapolating our perceptual
encounters to propositional content along these lines.}

\p{As such, I contend that such an (in some sense) innate
perception-to-predication instinct grounds the procedural
slicing of linguistic processes: \i{hot coffee} retraces in a 
linguistic construction the logical order of a coffee perception
which on one level is a raw perceptual encounter but is simultaneously
a predicative attribution. \q{Hot coffee} denotes a substance that
can be experienced in the mode of a base concept (coffee)
which is given predicative qualification (the coffee \i{is} hot).
The fact that there may be no noticed temporal gap
in \i{experience} between the sensate perception and the
epistemic posture does not preclude a certain logical
antecedent-subsequent ordering: the concept \i{coffee} is the
predicative base for my propositional attitude that what I
am dealing with here is hot coffee, not
hot-sensations-disclosing-coffee or coffee-I-experience-as-hot
(but who knows, maybe I'm hallucinating) or any other artificial
skeptifying of my actual experience, which is of raw perception
pregnant with propositional content.}

\p{So I wish to justify claims that (non-entrenched) phrase 
complexes like \q{hot coffee} are unions of disjoint cognitive
processes by noting that while such phrases evoke a certain perceptual
unity, they evoke a \i{kind} of unity
which we habitually regard \i{conceptually} as divided into
sensate givenness founding epistemic attitudes.  Cognitive
processes are not exclusively perceptual; they are
some mixture of perceptual and conceptual (and
enactive/kinaesthetic/operational).  A perceptual unity can
cover two conceptual aspects, like a sheet covering two
mattresses.  So the perceptual unity of hot coffee can
become the conceptual two-step of coffee as substance
and hot as attribute; committing this unity to
cognition as an overarching lifelong faculty involves registering
a thought-process of coffee as a substance which can, in acts of
logical predication, be believed to be hot or cold, black or
light, etc.  The apprehension of the substance is a different
cognitive process than the predication of the attribute, in terms of
how these mental acts fit within our epistemic models, even if
these two processes are experientially fused.  Typically we see the
coffee before we touch or taste it, so already the coffee has a logical
status apart from the heat we predicate in it.}

\p{Likewise, even if the black color is inextricable
from our perceiving (apart from
odd situations where we drink the coffee without looking at it),
we know the color will change if we add milk (even if just in
principle because, preferring black coffee, I don't actually do so);
so we know the coffee has a logical substrate apart from its color
too.  All of this ideation is latent in the coffee-perceptions
notwithstanding whatever perceptual unities we experience that
cloak logical forms like substance/attribute under the inexorable
togetherness of disclosure (the phenomenological impossibility of
spatial expanse without color, say).  In short, disjoint
cognitive processes can be required to reconstruct a perceptually
unified situation or episode, insofar as we are not just living
through the episode but prototyping, logically reconstructing,
signifying it {\mdash} the perceptual unity in the moment does not
propagate to procedural atomicity in absorbing the episode into
rational exercises.}

\p{Experience, then, presents \i{both} perceptual unities and
cognitive-propositional multiplicity; language can inherit
both holism on the perceptual side and compositionality
on the rational side, even in a single enactive/perceptual
episode.  Depending on how we via language want to
figure and express experience, we can bring either unity
or compositionality to the fore.  Our linguistic choices
will evoke perceptual unity if they select entrenched word-senses
or quasi-lexical forms; they will evoke compositionality
if they gravitate toward compound phrases and complex,
relatively rare lexicalizations and modes of expression.
To the degree that we are interested in a
cognitive-phenomenological \i{semantics}, we can attend
to the first part of this equation, to how the understood
atomicity of a word sense or a conventionalized phrase often
suggests an object or phenomenon consciously apprehended as
an integral whole; we can trace phenomenologically the
apperceptive unity that seems to drive the linguistic
community's accepting lexical atoms in this sense.
Conversely, to the degree that we are interested in a
cognitive-phenomenological \i{grammar}, we can attend to how
logically composite predication emerges even within
perceptual unity, because our encounter with phenomena is
not (save for exotic artistic or meditative pursuits) the
\q{ \i{dasein}} of irreflective sensory beings immersed in a
world of pure experience but the deliberate action of
epistemic beings carrying (modifiable but not random)
propositional attitudes to perceptual encounters.}

\p{Modeling grammar as a coordination between cognitive processes may
be an idealization, precisely because the compositive and integrative
faces of consciousness are two sides of the same
coin: it's not as if we work through a thought of
\i{coffee} or \i{toast}, abstract and without sensory
specificity, noticeably prelude to conceived/perceived
attributes like \i{hot}, \i{cold}, or \i{buttered}.
But we can still ascribe to linguistic-understanding
processes an idealized, \q{as if} temporality, treating the
elucidating of a sentence as a sequence of procedures leading from
bare concepts to well-rendered logical tableau,
suffused with some level of descriptive and situational
particularity.  So we go from \i{coffee} to \i{iced coffee} to
\i{buttered toast and iced coffee} to \i{snacking on
buttered toast and iced coffee}; each link in the chain
stepping up toward propositional totality.}

\p{My point is not that the logical form of the 
sentence is composed from
logically primitive and abstract parts, which is fairly trite;
my point is that such logical composition is only apparent
after a pattern of cognitive integration
that is more subtle and exceptional.  Extra-mentally, buttered
toast is just toast with butter on it, a fairly simplistic
logical conjunction.  Read as a baton passed between
two acts of mind, however {\mdash} conceiving toast and then conceiving
it buttered {\mdash} the conjunction is more elaborate;
the cognitive resources of \i{buttered} are not just
\q{something with butter on it} but the implication of a sensory
summation (the flavor, color, scent) and operational narrative
(we have seen or performed the deliberate act of applying the
butter).  Similarly a person dressed up is not just
someone whose torso is encircled by articles of clothing; a
barking dog is not just an animal making random noises;
a stray cat is different from a lost cat.  In their
interpenetration, cognitive processes develop (in the
photographer's sense) narrative
and causative threads that are latent in worldly situations
but reduced out of logical glosses; that is why it
seems incomplete, lacking nuance, or
beside the point to explicate semantic meanings in logical
terms, like \q{bachelor} as \q{unmarried man} (we can
certainly imagine a sentence like \i{My best friend
has been married for years but he's still a bachelor},
to imply he still has the habits and
attitudes of his single days).}

\p{A theory of sentences building from conceptual
underspecification to logical concreteness does
not preclude there being different scales
of specificity. \i{I snacked on toast and coffee}
is just as acceptable as \i{I snacked on buttered toast
and iced coffee}.  The communication conveys as much
situational detail as warranted in the conversational,
pragmatic context.  Language always has the \i{capability} to
push further and further into specificity; how
exhaustively the language user avails of this
capability is a matter of choice.  As theorists of language we
must then analyze how language possesses the \i{latent} capacity
to draw ever finer pictures; the adjectival \i{buttered} toast and
\i{iced} coffee takes the granularity of signifying at one
level (the level of the \i{I snacked on toast and coffee} sentence)
and layers on (or really layers \i{within}) a yet more
specific level.  The architecture of how this happens is
well addressed by type-theoretic methods (both coarse and mid-grained).}

\p{The remainder of this section will try to expand on this type-theoretic intuition.
My central thesis is that language understanding
involves integrating diverse \q{cognitive procedures},
each associated with specific words, word morphologies (plural
forms, verb tense, etc) and sometimes phrases.
The form of type theory appropriate in this context
is therefore one closely associated with procedural typing
and resolution: that is, assigning types to procedures,
and differentiating procedures based on the types of
their \q{arguments} or \q{parameters} (input and output data).}

\subsection{Interpretive Processes and Triggers}

\p{The type-theory/procedural perspective I will mostly work from
here contrasts with and adds nuance to
a more \q{logical} or \q{truth-theoretic} paradigm which
tends to interpret semantic phenomena via formal logic {\mdash}
for example, singular/plural in Natural Language
as a basically straightforward translation of the
individual/set distinction in formal logic.  Such
formal intuitions are limited in the sense that (to
continue this example) the conceptual mapping from
single to plural can reflect a wide range of
residual details beyond just quantity and multitudes.
Case in point are plurals: for each plural usage we have a conceptual
transformation of an underlying singular to a collective, but
how that collective is pictured varies in context.  One
dimension of this variation lies with mass/count: the
mass-plural \i{coffee} (as in \q{some coffee}) figures the
plurality of coffee (as liquid, or maybe coffee grounds/beans)
in spatial and/or physical/dynamic terms.  So we have:

\begin{sentenceList}

\sentenceItem{} \swl{itm:coffeeshirt}{There's some coffee on your shirt.}{ref}
\sentenceItem{} \swl{}{There's coffee all over the table.}{ref}
\sentenceItem{} \swl{}{She poured coffee from an ornate beaker.}{ref}
\sentenceItem{} \swl{}{There's too much coffee in the grinder.}{ref}
\sentenceItem{} \swl{}{There's a lot of coffee left in the pot.}{ref}
\sentenceItem{} \swl{}{There's a lot of coffee left in the pot
{\mdash} should I pour it out?}{ref}
\end{sentenceList}

These sentences use phrases associated with plurality (\i{all over},
\i{a lot}, \i{too much}) but with referents that on perceptual and
operational grounds can be treated as singular {\mdash} as in the
appropriate pairing of \i{a lot} and \i{it} in the last sentence.
With count-plurals the collective is figured more as an
aggregate of discrete individuals:

\begin{sentenceList}

\sentenceItem{} \swl{}{There are coffees all over the far wall at the espresso bar.}{ref}
\sentenceItem{} \swl{}{She poured coffees from an ornate beaker.}{ref}
\sentenceItem{} \swl{itm:coffeetable}{There are a lot of coffees left on the table
 {\mdash} shall I pour them out?}{ref}
\end{sentenceList}
}

\p{So mass versus count {\mdash} the choice of which plural form to use {\mdash}
triggers an interpretation shaping how the plurality is pictured and
conceived (which is itself triggered by the use of a plural to
begin with).  Compare \i{I sampled some chocolates} (where the count-plural
suggests \i{pieces} of chocolate) and
\i{I sampled some coffees} (where the count-plural implies
distinguishing coffees by virtue of grind, roast, and other
differences in preparation) (note that both are contrasted
to mass-plural forms like \i{I sampled some coffee} where
plural agreement points toward material continuity; there
is no discrete unit of coffee qua liquid).  Or compare
\i{People love rescued dogs} with \i{People fed the rescued dogs}
{\mdash} the second, but not the first, points toward
an interpretation that certain \i{specific} people
fed the dogs (and they did so \i{before} the dogs
were rescued).}

\p{But if we restrict attention to just, say,
count-plurals, there are still different schemata for intending
collections.  This is one dimension I left unexamined 
in the earlier \q{New York} examples:

\begin{sentenceList}

\sentenceItem{} \swl{itm:NYboroughs}{New Yorkers live in one of five boroughs.}{ref}
\sentenceItem{} \swl{itm:NYDem}{New Yorkers reliably vote for Democratic presidential candidates.}{ref}
\sentenceItem{} \swl{itm:NYcommute}{New Yorkers constantly complain about how long it takes to commute
to New York City.}{ref}
\end{sentenceList}

The first sentence is consistent with a reading applied to
\i{all} New Yorkers {\mdash} the five boroughs encompass the whole
extent of New York City.  The second sentence is only reasonable
when applied exclusively to the city's registered voters {\mdash} not all
residents {\mdash} and moreover there is no implication that the
claim applies to all such voters, only a proportion north of one-half.
And the final sentence, while perfectly reasonable, uses \q{New Yorkers}
to name a population completely distinct from the
first sentence {\mdash} only residents from the metro
area, but not the city itself, commute \i{to} the city.}

\p{The assumption that logical modeling can capture all the
pertinent facets of Natural-Language meaning can
lead us to miss the amount of situational reasoning
requisite for commonplace understanding.  In
\i{People fed the rescued dogs} there is an exception to the
usual pattern of how tense and adjectival modification
interact: we read \q{people fed} in
\i{People fed the rescued dogs} as occurring \i{before} the rescue;
because we assume that \i{after} being rescued the dogs would be
fed by veterinarians and other professionals (who would
probably not be designated with the generic \q{people}), and
also we assume the feeding helped the dogs survive.  We also
hear the verb as describing a recurring event; compare
with \i{I fed the dog a cheeseburger}.}

\p{To be sure, there are patterns and templates governing
scope/quantity/tense interactions that help us build logical models
of situations described in language.  Thus
\i{I fed the dogs a cheeseburger} can be read such that there
are multiple cheeseburgers {\mdash} each dog gets one {\mdash}
notwithstanding the singular form on \i{a cheeseburger}:
the plural \i{dogs} creates a scope that can elevate
the singular \i{cheeseburger} to an implied plural;
the discourse creates multiple reference frames each
with one cheeseburger.  Likewise the morphosyntax is
quite correct in: \i{All the rescued dogs are taken to an
experienced vet; in fact, they all came from the same
veterinary college} {\mdash} the singular on \i{vet} is properly
aligned with the plural \i{they} because of the scope-binding
(from a syntactic perspective) and space-building
(from a semantic perspective) effects of the \q{dogs} plural.
Or, in the case of \i{I fed the dog a cheeseburger every day}
there is an implicit plural because \q{every day} builds
multiple spaces: we can refer via the spaces collectively
using a plural (\i{I fed the dog a cheeseburger every day {\mdash}
I made them at home with vegan cheese}) or refer within
one space more narrowly, switching to the singular
(\i{Except Tuesday, when it was a turkey burger}).}

\p{Layers of scope, tense, and adjectives interact in complex ways that
leave room for common ambiguities: \i{All the rescued dogs are [/were]
taken to an experienced [/specialist] vet} is consistent with a reading
wherein there is exactly one vet, and she has or had treated every dog.
It is \i{also} consistent with a reading where there
are multiple vets and each dog is or was treated
by one or another.  Resolving such ambiguities
tends to call for situational reasoning and a \q{feel} for situations,
rather than brute-force logic.  If a large dog shelter describes
their operational procedures over many years, we might assume
there are multiple vets they work or worked with.  If instead the
conversation centers on one specific rescue we would be
inclined to imagine just one veterinarian.  Lexical and tense
variation also guides these impressions: the past-tense
form (\i{...the rescued dogs were taken...}) nudges us
toward assuming the discourse references one rescue (though it
could also be a past-tense retrospective of general operations).
Qualifying the vet as \i{specialist} rather than the vaguer
\i{experienced} also nudges us toward a singular interpretation.}

\p{What I am calling a \q{nudge}, however, is based on situational
models and arguably flows from a conceptual stratum outside
of both semantics and grammar proper; maybe it is even prelinguistic.
Consider

\begin{sentenceList}

\sentenceItem{} \swl{itm:pf}{People fed the rescued dogs.}{sem}
\sentenceItem{} \swl{itm:ve}{Vets examined the rescued dogs.}{sem}
\end{sentenceList}

There appears to be no explicit principle either in the semantics
of the lexeme \i{to feed}, or in the relevant tense agreements,
stipulating that the feeding in (\ref{itm:pf})
was prior to the rescue {\mdash} or conversely that
(\ref{itm:ve}) describes events
\i{after} the rescue.  Instead, we interpret the discourse through
a narrative framework that fills in details not provided by
the language artifacts explicitly (that abandoned dogs are
likely to be hungry; that veterinarians treat dogs in clinics, which
dogs have to be physically brought to).  For a similar case-study,
consider the sentences:

\begin{sentenceList}

\sentenceItem{} \swl{}{Every singer performed two songs.}{sem}
\sentenceItem{} \swl{}{Everyone performed two songs.}{sem}
\sentenceItem{} \swl{}{Everyone sang along to two songs.}{sem}
\sentenceItem{} \swl{}{Everyone in the audience sang along to two songs.}{sem}
\end{sentenceList}

The last of these examples strongly suggests that of potentially
many songs in a concert, exactly two of them were popular and singable
for the audience.  The first sentence, contrariwise, fairly strongly
implies that there were multiple pairs of songs, each pair performed
by a different singer.  The middle two sentences imply either
the first or last reading, respectively (depending on how we
interpret \q{everyone}).  Technically, the
first two sentences imply a multi-space reading and the latter two
a single-space reading.  But the driving force
behind these implications are the pragmatics of \i{perform} versus
\i{sing along}: the latter verb is bound more tightly to
its subject, so we hear it less likely that
\i{many} singers are performing \i{one} song pair, or conversely that
every audience member \i{sings along} to one song pair, but
each chooses a \i{different} song pair.}

\p{The competing interpretations for \i{perform} compared to
\i{sing along}, and \i{feed} compared to \i{treat}, are grounded
in lexical differences between the verbs.  I contend, though, that
the contrasts are not laid out in lexical specifications
for any of the words, at least so that the implied readings
follow just mechanically, or on logical considerations
alone.  After all, in more exotic but not implausible
scenarios the readings would be reversed:

\begin{sentenceList}

\sentenceItem{} \swl{}{The rescued dogs had been treated by vets in the past
(but were subsequently abandoned by their owners).}[-> sem][The rescued dogs had been treated by vets in the past.]
\sentenceItem{} \swl{}{Every singer performed (the last) two songs
(for the grand finale).}[-> sem][Every singer performed the last two songs for the grand finale.]
\sentenceItem{} \swl{}{Everyone in the audience sang along to two songs
(they were randomly handed lyrics to different songs when
they came in, and we asked them to join in when the song being
performed onstage matched the lyrics they had in hand).}{sem}
\end{sentenceList}

In short, it's not as if dictionary entries would specify that
\i{to feed} applies to rescued dogs before they are rescued,
and \i{to treat} applies after they are rescued.  Or
that \i{sing along} nudges scope interpretation in one direction
and \i{perform} nudges in a different direction.
These interpretations are driven by narrative
construals narrowly specific to given expressions.  The
appraisals would be very different for other uses of the verbs in
(lexically) similar (but situationally different) cases:
to \q{treat} a wound or a sickness, to \q{perform} a gesture or a
play.  We construct an interpretive scaffolding
for resolving issues like scope-binding and space-building based
on fine-tuned narrative construals that can vary alot
even across small word-sense variance:
As we follow along with these sentences, we have to build a narrative
and situational picture which matches the speaker's intent,
sufficiently well.}

\p{And that requires prelinguistic background
knowledge which is leveraged and activated (but not mechanically
or logically constructed) by lexical, semantic, or grammatical
rules and forms: \i{rescued dogs} all alone constructs a fairly
detailed mental picture where we can fill in many details by
default, unless something in the discourse tells otherwise
(we can assume such dogs are in need of food, medical care,
shelter, etc., or they would not be described as
\q{rescued}).  Likewise \i{sing along} carries a rich mental
picture of a performer and an audience and how they interact, one
which we understand based on having attended concerts rather than
by any rule governing \i{along} as a modifier to \q{sing}
{\mdash} compare the effects of \i{along} in \i{walk along},
\i{ride along}, \i{play along}, \i{go along}.  Merely
by understanding how \i{along} modifies \i{walk}, say
(which is basically straightforward; to
\q{walk along} is basically to \q{walk alongside}) we
would not automatically generalize to more idiomatic
and metaphorical uses like \q{sing along} or \q{play along}
(as in \i{I was skeptical but I played along (so as not to
start an argument)}).}

\p{We have access to a robust collection of \q{mental scripts} which
represent hypothetical scenarios and social milieus where
language plays out.  Language can activate various such
\q{scripts} (and semantic as well as grammatical formations
try to ensure that the \q{right} scripts are selected).
Nonetheless, we can argue that the conceptual and cognitive
substance of the scripts comes not from language per se
but from our overall social and cultural lives.
We are disposed to make linguistic inferences {\mdash} like
the timeframes implied by \i{fed the rescued dogs} or the scopes
implied by \i{sang along to two songs} {\mdash} because of
our enculturated familiarity with events like dog rescues
(and dog rescue organizations) and concerts
(plus places like concert halls).  These concepts are not
produced by the English language, or even by any dialect
thereof (a fluent English speaker from a different
cultural background would not necessarily make the
same inferences {\mdash} and even if we restrict attention to,
say, American speakers, the commonality of disposition
reflects a commonality of the relevant cultural
anchors {\mdash} like dog rescues, and concerts {\mdash} rather than
any homogenizing effects of an \q{American} dialect).
For these reasons, I believe that trying to account for
situational particulars via formal language models alone
is a dead end.  This does not mean that formal language
models are unimportant, only that we need to picture them
resting on a fairly detailed prelinguistic
world-disclosure.}

\p{There are interesting parallels in this thesis to the role
of phenomenological analysis, and the direct thematization
of issues like attention and intentionality: analyses
which are truly \q{to the things themselves} should take for
granted the extensive subconscious reasoning that undergirds
what we consciously thematize and would be aware of, in terms of
what we deliberately focus on and are conscious of
believing (or not knowing), for a first-personal \expose{}.
Phenomenological analysis should not consider itself as
thematizing every small quale, every little patch of
color or haptic/kinasthetic sensation which by some subconscious
process feeds into the logical picture of our surroundings that
props up our conscious perception.  Analogously, linguistic
analysis should not thematize every conceptual and inferential
judgment that guides us when forming the mental, situational
pictures we then consult to set the groundwork for linguistic
understanding proper.}

\p{These comments apply to both conceptual \q{background knowledge} and
to situational particulars of which we are cognizant in
reference to our immediate surroundings and actions.  This
is the perceptual and operational surrounding that gets
linguistically embodied in deictic reference and other
contextual \q{groundings}.  Our situational awareness therefore
has both a conceptual aspect {\mdash} while attending a concert,
or dining at a restaurant, say, we exercise cultural background
knowledge to interpret and participate in social events
 {\mdash} and also our phenomenological construal of our locales,
our immediate spatial and physical surroundings.
Phenomenological philosophers have explored in detail how these
two facets of situationality interconnect (David Woodruff Smith and
Ronald McIntyre in \i{Husserl and Intentionality:
A Study of Mind, Meaning, and Language}, for instance).
Cognitive Linguistics covers similar territory; the \q{cognitive}
in Cognitive Semantics and Cognitive Grammar generally tends
to thematize the conception/perception interface and
how both aspects are merged in situational understanding
and situationally grounded linguistic activity (certainly
more than anything involving Artificial Intelligence or
Computational Models of Mind as are connoted by terms like
\q{Cognitive Computing}).  Phenomenological and Cognitive
Linguistic analyses of situationality and perceptual/conceptual
cognition (cognition as the mental synthesis of
perception and conceptualization) can certainly enhance and
reinforce each other.}

\p{But in addition, both point to a cognitive and situational
substratum underpinning both first-person
awareness and linguistic formalization proper {\mdash} in other words,
they point to the thematic limits of
phenomenology and Cognitive Grammar and
the analytic boundary where they give way to
an overarching Cognitive Science.  In the case of
phenomenology, there are cognitive structures that suffuse
consciousness without being directly objects of attention
or intention(ality), just as sensate hyletic experience is
part our consciousness but not, as explicit content,
something we in the general case are conscious \i{of}.
Analogously, conceptual and situational models
permeate our interpretations of linguistic forms, but
are not presented explicitly \i{through} these
forms: instead, they are solicited obliquely and
particularly.}

\p{What phenomenology \i{should} explicate is not background situational
cognition but how attention, sensate awareness, and intentionality
structure our orientation \i{ \visavis{}} this background: how variations
in focus and affective intensity play strategic roles in our engaged
interactions with the world around us.  Awareness is a scale, and
the more conscious we are of a sense-quality, an attentional focus,
or an epistemic attitude, reflects our estimation of the
importance of that explicit content compared to a muted experiential
background.  Hence when we describe consciousness as a stream
of \i{intentional} relations we mean not that the intended
noemata (whether perceived objects or abstract thoughts)
are sole objects of consciousness (even in the moment)
but are that within conscious totality which we are most aware
of, and our choice to direct attention here and there reflects our
intelligent, proactive interacting with the life-world.
Situational cognition forms the background,
and phenomenology addresses the structure of intentional
and attentional modulations constituting the conscious
foreground.}

\p{Analogously, the proper role for linguistic
analysis is to represent how multiple layers or strands
of prelinguistic understanding, or \q{scripts}, or
\q{mental spaces}, are woven
together by the compositional structures of language.
For instance, \i{The rescued dogs were treated by an experienced vet}
integrates two significantly different narrative frames
(and space-constructions, and so forth): the frame implied
by \q{rescued dogs} is distinct from that implied
by \q{treated by a veterinarian}.  Note that both spaces are
available for follow-up conversation:

\begin{sentenceList}

\sentenceItem{} \swl{}{The rescued dogs were treated by an experienced vet.
One needed surgery and one got a blood transfusion.  We went there
yesterday and both looked much better.}{sem}
\sentenceItem{} \swl{}{The rescued dogs were treated by an experienced vet.
One had been struck by a car and needed surgery on his leg.  We
went there yesterday and saw debris from another car crash
{\mdash} it's a dangerous stretch of highway.}{sem}
\end{sentenceList}

In the first sentence \i{there} designates the veterinary clinic, while in
the second it designates the rescue site.  Both of these locales are
involved in the original sentence (as locations and also
\q{spaces} with their own environments and configurations:
consider these final three examples).

\begin{sentenceList}

\sentenceItem{} \swl{}{The rescued dogs were treated by an experienced vet.
We saw a lot of other dogs getting medical attention.}{sem}
\sentenceItem{} \swl{}{The rescued dogs were treated by an experienced vet.
It looked very modern, like a human hospital.}{sem}
\sentenceItem{} \swl{}{The rescued dogs were treated by an experienced vet.
We looked around and realized how dangerous that road is {\mdash}
for humans as well as dogs.}{sem}
\end{sentenceList}
}

\p{What these double space-constructions reveal is that accurate
language understanding does not only require
the proper activated \q{scripts} accompanying words and
phrases, like \q{rescued dogs} and \q{treated by a vet}.
It also requires the correct integration of each script,
or each mental space, tieing them together in accord with
speaker intent.  So in the current example we should read that
the dogs \i{could} be taken to the vet \i{because} they were
rescued, and \i{needed} to be taken to the vet \i{because} their
needing rescued was associated with being injured or in poor 
health.  Language structures guide us
toward how we should tie the mental spaces, and the
language segments where they are constructed, together: the
phrase \q{ \i{rescued} dogs} becomes the subject of the passive-voice
\i{were treated by a vet} causing the two narrative strands of the
sentence to encounter one another, creating a hybrid space
(or perhaps more accurately a patterning between
two spaces with a particular temporal and causal
sequencing; a hybrid narration bridging the spaces).
It is of course this hybrid space, this narrative
recount, which the speaker intends via the sentence.  This
idea is what the sentence is crafted to convey {\mdash} not just
that the dogs were rescued, or that they were taken to a vet, but
that a causal and narrative thread links the two events.}

\p{I maintain, therefore, that the analyses which are proper to linguistics
{\mdash} highlighting what linguistic reasoning contributes above and beyond
background knowledge and situational cognition {\mdash} should focus on
the \i{integration} of multiple mental \q{scripts},
each triggered by different parts and properties of the linguistic
artifact.  The \i{triggers} themselves can be individual words, but
also morphological details (like plurals or tense marking) and
morphological agreement.  On this theory, analysis has two distinct
areas of concerns: identification of grammatical, lexical, and
morphosyntactic features which trigger (assumedly prelinguistic)
interpretive scripts; and reconstructing how these scripts
interoperate (and how language structure determines such integration).}

\p{In the case of isolating triggers, a wide range of linguistic features
can trigger interpretive reasoning {\mdash} including base lexical choice;
word-senses carry prototypical narrative and situational templates that
guide interpretation of how the word is used in any given context.
\i{Rescued}, for example, brings on board a network of likely
externalities: that there are rescuers, typically understood to be
benevolent and intending to protect the rescuees from harm; that
the rescuees are in danger prior to the rescue but safe afterward;
that they need the rescuers and could not have reached safety themselves.
Anyone using the word \q{rescue} anticipates that their addressees will
reason through some such interpretive frame, so the speaker's role is
to fill in the details descriptively or deictically: who are the rescuees
and why they are in danger; who are the rescuers and why they are benevolent
and able to protect the rescuees.  The claim that
the word \i{rescue}, by virtue of its lexical properties, triggers an
interpretive \q{script}, is a proposal to the effect that when trying
to faithfully reconstruct speaker intentions
we will try to match the interpretive
frame to the current situation.}

\p{The \q{script} triggered by word-choice is not just an interpretive
frame in the abstract, but the interpretive \i{process} that matches
the frame to the situation.  This process can be exploited for
metaphorical and figurative effect, broadening the semantic scope
of the underlying lexeme.  In the case of \q{rescue} we have less
literal and more humorous or idiomatic examples like:

\begin{sentenceList}

\sentenceItem{} \hspace{-3pt}\swl{}{The trade rescued a star athlete from a losing team.}{idi}
\sentenceItem{} \swl{}{New mathematical models rescued her original research from obscurity.}{idi}
\sentenceItem{} \hspace{-3pt}\swl{}{Discovery of nearby earth-like planets rescued that
star from its reputation as ordinary and boring and revealed that its solar
system may actually be extraordinary.}{idi}
\end{sentenceList}

Each of these cases subverts the conventional \q{rescue} script by
varying some of the prototypical frame details:  maybe the
\q{danger} faced by the rescuee is actually trivial (as in the
first three), or the rescuee is not a living thing
whose state we'd normally qualify in terms of \q{danger} or \q{safety},
or by overturning the benevolence we typically attribute to
rescue events.  
But in these uses subverting the familiar script does not
weaken the lexical merit of the word choice; instead, the interpretive
act of matching the conventional \q{rescue} script to the matter at hand
reveals details and opinions that the speaker wishes to convey.  The
first sentence, for instance, uses \q{rescue} to connote
that being stuck on a losing team is an unpleasant (even if not life-threatening)
circumstance.  So one part of the frame (that the rescuee needs
outside intervention) holds while the other (that
the rescue is in danger) comes across as excessive but
(by this very hyperbole) communicating speaker sentiment.  By
both invoking the \q{rescue} script and exploiting mismatches between
its template case and the current context, the speaker
conveys both situational facts and personal opinions quite
economically.  Similarly, \i{rescue a paper from obscurity} is
an economical way of saying that research work has been rediscovered
in light of new science; and \q{rescued from a reputation} is a 
clever way of describing, with rhetorical force, 
how opinions have changed about someone or something.}

\p{All of these interpretive effects {\mdash} both conventional
and unconventional usages {\mdash} stem from the interpretive
scripts bound to words (and triggered by word-choice) at
the underlying lexical level {\mdash} we can assess these by reference
to lexical details alone, setting aside syntactic and morphological
qualities.  When morphosyntactic details \i{are} considered 
{\mdash} e.g. plurals, as in (\ref{itm:coffeeshirt})-(\ref{itm:coffeetable}) 
{\mdash} we then have a spectrum of other linguistic \q{triggers}, 
involving perceptual and enactive figurations (e.g. how 
plurality/multiplicities are conceived), alongside interpretive
\q{scripts}.  My essential point however is that language 
needs to trigger a collection of interpretive, perceptual, and 
enactive/operational processes, for complete understanding; 
and moreover that that language structures need to signal 
the \q{routing} of information between such processes, 
establishing patterns of priority/sequencing and 
information-supplementation among them.  These patterns, 
I believe, can be modeled to some approximation via 
computational type and process theories.}

\p{I contend, moreover, that these cognitive processes are not
\i{themselves} linguistic: while they may overlap with 
some language-relevant concerns (like conceptualization, and 
doxic specificity) they are not woven from the cloth of 
syntactic, semantic, or pragmatic elements internal to language.  
It is not within the purview of linguistics then to analyze 
interpretive scripts (except as a subsidiary case-study), or 
perceptual understanding, or situationally-mediated action.  
What \i{can} be left for linguistics proper is identifying the
\i{triggers} to these cognitive realities {\mdash} insofar as content or 
formations in language, within our goal-directed attempt to 
understand others' linguistic performances, compels us toward 
these extra-linguistic registers.}

\p{Linguistics on this perspective is necessarily and properly incomplete: 
we should not look to linguistic analysis to materially or structurally 
explain the cognitive processes triggered by language.  But we \i{can}
analyze the triggers themselves {\mdash} potentially via formal and 
even computational methods.  So the formal models I reviewed 
earlier {\mdash} such as the combination of Dependency graphs 
with typed S-Expressions encoded via double-indices {\mdash} can be adopted 
in a basically non-formalized, cognitive-linguistic paradigm, 
insofar as we ascribe to the global picture of language as an
\q{interface} or \q{trigger} to extralinguistic cognition.}

\thindecoline{}

\p{Suppose, for instance, that morphosyntax confirms the 
presence of a pluralizing operation, like
\i{New Yorkers}.  Against this minimal linguistic
\q{information}, we need
to invoke narrative frames, interpretive scripts, and prelinguistic
background knowledge to understand what \i{sort} of plurality
the speaker intends.  The phrase \i{New Yorkers} can refer 
to those who live in New York City (\ref{itm:NYboroughs}, \ref{itm:boroughs}); 
in the New York 
are but \i{outside} the city (\ref{itm:NYcommute}, \ref{itm:NYcommute}); 
or some combination.  
It can also mean \i{all} (\ref{itm:NYboroughs}, \ref{itm:boroughs}) or 
just \i{some} New Yorkers (\ref{itm:NYcommute}, \ref{itm:NYcommute}, \ref{itm:NYDem}).
Usually these alternatives are not explicitly 
spelled out, and have to be inferred from context 
and via background knowledge about which readings 
are plausible.    
As I argued from multiple angles earlier,
in the typical case {\mdash} i.e., stylistically
neutral, day-to-day language {\mdash} syntactic
composition does not neatly recapitulate logical form.}

\p{My prior analysis demonstrated warrants for this idea by
highlighting narrative and imagistic aspects
of language used to convey ideas, like \q{come out against}
providing the verb-phrases in reports of people
criticizing something.  The \q{New York} examples like
(\ref{itm:NYboroughs})-(\ref{itm:NYcommute}) point to a similar
conclusion, but from a more lexico-semantic orientation:
words like \i{borough} and \i{commute} carry a space
of logical details that tend to force logical
interpretations one way or another
(e.g., the detail that the territory of a city
is fully partitioned by boroughs, \i{so}, it
is \i{all} citizens who live in a borough).  This part of the
logic is however not reflected in sentence-structure; it is,
rather, latent
in lexical norms and assumed part of understanding
relevant sentences only because linguistic
competence is understood to include
familiarity with the logical
implications of the lexicalized concepts:
e.g. that the quantification in \q{New Yorkers
live in one of five boroughs} is \i{all},
but the quantification in \q{New Yorkers
vote democratic} is \i{most}.}

\p{Here I'll also add the
following: the current examples show
how if addressees \i{have} the requisite background knowledge,
linguistic structure does not have to replicate logical structure very
closely to be understood.  The content which addressees understand may
have a logical form, and language evokes this form {\mdash} guides
addressees toward considering specific propositional content {\mdash} but
this does not happen because linguistic structure in any precise way mimics,
replicates, reconstructs, or is otherwise organized propositionally.  Instead,
the relation of language to predicate structures is evidently
oblique and indirect: language triggers interpretive processes
which guide us toward propositional content, but the structure
of language is shaped around fine-tuning the activation of
this background cognitive dynamics more than around any need to model
predicate organization architecturally.   In the case of plurals,
the appearance of plural forms like \i{New Yorkers} or \i{coffees}
compels us to find a reasonable cognitive model for the
signified multitude, and this model will have a logical form
{\mdash} but the linguistic structures themselves do not in general
model this form for us, except to the limited degree needed to
activate prelinguistic interpretive thought-processes.}

\p{I make this point in terms of plural \i{forms}, and earlier made
similar claims in terms of lexical details (cf. \i{(live in a) borough}
vs. \i{vote (Democratic)}.  A third group of triggers I
outlined involved morphosyntactic \i{agreement}, which establishes
inter-word connections that themselves trigger interpretive processing.
Continuing the topic of plurals, how words agree with other words
in singular or plural forms evokes schema which guide situational
interpretations.  So for instance:

\begin{sentenceList}

\sentenceItem{} \swl{}{My favorite band gave a free concert last night.
They played some new songs.}{ref}
\sentenceItem{} \swl{}{There was some pizza earlier, but it's all gone.}{ref}
\sentenceItem{} \swl{}{There were some slices of pizza earlier, but it's all gone.}{ref}
\sentenceItem{} \swl{}{There were some slices of toast earlier, but there's none left.}{ref}
\sentenceItem{} \swl{}{There was some toast earlier, but they're all gone.}{ref}
\sentenceItem{} \swl{}{That franchise had a core of talented young players, but it
got eroded by trades and free agency.}{ref}
\sentenceItem{} \swl{}{That franchise had a cohort of talented players, but they
drifted away due to trades and free agency.}{ref}
\sentenceItem{} \swl{}{Many star players were drafted by that franchise, but it
has not won a title in decades.}{ref}
\sentenceItem{} \swl{}{Many star players were drafted by that franchise, but they
failed to surround them with enough depth.}{ref}
\sentenceItem{} \swl{}{Many star players were drafted by that franchise, but they
were not surrounded with enough depth.}{ref}
\sentenceItem{} \swl{}{Many star players were drafted by that franchise, but they
did not have enough depth (around them).}{ref}
\end{sentenceList}

Plurality here is introduced not only by isolated morphology (like \i{slices},
\i{players}, \i{songs}), but via agreements marked by
word-forms in syntactically significant pairings: was/were, it/they,
there is/there are.  Framing all of these cases is how we can usually schematize
collections both plurally and singularly: the same set can be cognized as a
collection of discrete individuals one moment and as an integral whole the next.
This allows language some flexibility when designating plurals
(cf. \i{Three times, students asked an interesting question}).  A sentence
discussing \i{slices of pizza} can schematically shift to treating
the pizza as a mass aggregate in \i{it's all gone}.  Here the antecedent
of \i{it} is \i{slices} (of pizza).  In the opposite direction, the mass-plural
\q{toast} can be refigured as a set of individual pieces in \i{they're all gone}.
The single \i{band} becomes the group of musicians in the band.
In short, how agreements are executed invites the addressee to reconstruct
the speaker's conceptualization of different referents discussed by
a sentence, at different parts of the sentence: linking
\i{it} to \i{slices} or \i{cohort}, or \i{they} to \i{the band} or
\i{the toast}, evokes a conceptual interpretation shaped in
part by how morphosyntactic agreement overlaps with \q{semantic}
agreement.  Matching \i{they} to \i{the band} presents agreement
in terms of how we conceive the aggregate (as a collection of
musicians); using \q{it} would also present an agreement, but
one schematizing other aspect of the band concept.}

\p{In the last five above cases, \i{it} similarly binds (being singular) to
\i{the franchise} seen as a single unit {\mdash} here basic grammar and
conceptual schema coincide {\mdash} but \i{it} also binds
to the \i{core of young players}.  The
players on a team can be figured as a unit or a multiple.  The franchise itself
can be treated as a multiple (the various team executives and decision-makers),
as in \i{they failed to surround the stars with enough depth}.  The last sentence
is ambiguous between both readings: \q{they} could designate either the players
or the franchise.  Which reading we hear alters the sense of \q{have}: asserting
that the star \i{players} lack enough depth implies that they cannot execute
plays during the game as effectively as with better supporting players;
asserting that the \i{franchise} lacks depth makes the subtly different point
that there is not enough talent over all.  The variant which would include
\q{around them} nudges toward the second reading; but it is
still permissible to, according to speaker intent, parse the last
\i{they} as designating the \i{franchise} and the last \i{them} as
the \i{players} {\mdash} i.e., that final \i{they}/\i{them} pair
having different antecedents.}

\p{The unifying theme across these cases is that when forming sentences we often
have a choice of how we figure plurality, and moreover these choices can be
expressed not only in individual word-forms but in patterns of agreement.
Choosing to pronominalize \i{slices of pizza} or \i{cohort of players}
as \i{it}, or alternatively \i{they}, draws attention to either the more
singular or more multitudinal aspects of the aggregate in question.
But this effect is not localized to the individual
\i{it}/\i{they} choice; it depends on
tracing the pronoun to its antecedent and construing
how the antecedent referent has both individuating and
multiplicity-like aspects.
Thus both individuation and plurality are latent in phrases like
\i{slices of} or \i{cohort of}, and this singular/plural co-conceiving
is antecedently figured by how subsequent morphosyntax agrees
with the singular or, alternatively, the plural.}

\p{Moreover, these patterns of agreement invoke new layers of
interpretation to identify the proper conceptual scope of plurals.
In \i{The band planned a tour, where they debuted new songs} we hear
the scope of \q{they} as narrower than its antecedent \q{the band},
because only the band's \i{musicians} (not stage crew, managers, etc.)
typically actually perform:

\begin{sentenceList}

\sentenceItem{} \swl{tour}{The band planned a tour, where
they debuted new songs.}{ref}
\sentenceItem{} \swl{itm:teamflew}{The team flew to
New York and they played the Yankees.}{ref}
\sentenceItem{} \swl{itm:theflies}{The city's largest theater company
will perform \q{The Flies}.}{ref}
\end{sentenceList}

Likewise in (\ref{itm:teamflew}), only the athletes are referenced via
\i{they played}; but presumably many other people (trainers, coaches, staff)
are encompassed by \q{the team flew}.  And in (\ref{itm:theflies})
we do not imply that the Board of Directors will
actually take the stage (the President
as Zeus, say).  Even in the course of one sentence, plurals
are reinterpreted and redirected:

\begin{sentenceList}

\sentenceItem{} The city's largest theater company
performed \q{The Flies} in French, but everyone's accent
sounded Quebecois.
\sentenceItem{} The city's largest theater company
performed \q{The Flies}; then they invited a professor
to discuss Sartre's philosophy when the play was over.
\end{sentenceList}

In the first sentence, the \q{space} built by the sentence is wider
initially but narrows to encompass only the actual actors on stage.
In the second, the \q{space} narrows in a different direction, since
we hear a programming decision like pairing a performance with a
lecture as made by a theater's administrators rather than its actors.
I discussed similar modulation in conceptual schemas related to plurality
and pluralization earlier; what is distinct in these last examples is how
the interpretive processes for cognizing plurality are shaped by
agreement-patterns (like \i{it} or \i{they} to a
composite antecedent) as
much as by lexical choice, or morphology, in isolation.}

\p{I have accordingly outlined a theory where lexical, morphological, and
morphosyntactic layers all introduce \q{triggers} for cognitive processes,
and it is these processes which (via substantially prelinguistic perception
and conceptualization) ultimately deliver linguistic meaning.  What is
\i{linguistic} about these phenomena is how specifically linguistic
formations {\mdash} word choice, word forms, inter-word agreements in form {\mdash}
trigger these (in no small measure pre- or extra-linguistic) interpretations.
But as I suggested this account is only preliminary to analysis of
how multiple interpretive processes are \i{integrated}.  Linguistic
\i{structure} contributes the arrangements through which the
crossing and intersecting between interpretive
\q{scripts} are orchestrated.  Hence at the higher linguistic scales and
levels of complexity, the substance of linguistic research, on this view,
should gravitate toward structural integration of interpretive
processes, even more than individual interpretive triggers themselves.}

\p{I have thus far emphasized types' structural role, on the 
premise that type theory offers technical models that can 
be applied to the study of \q{cognitive process integration}.  
Requirements on the transfer of information between 
processes {\mdash} how processes complement one another and 
take other's details as givens {\mdash} are suggestive analogies 
to latent interactions between mental \q{scripts} we are 
poised to apply to situations; and these interactions 
are the essential cognitive targets which linguistic structures 
have to trigger by semantic/lexical, and syntactic/morphosyntactic 
means.  The point of analyzing process preconditions 
type-theoretically is to leverage types as constructional  
media for structural theories of inter-process integration.}

\p{This approach is largely syntactic: while many lexical and semantic 
details may be analytically relevant, the style of analysis I 
just proposed is ultimately a form of grammar, identifying 
how organizational patterns among language-elements 
trigger (and are thereby explained by the intention of) 
corresponding patterns among cognitive processes.  
For sake of completeness, though, I will conclude by 
offering a few thoughts on the more \q{semantic} manifestation 
of types (and type theory) as premises of 
organizing concepts.}

\subsection{Types, Sets, and Concepts}
\p{In formal/computational contexts, types can be defined as sets of both values and
\q{expectations} \cite{MathieuBouchard} (meaning assumptions which may be made about
all values covered by the type); alternatively, we can (perhaps better) consider types as
\i{spaces} of values.  Types' extensions have internal structure; there
can be \q{null} or \q{invalid} values, default-constructed values, and
so forth, which are \q{regions} of the conceptual space spanned or 
encompassing types.  
There is definitional interdependence
between types and functions: a function is defined in terms of the types it accepts as parameters and
returns {\mdash} rather than its entire set of possible inputs and outputs, which can
vary across computing environments.\footnote{Moreover, expectations in a particular case
may be more precise than what is implied by the type itself {\mdash} it is erroneous
to assume that a proper type system will allow a correct \q{set of values} to
be stipulated for each point in a computation (the kind of contract enforced via
by documentation and unit testing).  So state-space in a given context may include many
\q{unreasonable} values, implying that within the overall space there is a \q{reasonable}
subspace, except that this subspace may not be crisply defined.} These are some reasons why in theoretical
Computer Science types are not \q{reduced} to underlying sets; instead, extensions
are sometimes complex spaces that model states of, or internal organization of comparisons
among, type instances.}

\p{Moreover, despite mathematical set theory's influence on 
formal logic (and philosophy, by extension), 
philosophers have long been troubled by 
the seeming arbitrariness and minimal structure of 
the essential notion of \i{sets} themselves,
and have explored other constructs to play 
analogous intellectually foundational roles, 
such as Michael Jubien's Property Theory
\cite{MichaelJubien}
or Jiri Benovsky \q{modal perdurants} \cite[p. 8]{JiriBenovsky}
(here citing two treatments I find particularly 
persuasive or thought-provoking).  
This is one \i{philosophical} interest in 
type theory as types have more formal structure 
and also, in a sense, more metaphysical coherence 
than \q{sets} (in the classical sense allowing 
set-composition from unrelated members).  
In a certain intellectual role, however, types are 
still somehow \q{set-like} in that we can 
consider the totality of values that may inhabit 
each type, and we need some theory or model 
of that totality {\mdash} which in turn engenders 
different versions of type theory.}

\p{An obvious paradigm is organizing type-extensions around prototype/borderline
cases {\mdash} there are instances which are clear examples of types and
ones whose classification is dubious.  I contend, however, 
that common resemblance is not always a good marker
for types being well-conceived {\mdash} many useful concepts are common
precisely because they cover many cases, which makes defining
\q{prototypes} or \q{common properties} misleading.  
Also, sometimes the clearest
\q{representative} example of a type or concept is actually not a
\i{typical} example: a sample latter or model home is actually not (in
many cases) a real letter or home.  So resemblance-to-prototype is at
best one kind of \q{inner organization} of concepts' and types' spaces
of extension.}

\p{Sets, concepts, and types represent three different primordial thought-vehicles for
grounding notions of logic and meaning.  To organize systems around \i{sets} is
to forefront notions of inclusion, exclusion, extension, and intersection,
which are also formally essential to mathematical logic and undergird the
classical interdependence of sets, logic, and mathematics.
To organize systems around \i{concepts} is to forefront practical engagement
and how we mold conceptual profiles, as collections of ideas and pragmas,
to empirical situations.  To organize systems around \i{types} is to forefront
\q{functions} or transformations which operate on typed values, the interrelationships
between different types (like subtypes and inclusion {\mdash} a type can itself
encompass multiple values of other types), and the conceptual abstraction
of types themselves from the actual sets of values they may exhibit
in different environments.  Sets and types are
formal, abstract phenomena; whereas concepts are characterized by
gradations of applicability, and play flexible roles in thought and language.
The cognitive role of concepts can be discussed with some rigor, but there is a
complex interplay of cognitive schema and practical engagements which
would have to be meticulously sketched in many real-world scenarios, if
our goal were to translate conceptual reasoning to formal structures
on a case-by-case basis.  We can, however, consider in general
terms how type-theoretic semantics can capture conceptual structures
as part of the overall transitioning of thoughts to language.}

\p{A concept does not merely package up a definition, like \q{restaurant} as
\q{a place to order food}; instead concepts link up with other concepts
as tools for describing and participating in situations.  Concepts are
associated with \q{scripts} of discourse and action, and find their
range of application through a variegated pragmatic scope.
We should be careful not to overlook these pragmatics, and
assume that conceptual structures can be simplistically
translated to formal models.
Cognitive Linguistics critiques
Set-Theoretic or Modal Logic reductionism (where a concept is just a set
of instances, or an extension across different possible worlds) {\mdash} George Lakoff and Mark Johnson,
prominently, argue for concepts' organization around
prototypes (\cite[p. 18]{LakoffJohnson} ; \cite[p. 171, or p. \textit{xi}]{Johnson})
and embodied/enactive patterns of interaction (\cite[p. 90]{LakoffJohnson} ;
\cite[p. 208]{Johnson}).

Types, by contrast, at least in linguistic applications of type theory, are abstractions
defined in large part by quasi-functional notions of phrase structure.
Nevertheless, the \i{patterns} of how types may inter-relate
(mass-noun or count-noun, sentient or non-sentient, and so forth)
provide an infrastructure for conceptual understandings to be
encoded in language {\mdash} specifically, to be signaled by which typed
articulations conversants choose to use.  A concept like
\i{restaurant} enters language with a collection of understood
qualities (social phenomena, with some notion of spatial location and
being a \q{place}, etc.) that in turn can be marshaled by sets of
allowed or disallowed phrasal combinations, whose parameters
can be given type-like descriptions.  Types, in this sense,
are not direct expressions of concepts but vehicles for
introducing concepts into language.}

\p{Concepts (and types also) are not cognitively the same as their
extension {\mdash} the concept \i{restaurant}, I believe, is distinct from
concepts like \i{all restaurants} or \i{the set of all restaurants}.
This is for several reasons.  First, concepts can be pairwise different
not only through their instances, but because they highlight different
sets of attributes or indicators.  The concepts \q{American President} and \q{Commander in Chief}
refer to the same person, but the latter foregrounds a military role.
Formal Concept Analysis considers \i{extensions} and \q{properties}
{\mdash} suggestive indicators that inhere in each
instance {\mdash} as jointly (and co-dependently) determinate: concepts
are formally a synthesis of instance-sets and property-sets \cite{YiyuYao},
\cite{Belohlavek}, \cite{Wille}.  Second,
in language, clear evidence for the contrast between \i{intension} and
\i{extension} comes from phrase structure: certain constructions specifically
refer to concept-extension, triggering a mental shift from thinking of the
concept as a schema or prototype to thinking of its extension (maybe in some context).
Compare:

\begin{sentenceList}

\sentenceItem{} \swl{itm:rhinor}{Rhinos in that park are threatened by poachers.}{sem}
\sentenceItem{} \swl{}{Young rhinos are threatened by poachers.}{sem}
\end{sentenceList}

Both sentences focus a conceptual lens in greater detail than \i{rhino} in general, but
the second does so more intensionally, by adding an extra indicative criterion; while
the former does so extensionally, using a phrase-structure designed to operate on
and narrow our mental construal of \q{the set of all rhinos}, in the sense of
\i{existing} rhinos, their physical place and habitat, as opposed to
the \q{abstract} (or \q{universal}) type.  So there is a familiar semantic
pattern which mentally transitions from a lexical type to its extension and
then extension-narrowing {\mdash} an interpretation that, if accepted, clearly
shows a different mental role for concepts of concepts' \i{extension} than the
concepts themselves.\footnote{There is a type-theoretic correspondence between intension and
extension {\mdash} for a type \Tnoindex{} there is a corresponding \q{higher-order} type
of \i{sets} whose members are \Tnoindex{}
(related constructions are the type of \i{ordered sequences} of \Tnoindex{};
unordered collections of \Tnoindex{} allowing repetition; and stacks, queues, and
deques {\mdash} double-ended queues {\mdash} as \Tnoindex{}-lists that can grow or shrink
at their beginning and/or end).  If we take this (higher-order)
type gloss seriously, the extension of a concept is not its \i{meaning}, but a
different, albeit interrelated concept.  Extension is not definition.
\i{Rhino} does not mean \i{all rhinos} (or \i{all possible rhinos}) {\mdash} though arguably
there are concepts \i{all rhinos} and \i{all restaurants} (etc.) along with the concepts
\i{rhino} and \i{restaurant}.}
}

\p{Concepts, in short, do not mentally signify sets, or
extensions, or sets-of-shared-properties.  Concepts, rather, are cognitive/dialogic tools.
Each concept-choice, as presentation device,
invites its own follow-up. \i{Restaurant} or \i{house} have meaning not via
idealized mental pictures, or proto-schema, but via kinds of things
we do (eat, live), of conversations we have, of qualities we deem relevant.  Concepts do not
have to paint a complete picture, because we use them as part of ongoing situations
{\mdash} in language, ongoing conversations.  Narrow concepts {\mdash} which may best exemplify
\q{logical} models of concepts as resemblance-spaces or as rigid designators to
natural kinds {\mdash} have, in practice, fewer use-cases \i{because} there
are fewer chances for elaboration.  Very broad concepts, on the other hand, can have,
in context, too \i{little} built-in \i{a priori} detail.
(We say \q{restaurant} more often than \i{eatery}, and
more often than \i{diner}, \i{steakhouse}, or \i{taqueria}).  Concepts dynamically play
against each other, making \q{spaces} where different niches of meaning, including
levels of precision, converge as site for one or another.  Speakers need freedom to choose
finer or coarser grain, so concepts are profligate, but the most oft-used trend toward middle
ground, neither too narrow nor too broad. \i{Restaurant} or \i{house} are useful because they are noncommittal, inviting more detail.
These dynamics govern the flow of inter-concept relations (disjointness, subtypes, partonymy, etc.).}

\p{Concepts are not rigid formulae (like instance-sets or even attributes fixing when
they apply); they are mental gadgets to initiate and
guide dialog.  Importantly, this
contradicts the idea that concepts are unified around instances' similarity (to each other or
to some hypothetical prototype): concepts have avenues for contrasting
different examples, invoking a \q{script} for further elaboration, or for building temporary filters.  
In, say,

\begin{sentenceList}

\sentenceItem{} \swl{}{Let's find a restaurant that's family-friendly.}{sem}
\end{sentenceList}

allowing such one-off narrowing is a feature
of the concept's flexibility.}

\p{In essence: no less important, than acknowledged similarities across all instances, are well-rehearsed ways
\visavis{} each concept to narrow scope by marshaling lines of \i{contrast}, of \i{dissimilarity}.
A \i{house} is obviously different from a \i{skyscraper}
or a \i{tent}, and better resembles other houses; but there are also more nontrivial \i{comparisons}
between houses, than between a house and a skyscraper
or a tent.  Concepts are not only spaces of similarity, but of \i{meaningful kinds of differences}.}

\p{To this account of conceptual breadth we can add the conceptual matrix spanned by
various (maybe overlapping) word-senses: to \i{fly}, for example, names
not a single concept, but a family of concepts all related to airborn
travel.  Variations highlight different features: the path of flight (\i{fly to Korea}, \i{fly over the mountain});
the means (\i{fly Korean air}, \i{that model flew during World War II});
the cause (\i{sent flying (by an explosion)}, \i{the bird flew away (after a loud noise)},
\i{leaves flying in the wind}).  Words allow different use-contexts
to the degree that their various \i{senses} offer an inventory of aspects for
highlighting by \i{morphosyntactic} convention.  Someone who says \i{I hate to fly} is not
heard to dislike hand-gliding or jumping off mountains.\footnote{People, unlike birds, do not fly {\mdash} so the verb, used intransitively
(not flying \i{to} somewhere in particular or \i{in} something in particular),
is understood to refer less to the physical motion and more to the socially
sanctioned phenomenon of buying a seat on a scheduled flight on an airplane. The construction
highlights the procedural and commercial dimension, not the physical mechanism and
spatial path.  But it does so \i{because} we know human flight is
unnatural: we can poetically describe how the sky is filled with flying leaves or birds,
but not \q{flying people}, even if we are nearby an airport.} Accordant variations
of cognitive construal (attending more to mode of action, or path, or motives, etc.),
which are elsewhere signaled by grammatic choices, are also spanned by a conceptual
space innate to a given word: senses are finer-grained meanings availing themselves to one construal or another.}

\p{So situational construals can be signaled by word- and/or
syntactic form choice (locative, benefactive, direct and indirect
object constructions, and so forth).  Whereas conceptual organization
often functions by establishing classifications, and/or invoking
\q{scripts} of dialogic elaboration, cognitive structure tends to apply more
to our attention focusing on particular objects, sets of objects, events, or
aspects of events or situations.  
So the contrast between singular, mass-multiples, and count-multiples,
among nouns, depends on cognitive
construal of the behavior of the referent in question (if singular, its
propensity to act or be conceived as an integral whole; if multiple, its
disposition to either be divisible into discrete units, or not).
Or, events can be construed in terms of their causes
(their conditions at the outset), or their goals (their conditions at
the conclusion), or their means (their conditions in the interim).
Compare \i{attaching} something to a wall (means-focused) to
\i{hanging} something on a wall (ends-focused); \i{baking} a cake
(cause-focus: putting a cake in the oven with deliberate intent to cook it)
to \i{burning} a cake (accidentally overcooking it).\footnote{We can express
an intent to bake someone a cake, but not (well, maybe comedically) to
\i{burn} someone a cake (\q{burn}, at least in this context, implies
something not intended); however, we \i{can} say
\q{I burnt your cake}, while it is a little jarring to say
\q{I baked your cake} {\mdash} the possessive implies that some
specific cake is being talked about, and there is less apparent reason
to focus on one particular stage of its preparation (the baking) once
it is done.  I \i{will} bake a cake, in the future, uses
\q{bake} to mean also other steps in preparation (like \q{make}), while,
in the present, \q{the cake \i{is} baking} emphasizes more its
actual time in the oven.  I \i{baked your cake} seems to focus
(rather unexpectedly) on this specific stage even after it is completed,
whereas \i{I baked you a cake}, which is worded as if the recipient
did not know about the cake ahead of time, apparently uses \q{bake} in
the broader sense of \q{made}, not just \q{cooked in an oven}.
Words' senses mutate in relation to the kinds of situations where they are used
{\mdash} why else would \i{bake} mean \q{make}/\q{prepare} in the past or future tense but
\q{cook}/\q{heat} in the present?}
These variations are not random assortments of polysemous words' senses:
they are, instead, rather predictably distributed according
to speakers' context-specific knowledge and motives.}

\p{I claim therefore that \i{concepts} enter language complexly, influenced by
conceptual \i{spaces} and multi-dimensional semantic and syntactic selection-spaces.
Concepts are not simplistically \q{encoded} by types, as if for
each concept there is a linguistic or lexical type that just
disquotationally references it {\mdash} that the type \q{rhino} means the concept
\i{rhino} (\q{type} in the sense that type-theoretic semantics would model lexical
data according to type-theoretic rules, such as \i{rhino} as subtype of \i{animal} or
\i{living thing}).
Cognitive schema, at least in the terms I just laid out, select particularly
important gestalt principles (force dynamics, spatial frames, action-intention)
and isolate these from a conceptual matrix.  On this basis, we can argue that
these schemata form a precondition for concept-to-type association; or,
in the opposite logical direction, that language users' choices to employ
particular type articulations follow forth from their prelinguistic
cognizing of practical scenarios as this emerges out of collections
of concepts used to form a basic understanding of and self-positioning within them.}

\p{In this sense I called types \q{vehicles} for concepts: not that types \i{denote}
concepts but that they (metaphorically) \q{carry} concepts into language.
\q{Carrying} is enabled by types' semi-formal rule-bound
interactions with other types, which are positioned to capture concepts' variations and
relations with other concepts.  
This is the relationship I would propose for integrating 
theories of concepts {\mdash} as cognitive and lexicosemantic 
phenomena {\mdash} with applications of type theory which, 
as I argued above, are fundamentally syntactic: 
types as structural invariants patterning the integration 
between processes operative in language understanding.}

\p{To express a noun in the benefactive case, for example, which can be seen as attributing to
it a linguistic type consistent with being the target of a benefactive,
is to capture the concept in a type-theoretic gloss.
It tells us, I'm thinking about this thing in such a way that it
\i{can} take a benefactive (the type formalism attempting to capture
that \q{such a way}).
A concept-to-type \q{map}, as I just
suggested, is mediated (in experience and practical reasoning) by
cognitive organizations; when (social, embodied) enactions take
linguistic form, these organizing principles can be encoded in how
speakers apply morphosyntactic rules.}

\p{So the linguistic structures,
which I propose can be formally modeled by a kind of type theory, work
communicatively as carriers and thereby signifiers of cognitive
attitudes. The type is a vehicle for the concept because it takes part in constructions
which express conceptual details {\mdash} the details
don't emerge merely by virtue of the type itself.
I am not arguing for a neat concept-to-type correspondence; instead, a type system provides a
\q{formal substrate} that models (with some abstraction and simplification) how
properties of individual concepts translate
(via cognitive-schematic intermediaries) to their
manifestation in both semantics and syntax.}

\p{Continuing with declension as a case study,
consider how an \q{ontology} of word senses 
can interrelate with the benefactive.
A noun as a benefactive target most often is a person or some other
sentient/animate being; an inanimate benefactive is most likely
something artificial and constructed (cf., \i{I got the car new tires}).
How readily hearers accept a sentence {\mdash} and the path they
take to construing its meaning so as to make it grammatically acceptable
{\mdash} involves interlocking morphological and type-related considerations;
in the current example, the mixture of benefactive case and which noun
\q{type} (assuming a basic division of nouns into e.g.
animate/constructed/natural) forces a broader or narrower
interpretation.  A benefactive with an \q{artifact} noun, for example,
almost forces the thing to be heard as somehow disrepaired:

\begin{sentenceList}

\sentenceItem{} \swl{}{I got glue for your daughter.}{sem}
\sentenceItem{} \swl{}{I got glue for your coffee mug.}{sem}
\end{sentenceList}

We gather (in the second case) that the mug is broken {\mdash} but this is never spelled out
by any lexical choice; it is implied indirectly by using benefactive case.  
It is easy to design similar examples with other cases:
a locative construction rarely targets \q{sentient} nouns, so in 
(reprising (\ref{itm:grandma})-(\ref{itm:press}))

\begin{sentenceList}

\sentenceItem{} \swl{}{We're going to Grandma!}{sem}
\sentenceItem{} \swl{}{Let's go to him right now.}{sem}
\sentenceItem{} \swl{}{Let's go to the lawyers.}{sem}
\sentenceItem{} \swl{}{Let's go to the press.}{sem}
\end{sentenceList}

we mentally substitute the person with the place where they live or work.}

\p{Morphosyntactic
considerations are also at play: \i{to the lawyers} makes \q{go} sound more like \q{consult with},
partly because of the definite article (\i{the} lawyers implies conversants have some prior involvement
with specific lawyers or else are using the phrase metonymically, as in \q{go to court} or
\q{to the courts}, for
legal institutions generally; either reading draws attention away from literal spatial implications of
\q{go}). \i{Go to him} implies that \q{he} needs
some kind of help, because if the speaker just meant going to wherever he's at, she probably would
have said that instead.}

\p{Similarly, the locative in \i{to the press} forces the mind to
reconfigure the landmark/trajector structure, where \i{going} is thought not as a literal
spatial path and \i{press} not a literal destination {\mdash} in other words, the phrase must be
read as a metaphor.  But the \q{metaphor} here is not \q{idiomatic} or removed from linguistic rules
(based on mental resemblance, not language structure); here it
clearly works off of formal language patterns: the landmark/trajector
relation is read abstracted from literal spatial movement because the locative is applied
to an expression (\i{the press}) which does not (simplistically) meet
the expected interpretation as \q{designation of place}.
In short, there are two different levels of \i{granularity} where we 
can look for agreement requirements: a more fine-grained level where e.g.
\i{locative} draws in a type-specification of a \i{place} or \i{location}; 
and a coarser level oriented toward Parts of Speech, and typologies 
of phrasal units.  The former analysis addresses the level I have called
\q{macrotypes}, while the latter scale is at the \q{macrotype} level.}

\p{I envision the unfolding that I have just sketched out as something Phenomenological
{\mdash} it arises from a unified and subjective consciousness, one marked by
embodied personal identity and social situation.  If there are structural stases
that can be found in this temporality of experience, these are not constitutive
of conscious reality but a mesh of rationality that supports it, like the veins in
a leaf.  Stuctural configurations can be lifted from language insofar as it is a
conscious, formally governed activity, and lifted from the ambient situations which
lend language context and meaning intents.  So any analytic emphasis on
structural fixpoints threaded through the lived temporality of consciousness is an
abstraction, but one that is deliberate and necessary if we want to make scientific
or in any other manner disputatable claims about how language and congition works.}

\p{To return to the example of \i{Student after student}, I 
commented that designating
one word to \q{represent} the phrase seemed arbitrary.  
If we consider functional-typing alone, \i{after} is the only non-noun,
the natural conclusion is that \q{after} should be typed \NNtoN{}
(which implies that \q{after} is analogous to the \q{functional} position, and
in a lambda-calculus style reconstruction would be considered the \q{head}
{\mdash} Figure ~\ref{fig:ESA} is an example of how
the sentence could be annotated, for sake of discussion).
This particular idiom depends however
on the two constituent nouns being the same word (a pattern I've also alluded to with
idioms like \i{time after time}).  
Technically, this appears to be an example of \i{dependent types}: 
specifically, a type-theoretic model of \i{after} in this would seem 
to require that the \NNtoN{} signature be constrained so that the second 
noun matches the first {\mdash} so the second \N{} type is actually constrained 
to be a singleton type dependent on the first \N{}'s value.  
By way
of illustration, Figure ~\ref{fig:ESA} shows a
destructuring with implicit type annotations. \begin{figure*}
\caption{Dependency-style graph with argument repetition}	
\label{fig:ESA}
\vspace{1em}
\hspace{0.15\textwidth}	
\begin{minipage}{0.7\textwidth}
\begin{tikzpicture}

%\tikzset{snake it/.style={decorate, decoration=snake, segment length=5mm, amplitude=15mm}}

%\draw

%\node [s1] at (0,0) {Student};

\node (s1) at (1,1) {\textbf{Student}};
\node (after) [right=5mm of s1] {\textbf{after}};
\node (s2) [right=5mm of after] {\textbf{student}};
\node (complained) [right=5mm of s2] {\textbf{complained}};

\node (s1Rep) [double,draw=black,shape=circle,thick,fill=gray!50,inner sep=.5em,below=3.5cm of s1] {};
\node (afterRep) [double,draw=black,shape=circle,thick,fill=gray!50,inner sep=.5em,below=2cm of after] {};
\node (s2Rep) [double,draw=black,shape=circle,thick,fill=gray!50,inner sep=.5em,below=3.5cm of s2] {};
\node (complainedRep) [double,draw=black,shape=circle,thick,fill=gray!50,inner sep=.5em,below=2.5cm of complained] {};

\draw [ |-,-|, <->, line width = .8mm, draw=gray!70, 
 dashed, double equal sign distance, >= stealth, shorten <= .25cm, shorten >= .25cm ]
 (s1) to (s1Rep);

\draw [ |-,-|, <->, line width = .8mm, draw=gray!70, 
 dashed, double equal sign distance, >= stealth, shorten <= .25cm, shorten >= .25cm ]
(after) to (afterRep);
 
\draw [ |-,-|, <->, line width = .8mm, draw=gray!70,  
 dashed, double equal sign distance, >= stealth, shorten <= .25cm, shorten >= .25cm ]
(s2) to (s2Rep);

\draw [ |-,-|, <->, line width = .8mm, draw=gray!70, 
 dashed, double equal sign distance, >= stealth, shorten <= .25cm, shorten >= .25cm ]
(complained) to (complainedRep);
 
\draw [shorten <= .25cm, shorten >= .25cm ] 
(afterRep) to node [draw=black,shape = star,star points=4,thick,inner sep = 0mm, above] {1} (s1Rep);

\draw [shorten <= .25cm, shorten >= .25cm ] 
(afterRep) to node [draw=black,shape = star,star points=4,thick,inner sep = 0mm,above] {1} (s2Rep);

\node (s1E) [left=-2mm of s1Rep.east]{};
\node (s2E) [left=1mm of s2Rep.west]{};

\draw [shorten <= 2cm, shorten >= .3cm, double, % bend left = 5, 
decorate, decoration={snake, segment length=5mm, amplitude=.5mm}] %, amplitude=15mm]  
(s1E) to node [draw=black,shape = star,star points=4,thick,inner sep = 0mm, below] {2} (s2E);

\draw [shorten <= .5cm, shorten >= .5cm ] 
(afterRep) edge [bend left=20,looseness=1] node [draw=black,shape = star,star points=4,thick,inner sep = 0mm, 
 above, near end ] {3} (complainedRep);

\node (frameTopLeft) [below left = 1.5cm and -.65 cm of s1] {};
\node (frameBottomLeft) [below = 2.5cm of frameTopLeft] {};
\node (frameBottomRight) [right = 3.95cm of frameBottomLeft] {};
\node (frameTopRight) [above = 2.5cm of frameBottomRight] {};

\draw [shorten <= 0.15cm, shorten >= 0.15cm ] 
(frameTopLeft) edge [bend right=30,looseness=1] (frameBottomLeft);

\draw [shorten <= 0.15cm, shorten >= 0.15cm ] 
(frameBottomLeft) edge [bend right=30,looseness=1] (frameBottomRight);

\draw [shorten <= 0.15cm, shorten >= 0.15cm ] 
(frameBottomRight) edge [bend right=35,looseness=1] (frameTopRight);

\draw [shorten <= 0.15cm, shorten >= 0.45cm ] 
(frameTopRight) edge [bend right=30,looseness=1]
node [draw=black,shape = regular polygon,regular polygon sides=3,thick,inner sep = .2mm, 
above, near start, shape border rotate = 180] {4} (frameTopLeft);

%node [draw=black,shape = star,star points=4,thick,inner sep = 0mm, above, 
%bend left=100,looseness=3] {3}

%;


\end{tikzpicture}
\end{minipage}

\hspace{0.1\textwidth}
\begin{minipage}{0.8\textwidth}
			\renewcommand{\labelitemi}{$\blacklozenge$}
	
\begin{itemize}\setlength\itemsep{-.3em}
\item 1 \hspace{12pt}  Head/dependent relation
\item 2 \hspace{12pt}  Argument repetition ({}\BlankAfterBlank{} idiom)
\item 3 \hspace{12pt}  Propositional completion ({}\VisNtoS{})	
			\renewcommand{\labelitemi}{$\blacktriangledown$}
\item 4 \hspace{12pt}  Phrase (modeled as applicative structure), typed as {}\NPl{} 
\end{itemize}
\end{minipage}
\end{figure*}
Earlier I opined that the repetition in \BlankAfterBlank{} is a 
rhetorical effect emphasizing that the recurrence of some
phenomenon is tedious or noteworthy {\mdash} in this one usage \i{after}
becomes a pluralizing adjective that takes two arguments, but
requires them to be the same.  This sense of \i{after} can be
given a function-like \POS{} notation as, say,
\AfterNSingAndNSingToNPl{} (using \NSing{} and \NPl{} to mean singular
and count-plural nouns, respectively), but with a further 
stipulation that the second argument duplicates the first.  
Formally, then, the parameters
for \i{after} are a dependent type pair
\cite{BernardyEtAl}, \cite{TanakaEtAl} satisfied by an identity comparison between
the two nouns.  This analysis captures a type-theoretic gloss on the 
structural contrast between \i{Student after student} and
\i{Many students}, phrases which are similar but not identical 
in meaning (so whose differences need explaining).}

\p{I have offered a more cognitive account focusing on 
the implicit temporality of \i{Student after student}; this later 
type-oriented model is more formal, or at least leaves open the 
possibility that language is organically taking in structures 
engineered into artificial (e.g., computer programming) languages. 
It is certainly possible to witness formalizable structures in 
language patterns {\mdash} Zhaohui Luo finds strong evidence for
dependent types being a good model for semantic norms in
\cite{LuoSoloviev}, for example.  Whether these kinds of 
formalisms have important causal influence on 
language acquiring its evident patterns, or are more like 
just convenient representational tools, is perhaps an 
open (and maybe case-by-case) question.}

\p{Consider alternatives for \q{many students}.  The phrase as
written suggests a type signature (with \q{many} as the \q{function-like} or
derivative type) \NpltoNpl{}, yielding a syntactic interpretation of the phrase; this
interpretation also suggests a semantic progression, an accretion of intended detail.
From \i{students} to \i{many students} is a conversion between two plural nouns
(at the level of concepts and semantic roles); but it also implies relative size,
so it implies some \i{other} plural, some still larger group of students from which
\q{many} are selected.  While rather abstract and formal, the \NpltoNpl{} representation
points toward a more cognitive grounding which considers this \q{function} as a form
of thought-operation; a refinement of a situational model, descriptive resolution,
and so forth.  If we are prepared to accept a cognitive underpinning to semantic
classification, we can make the intuition of part of speech signatures as \q{functions}
more concrete: in response to what \q{many} (for example) is a function \i{of},
we can say a function of propositional attitude, cognitive schema, or attentional
focus.}

\p{The schema which usefully captures the sense and picture of \i{students} is
distinct (but arguably a variation on) that for \i{many students}, and there is a
\q{mental operation} triggered by the \i{many students} construction which
\q{maps} the first to the second.  Similarly, \i{student after student} triggers a
\q{scheme evolution} which involves a more explicit temporal unfolding
(in contrast to how \i{many students} instead involves a more explicit
quantitative \i{many/all} relation).  What these examples show is that
associating parts of speech with type signatures is not just a formal
fiat, which \q{works} representationally but does not necessarily capture
deeper patterns of meaning.  Instead, I would argue, type signatures
and their resonance into linkage acceptability structures
(like singular/plural and mass/count agreement) \i{point toward} the
effects of cognitive schema on what we consider meaningful.}

\p{In \i{Student after student came out against the proposal},
to \i{come out}, for/against, lies in the semantic frame of attitude and expression
(it requires a mental agent, for example), but, as per my prior analysis, its reception
carries a trace of spatial form: to come out \i{to} a public place, to
go on record with an opinion.  Usually
\q{come out [for/against]}, in the context of a policy or idea, is similarly
metaphorical.  But the concrete spatial interpretation remains latent, as a kind
of residue on even this abstract rendition, and the spatial 
undercurrent is poised to emerge
as more literal, should the context warrant.  However literally or metaphorically
the \q{space} of the \q{coming out} is
understood, however explicit or latent its cogitative figuration,
is not something internal to the language; it is a potentiality which
will present in different ways in different circumstances.  This is not to say that
it is something apart from linguistic meaning, but it shows how linguistic meaning
lies neither in abstract structure alone, nor contextual pragmatics, but in their cross-reference.}

\p{}

\p{}

\p{}

\p{}

\p{}


%
\section{Conclusion}
\p{Of the three type levels I have proposed, the macrotype \q{functional} level is the most
quasi-mathematical; for other levels, formal type theory may provide interpretive
tools and methodological guides, but formally representable framings and
transformations may be only approximations of how people actually think, while
they are understanding language.  From this perspective, we are left with the
metatheoretical question of clarifying how different kinds of analyses, which
put different degrees of weight on formal or on interpretive argumentation,
are to be joined in overarching theories.  In particular, are the
linguistic phenomena which seem to demand more \q{interpretive} treatment actually
beyond formalization, or is it just impractical (but possible in theory) to provide
formal analysis of each individual case-study, each real-world language formation?
Is Natural Language actually no less formal than (for example) computer programming
languages, except that the former have a much larger set of semantic and syntactic
rules such that any analysis can uncover them only partially?  Or is any rule-based
model of language, no matter how complete, necessarily partial relative to real language?}

\p{We can consider at what point formal and computational methods reach a limit,
beyond which they fail to capture
the richess and expressiveness of Natural Language, or whether this limit itself
is an illusion {\mdash} whether even fully human
language competence is (perhaps in principle if not in practice) no less reducible
to formalizable patterns.  Whatever one's beliefs on this last question,
a progression of subdisciplines {\mdash} from formal-logical semantics through programming
languages and computational Natural Language Processing {\mdash} is a reasonable
scaffolding for a universe of formal methods that can build up, by progressive theoretical
sophistication or assembly of distinct analyses which piece together jigsaw-like, to model
real-world language understanding.  Perhaps real language is an \q{emergent property} of
many distinct algorithms that run and combine in the mind; or perhaps the relevant
algorithms are a precondition, presenting cognition with essential signifying givens
but fleshed out in other, more holistic ways, as we become conscious of language not
just as a formal system but an interactive social reality.}

\p{I have claimed that Cognitive Transform Grammar aims toward a 
theoretic nexus that plugs into several syntactic and semantic 
methodologies.  Both Depenedency Grammar word-pairs and 
functional type-attributions on lexemes (together with their
\q{arguments}) can be interpreted as Cognitive transforms.  
By itself, the superposition of
type-theoretic semantics on link-grammar graphs does not cross a hypothetical \q{barrier} between
the formal and the cognitive.  But I intend here to suggest a cognitive \i{interpretation}
for the formal structures; that they represent an outline of cognitive schema, or progressions,
or represent linguistic \q{triggers} that a cognitive language ability (taking language
as part of an environing world and produced by others, in rule-bound social situations,
to communicate ideas and sentiments) responds to.  This range of interpretations is
deliberately open-ended: we can say that a formal infrastructure grounds the cognitive
reception of language givens, without arguing specifically that formal structures identified
in language therefore model cognitive operations directly, or that these are instead
patterns identified in language that trigger a cognitive response, or any other
paradigm for mapping cognition as process and activity to language structure as model and
prototype.  Leaving these options open, however, I will focus in the remainder of this
paper on one interpretation, considering formal structures as \q{triggers} which
get absorbed into language understanding via observatory propensities: as language
users (on this proposal) we are disposed to identify certain formal structurations
operating in language as we encounter it, and respond to these observations by building
or refining mental models of the situations and signifying intentions we believe have been
implied by the discourse, in evolving and intersubjective dialogic settings that involve
joint practical activity as well as communication.}

\p{In this sense, I believe natural language reveals mutually-modifying juxtapositions
of concepts whose full semantic effects
are probably not \q{computable}: I would work on the assumption that language
\i{as a whole} and as human social phenomena are the precinct of a 
cultural fluency \i{beyond} Natural Language Processing.   
The aforementioned \q{linguistic side effects} can be \i{modeled} by tracing our reception
of linguistic meaning through syntactic and semantic formations, like Dependency Grammar
and Type Theory, but I argue for such models not as models \i{of} cognitive processes,
but rather models of \i{observations} which trigger cognitive follow-up.  Even if we
believe in and practice a rigorous formalization of morphosyntactic structure,
where the \i{pattern} of conceptual \q{side-effects} can be seen as
unfolding in algorithmic ways, the cognitive \i{details} of these
effects are too situational, and phenomenologically rich, for
computability as ordinarily understood.}

\p{But the formal structure is
not wholly irrelevant: to call up nuanced cognitive schema
{\mdash} or so I submit for consideration {\mdash} may not be possible without
algorithmically reproducible lexicosemantic and morphosyntactic triggers,
at least modulo some approximation.  A (perhaps non-computable) space
of cognitive schema may be projected onto a (perhaps computable)
set of affiliated morphological patterns, using notations like
link-grammar pairs and type signatures to catalog them.  For example, there may be a non-computable
expanse of possible construals of pluralization; but any such construal,
in context, is called into focus in conversants' minds by morphosyntactic
invitations, by speakers' choices of, say, \mbox{\NSingToNPl{}}-pattern
phrases.  The important balance is to take formalization as far as is reasonable
without being seduced into logico-symbolic reductionism.}

\p{Any word or usage invites various facets to either
emphasize or deemphasize, and these subsumed concepts or foci are
latent in potential meanings, brought into linguistic space
by the play of differentiation\footnote{Alluding, in part, to Sausurrean \q{system of differences}
\cite[p. 15]{EfePeker} {\mdash} to 
choose a reference which introduces
Sausurre in a rather unexpected context.}
: \i{baked}, not \i{made}; \i{flew}, not \i{traveled};
\i{spill}, not \i{pour}.
These under-currents of subsidiary concepts and foci are selectively hooked onto by
morphosyntactic selection, so in analyzing phrase
structure we also have to consider how using syntax
which constructs a given structure also brings to the forefront certain
nested concepts and construals, which are latent in word-sense options;
in the topos of lexicosemantic possibilia.}

\p{So, any talk about \q{side effects} of morphosyntactic functions
{\mdash} mapping verb-space to adjective-space, noun-space to
proposition-space, singularity to plurality, and so forth {\mdash} should consider
a type-theoretic gloss like \NtoN{} as sketching just the motivating
scaffold around an act of cognitive refocusing.  The interesting semantics
lies with \i{how} a sense crosses over, in conversants' minds,
to some other sense or concept, wherein other aspects are foregrounded
{\mdash} for example, within temporal event plurality: multiplicity as
frequency, or episodic distribution relative to some time span;
or suggesting something that is typical
or predominant; or relative count against some other
totality {\mdash} each such refocusing triggered by a phrasal construction
of the form \NtoNpl{} or \mbox{\NpltoNpl{}}.
Or we can map singulars, or count plurals, to mass nouns, and vice-versa (\i{shrubs} become \i{foliage};
\i{water} becomes \i{a glass of water}).
The plural and the singular are a coarse-grained semantic that has not yet arrived as \i{meaning}.
Conceptual complexes guide attention to classes and properties, defining a path of ascending
precision as speakers add descriptive detail;
cognitive construals negotiate relations between different kinds
of aggregates/individuals; individuality, aggregation and multiplicity as phenomena and
disposition.  These construals are practical and embodied, \i{and}
phenomenological {\mdash} they direct attention (\i{qua} transcendental universal of
mentality, if we like), to and fro, but in the course of intersubjective and
goal-driven practical action (and in that sense particular, world-bound, historicized).}

\p{Linguistically, the \q{effects} of language \q{functions} are
mutations/modifications in cognitive state, respondent to concrete
or abstract scenarios which are topics of dialog.  Sometimes, effects may
tolerate mathematical analysis; but such analytical thematics tend to peter out into the
ambient, chaotic worldliness of human consciousness.}


%
\begin{thebibliography}{99}
\phantomsection \label{References}
\addcontentsline{toc}{section}{References}
{\fontfamily{lmtt}\selectfont\scriptsize

\bibitem{RaubalAdams}
Benjamin Adams and Martin Raubal, 
\cq{A Metric Conceptual Space Algebra}.
\biburl{https://pdfs.semanticscholar.org/521a/cbab9658df27acd9f40bba2b9445f75d681c.pdf}

\bibitem{RaubalAdamsCSML}
Benjamin Adams and Martin Raubal, 
\cq{Conceptual Space Markup Language (CSML): Towards the Cognitive Semantic Web}.
\biburl{http://idwebhost-202-147.ethz.ch/Publications/RefConferences/ICSC_2009_AdamsRaubal_Camera-FINAL.pdf}

\bibitem{AsherPustejovsky}
Nicholas Asher and James Pustejovsky, 
\cq{A Type Composition Logic for Generative Lexicon}
\biburl{https://www.cs.brandeis.edu/~jamesp/classes/cs216-2009/readings2009/TCLforGL.pdf}


\bibitem{BarkerShanTG}
Chris Barker and Chung-Chieh Shan, 
\cq{Types as Graphs: Continuations in Type Logical Grammar}
\biburl{https://www.nyu.edu/projects/barker/barker-shan-types-as-graphs.pdf}

%
\bibitem{Belohlavek}
Radim B\v{e}lohl\'avek and Vladim{\i\OldI}r Sklen\'a\v{r}, 
\cq{Formal Concept Analysis Constrained by Attribute-Dependency Formulas}
B. Ganter and R. Godin, eds.,: ICFCA 2005, LNCS 3403, pp. 176-191, Berlin, Springer-Verlag, 2005.
\biburl{http://belohlavek.inf.upol.cz/publications/BeSk_Fcacadf.pdf}

\bibitem{JiriBenovsky}
Juri Benovsky, 
\cq{Two Concepts of Possible Worlds – or Only One?}.  
THEORIA, 2008, volume 74, pp. 318–330.
\biburl{http://citeseerx.ist.psu.edu/viewdoc/download?doi=10.1.1.464.3904&rep=rep1&type=pdf}



\bibitem{BernardyEtAl}
Jean-Philippe Bernardy, \i{et. al.},
\cq{Parametricity and Dependent Types}.
\biburl{http://www.staff.city.ac.uk/~ross/papers/pts.pdf}

\bibitem{BiskriDescles}
Isma{\"\OldI}l Biskri and Jean-Pierre Descles, 
\cq{Applicative and Combinatory Categorial Grammar 
(from syntax to functional semantics)}
\biburl{https://www.researchgate.net/publication/232754402_Applicative_and_Combinatory_Categorial_Grammar_from_syntax_to_functional_semantics}

\bibitem{BittnerSmithDonnelly}
Thomas Bittner, Barry Smith, and Maureen Donnelly,
\cq{The logic of systems of granular partitions.}
\biburl{http://ontology.buffalo.edu/smith/articles/BittnerSmithDonnelly.pdf}


\bibitem{MathieuBouchard}
Mathieu Bouchard, 
\cq{A Type Theory for the Documentation of PureData}
\biburl{http://artengine.ca/~catalogue-pd/43-Bouchard.pdf}

\bibitem{LineBrandt}
Line Brandt, 
\cq{The Communicative Mind.}
Newcastle-upon-Tyne: Cambridge Scholars Publishing, 2013

%
\bibitem{PerAageBrandt}
Per Aage Brandt, 
\cq{Spaces, Domains, and Meaning: Essays in Cognitive Semiotics}.
Newcastle-upon-Tyne: Cambridge Scholars Publishing, 2013
\biburl{http://semiotics.au.dk/fileadmin/Semiotics/pdf/per-aage-brandt/Per_Aage_Brandt__Spaces__Domains__and_Meaning._Essays_in_Co.pdf}



\bibitem{ChatzikyriakidisLuo}
Stergios Chatzikyriakidis and Zhaohui Luo, 
\cq{Individuation Criteria, Dot-types and Copredication: A View from Modern Type Theories.}
\intitle{Association for Computational Linguistics, Proceedings of the 14th 
Meeting on the Mathematics of Language (MoL 14)}, pp. 39–50, 2015. 
\biburl{http://www.aclweb.org/anthology/W15-2304}

\bibitem{ChoeCharniak}
Do Kook Choe and Eugene Charniak, 
\cq{Parsing as Language Modeling}.
\tinyurl{https://aclweb.org/anthology/D16-1257}


\bibitem{InteractingConceptualSpaces}
Bob Coecke, \i{et. al.}, 
\cq{Interacting Conceptual Spaces I:
Grammatical Composition of Concepts}.
\biburl{https://arxiv.org/pdf/1703.08314.pdf}

\bibitem{Descles2010}
Jean-Pierre Descl\'es, \cq{Reasoning in Natural Language in 
	Using Combinatory Logic and Topology: An Example with Aspect
	and Temporal Relations.}  
\biburl{https://www.aaai.org/ocs/index.php/FLAIRS/2010/paper/viewFile/1350/1736}


\bibitem{KitFine}
Kit Fine, \cq{Part-whole}, in Barry Smith and 
David Woodruff Smith, \intitle{The Cambridge Companion to Husserl}, 
pp. 463-486.
Cambridge University Press, New York, 1995.


\bibitem{Fauconnier}
Gilled Fauconnier, \cq{Mental Spaces: Aspects of Meaning Construction in Natural Language}.
Cambridge, 1994

\bibitem{Gardenfors}
Peter \Gardenfors{},  
\cq{Does Semantics Need Reality?}.
Riegler A., Peschl M., von Stein A. (eds),
\intitle{Understanding Representation in the Cognitive Sciences.}, 
pages 209-217.
Springer, Boston, MA.

\bibitem{Zenker}
Peter \Gardenfors{} and Frank Zenker,  
\cq{Theory Change as Dimensional Change: Conceptual Spaces 
	Applied to the Dynamics of Empirical Theories}.
\intitle{Synthese 190(6)}, pp. 1039-1058, 2013.  
\biburl{http://lup.lub.lu.se/record/1775234}

\bibitem{GoertzelPLN}
Ben Goertzel, 
\cq{Probabilistic Language Networks:
Integrating Word Grammar and Link Grammar 
in the Framework of Probabilistic Logic}.
\biburl{http://goertzel.org/ProwlGrammar.pdf}

\bibitem{KennethHolmqvist}
Kenneth Holmqvist,
\cq{Conceptual Engineering: Implementing cognitive semantics}, 
in Jens Allwood and Peter \Gardenfors, eds., 
\intitle{Cognitive Semantics}, pp 153 - 171, Amsterdam,
Philadelphia: John Benjamins, 1999.

\bibitem{HolmqvistDiss}
Kenneth Holmqvist, 
\cq{Implementing Cognitive Semantics: Image schemata, 
valence accommodation, and valence suggestion for  
AI and computational linguistics}.
PhD thesis, Dept. of Cognitive Science, Lund University, Lund, Sweden, 1993.

\bibitem{BlackwellPragmatics}
Laurence R. Horn and Gregory Ward, 
\cq{Handbook of Pragmatics}.
Wiley, 2004

\bibitem{Johnson}
Mark Johnson, \cq{The Body in the Mind: 
The Bodily Basis of Meaning, Imagination, and Reason}.  %Chicago, 1990
University of Chicago Press, 1990

\bibitem{MichaelJubien}
Michael Jubien, \cq{Ontology, Modality, and 
the Fallacy of Reference}.  
Cambridge University Press, New York, 1993.

\bibitem{KubotaLevine}
Yusuke Kubota and Robert Levine,
\cq{The syntax-semantics interface of \sq{respective} predication:
A unified analysis in Hybrid Type-Logical Categorial Grammar}
\biburl{https://www.asc.ohio-state.edu/levine.1/publications/kl-resp.pdf}


\bibitem{LakoffJohnson}
George Lakoff and Mark Johnson, 
\cq{Philosophy in the Flesh: the Embodied Mind and its Challenge to Western Thought}.
New York: Basic Books, 1999.


\bibitem{LangackerFoundations}
Ronald Langacker, 
\cq{Foundations of Cognitive Grammar, vol. 1}.  
Stanford University Press, 1991

\bibitem{ZhaohuiLuo}
Zhaohui Luo,
\cq{Type-Theoretical Semantics with Coercive Subtyping}.
\biburl{https://www.cs.rhul.ac.uk/home/zhaohui/ESSLLI11notes.pdf}

\bibitem{ZhaohuiLuoSignatures}
Zhaohui Luo,
\cq{Using Signatures in Type Theory to Represent Situations}.
\biburl{https://www.researchgate.net/publication/268079019_Using_Signatures_in_Type_Theory_to_Represent_Situations}


\bibitem{LuoSoloviev}
Zhaohui Luo and Sergei Soloviev,
\cq{Dependent Coercions}.
\biburl{https://www.sciencedirect.com/science/article/pii/S1571066105803147}

\bibitem{MeryMootRetore}
Bruno Mery, Richard Moot, and Christian Retor\'e, \cq{
	Plurals: Individuals and sets
	in a richly typed semantics}.  
\biburl{http://arxiv.org/pdf/1401.0660.pdf} \archiveDate{3 Jan 2014}

\bibitem{ErwanMoreau}
Erwan Moreau,
\cq{From link grammars to categorial grammars}
\biburl{https://hal.archives-ouvertes.fr/hal-00487053/document}


\bibitem{OsborneMaxwell}
Timothy Osborne and Daniel Maxwell, 
\cq{A Historical Overview of the Status of Function
Words in Dependency Grammar}
\biburl{https://www.aclweb.org/anthology/W15-2127}


\bibitem{EstherPascual}
Esther Pascual, 
\cq{Fictive Interaction: The conversation frame in thought, language, and discourse}.
Philadelphia, John Benjamins, 2014.

\bibitem{EfePeker}
Efe Peker,
\cq{Following 9/11: George W. Bush's Discursive Re-Articulation of American Social Identity}
Master's Dissertation, Link\"oping University, Sweden
\biburl{http://www.diva-portal.org/smash/get/diva2:21407/FULLTEXT01.pdf}

\bibitem{PetitotSyntaxe}
Jean Petitot, \cq{Syntax Topologique et Grammaire Cognitive}.
\intitle{Langages 25.103}, pp. 97-128, 1991. 

%?
\bibitem{JeanPetitot}
Jean Petitot, \cq{The morphodynamical turn of cognitive linguistics}.
\biburl{https://journals.openedition.org/signata/549}

\bibitem{Pinker}
Steven Pinker, \cq{The Stuff of Thought: Language As a Window Into Human Nature}.
Penguin, 2007.  

\bibitem{JamesPustejovsky}
James Pustejovsky, 
\cq{A Survey of Dot Objects}
Unpublished manuscript, 2006.

\bibitem{Ramstead}
Maxwell James Ramstead, 
\cq{Transcendental Phenomenology and Naturalistic Epistemology.}
\biburl{https://archipel.uqam.ca/8209/1/M14029.pdf}

\bibitem{SivaReddy}
Siva Reddy, \i{et. al},
\cq{Transforming Dependency Structures to Logical Forms for Semantic Parsing.}
\biburl{https://aclweb.org/anthology/Q16-1010}

\bibitem{TerryRegier}
Terry Regier, 
\i{The Human Semantic Potential: 
Spatial Language and Constrained Connectionism}.
Cambridge, MA: MIT Press, 2014.

\bibitem{Rossi}
\Aurelie{} Rossi, \cq{Applicative and Combinatory Categorial Grammar:
	Analysis of the French Interrogative Sentences}.
\intitle{FLAIRS Conference, Association for the Advancement of Artificial
Intelligence}, 2008.
\biburl{https://www.aaai.org/Papers/FLAIRS/2008/FLAIRS08-118.pdf}

\bibitem{Schneider}
Gerold Schneider, 
\cq{A Linguistic Comparison of Constituency, Dependency and Link Grammar}.
Zurich University, diploma, 2008.
\biburl{https://files.ifi.uzh.ch/cl/gschneid/papers/FINALSgeroldschneider-latl.pdf}


\bibitem{MattSelway}
Matt Selway, \i{et. al},
\cq{Configuring Domain Knowledge for Natural Language Understanding}
\biburl{http://ceur-ws.org/Vol-1128/paper9.pdf}


\bibitem{SleatorTamperley}
Daniel D. Sleator and Davy Tamperley, 
\cq{Parsing English with a Link Grammar}.
\biburl{https://www.link.cs.cmu.edu/link/ftp-site/link-grammar/LG-IWPT93.pdf}

\bibitem{DavidWoodruffSmith}
David Woodruff Smith,
\cq{Mind World}.  Cambridge University Press, 2004.

\bibitem{KiyoshiSudo}
Kiyoshi Sudo, \i{et. al.},
\cq{An Improved Extraction Pattern Representation Model
for Automatic IE Pattern Acquisition}
\biburl{https://www.aclweb.org/anthology/P03-1029}


\bibitem{TanakaEtAl}
Ribeka Tanaka, \i{et. al.}, \cq{Factivity and Presupposition
in Dependent Type Semantics}
\biburl{http://www.lirmm.fr/tytles/Articles/Tanaka.pdf}

\bibitem{OrlinVakarelov}
Orlin Vakarelov,
\cq{Pre-cognitive Semantic Information}.
\biburl{https://link.springer.com/article/10.1007/s12130-010-9109-5}

\bibitem{VakarelovAgent}
Orlin Vakarelov, 
\cq{The cognitive agent: Overcoming
informational limits}
\biburl{https://philarchive.org/archive/VAKTCAv1}


\bibitem{JanVanEijck}
Jan van Eijck, 
\cq{Dynamic Reasoning Without Variables}
\biburl{https://www.semanticscholar.org/paper/Dynamic-Reasoning-Without-EijckCWI/4fb3b23f81b8691fbdaf8d3eb24c086c5f826954}

Rik van Noord
\bibitem{RikVanNoord}
Rik van Noord, \i{et. al.},
\cq{Exploring Neural Methods for Parsing Discourse Representation Structures}
\biburl{https://www.aclweb.org/anthology/Q18-1043}


\bibitem{AnneVilnat}
Anne Vilnat,
\cq{Dialogue et analyse de phrases}
Habilitation M\'emoire, Universit\'e Paris-Sud  
\biburl{https://perso.limsi.fr/anne/HDR/MemoireHDR.pdf}

%\bibitem{WiegandMereology}
\bibitem{OlavKWiegand}
Olav K. Wiegand, 
\cq{A Formalism Supplementing Cognitive Semantics Based on a New Approach to Mereology}.
Ingvar Johansson, Bertin Klein and Thomas Roth-Berghofer, eds.,  
\intitle{Contributions to the Third International Workshop on Philosophy and Informatics}, 2006. 

\bibitem{WiegandGestalts}
\underlines
\cq{On referring to Gestalts}.
Mirja Hartimo, ed., \intitle{Phenomenology and Mathematics}, pp. 183-211.  Springer, 2010. 

\bibitem{Wille}
Rudolf Wille, 
\cq{Conceptual Graphs and Formal Concept Analysis}.
\intitle{Conceptual Structures: Fulfilling Peirce's Dream
Fifth International Conference on Conceptual Structures}, 1997.

\bibitem{YiyuYao}
Yiyu Yao, 
\cq{A Comparative Study of Formal Concept
	Analysis and Rough Set Theory in Data Analysis}.
\biburl{http://www2.cs.uregina.ca/~yyao/PAPERS/Rough_concept.pdf}

\bibitem{JordanZlatev}
Jordan Zlatev, 
%\underlines
\cq{The Dependence of Language on Consciousness}.
\biburl{http://www.mrtc.mdh.se/~gdc01/work/ARTICLES/2014/4-IACAP\%202014/IACAP14-GDC/pdf/Language-Consciousness-Zlatev.pdf}


}
\end{thebibliography}



\end{document}

