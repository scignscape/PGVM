\documentclass[12pt,letterpaper]{article}

%https://www.ncbi.nlm.nih.gov/pmc/articles/PMC3555311

\DeclareMathSizes{12pt}{14}{14}{14}

% pmml  arff  openannotation

%\usepackage[condensed,math]{anttor}
%\usepackage[T1]{fontenc}

%\usepackage[T1]{fontenc}
%\usepackage{tgtermes}

\usepackage[hang,flushmargin]{footmisc}

\usepackage{titlesec}

%\usepackage{sectsty}
%\sectionfont{\fontsize{13}{4}\selectfont}

\titleformat{\section}
  {\normalfont\fontsize{13}{15}\bfseries}{\thesection}{1em}{}

\let\OldPart\part

\renewcommand{\part}[1]{\OldPart{#1}%
%{\textcolor{darkRed}\hrule}
\vspace{-.5em}}

\titlespacing*{\section}
{0pt}{4.1ex plus 1ex minus .1ex}{-0.2ex plus .2ex}

\titlespacing*{\subsection}
{0pt}{3.1ex plus 1ex minus .1ex}{-.8ex plus .2ex}

%\usepackage{mathptmx}

\usepackage{eso-pic}

%\setlength\parindent{0pt}

\AddToShipoutPictureBG{%

\ifnum\value{page}>1{
\AtTextUpperLeft{
\makebox[20.5cm][r]{
\raisebox{-2.45cm}{%
{\transparent{0.3}{\includegraphics[width=0.29\textwidth]{e-logo.png}}	}} } }
}\fi
}

\AddToShipoutPicture{%
{
 {\color{blGreen!70!red}\transparent{0.9}{\put(0,0){\rule{3pt}{\paperheight}}}}%
 {\color{darkRed!70!purple}\transparent{1}\put(3,0){{\rule{4pt}{\paperheight}}}}
% {\color{logoPeach!80!cyan}\transparent{0.5}{\put(0,700){\rule{1cm}{.6cm}}}}%
% {\color{darkRed!60!cyan}\transparent{0.7}\put(0,706){{\rule{1cm}{.6cm}}}}
% \put(18,726){\thepage}
% \transparent{0.8}
}
}

\AddToShipoutPicture{%
\ifnum\value{page}=1
\put(257.5,942){%
	\transparent{0.7}{
		\includegraphics[width=0.2\textwidth]{logo.png}}}
\put(59,953){\textbf{{\fontfamily{phv}\fontsize{14}{14}\selectfont{}WHITE PAPER}}}
\fi
}	



\AddToShipoutPicture{%
\ifnum\value{page}>1
{\color{blGreen!70!red}\transparent{0.9}{\put(300,6){\rule{0.5\paperwidth}{.3cm}}}}%
{\color{inOne}\transparent{0.8}{\put(300,8){\rule{0.5\paperwidth}{.3cm}}}}%
{\color{inTwo}\transparent{0.3}\put(300,11){{\rule{0.5\paperwidth}{.3cm}}}}

\put(301,14){%
\transparent{0.7}{
\includegraphics[width=0.2\textwidth]{logo.png}}
}
%\fi

%\pgfmathparse{equal(\value{page},15)||equal(\value{page},20)?int(1):int(0)}

\ifnum\value{page}=14   %\pgfmathresult>0\relax
\put(41,56){
 {\setlength{\fboxsep}{.65em}\fontsize{9}{10}\selectfont

   {\color{black}{\colorbox{red!60!blGreen}{%
\parbox{7.1cm}{\hspace{-2pt}\colorbox{white}{\vspace{2pt}{\begin{minipage}{6.7cm}
	  {\textit{For more information please contact:}}\\\textbf{\color{blGreen!40!blbl}{Amy Neustein, Ph.D., Founder and CEO}}\\  
       \textbf{{\color{blGreen!20!black}Linguistic Technology Systems \\
      amy.neustein@verizon.net \textbullet{} \textbf{(917) 817-2184} \vspace*{-3pt} }}
	 \end{minipage}}}}}} } }
}
\fi

{\color{blGreen!70!red}\transparent{0.9}{\put(5.6,3){\rule{0.5\paperwidth}{.4cm}}}}%
{\color{inOne}\transparent{1}{\put(5.6,8){\rule{0.5\paperwidth}{.4cm}}}}%
{\color{inTwo}\transparent{0.3}\put(5.6,13){{\rule{0.5\paperwidth}{.4cm}}}}

\fi
}

%\pagestyle{empty} % no page number
%\parskip 7.2pt    % space between paragraphs
%\parindent 12pt   % indent for new paragraph
%\textwidth 4.5in  % width of text
%\columnsep 0.8in  % separation between columns

%\setlength{\footskip}{7pt}

\usepackage[paperheight=14in,paperwidth=8.5in]{geometry}

%\geometry{left=.72in,top=.39in,right=.68in,bottom=1.14in} %margins
\geometry{left=.76in,top=.46in,right=.74in,bottom=1.1in} %margins

\usepackage{etoolbox}% http://ctan.org/pkg/etoolbox
\makeatletter
% \patchcmd{<cmd>}{<search>}{<replace>}{<success>}{<failure>}
\patchcmd{\@part}{\par}{\quad}{}{}
\patchcmd{\@part}{\huge}{\Large}{}{}
\makeatother

\renewcommand{\partname}{\hspace{-1em}Part}

\renewcommand*\thepart{\Roman{part}:}

\renewcommand{\thepage}{\raisebox{2pt}{\arabic{page}}}

%\renewcommand{\footnoterule}{%
%	\kern 2pt
%	\hrule width .92\textwidth height .5pt
%	\kern 6pt
%}

%\setlength{\skip\footins}{84pt plus 3pt minus 3pt}

\usepackage[hyphens]{url}
\newcommand{\biburl}[1]{ {\fontfamily{gar}\selectfont{\textcolor[rgb]{.2,.6,0}%
{\scriptsize {\url{#1}}}}}}

%\linespread{1.3}

\newcommand{\sectsp}{\vspace{12pt}}

\usepackage{graphicx}
\usepackage{color,framed}

\usepackage{textcomp}

\usepackage{float}

\usepackage{mdframed}


\usepackage{setspace}
\newcommand{\rpdfNotice}[1]{\begin{onehalfspacing}{

\Large #1

}\end{onehalfspacing}}

\usepackage{xcolor}

\usepackage[hyphenbreaks]{breakurl}
\usepackage[hyphens]{url}

\usepackage{hyperref}
%\newcommand{\rpdfLink}[1]{\href{#1}{\small{#1}}}
%\newcommand{\dblHref}[1]{\href{#1}{\small{\burl{#1}}}}
%\newcommand{\browseHref}[2]{\href{#1}{\Large #2}}

\colorlet{blCyan}{cyan!50!blue}

\definecolor{darkRed}{rgb}{.2,.0,.1}


\definecolor{blGreen}{rgb}{.2,.7,.3}

\definecolor{darkBlGreen}{rgb}{.1,.3,.2}

\definecolor{oldBlColor}{rgb}{.2,.7,.3}

\definecolor{blColor}{rgb}{.1,.3,.2}

\definecolor{elColor}{rgb}{.2,.1,0}
\definecolor{flColor}{rgb}{0.7,0.3,0.3}

\definecolor{logoOrange}{RGB}{108, 18, 30}
\definecolor{logoGreen}{RGB}{85, 153, 89}
\definecolor{logoPurple}{RGB}{200, 208, 30}

\definecolor{logoBlue}{RGB}{4, 2, 25}
\definecolor{logoPeach}{RGB}{255, 159, 102}
\definecolor{logoCyan}{RGB}{66, 206, 244}
\definecolor{logoRed}{rgb}{.3,0,0}

\newcommand{\colorq}[1]{{\color{logoOrange!70!black}{\q{\small\textbf{#1}}}}}

\definecolor{inOne}{rgb}{0.122, 0.435, 0.698}% Rule colour
\definecolor{inTwo}{rgb}{0.122, 0.698, 0.435}% Rule colour

\definecolor{outOne}{rgb}{0.435, 0.698, 0.122}% Rule colour
\definecolor{outTwo}{rgb}{0.698, 0.435, 0.122}% Rule colour

\colorlet{linkcolor}{flColor!60!red}


\hypersetup{
	colorlinks=true,
	citecolor=blCyan!40!green,
	filecolor=magenta!30!logoBlue,
	urlcolor=blue,
    linkcolor=linkcolor!70!black,
%    allcolors=blCyan!40!green
}


\usepackage[many]{tcolorbox}% http://ctan.org/pkg/tcolorbox

\usepackage{transparent}

\newlength{\bsep}
\setlength{\bsep}{-1pt}
\let\xbibitem\bibitem
\renewcommand{\bibitem}[2]{\vspace{\bsep}\xbibitem{#1}{#2}}

\newenvironment{cframed}{\begin{mdframed}[linecolor=logoPeach,linewidth=0.4mm]}{\end{mdframed}}

\newenvironment{ccframed}{\begin{mdframed}[backgroundcolor=logoGreen!5,linecolor=logoCyan!50!black,linewidth=0.4mm]}{\end{mdframed}}


%\usepackage[T1]{fontenc}

%\usepackage{aurical}
% \Fontauri

%\usepackage{gfsdidot}
\usepackage[T1]{fontenc}

%\makeatletter
%\f@family,  cmr, T1, n, m,
%\f@encoding,
%\f@shape,
%\f@series,
%\makeatother


%\usepackage{aurical}
% \Fontauri

%\usepackage{LibreBodoni}
%\usepackage{fontspec}
%\setmainfont{QTBengal}

\usepackage{relsize}

%
%\newcommand{\bref}[1]{\hspace*{1pt}\textbf{\ref{#1}}}

\newcommand{\bref}[1]{\hspace*{1pt}\textbf{\ref{#1}}}

\newcommand{\pseudoIndent}{

\vspace{10pt}\hspace*{12pt}}

\newcommand{\YPDFI}{{\fontfamily{fvs}\selectfont YPDF-Interactive}}

%
\newcommand{\deconum}[1]{{\protect\raisebox{-1pt}{{\LARGE #1}}}}

\newcommand{\visavis}{vis-\`a-vis}

\newcommand{\VersatileUX}{{\color{red!85!black}{\Fontauri Versatile}}%
{{\fontfamily{qhv}\selectfont\smaller UX}}}

\newcommand{\NDPCloud}{{\color{red!15!black}%
{\fontfamily{qhv}\selectfont {\smaller NDP C{\smaller LOUD}}}}}

\newcommand{\MThreeK}{{\color{blGreen!45!black}%
{\fontfamily{qhv}\fontsize{10}{8}\selectfont {M3K}}}}


%\newcommand{\procitem}[1]{{\Large\texttt{#1}}}
\newcommand{\procitem}[1]{{#1}}
\newcommand{\dprocitem}[1]{{\color{black!60}\textbf{#1}}}


\newcommand{\lfNDPCloud}{{\color{red!15!black}%
{\fontfamily{qhv}\selectfont N{\smaller DP C{\smaller LOUD}}}}}

\newcommand{\textds}[1]{{\fontfamily{lmdh}\selectfont{%
\raisebox{-1pt}{#1}}}}

\newcommand{\dsC}{{\textds{ds}{\fontfamily{qhv}\selectfont \raisebox{-1pt}
{\color{red!15!black}{C}}}}}

\definecolor{tcolor}{RGB}{24,52,61}

\newcommand{\CCpp}{\resizebox{!}{8pt}{\AcronymText{C}}/\Cpp{}}
\newcommand{\NoSQL}{\resizebox{!}{8pt}{\AcronymText{NoSQL}}}
\newcommand{\SQL}{\resizebox{!}{8pt}{\AcronymText{SQL}}}

\newcommand{\SPARQL}{\resizebox{!}{8pt}{\AcronymText{SPARQL}}}

\newcommand{\NCBI}{\resizebox{!}{8pt}{\AcronymText{NCBI}}}

\newcommand{\HTXN}{\resizebox{!}{7.5pt}{\ATexttclr{HTXN}}}
\newcommand{\HGDM}{\resizebox{!}{7.5pt}{\ATexttclr{HGDM}}}



\newcommand{\sHTXN}{\resizebox{!}{7pt}{\ATexttclr{HTXN}}}

\newcommand{\TDM}{\resizebox{!}{8pt}{\AcronymText{TDM}}}

\newcommand{\lHTXN}{\resizebox{!}{8.5pt}{\ATexttclr{H}}%
\resizebox{!}{7.5pt}{\ATexttclr{TXN}}}

\newcommand{\lHGDM}{\resizebox{!}{8.5pt}{\ATexttclr{H}}%
\resizebox{!}{7.5pt}{\ATexttclr{GDM}}}

\newcommand{\lHGXF}{\resizebox{!}{8.5pt}{\ATexttclr{H}}%
\resizebox{!}{7.5pt}{\ATexttclr{GXF}}}

\newcommand{\sHGDM}{\resizebox{!}{6.5pt}{\ATexttclr{H}}%
\resizebox{!}{6.5pt}{\ATexttclr{GDM}}}


\newcommand{\lsHTXN}{\resizebox{!}{9.5pt}{\ATexttclr{HTXN}}}

\newcommand{\LAF}{\resizebox{!}{8pt}{\AcronymText{LAF}}}

\newcommand{\UDpipe}{\resizebox{!}{8pt}{\AcronymText{UDpipe}}}

\newcommand{\C}{\resizebox{!}{8pt}{\AcronymText{C}}}

\newcommand{\FCS}{\resizebox{!}{8pt}{\AcronymText{FCS}}}

\newcommand{\STL}{\resizebox{!}{8pt}{\AcronymText{STL}}}

\newcommand{\GAVI}{\resizebox{!}{8pt}{\AcronymText{GAVI}}}
\newcommand{\ArcGIS}{\resizebox{!}{8pt}{\AcronymText{ArcGIS}}}
\newcommand{\QGIS}{\resizebox{!}{8pt}{\AcronymText{QGIS}}}

\newcommand{\GIS}{\resizebox{!}{8pt}{\AcronymText{GIS}}}
\newcommand{\AngelScript}{\resizebox{!}{8pt}{\AcronymText{AngelScript}}}



\usepackage{mdframed}

\newcommand{\cframedboxpanda}[1]{\begin{mdframed}[linecolor=yellow!70!blue,linewidth=0.4mm]#1\end{mdframed}}


\newcommand{\PVD}{\resizebox{!}{8pt}{\AcronymText{PVD}}}

\newcommand{\TAGML}{\resizebox{!}{8pt}{\AcronymText{TAGML}}}
\newcommand{\sTAGML}{\resizebox{!}{6pt}{\AcronymText{TAGML}}}

\newcommand{\GTagML}{\resizebox{!}{8pt}{\ATexttclr{G{%
\small{TAG}}ML}}}

\newcommand{\lGTagML}{\resizebox{!}{10.5pt}{\ATexttclr{G{%
\scriptsize{TAG}}ML}}}

\newcommand{\SDK}{\resizebox{!}{8pt}{\AcronymText{SDK}}}
\newcommand{\NLP}{\resizebox{!}{8pt}{\AcronymText{NLP}}}

\newcommand{\AXF}{\resizebox{!}{8pt}{\ATexttclr{AXF}}}

\newcommand{\HyperGraphDB}{\resizebox{!}{8pt}{\AcronymText{HyperGraphDB}}}

\newcommand{\AllegroGraph}{\resizebox{!}{8pt}{\AcronymText{AllegroGraph}}}

\newcommand{\Graknai}{\resizebox{!}{8pt}{\AcronymText{Grakn.ai}}}


\newcommand{\sHyperGraphDB}{\resizebox{!}{7pt}{\AcronymText{HyperGraphDB}}}
\newcommand{\sGraknai}{\resizebox{!}{7pt}{\AcronymText{Grakn.ai}}}
\newcommand{\sLMNtal}{\resizebox{!}{7pt}{\AcronymText{LMNtal}}}
\newcommand{\sNeo}{\resizebox{!}{7pt}{\AcronymText{Neo4j}}}
\newcommand{\sWhiteDB}{\resizebox{!}{7pt}{\AcronymText{WhiteDB}}}


\newcommand{\lAXF}{\resizebox{!}{7.5pt}{\ATexttclr{A}}%
\resizebox{!}{6.5pt}{\ATexttclr{XF}}}


\newcommand{\lsAXF}{\resizebox{!}{8.5pt}{\ATexttclr{AXF}}}

\newcommand{\AXFD}{\resizebox{!}{8pt}{\ATexttclr{AXFD}}}

\newcommand{\CBICA}{\resizebox{!}{8pt}{\AcronymText{CBICA}}}

\newcommand{\IORT}{\resizebox{!}{8pt}{\AcronymText{IORT}}}

\newcommand{\openCyto}{\resizebox{!}{8pt}{\AcronymText{openCyto}}}
\newcommand{\FACSanadu}{\resizebox{!}{8pt}{\AcronymText{FACSanadu}}}
\newcommand{\cytoLib}{\resizebox{!}{8pt}{\AcronymText{cytoLib}}}


\newcommand{\sopenCyto}{\resizebox{!}{6pt}{\AcronymText{openCyto}}}
\newcommand{\sFACSanadu}{\resizebox{!}{6pt}{\AcronymText{FACSanadu}}}
\newcommand{\scytoLib}{\resizebox{!}{6pt}{\AcronymText{cytoLib}}}

\newcommand{\sJava}{\resizebox{!}{6pt}{\AcronymText{Java}}}
\newcommand{\sGUI}{\resizebox{!}{6pt}{\AcronymText{GUI}}}
\newcommand{\sCpp}{\resizebox{!}{6pt}{\AcronymText{C++}}}
\newcommand{\sR}{\resizebox{!}{6pt}{\AcronymText{R}}}
\newcommand{\sQt}{\resizebox{!}{6pt}{\AcronymText{Qt}}}

\usepackage{upquote}

\newcommand{\SeDI}{\resizebox{!}{8pt}{\AcronymText{SeDI}}}
\newcommand{\RSNA}{\resizebox{!}{8pt}{\AcronymText{RSNA}}}

\newcommand{\CER}{\resizebox{!}{8pt}{\AcronymText{CER}}}
\newcommand{\PACS}{\resizebox{!}{8pt}{\AcronymText{PACS}}}

\newcommand{\DCMTK}{\resizebox{!}{8pt}{\AcronymText{DCMTK}}}
\newcommand{\lsDCMTK}{\resizebox{!}{8pt}{\AcronymText{DCMTK}}}

\newcommand{\DICOM}{\resizebox{!}{8pt}{\AcronymText{DICOM}}}
\newcommand{\lsDICOM}{\resizebox{!}{8pt}{\AcronymText{DICOM}}}

\newcommand{\CytometryML}{\resizebox{!}{8pt}{\AcronymText{CytometryML}}}

\newcommand{\FCM}{\resizebox{!}{8pt}{\AcronymText{FCM}}}

\newcommand{\OMETIFF}{\resizebox{!}{8pt}{\AcronymText{OME-TIFF}}}
\newcommand{\OME}{\resizebox{!}{8pt}{\AcronymText{OME}}}

\newcommand{\sOMETIFF}{\resizebox{!}{6pt}{\AcronymText{OME-TIFF}}}
\newcommand{\sOME}{\resizebox{!}{6pt}{\AcronymText{OME}}}

\newcommand{\sXML}{\resizebox{!}{6pt}{\AcronymText{XML}}}

\newcommand{\CT}{\resizebox{!}{8pt}{\AcronymText{CT}}}

\newcommand{\LOINC}{\resizebox{!}{8pt}{\AcronymText{LOINC}}}

\newcommand{\RadLex}{\resizebox{!}{8pt}{\AcronymText{RadLex}}}

\newcommand{\DCC}{\resizebox{!}{8pt}{\AcronymText{DCC}}}


\newcommand{\OMOP}{\resizebox{!}{8pt}{\AcronymText{OMOP}}}
\newcommand{\PCORnet}{\resizebox{!}{8pt}{\AcronymText{PCORnet}}}
\newcommand{\FHIR}{\resizebox{!}{8pt}{\AcronymText{FHIR}}}

\newcommand{\CaPTk}{\resizebox{!}{8pt}{\AcronymText{CaPTk}}}

\newcommand{\WebGL}{\resizebox{!}{8pt}{\AcronymText{WebGL}}}


\newcommand{\VIOLIN}{\resizebox{!}{8pt}{\AcronymText{VIOLIN}}}



\newcommand{\lAXFD}{\resizebox{!}{7.5pt}{\ATexttclr{A}}%
\resizebox{!}{6.5pt}{\ATexttclr{XFD}}}


\newcommand{\IJST}{\resizebox{!}{8pt}{\AcronymText{IJST}}}

\newcommand{\BioC}{\resizebox{!}{8pt}{\AcronymText{BioC}}}

\newcommand{\CoNLL}{\resizebox{!}{8pt}{\AcronymText{CoNLL}}}
\newcommand{\CoNLLU}{\resizebox{!}{8pt}{\AcronymText{CoNLL-U}}}

\newcommand{\sapp}{\resizebox{!}{8pt}{\AcronymText{Sapien+}}}
\newcommand{\lsapp}{\resizebox{!}{8.5pt}{\AcronymText{Sapien+}}}
\newcommand{\lssapp}{\resizebox{!}{9.5pt}{\AcronymText{Sapien+}}}

\newcommand{\ePub}{\resizebox{!}{8pt}{\AcronymText{ePub}}}

%\lsLPF


\newcommand{\GIT}{\resizebox{!}{8pt}{\AcronymText{GIT}}}

%\definecolor{atColor}{RGB}{11, 71, 17}


\DeclareMathVersion{fordg}
\SetSymbolFont{letters}{fordg}{OML}{cmr}{b}{n}

%\DeclareMathSizes{12}{11}{12}{11}
%\DeclareMathSizes{12pt}{11pt}{11pt}{11pt}

\definecolor{atcColor}{RGB}{96, 17, 12}
%\textcolor{tcolor}{

\newcommand{\ATextCClr}[1]{\textcolor{atcColor}{\textbf{#1}}}

\newcommand{\ATexttclr}[1]{\textcolor{tcolor}{\textbf{#1}}}

\newcommand{\AIMConc}{\resizebox{!}{7.5pt}{\ATextCClr{AIM-Concepts}}}
\newcommand{\lAIMConc}{\resizebox{!}{8pt}{\ATextCClr{AIM-Concepts}}}

\newcommand{\HGXF}{{\resizebox{!}{8pt}{\ATexttclr{HGXF}}}}

%\newcommand{\lHGXF}{{\resizebox{!}{8.5pt}{\ATexttclr{HGXF}}}}

\newcommand{\sHGXF}{{\resizebox{!}{7pt}{\ATexttclr{HGXF}}}}

\newcommand{\CRtwo}{{\resizebox{!}{7.5pt}{\ATextCClr{CR2}}}}
\newcommand{\lCRtwo}{{\resizebox{!}{8pt}{\ATextCClr{CR2}}}}
\newcommand{\sCRtwo}{{\resizebox{!}{6pt}{\ATextCClr{CR2}}}}



\newcommand{\THQL}{\resizebox{!}{8pt}{\ATexttclr{THQL}}}
\newcommand{\lTHQL}{\resizebox{!}{7.5pt}{\ATexttclr{T}}%
\resizebox{!}{8pt}{\ATexttclr{HQL}}}

\newcommand{\HDICOM}{\resizebox{!}{7.5pt}{\ATexttclr{{\large h}-DICOM}}}

\newcommand{\hVaImm}{\resizebox{!}{7.5pt}{\ATexttclr{{\large h}-VaImm}}}


\newcommand{\PhaonVI}{\resizebox{!}{7.5pt}{\ATexttclr{Phaon-VI}}}



\definecolor{atColor}{RGB}{50, 22, 40}
\newcommand{\ATextClr}[1]{\textcolor{atColor}{\textbf{#1}}}

\newcommand{\DgDb}{{\mathversion{fordg}%
\makebox{\raisebox{-3pt}{\resizebox{!}{11pt}{\ATextClr{%
\rotatebox{17}{$\varsigma$}}}}\hspace{-4pt}%
\resizebox{!}{6.5pt}{\ATextClr{D\hspace{-2pt}B}}}}}


\newcommand{\lDgDb}{{\mathversion{fordg}%
\resizebox{!}{12pt}{\ATextClr{%
\rotatebox{17}{$\varsigma$}}}}\hspace{-4pt}%
\resizebox{!}{6.5pt}{\ATextClr{D\hspace{-2pt}B}}}}}

\newcommand{\URL}{\resizebox{!}{8pt}{\AcronymText{URL}}}
\newcommand{\CSML}{\resizebox{!}{8pt}{\AcronymText{CSML}}}
\newcommand{\sCSML}{\resizebox{!}{7pt}{\AcronymText{CSML}}}
\newcommand{\LPF}{\resizebox{!}{8pt}{\AcronymText{LPF}}}
\newcommand{\lLPF}{\resizebox{!}{8.5pt}{\AcronymText{LPF}}}
\newcommand{\lsLPF}{\resizebox{!}{9.5pt}{\AcronymText{LPF}}}

\newcommand{\AI}{\resizebox{!}{7.5pt}{\AcronymText{AI}}}
\newcommand{\lAI}{\resizebox{!}{8pt}{\AcronymText{AI}}}

\newcommand{\Jupyter}{\resizebox{!}{8pt}{\AcronymText{Jupyter}}}
\newcommand{\Python}{\resizebox{!}{8pt}{\AcronymText{Python}}}
\newcommand{\IDN}{\resizebox{!}{8pt}{\AcronymText{IDN}}}
\newcommand{\JPG}{\resizebox{!}{8pt}{\AcronymText{JPG}}}
\newcommand{\JPEG}{\resizebox{!}{8pt}{\AcronymText{JPEG}}}
\newcommand{\PNG}{\resizebox{!}{8pt}{\AcronymText{PNG}}}
\newcommand{\TIFF}{\resizebox{!}{8pt}{\AcronymText{TIFF}}}
\newcommand{\REPL}{\resizebox{!}{8pt}{\AcronymText{REPL}}}

\newcommand{\ITK}{\resizebox{!}{8pt}{\AcronymText{ITK}}}


\newcommand{\Pandore}{\resizebox{!}{8pt}{\AcronymText{Pandore}}}

\newcommand{\MIFlowCyt}{\resizebox{!}{8pt}{\AcronymText{MIFlowCyt}}}
\newcommand{\GatingML}{\resizebox{!}{8pt}{\AcronymText{Gating-ML}}}
\newcommand{\flowCL}{\resizebox{!}{8pt}{\AcronymText{flowCL}}}

\newcommand{\sMIFlowCyt}{\resizebox{!}{6.5pt}{\AcronymText{MIFlowCyt}}}



\makeatletter

\newcommand*\getX[1]{\expandafter\getX@i#1\@nil}

\newcommand*\getY[1]{\expandafter\getY@i#1\@nil}
\def\getX@i#1,#2\@nil{#1}
\def\getY@i#1,#2\@nil{#2}
\makeatother
	
\definecolor{BlueGreen}{rgb}{.1,.6,.4}

\newcommand{\ann}[9]{%
	\path [draw=#1,draw opacity=#2,line width=#3, fill=#4, fill opacity = #5, even odd rule] %
	(#6) ellipse(#7 and #8) ellipse(#7*#9 and #8*#9);}

\newcommand{\rectann}[9]{%
\path [draw=#1,draw opacity=#2,line width=#3, fill=#4, fill opacity = #5, even odd rule] %
(#6) rectangle(\getX{#6}+#7,\getY{#6}+#8)
({\getX{#6}+((#7-(#7*#9))/2)},{\getY{#6}+((#8-(#8*#9))/2)}) rectangle %
({\getX{#6}+((#7-(#7*#9))/2)+#7*#9},{\getY{#6}+((#8-(#8*#9))/2)+#8*#9});}

\definecolor{grammarArrowColor}{rgb}{.85,.85,.45}

\definecolor{pfcolor}{RGB}{94, 54, 73}

\newcommand{\EPF}{\resizebox{!}{8pt}{\AcronymText{ETS{\color{pfcolor}pf}}}}
\newcommand{\lEPF}{\resizebox{!}{8.5pt}{\AcronymText{ETS{\color{pfcolor}pf}}}}
\newcommand{\lsEPF}{\resizebox{!}{9.5pt}{\AcronymText{ETS{\color{pfcolor}pf}}}}


\newcommand{\XPDF}{\resizebox{!}{8pt}{\AcronymText{XPDF}}}

\newcommand{\GRE}{\resizebox{!}{8pt}{\AcronymText{GRE}}}
\newcommand{\CAS}{\resizebox{!}{8pt}{\AcronymText{CAS}}}

\newcommand{\lMOSAIC}{%
\resizebox{!}{8pt}{\ATexttclr{M}}%
\resizebox{!}{6pt}{\ATexttclr{OSAIC}}}

\newcommand{\llMOSAIC}{%
\resizebox{!}{8.5pt}{\ATexttclr{MOSAIC}}}

\newcommand{\ConceptsDB}{\resizebox{!}{7.5pt}{\ATexttclr{ConceptsDB}}}

\newcommand{\sConceptsDB}{\resizebox{!}{8pt}{\ATexttclr{ConceptsDB}}}

\newcommand{\lConceptsDB}{\resizebox{!}{8pt}{\ATexttclr{C}}%
\resizebox{!}{7.5pt}{\ATexttclr{onceptsDB}}}

\newcommand{\llConceptsDB}{\resizebox{!}{8.5pt}{\ATexttclr{C}}%
\resizebox{!}{8pt}{\ATexttclr{onceptsDB}}}

\newcommand{\CsAF}{\resizebox{!}{8pt}{\ATexttclr{CsAF}}}
\newcommand{\lCsAF}{\resizebox{!}{8pt}{\ATexttclr{C}}%
\resizebox{!}{7.5pt}{\ATexttclr{sAF}}}


\newcommand{\XML}{\resizebox{!}{8pt}{\AcronymText{XML}}}

\newcommand{\lXML}{\resizebox{!}{8pt}{\AcronymText{XML}}}

\newcommand{\RDF}{\resizebox{!}{8pt}{\AcronymText{RDF}}}
\newcommand{\DOM}{\resizebox{!}{8pt}{\AcronymText{DOM}}}

\newcommand{\Java}{\resizebox{!}{8pt}{\AcronymText{Java}}}


\newcommand{\ParaView}{\resizebox{!}{8pt}{\AcronymText{ParaView}}}
\newcommand{\Octave}{\resizebox{!}{8pt}{\AcronymText{Octave}}}
\newcommand{\ROOT}{\resizebox{!}{8pt}{\AcronymText{ROOT}}}
\newcommand{\CERN}{\resizebox{!}{8pt}{\AcronymText{CERN}}}
\newcommand{\MQFour}{\resizebox{!}{8pt}{\AcronymText{MQ4}}}
\newcommand{\VISSION}{\resizebox{!}{8pt}{\AcronymText{VISSION}}}

\newcommand{\sVISSION}{\resizebox{!}{6pt}{\AcronymText{VISSION}}}


\newcommand{\ReproZip}{\resizebox{!}{8pt}{\AcronymText{ReproZip}}}
\newcommand{\BioCoder}{\resizebox{!}{8pt}{\AcronymText{BioCoder}}}

\newcommand{\Covid}{\resizebox{!}{8pt}{\AcronymText{Covid-19}}}


\newcommand{\HMCL}{{\resizebox{!}{7.5pt}{\ATexttclr{HMCL}}}}
\newcommand{\DSPIN}{{\resizebox{!}{7.5pt}{\ATexttclr{D-SPIN}}}}

%\newcommand{\sPandore}{{\resizebox{!}{6.5pt}{\ATexttclr{Pandore}}}}
%\newcommand{\sMIBBI}{{\resizebox{!}{6.5pt}{\ATexttclr{MIBBI}}}}

\newcommand{\lDSPIN}{{\resizebox{!}{8pt}{\ATexttclr{D-SPIN}}}}
\newcommand{\lsDSPIN}{{\resizebox{!}{8.5pt}{\ATexttclr{D-SPIN}}}}

\newcommand{\sDSPIN}{{\resizebox{!}{6.5pt}{\ATexttclr{D-SPIN}}}}

\newcommand{\CLang}{\resizebox{!}{8pt}{\AcronymText{C}}}

\newcommand{\HNaN}{\resizebox{!}{8pt}{\AcronymText{HN%
\textsc{a}N}}}

\newcommand{\JSON}{\resizebox{!}{8pt}{\AcronymText{JSON}}}
\newcommand{\UV}{\resizebox{!}{8pt}{\AcronymText{UV}}}

\newcommand{\sJSON}{\resizebox{!}{6pt}{\AcronymText{JSON}}}

\newcommand{\PET}{\resizebox{!}{8pt}{\AcronymText{PET}}}
\newcommand{\MRI}{\resizebox{!}{8pt}{\AcronymText{MRI}}}


\newcommand{\MeshLab}{\resizebox{!}{8pt}{\AcronymText{MeshLab}}}
\newcommand{\IQmol}{\resizebox{!}{8pt}{\AcronymText{IQmol}}}

\newcommand{\SGML}{\resizebox{!}{8pt}{\AcronymText{SGML}}}

\newcommand{\WhiteDB}{\resizebox{!}{8pt}{\AcronymText{\makebox{WhiteDB}}}}

\newcommand{\CrossRef}{\resizebox{!}{8pt}{\AcronymText{CrossRef}}}

\newcommand{\ASCII}{\resizebox{!}{8pt}{\AcronymText{ASCII}}}

\newcommand{\IAT}{\resizebox{!}{8pt}{\AcronymText{IAT}}}

\newcommand{\GUI}{\resizebox{!}{8pt}{\AcronymText{GUI}}}
\newcommand{\UI}{\resizebox{!}{8pt}{\AcronymText{UI}}}

\newcommand{\DSL}{\resizebox{!}{8pt}{\AcronymText{DSL}}}

\newcommand{\URI}{\resizebox{!}{8pt}{\AcronymText{URI}}}
\newcommand{\DTD}{\resizebox{!}{8pt}{\AcronymText{DTD}}}
\newcommand{\sDTD}{\resizebox{!}{6pt}{\AcronymText{DTD}}}

\newcommand{\API}{\resizebox{!}{8pt}{\AcronymText{API}}}

\newcommand{\JATS}{\resizebox{!}{8pt}{\AcronymText{JATS}}}


\newcommand{\SDI}{\resizebox{!}{8pt}{\AcronymText{SDI}}}
\newcommand{\SDIV}{\resizebox{!}{8pt}{\AcronymText{SDIV}}}

\definecolor{atColor}{RGB}{50, 22, 40}
\newcommand{\ATextClr}[1]{\textcolor{atColor}{\textbf{#1}}}

\newcommand{\DgDbx}{\makebox{\raisebox{-3pt}{\resizebox{!}{11pt}{\ATextClr{%
\rotatebox{17}{$\varsigma$}}}}\hspace{-4pt}%
\resizebox{!}{6.5pt}{\ATextClr{D\hspace{-2pt}B}}}}

\newcommand{\lDgDbx}{\makebox{\raisebox{-3pt}{%
\resizebox{!}{12pt}{\ATextClr{%
\rotatebox{17}{$\varsigma$}}}}\hspace{-4pt}%
\resizebox{!}{6.5pt}{\ATextClr{D\hspace{-2pt}B}}}}


\newcommand{\IDE}{\resizebox{!}{8pt}{\AcronymText{IDE}}}

\newcommand{\OWL}{\resizebox{!}{8pt}{\AcronymText{OWL}}}

\newcommand{\Kaggle}{\resizebox{!}{8pt}{\AcronymText{Kaggle}}}


\newcommand{\ViSion}{\resizebox{!}{8pt}{\AcronymText{ViSion}}}

\newcommand{\CWL}{\resizebox{!}{8pt}{\AcronymText{CWL}}}

\newcommand{\ThreeD}{\resizebox{!}{8pt}{\AcronymText{3D}}}
\newcommand{\TwoD}{\resizebox{!}{8pt}{\AcronymText{2D}}}

\newcommand{\medInria}{\resizebox{!}{8pt}{\AcronymText{medInria}}}
\newcommand{\ThreeDimViewer}{\resizebox{!}{8pt}{\AcronymText{3DimViewer}}}

\newcommand{\FAIR}{\resizebox{!}{8pt}{\AcronymText{FAIR}}}

\newcommand{\MIAPEGI}{\resizebox{!}{8pt}{\AcronymText{MIAPE-GI}}}
\newcommand{\MIAPE}{\resizebox{!}{8pt}{\AcronymText{MIAPE}}}



\newcommand{\QNetworkManager}{\resizebox{!}{8pt}{\AcronymText{QNetworkManager}}}
\newcommand{\QTextDocument}{\resizebox{!}{8pt}{\AcronymText{QTextDocument}}}
\newcommand{\QWebEngineView}{\resizebox{!}{8pt}{\AcronymText{QWebEngineView}}}
\newcommand{\HTTP}{\resizebox{!}{8pt}{\AcronymText{HTTP}}}


\newcommand{\lAcronymTextNC}[2]{{\fontfamily{fvs}\selectfont {\Large{#1}}{\large{#2}}}}

\newcommand{\AcronymTextNC}[1]{{\fontfamily{fvs}\selectfont {\large #1}}}


\colorlet{orr}{orange!60!red}

\newcommand{\textscc}[1]{{\color{orr!35!black}{{%
%						\fontfamily{Cabin-TLF}\fontseries{b}\selectfont
{\textsc{\textbf{#1}}}}}}}


\newcommand{\textsccserif}[1]{{\color{orr!35!black}{{%
				\scriptsize{\textbf{#1}}}}}}


\newcommand{\iXPDF}{\resizebox{!}{8pt}{\textsccserif{%
\textit{XPDF}}}}

\newcommand{\iEPF}{\resizebox{!}{8pt}{\textsccserif{%
\textit{ETSpf}}}}

\newcommand{\iSDI}{\resizebox{!}{8pt}{\textsccserif{%
\textit{SDI}}}}

\newcommand{\iHTXN}{\resizebox{!}{8pt}{\textsccserif{%
\textit{HTXN}}}}


\newcommand{\AcronymText}[1]{{\textscc{#1}}}

\newcommand{\AcronymTextser}[1]{{\textsccserif{#1}}}


\newcommand{\mAcronymText}[1]{{\textscc{\normalsize{#1}}}}

\newcommand{\FASTA}{{\resizebox{!}{8pt}{\AcronymText{FASTA}}}}
\newcommand{\SRA}{{\resizebox{!}{8pt}{\AcronymText{SRA}}}}
\newcommand{\DNA}{{\resizebox{!}{8pt}{\AcronymText{DNA}}}}
\newcommand{\MAP}{{\resizebox{!}{8pt}{\AcronymText{MAP}}}}
\newcommand{\EPS}{{\resizebox{!}{8pt}{\AcronymText{EPS}}}}
\newcommand{\CSV}{{\resizebox{!}{8pt}{\AcronymText{CSV}}}}
\newcommand{\PDB}{{\resizebox{!}{8pt}{\AcronymText{PDB}}}}

%\newcommand{\WebGL}{{\resizebox{!}{8pt}{\AcronymText{WebGL}}}}

\newcommand{\Docker}{{\resizebox{!}{8pt}{\AcronymText{Docker}}}}


\newcommand{\OBO}{{\resizebox{!}{8pt}{\AcronymText{OBO}}}}

\newcommand{\XOCS}{{\resizebox{!}{8pt}{\AcronymText{XOCS}}}}

\newcommand{\RNA}{{\resizebox{!}{8pt}{\AcronymText{RNA}}}}

\newcommand{\ChemXML}{{\resizebox{!}{8pt}{\AcronymText{ChemXML}}}}

\newcommand{\TeXMECS}{\resizebox{!}{8pt}{\AcronymText{TeXMECS}}}

% pmml  arff  openannotation

\newcommand{\PMML}{\resizebox{!}{8pt}{\AcronymText{PMML}}}
\newcommand{\ARFF}{\resizebox{!}{8pt}{\AcronymText{ARFF}}}
\newcommand{\IeXML}{\resizebox{!}{8pt}{\AcronymText{IeXML}}}

\newcommand{\SeCo}{\resizebox{!}{8pt}{\AcronymText{SeCo}}}


\newcommand{\HDFFive}{\resizebox{!}{8pt}{\AcronymText{HDF5}}}

\newcommand{\NGML}{\resizebox{!}{8pt}{\ATexttclr{NGML}}}

\newcommand{\SDRM}{\resizebox{!}{8pt}{\ATexttclr{SDRM}}}
\newcommand{\sSDRM}{\resizebox{!}{6pt}{\ATexttclr{SDRM}}}
\newcommand{\lSDRM}{\resizebox{!}{8pt}{\ATexttclr{S}}%
\resizebox{!}{8pt}{\ATexttclr{DRM}}}


%\newcommand{\SDRF}{\resizebox{!}{8pt}{\ATexttclr{SDRF}}}
%\newcommand{\sSDRF}{\resizebox{!}{6pt}{\ATexttclr{SDRF}}}
%\newcommand{\lSDRF}{\resizebox{!}{8pt}{\ATexttclr{S}}%
%\resizebox{!}{8pt}{\ATexttclr{DRF}}}

\newcommand{\SDRFx}{\resizebox{!}{7.5pt}{\ATexttclr{SDR%
\hspace{1pt}{\raisebox{-1pt}{\resizebox{!}{8.5pt}{\fontfamily{qhv}\fontseries{b}\selectfont{}{F}}}%
}}}}


\newcommand{\lSDRF}{\resizebox{!}{8pt}{\ATexttclr{S}}\resizebox{!}{8pt}{\ATexttclr{DR%
\hspace{1pt}{\raisebox{-.5pt}{\fontfamily{qhv}\fontseries{b}\selectfont{}\Large{F}}%
}}}}

\newcommand{\SDRF}{\resizebox{!}{8pt}{\ATexttclr{S}}\resizebox{!}{8pt}{\ATexttclr{DR%
\hspace{1pt}{\raisebox{-1pt}{\fontfamily{qhv}\fontseries{b}\selectfont{}\Large{F}}%
}}}}

\newcommand{\sSDRF}{\resizebox{!}{6.5pt}{\ATexttclr{S}}\resizebox{!}{6.5pt}{\ATexttclr{DR%
\hspace{1pt}{\raisebox{-.5pt}{\fontfamily{qhv}\fontseries{b}\selectfont{}\Large{F}}%
}}}}


\newcommand{\Cpp}{\resizebox{!}{8pt}{\AcronymText{C++}}}

%\newcommand{\\WhiteDB{}}{\resizebox{!}{8pt}{\AcronymText{\WhiteDB{}}}}

\colorlet{drp}{darkRed!70!purple}

%\newcommand{\MOSAIC}{{\color{drp}{\AcronymTextNC{\scriptsize{MOSAIC}}}}}

\newcommand{\MOSAIC}{\resizebox{!}{8pt}{\ATexttclr{MOSAIC}}}
\newcommand{\sMOSAIC}{\resizebox{!}{6pt}{\ATexttclr{MOSAIC}}}


\newcommand{\mMOSAIC}{{\color{drp}{\ATexttclr{\normalsize{MOSAIC}}}}}

\newcommand{\lmMOSAIC}{\ATexttclr{\Large{M}\normalsize{OSAIC}}}

\newcommand{\MOSAICVM}{\mMOSAIC-\AcronymTextNC{VM}}
\newcommand{\MOSAICSR}{\resizebox{!}{8pt}{%
\mMOSAIC-\AcronymTextNC{\textcolor{drp!25!black}{SR}}}}

\newcommand{\MOSAICSD}{\resizebox{!}{8pt}{%
\mMOSAIC-\AcronymTextNC{\textcolor{drp!25!black}{SD}}}}

\newcommand{\MOSAICGIS}{\resizebox{!}{8pt}{%
\mMOSAIC-\AcronymTextNC{\textcolor{drp!25!black}{GIS}}}}


\newcommand{\lMOSAICSR}{\resizebox{!}{7.5pt}{%
\lmMOSAIC-\AcronymTextNC{\textcolor{drp!25!black}{SR}}}}

\newcommand{\lMOSAICSD}{\resizebox{!}{8pt}{%
\lmMOSAIC-\AcronymTextNC{\textcolor{drp!25!black}{SD}}}}


\newcommand{\lsMOSAICSR}{\resizebox{!}{9.5pt}{%
\lMOSAIC-}\resizebox{!}{8pt}{\AcronymTextNC{\textcolor{drp!45!black}{SR}}}}

%\newcommand{\lMOSAICSR}{\resizebox{!}{8pt}{%
%\mMOSAIC-\AcronymTextNC{SR}}}


\newcommand{\sMOSAICVM}{\resizebox{!}{8pt}{\MOSAICVM}}


\newcommand{\LDOM}{\resizebox{!}{8pt}{\AcronymText{LDOM}}}
\newcommand{\Cnineteen}{\resizebox{!}{8pt}{\AcronymText{CORD-19}}}

\newcommand{\lCnineteen}{\resizebox{!}{7.5pt}{\AcronymText{CORD-19}}}


\newcommand{\MOL}{\resizebox{!}{8pt}{\AcronymText{MOL}}}

\newcommand{\ACL}{\resizebox{!}{8pt}{\AcronymText{ACL}}}

\newcommand{\LXCR}{\resizebox{!}{8pt}{\AcronymText{LXCR}}}
\newcommand{\lLXCR}{\resizebox{!}{8.5pt}{\AcronymText{LXCR}}}
\newcommand{\lsLXCR}{\resizebox{!}{9.5pt}{\AcronymText{LXCR}}}

%\newcommand{\lMOSAIC}{{\color{drp}{\lAcronymTextNC{M}{OSAIC}}}}
\newcommand{\lfMOSAIC}{\resizebox{!}{8pt}{{\color{drp}{\lAcronymTextNC{M}{OSAIC}}}}}

\newcommand{\Mosaic}{\resizebox{!}{8pt}{\MOSAIC}}
\newcommand{\MosaicPortal}{{\color{drp}{\AcronymTextNC{MOSAIC Portal}}}}

\newcommand{\RnD}{\resizebox{!}{8pt}{\AcronymText{R\&D}}}

\newcommand{\DMS}{\resizebox{!}{8pt}{\AcronymText{DMS}}}

\newcommand{\sDMS}{\resizebox{!}{6pt}{\AcronymText{DMS}}}

\newcommand{\AXFI}{\resizebox{!}{8pt}{\ATexttclr{AXF%
\hspace{-2pt}{\fontfamily{qcr}\fontseries{b}\selectfont{}\Large\textit{i}}%
}}}

\newcommand{\AXFt}{\resizebox{!}{8pt}{\ATexttclr{AXF}}\resizebox{!}{8pt}{%
\hspace{1pt}{\fontfamily{lmdh}\fontseries{b}\selectfont{}\LARGE{\ATexttclr{t}}}%
}}}

\newcommand{\MIBBI}{\resizebox{!}{8pt}{\AcronymText{MIBBI}}}

\newcommand{\JVM}{\resizebox{!}{8pt}{\AcronymText{JVM}}}
\newcommand{\ECL}{\resizebox{!}{8pt}{\AcronymText{ECL}}}

\newcommand{\ChaiScript}{\resizebox{!}{8pt}{\AcronymText{ChaiScript}}}

\newcommand{\TCP}{\resizebox{!}{8pt}{\AcronymText{TCP}}}

\newcommand{\lQt}{\resizebox{!}{8.5pt}{\AcronymText{Qt}}}
\newcommand{\QtCpp}{\resizebox{!}{8.5pt}{\AcronymText{Qt/C++}}}
\newcommand{\Qt}{\resizebox{!}{8pt}{\AcronymText{Qt}}}

\newcommand{\QtSQL}{\resizebox{!}{8pt}{\AcronymText{QtSQL}}}

\newcommand{\HTML}{\resizebox{!}{8pt}{\AcronymText{HTML}}}
\newcommand{\PDF}{\resizebox{!}{8pt}{\AcronymText{PDF}}}

\newcommand{\sPDF}{\resizebox{!}{6pt}{\AcronymText{PDF}}}
\newcommand{\sGUI}{\resizebox{!}{6pt}{\AcronymText{GUI}}}
\newcommand{\sIDE}{\resizebox{!}{6pt}{\AcronymText{IDE}}}
\newcommand{\sBioCoder}{\resizebox{!}{6pt}{\AcronymText{BioCoder}}}
\newcommand{\sMIBBI}{\resizebox{!}{6pt}{\AcronymText{MIBBI}}}
\newcommand{\sPandore}{\resizebox{!}{6pt}{\AcronymText{Pandore}}}


\newcommand{\R}{\resizebox{!}{8pt}{\AcronymText{R}}}
\newcommand{\SciXML}{\resizebox{!}{8pt}{\AcronymText{SciXML}}}

\newcommand{\SciData}{\resizebox{!}{8pt}{\AcronymText{SciData}}}


\newcommand{\MPF}{\resizebox{!}{8pt}{\ATexttclr{MPF}}}


\newcommand{\lGRE}{\resizebox{!}{7.5pt}{\AcronymText{GRE}}}

\newcommand{\p}[1]{

\vspace{.7em}#1}

\newcommand{\q}[1]{{\fontfamily{qcr}\selectfont ``}#1{\fontfamily{qcr}\selectfont ''}} 

%\newcommand{\deconum}[1]{{\textcircled{#1}}}

\renewcommand{\thesection}{\protect\hspace{-1.5em}}
%\renewcommand{\thesection}{\protect\mbox{\deconum{\Roman{section}}}}
\renewcommand{\thesubsection}{\protect\hspace{-1em}}

\newcommand{\llMOSAIC}{\mbox{{\LARGE MOSAIC}}}
%\newcommand{\lfMOSAIC}{\mbox{M\small{OSAIC}}}

\newcommand{\llMosaic}{\llMOSAIC}
\newcommand{\lMosaic}{\lMOSAIC}
\newcommand{\lfMosaic}{\lfMOSAIC}

%\newcommand{\dsC}{}

\newcommand{\textds}[1]{{\fontfamily{lmdh}\selectfont{%
\raisebox{-1pt}{#1}}}}

\newcommand{\ltextds}[1]{{\fontfamily{lmdh}\fontsize{12}{11}\selectfont{%
\raisebox{-1pt}{#1}}}}

\newcommand{\dsC}{{\textds{ds}{\fontfamily{qhv}\selectfont \raisebox{-1pt}{C}}}}
\newcommand{\ldsC}{{\textds{ds}{\fontfamily{qhv}\selectfont \raisebox{-1pt}{C}}}}

\newcommand{\MdsX}{\resizebox{!}{8pt}{\ATexttclr{\raisebox{-1pt}{{\fontfamily{lmdh}\selectfont M}}\fontfamily{qhv}\selectfont dsX}}}
\newcommand{\lsMdsX}{\resizebox{!}{10.5pt}{\ATexttclr{\raisebox{-1pt}{{\fontfamily{lmdh}\selectfont M}}\fontfamily{qhv}\selectfont dsX}}}
\newcommand{\lMdsX}{\resizebox{!}{9.5pt}{\ATexttclr{\raisebox{-1pt}{{\fontfamily{lmdh}\selectfont M}}\fontfamily{qhv}\selectfont dsX}}}


\newcommand{\llWC}{\mbox{{\LARGE WhiteCharmDB}}}

\newcommand{\llwh}{\mbox{{\LARGE White}}}
\newcommand{\llch}{\mbox{{\LARGE CharmDB}}}

\usepackage{enumitem}
%\usepackage{listings}

\colorlet{dsl}{purple!20!brown}
\colorlet{dslr}{dsl!50!blue}

\setlist[description]{%
  topsep=11pt,
  labelsep=22pt, leftmargin=6pt,
  itemsep=8pt,               % space between items
  %font={\bfseries\sffamily}, % set the label font
  font=\normalfont\bfseries\color{dslr!50!black}, % if colour is needed
}

\setlist[enumerate]{%
  topsep=3pt,               % space before start / after end of list
  itemsep=-2pt,               % space between items
  font={\bfseries\sffamily}, % set the label font
%  font={\bfseries\sffamily\color{red}}, % if colour is needed
}

%\usepackage{tcolorbox}

\newcommand{\slead}[1]{%
\noindent{\raisebox{2pt}{\relscale{1.15}{{{%
\fcolorbox{logoCyan!50!black}{logoGreen!5}{#1}
}}}}}\hspace{.5em}}


\let\OldLaTeX\LaTeX

\renewcommand{\LaTeX}{\resizebox{!}{8pt}{\color{orr!35!black}{\OldLaTeX}}}
\renewcommand{\sLaTeX}{\resizebox{!}{6pt}{\color{orr!35!black}{\OldLaTeX}}}

\let\OldTeX\TeX

\renewcommand{\TeX}{\resizebox{!}{8pt}{\color{orr!35!black}{\OldTeX}}}


\newcommand{\LargeLaTeX}{\resizebox{!}{8.5pt}{\color{orr!35!black}{\OldLaTeX}}}

\setlength\parindent{0pt}
%\setlength\parindent{24pt}
%
\let\inodot\i
\renewcommand{\i}[1]{\textit{#1}}

\newcommand{\mdash}{---}




\newcommand{\lun}[1]{\raisebox{-4pt}{\fontfamily{qcr}\selectfont{%
\LARGE{\textbf{\textcolor{tcolor}{#1}}}}}\vspace{-2pt}}

\newcommand{\inditem}{\itemindent10pt\item}

\usepackage{soul}

\definecolor{hlcolor}{RGB}{114, 54, 203}
\colorlet{hlcol}{hlcolor!35}
\sethlcolor{hlcol}

\makeatletter
\def\SOUL@hlpreamble{%
	\setul{}{3ex}%         !!!change this value!!! default is 2.5ex
	\let\SOUL@stcolor\SOUL@hlcolor
	\SOUL@stpreamble
}
\makeatother

\usepackage{scrextend}
%\vspace*{3em}
\newenvironment{mldescription}{\vspace{1em}%
  \begin{addmargin}[4pt]{1em}
    \setlength{\parindent}{-1em}%
    \newcommand*{\mlitem}[1][]{\vspace{5pt}\par\medskip%
%\colorbox{hlcolor}{\textbf{##1}}\quad}\indent
\hl{ \textbf{##1} }\quad}\indent
}{%
  \end{addmargin}
  \medskip
}

\usepackage{marginnote}

\newcommand{\mnote}[1]{%
\vspace*{-2em}
\reversemarginpar
\raisebox{1em}{\marginnote{\parbox{4em}{%
\begin{mdframed}[innerleftmargin=4pt,
	innerrightmargin=1pt,innertopmargin=1pt,
	linecolor=red!20!cyan,userdefinedwidth=4em,
	topline=false,
	rightline=false]
{{\fontfamily{ppl}\fontsize{12}{0}\selectfont
		\textit{#1}}}
\end{mdframed}}
	}[3em]}}

\newcommand{\mnotel}[1]{%
\vspace*{-2em}
\reversemarginpar
\raisebox{-4em}{\marginnote{\parbox{4em}{%
\begin{mdframed}[innerleftmargin=4pt,
	innerrightmargin=1pt,innertopmargin=1pt,
	linecolor=red!20!cyan,userdefinedwidth=4em,
	topline=false,
	rightline=false]
{{\fontfamily{ppl}\fontsize{12}{0}\selectfont
		\textit{#1}}}
\end{mdframed}}
	}[3em]}}

\newcommand{\mnoteh}[3]{%
	\vspace*{#1}
	\reversemarginpar
	\raisebox{#2}{\marginnote{\parbox{4em}{%
				\begin{mdframed}[innerleftmargin=4pt,
					innerrightmargin=1pt,innertopmargin=1pt,
					linecolor=red!20!cyan,userdefinedwidth=4em,
					topline=false,
					rightline=false]
					{{\fontfamily{ppl}\fontsize{12}{0}\selectfont
							\textit{#3}}}
				\end{mdframed}}
			}[3em]}}


\newcommand{\mnoteb}[1]{%
	\vspace*{1em}
	\reversemarginpar
	\raisebox{1em}{\marginnote{\parbox{4em}{%
				\begin{mdframed}[innerleftmargin=4pt,
					innerrightmargin=1pt,innertopmargin=1pt,
					linecolor=red!20!cyan,userdefinedwidth=4em,
					topline=false,
					rightline=false]
					{{\fontfamily{ppl}\fontsize{12}{0}\selectfont
							\textit{#1}}}
				\end{mdframed}}
			}[3em]}}
	
\usepackage{wrapfig}

\usetikzlibrary{arrows, decorations.markings}
\usetikzlibrary{shapes.arrows}

\newcommand{\colorarr}[8]{
	\draw [#1, draw=#2,draw opacity=#3,
	fill=#4,fill opacity=#5,line width=#6] (#7) to (#8);
}

\definecolor{blGreen}{rgb}{.2,.7,.3}
\definecolor{darkRed}{rgb}{.2,.0,.1}

\definecolor{postLinkColor}{rgb}{.5,.5,.1}

\definecolor{fcBoxColor}{rgb}{.8,.6,.3}


\newcommand{\curicon}[2]{%
	\node at (#1,#2) [
	draw=black,
	%minimum width=2ex,
	inner sep=.7pt,
	fill=white,
	single arrow,
	single arrow head extend=3pt,
	single arrow head indent=1.5pt,
	single arrow tip angle=45,
	line join=bevel,
	minimum height=4.6mm,
	rotate=115
	] {};
}

\makeatletter
\def\@cite#1#2{[\textbf{#1\if@tempswa , #2\fi}]}
\def\@biblabel#1{[\textbf{#1}]}
\makeatother


%\let\origref\ref
%\renewcommand{\ref}[1]{{\LARGE #1}}

%\def\ref#1{\textbf{\origref{{\LARGE #1}}}}

%\setlength{\footnotesep}{-8pt}
\usepackage[originalparameters]{ragged2e}


\renewcommand{\thefootnote}{\textcolor{logoGreen!80!logoBlue}{{\fontfamily{qcr}\fontseries{b}\fontsize{10}{4}\selectfont\arabic{footnote}}}}


\newcommand{\LVee}{{\colorbox{cyan!40!yellow}{\textcolor{red!70!navy}{\textbf{\LARGE$\vee$}}}}}
\newcommand{\LWedge}{{\colorbox{cyan!40!yellow}{\textcolor{red!70!navy}{\textbf{\LARGE$\wedge$}}}}}

\renewcommand{\LVee}{}
\renewcommand{\LWedge}{}


\urlstyle{same}

\newcommand{\lpageref}[1]{\raisebox{-2pt}{\pageref{#1}}}

\usepackage[preserveurlmacro]{breakurl}

%
\newcommand{\bhref}[1]{\href{#1}{\burl{#1}}}

%\newcommand{\bhref}[1]{} 

%\setmainfont{QTChanceryType}

%\usepackage{array}
%\newcolumntype{RR}[1]{>{\raggedright\let\newline\\\arraybackslash\hspace{0pt}}m{#1}}

\usepackage{caption} 
\captionsetup[table]{skip=2pt}

\newcommand{\cframedboxx}[3]{\begin{mdframed}[linecolor=rb!85!red,linewidth=#2]\hspace{#1}#3\end{mdframed}}

%\usepackage{aurical}
% \Fontauri

%\usepackage{LibreBodoni}
%\usepackage{fontspec}

%\renewcommand{\rmdefault}{phv}

\renewcommand{\rmdefault}{ppl}
\renewcommand{\familydefault}{\rmdefault}

%\usepackage[english]{babel}
\hyphenation{hyper-node}
\hyphenation{hypo-node}
\hyphenation{hyper-nodes}
\hyphenation{hypo-nodes}

\begin{document}

\setlength{\skip\footins}{18pt}	
	
{\linespread{1.25}\selectfont

\vspace*{1em}

\begin{center}
%{\relscale{1.2}{\fontfamily{qcr}\fontseries{b}\selectfont 
%{\colorbox{black}{\color{blue}{\llWC{} Database Engine \\and 
%\llMOSAIC{} Native Application Toolkit}}}}}

\colorlet{ctmp}{logoPeach!20!gray}
\colorlet{ctmpp}{ctmp!90!yellow}
\colorlet{ctmppp}{ctmpp!50!black}
\colorlet{ctmpppp}{ctmppp!90!logoRed}
\colorlet{ctmcyan}{ctmpppp!70!cyan}

\colorlet{ctmppppy}{ctmppp!60!orange}

%{\colorbox{darkBlGreen!30!darkRed}{%
\begin{tcolorbox}
[
%%enhanced,
%%frame hidden,
%interior hidden
arc=2pt,outer arc=0pt,
enhanced jigsaw,
width=\textwidth,
colback=ctmppppy!40,
%colback=ctmcyan!50,
colframe=logoRed!30!darkRed,
drop shadow=logoPurple!50!darkRed,
%boxsep=0pt,
%left=0pt,
%right=0pt,
%top=2pt,
]
%\hspace{22pt}
\begin{minipage}{\textwidth}	
\begin{center}	
{\setlength{\fboxsep}{32pt}
	\relscale{1.1}{{\fontfamily{qcr}\fontseries{b}\selectfont%
{\Large{The HGDM Protocol  \\

\vspace{-5pt}(Hypergraph Data Modeling)}\\

\vspace{5pt}
\textit{\large{Unifying Hypergraph Database Architecture 
with Scientific Research-Oriented Data Sharing}} }}}}
\end{center}
\end{minipage}
\end{tcolorbox}
\end{center}

\vspace*{-.5em}

%\textcolor{darkRed}{\hrule}


\section{Introduction}

%\p{With the rise of digital health records, there has been a parallel rise in standards for sharing medical and bioinformatic data between institutions.  Sometimes, such data sharing is directly involved in clinical treatments, 
% --- %
%for example, when an emergency room patient's medical history is obtained from their primary-care physician.  In these immediate scenarios, data communications are governed by tried-and-tested standards and technology, with little room for innovative or experimental solutions.  However, medical data sharing also covers a spectrum of use-cases which are not directly implicated in \q{real time} patient care --- cases such as sharing observational studies, case series, clinical trial results, evaluating patient outcomes and cost/benefit analysis for different treatment options, and similar kinds of clinical, interventional, diagnostic, epidemiological, or translational research.   In these scenarios, there are still guidelines and requirements in effect --- e.g., patient data needs to be suitably anonymized for research purposes; and the reporting for clinical trials needs to accurately reflect trial design.  Nonetheless, even factoring in such parameters, there is considerable flexibility in how research-oriented medical data can be structured, modeled, and communicated.}

\p{Recent years have seen a proliferation of research in bioinformatic systems or \q{biomedical knowledge engineering,} yielding new 
insights into optimal database design and data-mining strategies.\footnote{See 
e.g. 
(among many works that could be cited) \bhref{https://www.ncbi.nlm.nih.gov/pmc/articles/PMC3555311/} or 
\bhref{http://nemo.nic.uoregon.edu/wiki/images/8/88/978-3-938793-98-5_Munn_Ontology.pdf}.}
What has been emphasized as a consequence 
this research is that clinically and scientifically important 
information is intrinsically \textit{heterogeneous}, spanning a diversity of formats and 
subject areas (including bioimaging, genomics, epidemiology, biochemistry, sociodemographics, 
immunology, clinical outcomes, and patient \q{quality of life} or related patient-centered 
evaluations).  
Medical data sharing --- in addition to the quotidion exchange of 
patient histories for \q{real time} patient care (which is governed 
by strict regulations that inhibit novel, experimental technologies) 
--- encompasses use-cases such as sharing observational studies, case series, clinical trial results, evaluating patient outcomes and cost/benefit analysis for different treatment options, and similar kinds of clinical, interventional, diagnostic, epidemiological, or translational research.
In this research-oriented setting, there are still guidelines and requirements in effect (e.g., patient data must be suitably anonymized, 
and the reporting for clinical trials must accurately reflect trial design); but there is nonetheless considerable flexibility in how research-oriented medical data can be structured, modeled, and communicated.}

\p{In order to integrate all available information relevant 
for a given biomedical research/scientific project --- both within single
institutions and across multiple institutions --- technology must be 
flexible enough to adapt to different data formats and profiles.  This has led to 
two related (but philosophically distinct) paradigms: first, large-scale 
adoption of \q{Semantic Web} techniques oriented to knowledge-acquisition and 
Artificial Intelligence; and, second, the emergence of 
\q{hypergraph} database engines, which aspire to unify many different 
database architectures into a multi-purpose totality.}

\p{Both 
of these approaches offer the promise of a more tightly integrated 
biomedical information ecosystem: data and files reflecting many 
disparate scientific paradigms accessible through one 
platform that is shared across multiple institutions.  With robust 
multi-institutional data sharing in place, technology  
can start to redress shortcomings in existing biomedical research 
practices.  To wit, more extensive documentation of patient 
outcomes can assess the effectiveness of treatments within a broad 
spectrum of quality-of-life concerns (not just noting short-term clinical 
consequences of treatment modalities); and the 
merging of observational studies may provide a statistically 
diversified array of interventions and demographic/diagnostic profiles 
to compare/contrast, which can compensate for a paucity of clinical trials 
in emergent subject areas (Covid-19 being a case in point).}

\p{Despite these potential benefits, the large majority of biomedical 
data continues to be stored via conventional database 
and/or filesystem technologies.  Moreover, there are nontrivial 
differences between how each hypergraph database engine is designed, 
as well as between hypergraph database architecture in general and the 
Semantic Web.  These differences present a hindrance to 
fully leveraging the data-modeling innovations intrinsic to the 
hypergraph and Semantic Web paradigms, particularly in the context 
of multi-institution data integration --- even though 
such integration is a primary motivation for the paradigms 
as outlined above.} 

\section{Creating a Truly Unified Data-Sharing Framework}
\p{Against this backdrop, Linguistic Technology Systems 
proposes a new \q{Hypergraph Data Modeling} 
(\HGDM{}) protocol, whose goal is to unify the principal structures 
of multiple hypergraph and Semantic Web frameworks, yielding a 
truly general-purpose data-sharing system.  The \HGDM{} protocol is 
focused on research-oriented communications between biomedical 
institutions.\footnote{There are actually no technical details in \sHGDM{} 
which restrict it to a clinical, health-care, or bioinformatic context, 
so implementations could potentially be beneficial in other sectors.}  
\lHGDM{} aims to compensate for limited industry adoption of 
hypergraph-database and (to some extent) Semantic Web technology 
--- not by deploying hypergraph database engines in place of 
legacy systems, but by facilitating the implementation of software 
layers which allow information spaces to mimic the behavior 
of hypergraph database engines with respect to multisite data 
sharing.  That is, \HGDM{} \textit{first and foremost} 
defines a protocol whose canonical 
use-case is one of remote applications accessing hypergraph 
database content via a semantically-constrained network.\footnote{According 
to semantics defined by the protocol: hypergraph structures 
are embraced as representations for content 
shared between applications, whether or not the 
communicating end-points adopt hypergraphs natively.}  However, the protocol 
only stipulates a contract between server and client software 
\textit{applications}, without explicit requirements on the server's 
physical data storage.  An \HGDM{} server could therefore be a 
software layer adapting filesystems or database instances of different 
kinds, and not only hypergraphs.  This need not entail  
programming encompassing the entirety of an institution's local data; 
instead, developers could selectively curate information fitting 
some constrained criteria, as part of a targeted data-sharing project.}

\p{In short, \HGDM{} governs data-sharing activity 
coordinated between three different software entities 
(\textit{see} Figure~\ref{fig:s1}).  That is, 
a \q{server} (which is not necessarily a 
\TCP{} server in the conventional internet sense) responds to 
requests for information against some data space (potentially 
but not necessarily a hypergraph database instance).  After 
pulling relevant content from this data space, the server 
serializes the response data and sends it  
to an intermediate software component, which parses the 
response into a hypergraph data structure.  \lHGDM{} introduces 
a novel \q{Hypergraph Exchange Format} (\HGXF{}) for 
encoding/serializing hypergraph data.  %Then, 
The intermediate 
software, as such, then receives \HGXF{}-encoded data and parses this 
content to create an \q{infoset,} similar in purpose to an 
\XML{} post-processing infoset.  \lHGXF{} utilizes 
an extended version of the \TAGML{} markup 
language\footnote{See \bhref{https://pdfs.semanticscholar.org/bbd9/9215f6bed393c9274f8e0642bebf42d8f633.pdf}.} for a hypergraph-based 
document syntax, and derives a hypergraph-based semantics 
from the \HGDM{} infoset protocol.   The information contained 
in this infoset is then translated into \q{application-level} 
objects --- viz., instances of data types which are natively 
recognized or implemented by the client application that 
initially formulated the \HGDM{} request.  Accordingly, the third 
end-point for \HGDM{} components would be a library embedded 
in applications, one which implements functionality to initiate 
\HGDM{} requests and can 
translate \HGXF{} response-data 
infosets into application-specific data.} 

\p{The \HGDM{} protocol establishes guidelines or requirements 
for each of these three data-sharing endpoints/layers.  The computational 
units within these components will be generally refered to, below, as 
\q{objects,} so that we have \textit{server objects}, 
\textit{infoset objects}, and \textit{client objects}.  
\lHGDM{} regulates communications between server, infoset, and 
client components by stipulating or recommending that 
these objects (and the data types of which they are 
instances) support specific methods and procedures, 
many of which are oriented toward sustaining hypergraph-based 
data models across each phase in the data-sharing 
process.\footnote{According to the goal of representing, 
at a minimum, the important structural contributions 
of major hypergraph database engines, one question then 
arises as to how to identify such engines as a subset 
of overall database technology.  Considering a range 
of commercial and academic projects, \sHGDM{} draws 
from (in particular) \sHyperGraphDB{}, \sGraknai{}, AtomSpace 
(see \bhref{https://aiatadams.files.wordpress.com/2016/01/cogprime-overview.pdf}), 
\sLMNtal{} 
(see \bhref{https://waseda.pure.elsevier.com/en/publications/lmntal-a-language-model-with-links-and-membranes}), 
\sNeo{} (see \bhref{http://www.we-yun.com/doc/books/Neo4j\%20in\%20Action.pdf} 
or \bhref{https://academic.oup.com/nar/article/48/D1/D344/5580911}), 
and \sWhiteDB{} (see \bhref{http://whitedb.org/} 
pr \bhref{https://arxiv.org/pdf/1910.09017.pdf}), as well as several runtime 
(non-persistent) hypergraph libraries.}}

\begin{figure}
\begin{tikzpicture}
\usetikzlibrary{shapes}
\usetikzlibrary{positioning}
%\node[inner sep=0pt] (tbl) at (0,0)

\node [anchor=west,bottom color=yellow!30!blue,top color=pink!70!green, 
inner sep=3, shading angle=50, text width=2.5cm]
  (server) at (0, 0) {\vspace{-8pt}%  %{\color{rb!85!red}{
{\cframedboxx{0mm}{2pt}{\large\raggedright\textbf{Server}}}};

\node [%ellipse,draw
ellipse, draw, text width=2cm
] (hdb) at (1,-2) {Hypergraph\\

Database};

\draw [>=latex, ->, 
draw=fcBoxColor!40!purple,draw opacity=1,shorten >=4pt,
	fill=blGreen!30!red,fill opacity=1,line width=1mm
] (hdb.north east) to (server.south east);

 \node[regular polygon, regular polygon sides=5,
        minimum size=4.5cm, draw=blue!20!black,dotted,line width=.5mm, 
        fill=blue!24, fill opacity=.5, xscale=1.8,
        outer sep=0pt] at (1.05,-5.2) (hex){};


\node [ellipse,draw,text width=2.5cm] (cdb) at (1,-4.2) {Conventional\\

Database};
\node [ellipse,draw,text width=2cm] (rdf) at (1,-6) {RDF Data\\

Space};

%\node [draw, regular polygon, regular polygon sides=6, minimum width=2cm] 
%(ad) at (5,-3) {}; 

%\draw[dotted] (0,0) rectangle (5,-4);


\node [draw=red!20!black,fill=red!30] 
(adt) at (4.5,-4) {Adapters}; 

\node [above=3cm of adt.north] 
(adtup) {}; 


\draw [>=latex, ->, 
draw=purple!40!orange,draw opacity=1,
	fill opacity=1,line width=1mm
] (adt.north) to (adtup.south);


\node [anchor=west, bottom color=blGreen!60!yellow,top color=pink!70!orange,
inner sep=3, shading angle=150, text width=2.5cm]
  (infoset) at (7, 0) {\vspace{-8pt}%  %{\color{rb!85!red}{
{\cframedboxx{0mm}{2pt}{\large\raggedright\textbf{Infoset}}}};

\draw [>=latex, ->, 
draw=fcBoxColor!60!black,draw opacity=1,
	fill=blGreen!30!red,fill opacity=1,line width=2mm
] (server.east) to (infoset.west) node [below] 
{\hspace{-5.3cm}%
{\small\textbf{\textcolor{orange!25!black}{HGXF/GTagML}}}};
   
\node [anchor=west,bottom color=purple!30!blue,top color=magenta!70!red, 
inner sep=3, shading angle=250, text width=2.5cm]
  (client) at (11, 0) {\vspace{-8pt}%  %{\color{rb!85!red}{
{\cframedboxx{0mm}{2pt}{\large\raggedright\textbf{Client}}}};

\draw [>=latex, ->, 
draw=purple!40!orange,draw opacity=1,
	fill opacity=1,line width=1mm
] (infoset.east) -- (client.west) node [midway] (larr) {};

\node[below = 2cm of larr, text width=5cm, inner sep=14pt,fill,
fill=magenta!20,
draw=magenta!40!black,dotted,line width=1mm] (exp) {Server libraries 
need to convert database \makebox{structures} to 
hypergraphs; client libraries need to convert 
hypergraph-based infosets to application-level 
data types.};

\draw [>=latex, ->, 
draw=purple!40!orange,draw opacity=1,
	fill opacity=1,line width=1mm
] (larr.south) to (exp.north);


\end{tikzpicture}
\caption{Outline of HGDM Server, Infoset, and Client Layers}
\label{fig:s1}
\end{figure}

\p{The following sections will outline some of the 
procedures standardized for the three protocol layers (server, 
infoset, and client) and will also address some details 
concerning how data within the hypergraphs should be encoded.}


\section{Protocol Outline for Server, Infoset, and Client Objects}

\p{As indicated in the Introduction, there are three different 
contexts or stages where \HGDM{} components would be implemented: 
\textit{servers,} \textit{infosets} 
(objects parsed directly from serializations), and \textit{client} applications 
(values used directly by user-facing software components, 
which are initialized from objects of the prior two kinds).  
Server objects may be directly obtained from a 
hypergraph database, or alternatively from 
\q{adapters} (components that access non-hypergraph 
databases in a manner which emulates hypergraphs).  
The server, infoset, and client components 
will sometimes be referred to as \q{layers} or 
\q{endpoints} (considering that each is potentially 
the endpoint of a network communication governed by 
\HGDM{}).  Objects within each of these three layers can be 
conceptualized as \textit{hypernodes}, meaning that 
they have both internal structure and labeled connections 
with other objects; although these hypernode structures 
are completely formalized only at the infoset layer.}

\p{The material below provides a summary of \HGDM{} 
by enumerating certain procedures within the protocol specifications 
for each layer.  The procedures are presented as grouped according to 
the layer where they are most directly relevant; however, objects 
may also implement procedures from other layers.  
Of the three endpoints, the intermediate \q{infoset} layer is 
the one most strictly regulated by \HGDM{}.  By contrast, 
the server and 
client components have more latitude, recognizing that 
these layers may be implemented in the context of disparate 
application and data-persistence environments.  
Nevertheless, it is consistent with \HGDM{} for the 
server and client layers to implement procedures 
and data structures mimicking the hypergraph constructions 
of the infoset layer.}

\subsection[Server-Layer Procedures]{\textit{Server-Layer Procedures}}
\p{This subsection will review procedures associated 
with server objects.  Some of these procedures are 
associated with specific database architectures, so 
therefore they may not be implemented in every \HGDM{} 
server-layer component --- although providing 
capabilities concordant with multiple database 
architectures allows components to serve as adapters 
integrating various databases into an \HGDM{} 
network; as such, these procedures can be seen 
as a guideline for integration across heterogeneous 
information systems. 

\begin{description}

\item[\procitem{get\_binary\_encoding()}]  Provides either a 
raw byte array or base-32 encoded byte array, which 
is a binary serialization of a 
server object.  This encoding should be reversible, 
in that a procedure exists to recreate the serialized 
object through the byte array.

\descindent{}  \HGDM{} recommends a base-32 encoding scheme 
designed to be compatible with \textbf{QString} lexical casts 
(note that this encoding differs from other base-32 systems 
commonly used).\footnote{\textbf{QString} refers to the 
principle character string/text class of the \sQt{} \sCpp{} 
application development framework.  The encoding 
is structured so that base-32 strings can be mapped 
to decimal or hexadecimal numbers via \textbf{QString} 
methods such as \textbf{.toInt(...)}.}  Base-32 streams 
in \HGDM{} are case-insensitive, little-endian, 
and use digits \textbf{0}-\textbf{9} and letters 
\textbf{a}-\textbf{v}.  
The additional alphanumeric characters \textbf{w}, \textbf{x}, 
\textbf{y}, \textbf{z}, and 
underscore (\q{\textbf{\_}}) are employed as end-of-stream separators/markers, 
indicating 0-4 padding bytes.

\item[\procitem{get\_decoder\_class()}]  Returns the name of a 
class whose objects are binary-encoded (and consequently decoded) 
by \dprocitem{get\_binary\_encoding}.  

\item[\procitem{get\_decoder\_procedure()}]  Returns/provides  
a data structure through which one obtains a procedure 
constructing decoder-class objects from byte arrays.  

\item[\procitem{get\_indexes()}]  Returns a list 
of fields within a server object (construed as a 
hypernode) which are indexed for querying, meaning 
that one can identify individual objects based 
on the value for that field (and also potentially 
sort a collection of objects via that field).

\descindent{}  Borrowing terminology from \HyperGraphDB{}, 
\HGDM{} refers to the smaller units within hypernodes 
as \q{projections.}\footnote{See \bhref{http://www.hypergraphdb.org/docs/hypergraphdb.pdf}.}  At the infoset layer, all 
objects are fully-formed hypernodes wherein all 
data contained by each object is available 
within \q{hyponodes} --- the nodes inside hypernodes; 
the hypernode is said to \textit{project onto} each 
of its hyponodes.  In the server-layer context, 
using a design analogous to \HyperGraphDB{}, 
only some of the relevant information encoded via an 
object may be available via projections 
(the remainder is binary-encoded and therefore not 
accessible to query evaluators outside the object itself).  
In general, the data fields which have projections within 
server objects are those that have indices.    

\item[\procitem{get\_table\_structure()}]  Returns/provides 
a description of how tabular data can be mapped to a 
collection of hypernodes.

\descindent{}  This procedure is associated with adapters 
for \SQL{}-style databases.  In general, data layouts which 
are technically distinct from the hypergraph perspective 
may all be embodied by relational tables in the \SQL{} 
paradigm, obscuring the hypergraph-structural variations.   
This procedure compensates for that ambiguity 
by providing an outline of how a given table should be 
mapped to hypergraph formations.  In the simplest mapping, 
each record becomes one hypernode, and each column 
corresponds to one projection.  However, more 
complex mappings are also possible.  For example, 
one column in the table might hold map keys, while the 
other columns are aggregated into hypernodes, so 
that the overall data structure is treated as a map 
(associative array) whose values are hypernodes.  
Or, columns may be aggregated into two sets of hypernodes, 
with the overall data structure treated as a graph 
asserting relations between pairs of hypernodes 
selected from those sets.

\item[\procitem{get\_rdf\_mapping()}]  Returns/provides 
a data structure encapsulating how \RDF{}-style 
graph data may be interpreted in a hypergraph context.  
This procedure is associated in particular 
with adapters working with Semantic Web services.

\descindent{}  In general, mapping 
Semantic Web or \RDF{} ontologies to hypergraphs 
is non-trivial (certainly too complex for an 
automated process or a single algorithm), 
because there are numerous sorts 
of inter-node relations in a hypergraph 
context which may all be embodied as generic 
graph-edges in \RDF{}.  Therefore, adapter schemes 
for \RDF{}-style data should model how \RDF{} 
relations translate to hypergraph constructions --- including 
hypernode-to-hypernode connections, hypernode-to-hyponode 
projections, hyponode-to-hypernode proxies, and various 
combinations of these links.  

\item[\procitem{get\_file\_type\_description()},  
\procitem{get\_file\_type\_class()}, 
\procitem{get\_file\_type\_decoder()}]  These procedures 
are principally intended for use when adapters 
encapsulate access to a local filesystem, or individual files 
therein, in accordance with a hypergraph protocol.  


\end{description}


}

\subsection[Infoset-Object Procedures]{\textit{Infoset-Object Procedures}}

\p{In \HGDM{}, infoset objects are hypernodes 
containing sequences of individual parts 
(hyponodes), and also connected to other hypernodes 
via directed edges, or \q{connections.}  Each 
connection has a \q{connector} object which is itself 
a hypernode, but is not (directly) connected to 
other hypernodes, apart from the source-connector-target 
triple.  All hypernodes and hyponodes have a type, 
and the hypernode type constrains the type of its 
hyponodes.}

\p{In \HGDM{}, hypernodes may have a dynamically changing 
sequence of hyponodes --- the length of this sequence 
may enlarge or contract.  However, new hyponodes must be 
added according to a fixed type pattern.  To be precise, 
all hypernodes have a (possibly empty) list of 
\textit{structure fields}, which are hyponodes each 
of which is associated with a predetermined 
type and a string label.  Each of these fields must have a 
value (perhaps a \q{null} value if such a value exists 
in the hyponode type); and hyponodes cannot be added 
or removed from the structure-field part of a hypernode.  
In addition to (and sequentially following) the structure 
fields, hypernodes may have \textit{array fields}, 
which in the simplest case is zero or more hyponodes all 
with the same type.  Within the span of the array fields, 
hyponodes may be added to or removed from the sequence.  
\lHGDM{} permits array fields to have multiple types, 
but in this case the types must conform to a predictable 
\q{cycle,} meaning that every $n$th hyponode (for some $n$) 
has the same type.  For instance, a map between strings 
and integers can be represented by a length-two cycle 
alternating between string-typed hyponodes and integer-typed 
hyponodes.}

\p{Many of the canonical \HGDM{} procedures in the infoset 
context involve hypernodes, hyponodes, array fields, and 
structure fields.  These include: 


\begin{description}

\item[\procitem{get\_hypernode\_type(...)}, 
\procitem{get\_hyponode\_type(...)}]  Returns a 
name or a data structure representing the corresponding 
node's type.

\descindent{}  In general, each hypernode's type determines 
both the label used to access items in its sequence of hyponodes, 
and also the corresponding hyponode's type (note that only 
structure fields have string labels, however). 
   
\item[\procitem{get\_structure\_field(...)}, 
\procitem{get\_structure\_fields(...)}]  Provides 
access to one or more of a hypernode's projections 
(i.e., its contained hyponodes).  Each hyponode may be 
obtained via an index into the containing hypernode's 
hyponode-sequence, but using a descriptive label 
is more portable and self-explanatory than using raw 
indices.  Variants on these procedures which take 
multiple labels will return or expose multiple 
projections accordingly.

\descindent{}  \lHGDM{} recommends that access to 
projections be provided via a \q{functional} style 
as well as (or in place of) returning a handle to 
hyponodes directly.  This means that the procedures 
have variants which accept a procedural value 
(e.g., a code block) which the called procedure 
will execute (assuming a proper hyponode can be provided), 
passing the relevant hyponode as a value to the 
code passed as an argument.  If multiple hyponodes 
are requested, this code should be called multiple 
times, one for each hyponode.  

\item[\procitem{get\_array\_field(...)}, 
\procitem{get\_array\_fields(...)}]  Provides 
access to one or more hyponodes from the 
\q{cyclic} part of the hypernode.

\descindent{}  Array fields are one-indexed 
(meaning that the field corresponding to the 
number \q{1} is the first field in the cycle, 
and so forth).  Note that the array index 
is not (in general) the same as the index 
in the overall hyponode sequence.

\descindent{}  Even if the primary role of a hypernode 
is to hold a collection of values --- i.e., 
a resizable array whose hyponodes are indexed 
via array fields --- hypernodes may pair these 
arrays with one or more structure fields 
carrying holistic info about the array.  \lHGDM{} 
stipulates that hypernode implementations should 
indicate the length of a hypernode's total hyponode 
sequence; the boundary between the structure fields 
and array fields; and the \q{cycle length,} which 
derives from the recurring type-pattern governing 
how hyponodes may be added as array fields.  
However, depending on how hypernodes encode their
data, the array fields may embody a \q{logical} 
sequence of implicit values spread over multiple 
hyponodes.  In some of these cases the number of distinct 
logical values encoded within a hypernode cannot 
be determined just from sequence- and cycle-length 
data alone.  This is an example of the sort of 
holistic data which might reasonably be stored 
in structure fields preceding an array-field cycle.

\descindent{}  As with structure fields, array-field 
access can be provided through functional variants, 
and through variants which return/expose multiple 
hyponodes.  Variants should be provided which expose 
multiple array fields both via a list of indices 
and via a range of indices, indicating interest 
in each field indexed between the range start and end (inclusive). 

\item[\procitem{get\_field(...)}, 
\procitem{get\_fields(...)}]  These are lower-level 
procedures which expose fields based on raw (zero-indexed) 
sequence numbers.  These procedures do not distinguish 
whether the accessed hyponode is a structure field 
or an array field.

\item[\procitem{get\_role\_projection(...)},
\procitem{get\_conceptual\_role\_structure(...)}]  Exposes 
a hyponode via a description of the \q{role} played 
by that value in a hypernode, which is not (necessarily) 
the same as a structure-field label.

\descindent{}  Following \textbf{Grakn.ai}, \HGDM{} recognizes 
that some hypernodes are representations of 
events or situations which comprise multiple parts or 
\q{actors} playing different roles.\footnote{See \bhref{http://klarman.me/academic-work/resources/MesEtAlCISIS17.pdf}.}  Labeling these roles 
provides an alternative interface for accessing the hyponodes 
onto which the roles project, in the context of a given 
hypernode.  These projections may potentially be array 
fields as well as structure fields, particularly when 
the same role is played by multiple parties.

\descindent{}  \lHGDM{} allows this concept of \q{roles} to 
be generalized to role-aggregates in a manner which 
borrows from Conceptual Space theory as well as conceptual-role 
semantics, which motivates the \dprocitem{get\_conceptual\_role\_structure(...)} 
variant.  Conceptual Space representations are also relevant to the next item.

\vspace*{6pt}
\item[\parbox{.8\textwidth}{\procitem{get\_dimension\_structure(...)}, 
\procitem{get\_domain\_structure(...)},\\ 
\procitem{get\_conceptual\_domain\_structure(...)},
\procitem{get\_hybrid\_conceptual\_structure(...)}}]\

\vspace*{6pt}Returns/provides 
dimensional information about a projection, including numerical/statistical 
detail such as level of measurement (Nominal, Ordinal, Interval, Ratio), 
units of measurement, scale, ranges, or expected distribution 
within a range.  The \HGDM{} semantics of dimensions (and domains, which are 
dimension aggregates) is derived from the Conceptual Space Markup Language 
\CSML{}.\footnote{See {%\scriptsize
\bhref{https://opus.lib.uts.edu.au/bitstream/10453/31756/1/2013007531OK2.pdf},
\bhref{https://www.researchgate.net/publication/221406067_Conceptual_Space_Markup_Language_CSML_Towards_the_cognitive_SemanticWeb},
\bhref{https://www.semanticscholar.org/paper/A-Metric-Conceptual-Space-Algebra-Adams-Raubal/521acbab9658df27acd9f40bba2b9445f75d681c},
\bhref{https://citeseerx.ist.psu.edu/viewdoc/download?doi=10.1.1.302.8106&rep=rep1&type=pdf}, 
\bhref{http://ceur-ws.org/Vol-2090/paper4.pdf},} 
or {\bhref{https://arxiv.org/pdf/1703.08314.pdf}}.}

\descindent{}  In accord with Conceptual Space theory, dimensions may be 
grouped into \textit{domains}, and domains grouped into 
\q{concepts.}  These aggregative structures can be defined or 
indicated via 
\dprocitem{get\_domain\_structure} and  
\dprocitem{get\_conceptual\_domain\_structure}.  \lHGDM{} 
also allows domains to be characterized in terms of 
roles as well as dimensions; as such, \dprocitem{get\_hybrid\_ conceptual\_structure(...)} 
yields a structural articulation which combines the features/details 
of \makebox{\dprocitem{get\_conceptual\_domain\_structure(...)}} 
and \dprocitem{get\_conceptual\_role\_structure(...)}.

\item[\procitem{get\_connection(...)}]  Given a hypernode (the source) and 
a description of a hypernode attached to a hyperedge (the connector), 
returns a hypernode which is the target of the edge whose source 
and connector matches the arguments.

\descindent{}  If multiple edges match, and therefore multiple targets, 
the procedure can either return all matching hypernodes or else 
(assuming it is possible to order the targets) the first target which 
matches.

\descindent{}  \lHGDM{} has a notion of \textit{frames} and of 
\textit{contexts} which might constrain connection matches.  When 
contexts are in effect, certain connections (i.e., connections 
between hypernodes) may only be considered applicable when 
one or more contexts are \q{active.}  Similarly, \q{frames} are 
sets of hypernodes and/or hyperedges, and a certain connection 
may \q{exist} within one frame but not outside it.  When 
frames and/or contexts could be in place which filter 
connections, any \dprocitem{get\_connection(...)} procedure 
should take frames or contexts as parameters, and 
attempt to match edges within the confines of those structures.

\item[\procitem{get\_proxy(...)}]  Returns a 
hypernode which is the target of a hyponode \q{proxy.}  
Proxying means that a hyponode has a \textit{value} 
which designates or points to a hypernode.  \lHGDM{} 
uses proxies as an alternative form of linkage between 
hypernodes --- via one hyponode which holds the proxy 
value --- in contrast to connection-based hyperedges.  

\item[\procitem{consruct\_hypernode\_via\_proxies(...)}]  Given a 
sequence of hypernodes, constructs a new hypernode whose 
projections are proxies to each of those hypernodes, in 
turn.  A variation of this procedure should be one which 
also takes a pre-existing hypernode as a parameter, and 
appends the proxies as array fields to that hypernode.
  
\descindent{}  Note that for each hypernode type, proxies 
to that type represents a distinct type applicable to 
hyponodes.  Hypernodes can only have fields which are 
proxies if those proxy types are part of the hypernode's 
type profile.

\item[\procitem{get\_binary\_proxy(...)}]  The 
notion of \textit{binary proxies} is available in 
an \HGDM{} context if it is possible to encode 
sequences of hyponode values to a compact binary 
representation.  In that case, it is possible to 
indirectly encode the data within a hyponode-set 
as a byte array.  One form of \dprocitem{get\_binary\_proxy} 
should therefore construct a hypernode whose projections 
are translated from such a binary encoding.  Note that 
the resulting hypernode is not considered to be part of 
the graph --- it is not connected to any hypernode 
via either hypernode connections or actual proxies.

\descindent{}  A variation on this construction 
combines the properties of bonafide proxies and 
binary proxies.  In this case, a hyponode contains 
both a proxy value and a binary value compactly 
encoding that proxied hypernode's content.  When 
the proxied hypernode changes, the binary proxy 
is automatically updated.  

\descindent{}  Whether used alone or paired with a 
bonafide proxy, binary-proxy values may be 
part of a union type (i.e., a type which can be 
declared which \textit{either} holds a 
binary proxy \textit{or} some other value, such as an integer).    
    
\item[\procitem{get\_channel\_structure(...)}]  In \HGDM{}, 
a \textit{channel} is in effect a special form of 
frame which aggregates some connectors that share a target.  
Channels may have labels, and be regulated by a 
\textit{channel system} which identifies what sort of 
channels may be applied and whether there are restrictions 
on how they are combined (e.g., whether a non-empty channel --- 
of a given kind --- leading toward one hypernode precludes, or alternatively 
requires, an additional non-empty channel leading to 
the same hypernode).  The \dprocitem{get\_channel\_structure} 
procedure should, given a target node, identify the set of 
source nodes which connect to the target, grouped by channel.

\descindent{}  Note that efficiently finding source nodes 
from target nodes is only possible in general if 
the \HGDM{} software implements \q{reversible} connections, 
meaning that target-to-source relations are stored alongside  
source-to-target.  Whether or not such reverse-connections 
are modeled in general, they should always be provided 
once connections are associated with channels (i.e., reverse 
edges should be represented as a data structure attached 
to channels' target hypernodes).
     
\item[\procitem{get\_hyponode\_set(...)}]  Given a hyperedge, 
returns a sequence of hyponodes comprising all 
hyponodes in the source hypernode and then the target 
hypernode, effectively erasing the boundary between 
the two hypernodes in the edge context.  This is 
an example of mapping hyperedges to hypernodes, 
which is a basic operation of hypergraph analysis.  

\end{description}

}

\subsection[Client-Object Procedures]{\textit{Client-Object Procedures}}
\p{In general, application-level objects are initialized 
from \HGDM{} infoset objects, whose hypergraph-based 
structures are designed to encode many data structures 
in a flexible, expressive fashion.  Client objects, 
in contrast to infosets, are not intended primarily 
for sharing between disparate applications; as such, 
these objects can be modeled directly according to 
the needs of individual applications, in terms 
of building compelling \GUI{} front-ends and 
effective domain-specific analytic capabilities.  
Constructions which are specific to hypergraphs 
--- e.g., role-projections, proxies, and inter-hypernode 
connections --- are not necessarily appropriate 
in the application-level context (often they can be 
replaced by ordinary application-level type structures, 
such as pointers and object members with \textbf{get}/\textbf{set} 
accessors).}

\p{Nevertheless, the \HGDM{} protocol recommends several design patterns 
or practices which can be enacted at the application level 
to help application code translate infosets to local type-instances, 
and in general to interoperate effectively 
with \HGDM{} networks.  These patterns are particularly 
applicable to \q{collections} types (the sorts of values 
encoded/decoded at the infoset layer via array fields) 
and to string/textual data (particularly in contexts 
where words or names from multiple languages may be involved).}

\p{With respect to data types considered in most contexts 
to be \q{atomic} values --- e.g., strings, integers, 
and decimals --- \HGDM{} encourages developers to 
use types and encodings that transparently 
document scientific and implementational assumptions and 
requirements.  For instance, magnitudes can be 
decorated with dimension units and range-restrictions.  
Floating-point values can be expressed with extra precision 
by employing ratios (integer quotient-pairs) or \q{posits} 
(a special number system invented by John 
Gustafson)\footnote{See \bhref{http://www.johngustafson.net/pdfs/BeatingFloatingPoint.pdf}.} in lieu of 
traditionally-encoded floating-point representations.}

\p{With respect to non-atomic \q{struct} types, whose contents 
are accessed by named data fields, \HGDM{} recommends 
accessors in which the bare field-name serves as a 
get-accessor and the form \q{\textbf{set\_}} provides 
setters.  Procedures named \q{\textbf{get\_}...} should not be used 
for accessors (which logically serve merely to return some 
member data) but for obtaining a value which requires some 
intermediate computation, or a value which may vary 
according to some context beyond the prior explicit setting of one 
single member value.}  

\p{In terms of non-atomic collections types, \HGDM{} recommends a 
collections protocol which overlaps with some patterns 
applicable to infoset array fields.  In particular, 
\HGDM{} recommends collections types whose underlying 
sequential data is one-indexed, although with a 
\q{zero} value allocated in memory to represent \q{default} 
or \q{null} values.  Internally, that is, the sequences 
are zero-based --- usually it is not necessary to 
modify access indices --- but the actual value present 
at position zero is considered not a \textit{logical} 
part of the sequence, but rather a placeholder for a 
value to use for out-bounds or other exceptional 
circumstances.  If \textbf{arr} is an array and 
$n$ is an integer, the expression \textbf{arr[n]} 
would then return \textbf{arr[0]} for any $n$ greater 
than \textbf{arr}'s size.}

\p{This system addresses several suboptimal properties 
of both zero-indexed and one-indexed arrays.  On the 
one hand, there is a correspondence between the 
length of an array and the index of it's last element: 
if $n$ is the length of \textbf{arr}, then \textbf{arr[n]} 
is the last element in \textbf{arr}.  Moreover, if one searches 
for an element in \textbf{arr} which is not found, the 
return value can be naturally \textbf{0}, numerically 
the same as boolean \textbf{false} --- and not (as is 
common practice in zero-indexed environments) \textbf{-1}.  
Employing \textbf{-1} as a \q{sentinel} not-found value 
is problematic in several ways, one of which is that 
\textbf{-1} evaluates under boolean conversion to 
\textbf{\textit{true}} rather than \textbf{false}.   
Indeed, \textbf{-1} does not even technically 
exist if the array index is an unsigned integer.  
Moreover, some systems allow negative indices to 
support counting from the \textit{end} of the sequence 
--- \textbf{arr[-1]} would be \textbf{arr}'s last 
element, \textbf{arr[-2]} its next-to-last, etc.  
In this case it is dubious to use \textbf{-1} both 
as a valid index value and as a flag for a index 
not found --- i.e., for the \textit{lack} of any 
index which returns a value that would satisfy some 
condition.  With any logic, the proper 
flag value for a \q{not-found} condition would be zero.  
On the other hand, computationally offsetting every 
index by one is at least a little inefficient 
and potentially error-prone.  A reasonable compromise 
is to leave an array internally zero-indexed, but 
treat the first (zero-index) value as a \q{dummy} 
value which is not logically part of the array.  
In that case, then, one can utilize this initial value 
as a default to be provided in out-of-bounds situations.  
With that practice, \textbf{arr[n]} becomes defined for 
all $n$ --- defaulting to \textbf{arr[0]} for $n$ 
out of the proper index range.}

\p{The \HGDM{} protocol recommends employing such a one-indexed system 
as a basis for multiple collections types that 
may be based internally on arrays, including lists, 
stacks, queues, and even matrices (two-dimensional 
arrays).  \lHGDM{} also recommends 
providing \q{functional} accessors to collections 
types (that is, implementing variants of procedures 
which call a passed code-block with a value 
obtained from a collection, or potentially multiple 
calls with multiple values, in lieu of return 
values directly).  Other guidelines, particularly 
those related to textual data, will be addressed 
in a later section.

\begin{description}

\item[\procitem{at(...)}, 
\procitem{get(...)}, \procitem{value(...)}, 
\procitem{get\_at(...)}]  These procedures provide 
indexed-based access to collections which support 
access at any index (not just the start or end of a 
list).  The first element is at position \textbf{1}.  

\descindent{}  When implementing indexed-based access, 
the basic question to address is how to handle 
out-of-range indices.  These distinct procedures use 
different strategies for such cases.  Specifically, 
\dprocitem{value} (similar to \Qt{}'s collections) 
requires a default value to be provided for out-of-bounds
indices (a default-constructed value may be used for types 
which have one, so this default value would often not 
be explicitly passed as an argument).  The 
\dprocitem{at} variant is recommended to return a reference 
to a value, defaulting to the \q{dummy} zeroth value if 
necessary.  The \dprocitem{get} variant returns a pointer 
to a value, or a \textbf{null} pointer for out-of-bounds 
indices.  Finally, \dprocitem{get\_at} can mimic the 
behavior of either \dprocitem{at} or \dprocitem{value}, 
but in either case \dprocitem{get\_at} should provide a 
method syntax for the same operation achieved by 
overloading square brackets: i.e., \textbf{arr[n]} 
should be the same as \textbf{arr.get\_at(n)}.

\descindent{}  Note that programmers seem to differ 
on whether (in an ideal language) bracket access 
(viz., \textbf{arr[n]}) should return a (copied) value, 
a mutable reference, or a constant reference.  \lHGDM{} 
allows for each of those options, to accommodate 
different computing environments.  One note, however: 
in a \Cpp{} context, it is possible to mix all three 
styles by an expanded implementation of 
\textbf{operator []}.  The simple step here is using 
\textbf{const}/non-\textbf{const} overloading to 
provide two different forms of the procedure, one 
returning a constant reference and one a mutable reference. 
The more complex step (and one which some developers might 
deem too much a departure from idiomatic \Cpp{}) is 
to associate \textbf{operator []} with an intermediate 
\textit{functor} type which has a \textit{cast} operator 
to a reference but a \textit{call} operator to a value, 
via a provided default value.  The details of this 
implementation are outside the scope of this summary, 
but the upshot is that with operator-overloading along 
these lines default values can be provided 
even with bracket notation via code such as 
\textbf{arr[n](1)} --- observe the trailing \q{\textbf{(1)}} 
here which indicates that the number \textbf{1} should 
be used as a default if $n$ is out of range.

\item[\procitem{push(...)}, \procitem{pop()}, \procitem{top()}]  
These procedures apply to collections behaving like stacks 
--- \q{Last In, First Out} data structures in which the 
element most recently added to the collection is removed 
by default, when no other criteria for an element to 
be removed is provided.  All stack-like types should provide 
\dprocitem{push} (add an element to the stack), 
\dprocitem{top} (get the most recently added element), 
and \dprocitem{pop} (remove the top element).  Stacks 
\textit{may} also provide procedures to access and append/remove 
elements in other positions as well, if the stack can also 
be used as a different kind of collection; however, 
\dprocitem{push}, \dprocitem{pop}, and \dprocitem{top} 
are the characteristic functions associated with stacks.  

\item[\procitem{enqueue(...)}, \procitem{dequeue()}, \procitem{head()}]  
These procedures apply to collections behaving in the manner of queues 
--- \q{First In, First Out} data structures in which items 
that have been present in the collection longest are removed 
first, by default.  Every item added to the queue is placed 
logically \q{behind} the prior elements.  This kind of 
collection is visualized like a line-up of objects, rather 
than as a vertical \q{stack,} which explains the common 
terminology used for queue-oriented procedures rather than 
stack procedures.  As with stacks, elements \textit{may} 
be accessed out-of-turn, but \dprocitem{enqueue}, 
\dprocitem{dequeue}, and \dprocitem{head} are canonical 
functions associated with queues. 

\item[\procitem{push(...)}, 
\procitem{enlist(...)}, \procitem{pop()}, \procitem{delist()}, 
\procitem{top()}, \procitem{rear()}]  \lHGDM{}'s 
terminology for deques (double-\-end\-ed queues) 
differs from conventional procedure-names more so than 
for stacks or queues.  \lHGDM{} considers deques to be a 
combination of two stacks, one representing the \q{front} 
of a collection and one representing the \q{back.}  
Deques in general are queues in which elements may be 
added to the front --- where they are the first to be 
removed --- as well as to the rear; in effect, an algorithm 
can select to allow some element to \q{skip the queue} 
when it is added.  Logically, this has the effect of 
forming a \q{front} and a \q{back} stack, where the former 
is enlarged whenever an element is added which may 
\q{skip the queue} and where the latter is enlarged whenever an 
element is added in conventional \q{First In, First Out} 
fashion.\footnote{It is possible to \textit{literally} 
implement queues as double-stacks, but one must then 
track the position where the two stacks meet and accommodate 
the possibility that one stack may be empty.}

\descindent{}  \lHGDM{} accordingly uses non-standard 
procedure-names for the \q{back} stack, intended to 
convey the stack-like behavior: \dprocitem{enlist} 
adds an item to the rear, \dprocitem{delist} removes the 
item at the rear, and \dprocitem{rear} returns this item.  
Note that an item added via \dprocitem{enlist} may 
eventually be removed via \dprocitem{pop} (if all 
items ahead of it are removed first); however 
\dprocitem{delist} removes the rear-most item \textit{first} 
(similarly, an item added via \dprocitem{push} might 
eventually be removed via \dprocitem{delist}).

\vspace*{6pt}   
\item[\parbox{\textwidth}{\procitem{is\_row\_major()}, 
\procitem{is\_column\_major()}, \procitem{is\_diagonal()}, 
\procitem{is\_symmetric()}, \procitem{is\_skew\_symmetric()},
\procitem{get\_matrix\_length()}, 
\procitem{get\_matrix\_size()}}]\ 

\vspace*{-6pt}  
These procedures are applicable to matrices (or two-dimensional 
arrays), which in \HGDM{} are assumed to be represented 
internally as a one-dimensional array.  Most 
numerical or linear-algebra libraries adopt this convention, 
translating two-dimensional indices to a one-dimensional 
index into an internal array.  Matrices in 
\textit{row major} order store each row in contiguous 
memory, so the whole first role is at the beginning 
of the internal array, then the whole second row, etc.  
Conversely, matrices in \textit{column major} order 
store columns in contiguous memory.  Libraries differ in 
whether they employ row-major or column-major order, 
so \HGDM{} recommends matrix classes support \textit{both} 
options, to minimize memory-copying when importing 
data from other contexts.  In general, row-major order is 
more efficient when it is desired to copy or reference 
individual rows, and column-major is better for copying 
or referencing individual columns.

\descindent{}  Note that some matrices are neither 
row-major nor column-major.  In particular, diagonal, 
symmetric, and skew-symmetric matrices have redundancies 
which make it unnecessary to store every item on its own.  
For instance, for diagonal matrices it is only necessary 
to have an array which is the length of the diagonal; all 
other elements (i.e., positions $r$, $c$ for $r \ne{} c$) 
are zero.  When the matrix in these cases is represented 
in a manner which minimizes the length of the internal 
array, neither rows nor columns are contiguous in memory.

\descindent{}  Note also that these cases reveal a 
divergence between the logical number of elements in 
the matrix (the product of the number or rows and number 
of columns) with the length of the internal 
array.  As such, \HGDM{} recommends distinguishing 
matrix \textit{size} (logical size) from matrix 
\textit{length} (the actual array length).  For 
instance, for a typical square diagonal matrix 
with $n$ rows/columns, 
\dprocitem{get\_matrix\_length} would return 
$n$, while \dprocitem{get\_matrix\_size} would 
return $n$^{\scalebox{.5}{2}}.           

\item[\procitem{get\_dimension\_structure()}, 
\procitem{get\_column\_dimension\_structure()}, 
\procitem{get\_pseudo\_constructor()}]  Except 
where computing speed has to be strictly optimized, 
\HGDM{} recommends using more detailed data types 
--- even for numeric values --- than conventional 
integers or floating-point decimals.  
For natural numbers, range and/or unit-restricted 
types ensure that values are used in scientifically 
reasonable ways.  For a collections type, \dprocitem{get\_dimension\_structure} 
and related procedures would 
then return information about the range, scale, and     
units of measurement applicable to all elements in the 
collection.  Along similar lines, 
\dprocitem{get\_column\_dimension\_structure} would 
provide this information relative to an individual 
matrix column, in a situation where different 
columns could potentially have different dimensional details.

\descindent{}  For each instance of a range-bound numeric type, 
\HGDM{} also recommends proving a static factory function 
(a \q{pseudo} constructor) which confirms that the requested 
initial value is in the allowable range before calling 
an actual constructor.  A pointer to that function 
--- which may be obtained via a templated 
\dprocitem{get\_pseudo\_constructor} procedure --- 
may then serve as a \q{passkey} value signaling that 
the provided constructor argument lies within the requisite range. 
\end{description}
}

\section{The Hypergraph Text Encoding Protocol}
\p{The prior sections in this paper have considered 
hypergraph-based information encoding/serializ\-ation 
with an emphasis on numeric 
and collections types, recommending representational 
practices which document scientific and programming 
assumptions and requirements.  The underlying principle --- 
that data encoding should be rigorous and transparent 
so as to clarify the data's scientific background --- 
also applies to textual data, implying the need for 
carefully designed text-representation protocols.}

\p{Properly encoding textual, Natural Language content is 
one of the more complex and challenging aspects of 
rigorous data sharing.  The Unicode standard theoretically 
provides a consistent framework for representing all 
the world's languages, even under-resourced languages 
(plus many other symbols, e.g. 
those used in mathematics).  However, there are 
several different Unicode encodings, with no guarantees 
that two separate communicating applications would employ 
compatible encodings.  Unicode also has certain 
representational limitations (outlined below), 
so it is not a definitive textualization solution.}

\p{Most textual data needs fewer than 256 distinct characters 
--- the most that can be represented with one byte --- even 
if the text contains some special symbols, such as accented 
letters occurring in foreign names.  It therefore wastes 
a lot of memory to allocate more than one byte per character, 
especially for long text documents.  However, \textit{which} 
256 characters are needed can easily vary from one text to 
another.  Generic types holding textual data, such as 
\textbf{QString}s, often employ two bytes per character instead, 
for greater flexibility.  One limitation of this solution, 
however --- apart from extra memory-usage --- is that 
characters cannot be paired one-to-one with two-byte 
units: in \Qt{}, for instance, some Unicode glyphs 
actually require \textit{two} \textbf{QChar}s in sequence.  
The full Unicode system utilizes a combination of one-byte, 
two-byte, and four-byte glyphs; as such, there is no 
correlation between the index of a glyph in a character 
stream and the position of a byte in a corresponding 
Unicode array.  To retrieve the $n$th character in an 
unrestricted Unicode text, for example, it may be 
necessary to examine every prior character so as to 
determine which byte(s) actually holds the character 
requested.  A further consideration is that users often 
want to copy-and-paste across applications; given 
that distinct applications often employ incompatible 
character-encoding schemes, a thorough encoding system should 
support the notion of a \q{copyable} range which can convert 
the indexed characters to a neutral format.}

\p{In short, efficiently indicating individual 
characters or character-sequences by positional indices is 
oftentimes more important than strict adherence to Unicode (or any other) 
standards.  For example, Natural Language text may need 
to be annotated wherein mentions of names or concepts 
are identified via index-ranges.  In a biomedical 
context, strings representing proper names (e.g. patients or 
doctors), diagnostic codes, symptoms, clinical 
procedures, chemical formulae, and so forth, may all 
be detected via text mining (Named Entity Recognition, 
Lemmatization, Diarization, etc.).  
The representation of such annotation data could become especially  
complicated without a straightforward indexing scheme 
--- that is, without there being at least one view onto a document 
such that the overall textual content is a list of characters, 
with text segments identified by providing a start and 
end index.  For efficient indexing, every character should 
accordingly have \textit{the same number} of bytes.} 

\p{The technical question therefore emerges of how to 
flexibly encode a variety of characters (including certain 
rare glyphs only used once or twice in a document) while 
enforcing that all glyphs span the same length of bytes 
(preferably just one byte) and also preventing unnecessary 
wasted memory.  There are two possible (but not 
exclusive) solutions to this problem.  One option is 
to reserve a certain number of character codes which are 
not assigned, except for within individual documents --- 
allowing them to locally map those codes to a more expansive 
character set, such as Unicode.  Another option is to 
maintain a separate index --- outside of the principal 
character array --- for the handful of special glyphs which 
a document may need outside its normal 
character set.  For instance, suppose three characters 
out of several thousand in a document require glyphs that 
do not fit within a one-byte character set.  This document 
could use a special code (e.g., \textbf{255}) for these characters 
in the main array and then, separately, maintain a mapping 
from the three index points to (say) Unicode glyphs intended 
to be inserted into those three positions.}

\p{Sometimes these strategies can be combined.  
For example, many non-English letters appearing 
in English-language texts are accented characters 
from other European languages, belonging to proper 
names or special phrases (cf. the \q{\`{a}} in 
\textit{vis-\`{a}-vis}).  These glyphs are typically 
described by providing a \q{base} letter plus a 
\q{diacritic} (accent) mark.  In a one-byte-per-character 
encoding, one could place only the diacritic code in 
the main character array, and use a separate mapping 
to fill in those letters (e.g. a data structure which 
would assert that the acute accent at position 102, say, 
should be placed atop an \q{e}).  Only those character 
codes which could potentially need supplemental bytes 
(e.g. diacritic codes) would trigger lookup in that 
external mapping.  Such a system combines the ideas 
of having a designated subset of codes for \q{locally defined} nonstandard 
glyphs and of having an indexed-based lookup for nonstandard 
glyphs at specific index-positions in a character array.}

\p{Aside from encoding limitations of Unicode and other 
popular character formats (including \ASCII{}), another 
problem with these existing encodings is that they 
tend to address the visual appearance of glyphs 
rather than their semantic meaning.  Consider an 
ordinary period, which could have several different 
interpretations: the end of a sentence, part of an 
abbreviation, the decimal point in a printed number, 
part of an ellipses, or an operator in computer code.  
Failure to encode such differences can make 
text mining more complex than necessary --- for instance, 
using separate character codes for end-of-sentence 
punctuation than for other uses of periods, exclamations, 
and question marks would eliminate the need for 
\AI{}-driven \q{Sentence Boundary Detection.}  In some 
contexts, differences in glyphs' linguistic meaning 
may translate to visual typesetting as well: periods 
marking an end of sentence, say, should be followed 
by larger space-gaps than those inside 
abbreviations (e.g., \q{Dr.}).  Hyphens are often printed 
with different lengths when used as punctuation (dashes) 
as opposed to within words or phrases (cf. \textit{vis-\`{a}-vis} 
again) and within numeric literals (cf. \q{-1}).  Quote 
marks are typically typeset as curved indicators when actually 
used to surround quotes, but as straight marks when used 
for feet and inches (\q{6\textquotesingle{}2\textquotedbl{} tall}).  
Such presentational details may only be considered in 
depth at the camera-ready phase of a scholarly publication, 
but in fact there is an overlap between the proper 
visual cues expressing certain punctuation or other 
non-alphabetic marks and their semantic meaning, which is 
relevant for text mining.  As such, these semantic 
differences should optimally be encoded in character 
sets themselves (rather than relegated to special 
markup, such as \LaTeX{} commands).}

\p{For a further consideration, the basic 
Latin-1 character set is also inefficient in 
including certain obsolete \ASCII{} codes (such as the 
\q{bell} character, which literally rings a bell via the 
computer's audio) that are almost never currently used.  
There are, in short, \textit{de facto} unassigned 
code points, leaving room for duplicate glyphs which 
may have the same appearance but alternate 
interpretations (e.g. abbreviation-period 
vs. punctuation-period).  Combining all of these 
ideas, then, \HGDM{}'s recommended text-encoding 
protocol --- dubbed \HTXN{} (\q{Hypergraph Text 
Encoding}) --- uses diacritics, \q{interpretation 
codes} (disambiguating semantically distinct but 
visually similar glyphs) and the option to 
extend character sets arbitrarily for individual 
documents, to yield a flexible and expressive 
encoding.  This encoding, in its raw form, requires more than 
one byte per character, but \HTXN{} employs 
certain tricks to compress the character set 
so that it can typically fit into a single-byte context.}

\subsection{Character Encoding and TAGML}
\p{Having discussed the encoding of individual 
characters, it is appropriate to address \HGDM{} recommendations 
for encoding markup and presentation details, 
at a higher scale than individual glyphs.  As a general-purpose 
serialization format, \HGDM{} suggests 
\TAGML{}, which is similar to \XML{} but allows 
certain constructions that are more 
expressive than \XML{} (e.g., concurrent markup).
However, \HGDM{} describes a modified version of 
\TAGML{}, one engineered to work with \HTXN{} at the 
character-encoding level.}

\p{Intrinsically, \TAGML{} does not specifically 
address character encoding; instead, this is assumed 
to be a property of each \q{node} which holds 
character data.  In \TAGML{}, certain nodes 
(called \q{prenodes} in \HGDM{}) represent 
character-sequences, while other hypernodes 
embody markup, applied to groups of prenodes.  
The overall character sequence of a document 
is assumed to be retrieved by 
collating characters from every prenode, in order.  
This can be inefficient when it is necessary 
to define annotations, or other index-based 
ranges, which may span multiple prenodes 
(one cannot readily map single index numbers --- assuming 
these quantify over a whole document --- 
to character indices within individual prenodes).  
For this reason, \HGDM{} recommends that 
a \TAGML{} runtime hold character data 
outside of prenodes proper --- specifically, 
that characters be stored in \q{text blocks} 
and prenodes use indices into these blocks 
to establish their character data.  Each 
text block, in turn, uses \HTXN{} to 
encode characters via single bytes 
(different text blocks may use variations 
on the basic \HTXN{} encoding).}

\p{Note that this setup does not operationally 
alter the \TAGML{} structure: while a text block 
\textit{may} encode an entire document via a character 
array that can be accessed outside any prenode, 
this is only one possible use-case.  Absent specific 
designs to the contrary, it should be assumed that the 
definitive character data for any document is 
accessible only via prenodes, and the use of text blocks 
is merely a memory optimization to streamline the initialization 
of prenode contents.  In particular, prenodes which 
have overlapping indices to the same block are not overlapping 
markup --- instead, each prenode holds multiple copies of a 
given character string, which have no logical relation to 
one another.  For example, a distinct text block could be 
set up to hold proper names that have foreign characters.  
Individual names could then be carried within a prenode by 
giving the start and end indices of the name in its 
containing block.  If a name appears multiple times in 
a document, several prenodes might then hold duplicate 
data --- the same indices against the same block.  But 
these prenodes do not overlap; each is understood to 
have a logically isolated copy of that name, as if 
the characters in the name were stored internally in
the prenode's character data without the use of an 
external text block as the in-memory storage.}

\p{Separate and apart from the character-encoding 
issues analyzed above, \HGDM{} also proposes extending \TAGML{} 
in ways consistent with the intended use of 
\TAGML{} as a general hypergraph notation format 
--- in short, as the markup language for \HGXF{} 
(when hypergraph serialization is done via 
text documents).  Specifically, \HGDM{} proposes 
a variant dubbed \q{Grounded} \TAGML{} (\GTagML{}) 
in which \TAGML{} hypergraph structures are 
refined to notate hypergraph relations within 
serialized data, not only (as in the current \TAGML{}) 
among markup elements.  That is, \GTagML{} structures 
are \q{grounded} in hypergraph relations intrinsic 
to the data being conveyed by \GTagML{} documents.  
Further information about such \q{grounding} is outside 
the scope of this outline, but may be provided by 
Linguistic Technology Systems on request.}

}
\end{document}



