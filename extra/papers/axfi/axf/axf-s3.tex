
\part{AXFI Library Architecture}
\vspace{.5em}
\section{AXFI Object Libraries and Serialization}

\p{\lAXFI{} is designed around the premise that the 
primary goal of any data modeling system (and, as a 
surface-level encoding of data to be shared, 
any markup language) is to encapsulate data 
which is used by and present in one application, 
to be subsequently re-used by a different 
application --- or the same application at a future 
time.  There is an application which 
\textit{creates} an annotation, and a (not 
necessarily different) application which 
\textit{consumes} an annotation.  \q{Consuming} 
in this context means rendering the annotation 
so it may be viewed by a human user and/or 
reading annotation data so that such data may be 
incorporated into some automated computational 
process.  Of course, the consumer-application 
may also be a creator, adding new annotations 
or revising the ones it has consumed.}

\p{The simplest use-case is where the same 
application both \textit{creates} and 
\textit{consumes} each annotation.  However, 
there would be less need for standard 
models if every application only shared 
data with itself.  We can assume, then, 
that any standard such as \AXFI{} 
is primarily intended to ensure 
that two \textit{different} applications 
correctly share data.  Data sharing is 
\q{correct} insofar as the consumer-application 
reads, understands, displays, and allows 
human users (as well as, potentially, 
computer algorithms) to manipulate the 
annotation in ways which are functionally 
equivalent to the creator-application.  
When this functional equivalence 
is achieved, the two applications 
can be said to be \textit{aligned}.}

\p{Creator-to-consumer alignment can happen in 
one of two ways.  One option is that the 
data model shared between the two applications is 
formalized so that different applications 
are naturally interoperable because they are 
implemented according to similar guidelines.  
This may be called \q{coincidental alignment.}  
There are, in turn, two pathways toward 
coincidental alignment.  On the one hand, 
data-model formalization may stipulate criteria 
on creator and consumer applications to ensure 
that any creator is aligned with any 
consumer.  Such a data model would essentially 
provide a \q{checklist} of requirements 
to be satisfied, procedures to be implemented, 
and so forth, so that applications are 
only considered to be creators and/or consumers 
by fulfilling the checklist contract.}

\p{On the other hand, the formalization may 
focus on the medium through which 
data shared between creator and consumer 
is structured and encoded.  This second mode of 
standardization --- coincidental 
alignment based on data-serialization formats --- 
is arguably the most common strategy for 
enforcing alignment (even if it is not 
identified in these terms).  For example, 
\XML{} schemata represent constraints 
on any creator of information encoded by 
conformant \XML{} documents; given these 
constraints, the \XML{} \textit{consumers} 
can guarantee alignment with creators 
by incorporating the schema as a contract 
the creator obeys.  The consumers need 
not have any knowledge of the creators 
\q{apart from} the fact the \XML{} they 
create conforms to the schema.}

\p{Whether by explicit application-level requirements 
or via contracts on the medium by which shared 
data is encoded (e.g., \XML{} schemata), 
coincidental alignment can occur without 
any explicit coordination between the 
code used by creator and consumer applications, 
respectively.  As an alternative model, 
however, we can identify \textit{deliberate} 
alignment, wherein the consumer aligns with 
a creator by explicitly referring to and 
mimicking the creator's data models and 
procedures.  To facilitate deliberate 
alignment in this sense, creator 
applications can architect the relevant 
code to prioritize transparency and reuse --- 
e.g., factoring this code into a 
semi-autonomous library which the 
consumer application can use directly, 
or use as the basis for its own library.}

\p{In contrast to many data models, 
\AXFI{} openly embraces deliberate 
rather than coincidental alignment in this sense.  
This is expressed through the idea of 
\AXFI{} Object Libraries: a creator 
of \AXFI{} is encouraged to provide 
(or reuse) a code library which implements 
logic for constructing, serializing, and 
deserializing \AXFI{} objects.  Each 
\AXFI{} Object Library may add its own 
data profiles and capabilities to the 
underlying \AXFI{} system.  Linguistic 
Technology Systems provides a 
default \AXFI{} library, called 
\q{\DSPIN{}} (Data Structuring Protocol 
for Image-Analysis Networking).  This 
\DSPIN{} library serves as a 
\textit{reference implementation} 
for a generic or canonical \AXFI{} 
format.  Other \AXFI{} Object Libraries 
can provide reference implementations 
for alternative/extended versions of 
\AXFI{} by adding their own features 
which are not present in \DSPIN{} itself.}


\p{Unlike other frameworks, then, \AXFI{} 
does not define a standard data structure or 
serialization format except insofar as 
each reference implementation, exposing  
a given \AXFI{} Object Library for 
cross-application reuse, yields an 
archetype that can be copied or modified.  
However, any data shared between 
heterogeneous endpoints needs to be 
encoded somehow, and \AXFI{} does 
model certain patterns in how 
\AXFI{} data should be represented 
in textual or binary files/streams.}

\p{All \AXFI{} data exists in the 
context of some \AXFI{} Object Library.  
When networking between separate applications, 
\AXFI{} assumes that the creator and 
consumer either uses the same 
Object Library or two libraries which 
are functionally equivalent, at 
least in all ways affecting the shared 
data.  So the role of an \AXFI{} \textit{package} 
--- that is, a file, or some similar text 
or binary stream carrying \AXFI{} data 
--- is to allow objects created 
by an Object Library to be serialized/encoded 
and then later deserialized by the same 
(or an equivalent) Object Library.  
Any binary or textual format which 
permits this \q{co-serialization} --- i.e., 
proper serialization and deserialization 
between aligned libraries --- is a 
reasonable surface-level encoding of 
\AXFI{} data.}

\p{In short, for each \AXFI{} data package 
there is a collection of \AXFI{} objects 
being serialized and (later) deserialized, 
and the package has a \textit{surface 
representation} whose primary importance 
is to be a convenient vehicle for the 
relevant co-serialization.  Serialization 
formats can be chosen to minimize the 
amount of biolerplate code needed for the 
Object Library to marshal \AXFI{} 
data between runtime and serial encodings.}

\p{While any markup language, or other 
encoding mechanism, could potentially 
be used for co-serialization, \AXFI{} 
also recommends that Object Libraries 
employ serialization formats 
which document how the data-sharing 
logic is implemented.  In other words, 
the data format should make the 
process of marshaling information 
between runtime and co-serialization 
environments transparent and reusable.  
This is different than (say) 
an \XML{} schema, where the serialization 
format is the \textit{basis} of 
multi-application alignment.  In \AXFI{}, 
the Object Libraries, not any markup 
standardization, are responsible 
for application alignment.  Nevertheless, 
while serialization format has no 
constraints to enforce alignment 
on a technical level, \AXFI{} Object 
Libraries should be considered easier 
to re-use if they employ a conceptually 
nuanced and self-documenting format 
for expressing/sharing data.}

\p{To this end, Linguistic 
Technology Systems provides a new 
markup format, in conjunction with 
the \DSPIN{} reference implementation 
for serializing \AXFI{} data, called 
\GTagML{} (\q{Grounded} \TagML{}), 
where \TagML{} is an existing 
\XML{}-like language (but with a 
non-\XML{} syntax and a richer 
semantics, e.g., concurrent markup).  
\lGTagML{} is \q{grounded} in the 
sense that each \GTagML{} file 
presents the co-serialization of 
certain instances of certain data 
types, where the relation of the 
explicit contents of the \GTagML{} 
code and the data types which 
are present as the code is generated 
--- and will again be present (or 
have aligned, corresponding types which 
are instantiated) when the 
code is de-serialized --- is explicitly 
modeled as part of the \GTagML{} expression.  
That is, \GTagML{} data packages 
are \q{grounded} in explicitly-referenced 
data types, co-serialization logic, and 
code libraries using \GTagML{} for 
inter-application communications.}

\p{A thorough discussion of \GTagML{} is 
outside the scope of \AXFI{}, but a 
brief summary of how \GTagML{}'s notion 
of grounding applies to \AXFI{} can 
be sketched out.  Any \AXFI{} data 
package represents the serialized 
encoding of one or more objects 
defined/constructed by an \AXFI{} Object 
Library.  The \GTagML{} code which 
serializes these \AXFI{} objects 
is therefore a tool to allow the 
objects to be lifted from their 
runtime context and re-created at 
some other time and/or place.  
But the \GTagML{} code also 
bears a formal relationship 
to the data type instances which 
it serializes: a given set of 
\GTagML{} nodes, for instances, 
encodes some particular object 
(e.g., some annotation or some geometric 
primitive), an object which is 
an instance of a given data type, 
provided by a given code library, 
linked in to a given creator 
application.  The serialization occurs at 
a given moment in time, and in a given 
application context (the specific session, 
user, or provenance data which might be 
relevant for the overall co-serialization 
context).  The idea behind \GTagML{} is 
that such semantic relations between 
\GTagML{} markup and the data type/application-level 
context where this markup is generated is 
formally modeled within the markup itself.  
When serialized via \GTagML{}, \AXFI{} data 
includes a structured description of how 
the \AXFI{} objects relate to the \GTagML{} 
code which encapsulates them.}

\p{The relation between \AXFI{} and \GTagML{} 
can be further illustrated by considering 
domain-specific extensions or versions of 
\AXFI{}, which are discussed in the 
following section.}









