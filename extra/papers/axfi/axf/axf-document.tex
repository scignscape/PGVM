
%\part{Hypergraph Text Encoding Protocol}
%\section{Overview}

\p{The LTS Annotation Exchange Format (\AXF{}) and 
Hypergraph Text Encoding Protocol (\HTXN{}) are components 
of \ConceptsDB{} Application Framework (\CsAF{}), 
which is based on the \ConceptsDB{} Hypergraph Database Engine.  
\lCsAF{} is designed with an emphasis on curating 
scientific data sets and facilitating scientific research.  
When research data is published in conjunction 
with books are articles, \AXF{} and \HTXN{} can 
be used to cross-reference publication texts with 
the associated data sets.  Here, \AXF{} annotatons 
can link specific manuscript locations or elements 
--- sentences, paragraph, quotations, figure illustrations, 
or other isolated components within an overall document 
--- to corresponding parts of a data set.  The 
\HTXN{} protocol, for its part, allows granular 
publication elements to be identified, so as to be 
the basis for \AXF{} annotations.}

\p{In addition to the overall \AXF{} framework, with its 
orientation to research publications and data, \AXF{} 
includes more specific specializations for 
images (\AXFI{}) and for linguistic analysis 
(\AXFT{}).  The former extensions, focused on 
bioimaging and general image-analysis, are discussed 
further in the second part of this outline.} 

