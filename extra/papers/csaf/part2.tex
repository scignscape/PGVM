
\part{Interoperating with Data Sets and Scientific Repositories}

\p{Systematic use of published scientific data requires 
accessing information on several different levels, 
including outlines of research methods and protocols 
as well as raw research data.  Several standard 
models exist for how research data sets 
should be described, including Research Objects\footnote{See 
\bhref{https://eprints.soton.ac.uk/271587/1/research-objects-final.pdf}}, 
the \FAIR{} (Findable, Accessible, 
Interoperable, Reusable) initiative\footnote{See \bhref{https://www.researchgate.net/publication/331775411_FAIRness_in_Biomedical_Data_Discovery}.}, 
\SciXML{},\footnote{See \bhref{http://able.myspecies.info/scixml} } 
and \SciData{}\footnote{See \bhref{https://stuchalk.github.io/scidata/}.}.  
Also relevant to scientific data sharing, though less general 
in scope, are formats for composing and/or annotaging scientific 
publications, such as 
\IeXML{}\footnote{See \bhref{https://www.semanticscholar.org/paper/IeXML\%3A-towards-an-annotation-framework-for-semantic-Rebholz-Schuhmann-Kirsch/1d72a56b6576117c62f388a5f2193965e4c7e293}.} and \JATS{} (Journal Article Tag Suite).  Despite (or perhaps beause of) this diversity of 
representational possibilities, guidelines for construsting 
and annotating scentific data remain open-ended: researchers 
who want to rigorously annotate/document their published 
data can do so in many different ways, which 
presents challenges to software that interacts with research 
data.}

\p{Several different kinds of applications interface with 
scientific data, including software where such data 
is \textit{created}; web services (and potentially \API{}s) where 
data is \textit{hosted}; and programs which researchers 
use to find and download (or otherwise access) scientific 
data.  Because data-representation paradigms vary, 
these applications need to be flexible enough to 
manipulate data and metadata packages structured in 
divergent ways.  For example, each \MdsX{} module 
is designed around a particular scientific 
repository/portal's protocols, so the structure 
of distinct \MdsX{} modules may be noticeably 
different.  Nevertheless, some general structuring 
principles can be developed with respect 
to querying and consuming repositories' resources.  
This motivates the idea of a general 
\q{Scientific Data Repository Model,} 
(\SDRM{}) which 
is concretized within code implementing 
clients for individual portals (e.g., 
\MdsX{} modules).  The following section 
will present an overview of this 
\SDRM{} representation, and subsequent 
sections will specify how \SDRM{} 
is related to \ConceptsDB{} and other 
technologies discussed here.}

\section{Interoperating with Science Portals via 
the Scientific Data Repository Model}

\p{}



