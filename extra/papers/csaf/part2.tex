
\part{Interoperating with Data Sets and Scientific Repositories}

\p{Systematic use of published scientific data requires 
accessing information on several different levels, 
including outlines of research methods and protocols 
as well as raw research data.  Several standard 
models exist for how research data sets 
should be described, including Research Objects\footnote{See 
\bhref{https://eprints.soton.ac.uk/271587/1/research-objects-final.pdf}}, 
the \FAIR{} (Findable, Accessible, 
Interoperable, Reusable) initiative\footnote{See \bhref{https://www.researchgate.net/publication/331775411_FAIRness_in_Biomedical_Data_Discovery}.}, 
\SciXML{},\footnote{See \bhref{http://able.myspecies.info/scixml} } 
and \SciData{}\footnote{See \bhref{https://stuchalk.github.io/scidata/}.}.  
Also relevant to scientific data sharing, though less general 
in scope, are formats for composing and/or annotaging scientific 
publications, such as 
\IeXML{}\footnote{See \bhref{https://www.semanticscholar.org/paper/IeXML\%3A-towards-an-annotation-framework-for-semantic-Rebholz-Schuhmann-Kirsch/1d72a56b6576117c62f388a5f2193965e4c7e293}.} and \JATS{} (Journal Article Tag Suite).  Despite (or perhaps beause of) this diversity of 
representational possibilities, guidelines for construsting 
and annotating scentific data remain open-ended: researchers 
who want to rigorously annotate/document their published 
data can do so in many different ways, which 
presents challenges to software that interacts with research 
data.}

\p{Several different kinds of applications interface with 
scientific data, including software where such data 
is \textit{created}; web services (and potentially \API{}s) where 
data is \textit{hosted}; and programs which researchers 
use to find and download (or otherwise access) scientific 
data.  Because data-representation paradigms vary, 
these applications need to be flexible enough to 
manipulate data and metadata packages structured in 
divergent ways.  For example, each \MdsX{} module 
is designed around a particular scientific 
repository/portal's protocols, so the structure 
of distinct \MdsX{} modules may be noticeably 
different.  Nevertheless, some general structuring 
principles can be developed with respect 
to querying and consuming repositories' resources.  
This motivates the idea of a general 
\q{Scientific Data Repository Model,} 
(\SDRM{}) which 
is concretized within code implementing 
clients for individual portals (e.g., 
\MdsX{} modules).  The following section 
will present an overview of this 
\SDRM{} representation, and subsequent 
sections will specify how \SDRM{} 
is related to \ConceptsDB{} and other 
technologies discussed here.}

\section{Interoperating with Science Portals via 
the Scientific Data Repository Model}

\p{According to the \SDRM{} representation, each 
data set is accompanied/constituted by a collection 
of \q{assets.}  These include raw data files; 
publication texts; research method descriptions; 
code for accessing/examining/analyzing research 
data; and annotations defining interconnections 
between these different elements.  
All material within or associated with a data set 
is classified into one of \SDRM{} \q{units,} including 
the five just enumerated and several 
others.\footnote{For instance, data in some fields --- such 
as genomics --- sometimes becomes too large to 
be conveniently downloaded to an ordinary computer; 
in this case, portals such as GenBank offer query services 
to restrict vary large data sets into small result-sets 
that researchers can examine directly.  In these 
cases, the steps required to obtain information via 
large, remotely-hosted data steps would potentially be 
encapsulated into a distinct \sSDRM{} unit classified 
outside the main \sSDRM{} categories.  Another 
potential example of \sSDRM{} would be 
statements related to funding, potential conflicts 
of interest, and so forth.}  
\lSDRM{} units represent both serializations of 
data-set information, as well as \GUI{} components 
(such as Dialog Boxes) allowing readers of 
publicatons (and users of corresponding data sets) 
to view the unit's information.   Moreover, 
\SDRM{} units, by design, encapsulate their 
data in type-instances (e.g., objects of 
\Cpp{} classes) that can be manipulated at 
runtime by computer code.  In short, each 
unit has at least three representations 
(serial, \GUI{}, runtime), giving rise to a 
spectrum of \SDRM{} data that can be 
summarized in a table (see Table~\hyperref[tab:sdrm]{1}).}  

\begin{table}
\renewcommand{\arraystretch}{2}
\label{tab:sdrm}
\begin{scriptsize}
\begin{center}
\begin{tabular}{|p{.1\textwidth}|p{.25\textwidth}|p{.25\textwidth}|p{.2\textwidth}|}
\multicolumn{1}{c}{\textbf{\sSDRM{} Unit}} &%
\multicolumn{1}{c}{\textbf{Runtime Object}} &% 
\multicolumn{1}{c}{\textbf{Serialization}} &%
\multicolumn{1}{c}{\textbf{\sGUI{} Component}} \\%
\midrule
\RaggedRight{}Raw Data Files &\RaggedRight{} File and archive objects identifying file types, 
preferred applications, software prerequisites, etc. 
&\RaggedRight{} Structured review of archive layout (e.g., the Research Object 
Bundle \textbf{manifest} file) 
&\RaggedRight{} Filesystem archive view plus information on file types and sizes  \\%
Publication &\RaggedRight{} Machine-Readable Text (e.g., \sHTXN{}) 
&\RaggedRight{}  
Machine-Readable Serializaton (e.g., \sHTXN{}-compatible \LaTeX{}) 
&\RaggedRight{} \sPDF{} Viewer  \\%
\RaggedRight{}Research Method &\RaggedRight{} Research Protocol Encoding (e.g., \sBioCoder{}) 
&\RaggedRight{} Research Protocol Serialization (e.g., an \sMIBBI{}-based markup language) 
&\RaggedRight{} Dialog Box to view \q{Minimum Information} checklist   \\
\midrule
\RaggedRight{}Dataset-Specific Code Library &\RaggedRight{} Descriptions of library dependencies, compile 
steps, inter-type relations (e.g. corresponences between data object and 
\sGUI{} types), etc. 
&\RaggedRight{} Serialized build information  
&\RaggedRight{} Dialog to set compile/user options; \sIDE{} integration  \\
\midrule
\RaggedRight{}Annotations on Data, Code, and Text &\RaggedRight{} Integrated Annotation Object 
Model &\RaggedRight{} Annotation Encoding (e.g. LTS's \q{Annotation Exchange Format}   
&\RaggedRight{} Annotation-related context menus within \sPDF{} viewers and 
dataset \sGUI{} elements 
\bottomrule
\end{tabular}
\end{center}
\end{scriptsize}
\caption{Representations for \SDRM{} Units}
\end{table}

\p{In general, the data within \SDRM{} units might be 
obtained from different sources, depending on the 
module --- e.g., from publishers' \API{}s, or 
from downloaded data sets, or some combination.  
It is better to make at least part of the \SDRM{} data available 
without having to download data sets as a whole, because 
a useful overview of the data set may factor 
in whether users choose to download it to begin with.} 
 
\p{}



