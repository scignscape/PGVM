\part{MOSAIC Components}

\p{In recent years --- as publishing has become more \q{digitized} --- 
many online platforms have emerged for scientists to share and publish
their research work, including both raw data and scientific papers.
These portals generally provide an index/search feature where readers
can locate research based on the name of the author, title of 
the publication, keywords, or subject matter, along with a document 
viewer where readers can view the abstract of each 
publication (and sometimes full text of the article) and other
publication details (such as co-authors, date of 
publication, bibliographic references, etc.).
However, these portals offer only limited features for interactively
exploring publications, particularly when it comes to examining data
sets and/or browsing multimedia assets associated with publications,
such as audio, video, interactive diagrams, or \ThreeD{} graphics.}

\p{The \MOSAIC{} system is envisioned, in its totality, 
as a suite of code libraries developed by LTS allowing 
institutions to host publication repositories --- and 
for individual users to access publication repositories --- 
free of the limitations of existing scientific document portals.  
In the immediate future, LTS is focused on the more modest 
goal of implementing software to access existing  
portals (via \MdsX{}) and to create individual 
data sets (via \dsC{}).  With \MOSAIC{}, each publication can be
linked to a \text{supplemental archive} that contains information about
the author's research methods, data sets, and focal topics.  If desired, 
these archives can include machine-readable representations of 
full publication text, to support advanced text-mining techniques 
across the repository.  The supplemental 
archive can be explicitly linked to publications within 
their host repository, or it may be maintained 
in a decentralized manner external to the hosting platform.}

\p{Using \MOSAIC{}, developers can implement a hosting platform 
and/or a client platform 
for this supplemental material, in addition to the publications 
themselves, where the platform provides software
enabling readers to browse and access supplemental archives and 
their data sets and/or methodological 
descriptions.\footnote{A \sMOSAIC{} 
publication repository or \q{portal} 
is a structured collection of data and documents which can be 
hosted via web servers (including fully encapsulated and 
containerizable cloud services) using technology related 
to \sConceptsDB{}. \sMOSAIC{} repositories can 
serve as general-purpose portals, hosting academic papers covering a broad range 
of topics.  Alternatively, \sMOSAIC{} repositories 
can serve as targeted portals hosting papers focused 
on more narrowly focused subject domains.  
Organizations can utilize \sMOSAIC{} internally as the 
basis of a private document management system (\sDMS{}), or 
to provide (public) open access to 
complete publications, including human-readable and machine-readable 
full text and all supplemental content, with no restrictions.  
Excepting sensitive/private information which might be provided via 
a dedicated component within the portal with specific features 
for authors, editors, and administrators.  
Any project which \textit{does} restrict 
access to some part of any document requires a commercial 
\sMOSAIC{} license.}  \lMOSAIC{} can be customized 
for different subject areas by incorporating domain-specific 
ontologies into document-search features as well as 
machine-readable document-text annotations.}

\p{\lMOSAIC{} is designed so that the software to access 
publications and publications' supplemental archives can be embedded in 
scientific-computing applications via \MOSAIC{} \textit{plugins}.  
This ensures that publications and data sets can be 
examined interactively with the same software that scientists 
employ to conduct research.  \lMOSAIC{} also introduces a 
\textit{structured reporting system} (\MOSAICSR{}) for describing 
research/experimental/lab methods/protocols.  Supplemental archives, 
plugins, and the structured reporting system 
are outlined below.}


\section{Supplemental Archives} 
\p{\lMOSAIC{} supplemental archives are additional resources 
paired with \MOSAIC{} publications.  In general, supplemental archives 
may include raw data, descriptions of research methods, 
or annotated data linked to publication texts.  Each supplemental archive 
could have any of the following resources:

\begin{enumerate}[leftmargin=12pt]
\item Interactive versions of the publication, with annotations
indicating important concepts and phrases, perhaps aggregated into a
\q{glossary} defining technical terms;

\item Machine-readable representations of document texts, with 
special-purpose character encodings designed to facilitate 
Text and Data Mining (\TDM{});

\item Structured files containing raw data discussed in the publication,
along with interactive software allowing scientists to access and
reuse the data;

\item Detailed reports of the author's research methods and experimental
setup and/or protocols, conforming to the relevant standards with
respect to the publication's subject classification --- for instance,
the \q{Minimum Information for Biological and Biomedical Investigations}
(\MIBBI{}) includes about 40 specific standards for different branches of
biology and medicine;

\item Representations of analytic methods and algorithms underlying 
the research findings, which are provided directly via computer code or
indirectly via formal descriptions of computational workflows;

\item Self-contained computer software which demonstrates code that the
author developed and/or used for analyzing/curating research data; and 

\item Multi-media assets such as audio or video files, annotated images,
\ThreeD{} graphics, interactive \TwoD/\ThreeD{} plots and diagrams, 
or other kinds of non-textual content which needs to be 
viewed with special multi-media software.

\end{enumerate}
}

\p{The contents of a supplemental archive will be different 
for different publications, depending on whether the archive 
contains specific raw data or just a summary of the methods 
used to obtain the raw data.  In the former case, a typical 
archive will include a \textit{Dataset Application}, or 
a semi-autonomous software component allowing 
researchers to study and visualize 
the data set, typically provided in turn via data files that are 
also located in the archive.  The Dataset Application will, in 
general, provide both a visual interface for raw data and code 
libraries for computationally manipulating this data; typically 
raw data files will encode serialized data structures that 
can be deserialized by functionality (e.g., \Cpp{} classes) included 
in the application code (Dataset Applications are 
discussed further in Part II, Page~\lpageref{sec:dsa}).  
The \MOSAIC{} Dataset Explorer 
(\MdsX{}) libraries  
can be adapted to whatever specific kinds of information must 
be serialized for a given publication, providing an 
implementational starting-point for Dataset Applications.}

\section{\protect\llMOSAIC{} Plugins}
\p{As an alternative to fully self-contained and 
autonomous Dataset Applications, \MdsX{} explorers 
can also be deployed as plugins 
for existing scientific software, allowing publications and
supplemental archives to be read and examined 
within computer software associated with each publication's subject
matter.  For example, articles about chemistry could be read within
\IQmol{}, a molecular visualization program; articles about cellular
biology and bioimaging could be read within \CaPTk{} 
(the Cancer Imaging and Phenomics Toolkit), an image-processing
application; articles discussing novel computer code could be read
within \Qt{} Creator, an Integrated Development Environment (\IDE{}) for
programming; etc.  The advantage of accessing 
a publication and a supplemental archive within actual scientific
software is that it allows research work to be understood, evaluated, 
and reused within the computing environments which scientists typically use to 
conduct professional research.  This is different 
than existing science publication portals, which 
generally rely on web browsers to
access such supplemental materials as data sets and multimedia files ---
in this standard setup the software ecosystem wherein readers examine
published research is fundamentally separate from the software
platforms where research is actually conducted.  This example 
demonstrates how existing portals are limited 
in their ability to share research
in rigorously reusable and replicable ways, motivating 
\MOSAIC{}'s contrarian design.}

\p{Embedding presentations of their research within existing
scientific software has the added benefit for authors of making their work 
more practically and immediately useful for the scientific community.  
Such presentations establish the computational framework for pragmatically 
deploying their techniques in real-world scientific contexts, 
accelerating the pace at which research work can be translated to 
concrete scientific (and clinical/lab/experimental) practice.
As one example, research involving novel image analysis techniques 
could be packaged so as to target a \MOSAIC{} plugin for bioimaging
software such as \CaPTk{}, so that 
readers could actually run the author's code as a \CaPTk{} 
module.\footnote{This toolkit provides a good case-study for research 
publication because it has an innovative \sQt{} and Common Workflow 
Language based extension mechanism; cf. \bhref{https://cbica.github.io/CaPTk/tr_integration.html}.}  This is important because such 
functional assessment and adoption of novel contributions is harder to carry
out if a body of research work is described indirectly within article
text, as opposed to being concretely implemented within a
specific scientific application.}

\p{Another benefit of using plugins to 
access supplemental archives is that the host application will usually
provide more sophisticated multimedia and data visualization
capabilities, compared to static \PDF{} images or even interactive web
portals.  Publishers have begun to develop online platforms for
browsing research papers in conjunction with multimedia content such
as interactive diagrams and \ThreeD{} graphics --- a physical model of a
protein or a chemical compound, for instance, can be 
viewed online via \WebGL{};
such graphics could even be embedded as an \q{\textbf{iframe}} within an \HTML{}
version of the publications.  Publishers consider this to be
cutting-edge technology.  However, the same molecular \ThreeD{} model, when
viewed in \IQmol{}, can be enhanced with many additional visual features,
representing bonds, orbitals, torsion angles, etc.  The multimedia
experience of exploring chemistry data in custom software like \IQmol{} 
is therefore much greater than the experience of generic web
multimedia, which means that the scientific software is a better 
forum for showcasing novel research.}

\section{\protect\llMOSAIC{} Structured Reporting (\protect\lsMOSAICSR{})}

\p{The \MOSAIC{} structured reporting framework (\MOSAICSR{}) 
includes tools to help authors develop interactive 
presentations supplementing academic documents, and specifically 
to use supplemental archives to document how their research 
has been conducted.  With \MOSAICSR{}, authors can implement or 
reuse code libraries that report on 
research/experiment methods, workflows, and 
protocols.  The \MOSAICSR{} information may be structured 
as a \q{minimum information checklist} conformant 
to standards such as those collectively 
gathered into the \MIBBI{} recommendations;  
in this case \MOSAICSR{} would be applied by  
implementing object models instantiating \MIBBI{} 
policies.  Alternatively, \MOSAICSR{} reports can be 
derived from actual computer programs simulating 
research workflows, similar to \BioCoder{}\footnote{\label{fnt:bioc}See 
\bhref{https://jbioleng.biomedcentral.com/articles/10.1186/1754-1611-4-13}.} (which is included by LTS, in an updated \Cpp{} version, as one 
\MOSAIC{} library).  Finally, \MOSAICSR{} presentations 
may be based on annotations applied to research/analytic 
code.  For example, in the context of image analysis, 
the \Pandore{} project (an image-processing application) 
provides a suite of image-processing 
operations that can be called from computer code 
(together with the \q{Image Processing 
Objectives} Ontology mentioned earlier). 
Image-analysis pipelines can therefore be explained 
by annotating the pertinent function-calls 
(which \Pandore{} calls \q{operators}) with terms from 
the \Pandore{} controlled vocabulary, providing 
annotation targets for \MOSAICSR{} presentations.}

\p{\lMOSAICSR{} can express both computational 
workflows that are fully encapsulated by published 
code and real-world protocols concerning 
laboratory equipment and physical materials 
or samples under investigation.  In the latter 
guise, \MOSAICSR{} code can employ or instantiate 
standardized terminologies and data structures 
for describing experiments --- such as 
\MIBBI{} policies or \BioCoder{} functions.  
In this case, the role of \MOSAICSR{} code is 
to serve as a serialization/deserialization 
endpoint for sharing research metadata.  
Conversely, when workflows are fully implemented 
within software developed as part of a body 
of research, \MOSAICSR{} can provide a functional 
interface allowing this code to be embedded 
in scientific software.  For the latter,  
\MOSAICSR{} provides a framework for modeling 
how a software component specific to a given 
research project exposes its functionality to 
host and/or networked peer applications.  There 
are also instances where both scenarios are 
relevant --- the \MOSAICSR{} code would simultaneously 
document real-world experimental protocols and 
construct a digital interface as part of 
a workflow which is part digital (\q{\textit{in silico}}) and part 
\q{real-world} (\q{in the lab}).}

\section{\protect\llMOSAIC{} Annotations}
\p{Included in any \MOSAIC{} plugin would be specially-designed 
\PDF{} viewers for interactively 
reading authors' papers.  In particular, these \PDF{}
applications would recognize cross-references pointing between 
publications and their associated supplemental archives.  This 
cross-referencing would allow authors to identify concepts which are 
discussed and/or represented both in the research paper and in the 
archive, allowing them to annotate 
their papers more thoroughly.  For example, 
the concept \q{\RNA{} Extraction} may be discussed in a
publication text, and also formally declared as one step in the lab
processing as represented via \BioCoder{}, summarized in a
\BioCoder{}-generated chart and included in the supplemental archive 
(a similar example is used in the documentation for \BioCoder{}).
The \PDF{} viewer would then ensure that the phrase \q{\RNA{} Extraction} in
the text is interactively linked to the concordant step in the
experimental process, so that readers would then be able 
to view the \BioCoder{} summary as a context-menu action associated 
with the phrase, precisely where it appears in the \PDF{} file 
(in other words, the annotation ground is a granular 
location in the text).  Another 
example would be the phrase \q{Oxygenated Airflow}, which refers to airflow 
in assisted-breathing devices, such as
ventilators.  To ensure that the device works properly, the
equipment must be monitored to check that a steady stream of oxygen
reaches the patient.  Research into the design and manufacturing of
ventilators and similar devices may then include \q{Oxygenated Airflow}
both as a phrase within the article text itself and as a parameter in data
sets evaluating the device's performance.  In this situation, the
publication-text location of the \q{Oxygenated Airflow} phrase should once
again be annotated with links to the relevant part of the data set
(e.g., a table column) where measurements of airflow levels are
recorded.}

\p{In order to establish granular links between 
publication texts and data sets, \MOSAIC{} introduces 
a novel \q{Hypergraph Text Encoding} (\HTXN{}) 
system.  This includes \LaTeX{} code generators 
which output both \LaTeX{} and \HTXN{} representations 
of manuscripts, where the \LaTeX{} files are 
provisioned with processing steps during \PDF{} 
creation that generates geometric mappings between 
\PDF{} viewport coordinates and \HTXN{} annotations.  
Such information is then used by \MOSAIC{} \PDF{} 
viewers to cross-references publications 
and data sets according to a protocol which 
allows users to convenient switch back and 
forth between reading documents and visualizing 
data.}

\p{In addition to this technology for linking 
annotations with \PDF{} and data-set annotations, 
LTS is working on frameworks for constructing 
annotations themselves with a rigorous semantics 
grounded on general-purpose models such 
as OpenAnnotation and the Linguistic Annotation 
Framework.  We are, in particular, directing 
attention to bioimaging and other image-related 
use-cases via an \q{Annotation Exchange Format 
for Images} (\AXFI{}), and to linguistic 
use-cases via an \q{Annotation Exchange Format 
for Text} (\AXFt{}).  The \AXFt{} system 
supports graph-style annotations marking 
inter-textual connections on different 
linguistic scales.  At the sentence level 
and smaller, these annotations are oriented 
to discourse-grounding models based on 
Cognitive Grammar; at the sentence level 
and larger, the primary model is 
Sequence Package Analysis, a discourse-representation 
methodology pioneered by LTS's founder Amy Neustein.  
[Recommend here giving links to refs on HTXN and SPA]}






