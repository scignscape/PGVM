
\section{Overview}

\p{\lConceptsDB{} is a hypergraph database engine being developed 
by Linguistic Technology Systems (LTS).  \lConceptsDB{} features 
a multi-paradigm data representation strategy which 
is informed by contemporary research into \AF{} and 
information semantics, allowing theoretical frameworks 
such as Conceptual Spaces, Conceptual Role Semantics, 
and Petri Nets to be applied toward the computational 
modeling of information spaces.  In addition, 
\ConceptsDB{} is designed to prioritize application 
development, particularly naive, desktop-style applications: 
data about application/session/user state can be 
directly stored in the database, so that \ConceptsDB{} 
provides a convenient scaffolding for implementing 
desktop software.  Hypergraph-based data modeling 
ensures that information needed by the application 
can be conveniently marshaled between runtime formats 
and whatever persistent/serial/\GUI{} representations which 
are necessary for application-level capabilites.}

\p{Preliminary to the commercial release of 
\ConceptsDB{}, LTS is deploying a version of \ConceptsDB{} 
as part of the \q{\ConceptsDB{} Application Framework} 
(\CsAF{}) which is oriented to scientific 
software and computing.  \lCsAF{} database instances 
can be employed at several sites within a 
scientific-computing and/or data-sharing platform, 
included as an embedded component of published 
data sets, as a cloud service managing data and/or 
publication repositories, and as a tool for 
individual or institutional users to 
track data sets and publications.  
Our \q{\MOSAIC{}} code libraries 
(which stands for \q{Multi-Paradigm 
Ontologies for Scientific and Technical Publications}) 
encapsulate functionality applicable to 
using \ConceptsDB{} in publishing contexts.  
A \MOSAIC{} \q{Portal} is a data/publication 
repository where users can query and download 
resources following a protocol associated 
with \ConceptsDB{}.  The \q{\MOSAIC{} Dataset Explorer} 
(\MdsX{}) libraries are designed for applications 
which interoperate with data/publication 
archives (including but not limited to \MOSAIC{} Portals).  
Individual data sets may also embed a version of 
\ConceptsDB{} --- specifically, \q{DigammaDB} 
(\DgDb{}), which is a light-weight, self-contained 
\ConceptsDB{}-compatible engine suitable to be distributed 
as source code within a data set/code repository.  
LTS's \q{Dataset Creator} (\dsC{}) is a technology 
that can help researchers deploy data sets 
with customized desktop software, potentially 
using \DgDb{}.  In short, \MdsX{}, \DgDb{}, 
and \dsC{} form a trio of libraries 
helpful for the implementation of published 
data sets with special-purpose applications that are 
customized for viewing and examining information 
specific to the research methods and paradigms 
from which a data set originates; and 
for the implementation of tools to 
acquire and keep track of data sets and their 
associated publications.}

\p{Collectively, then \MdsX{}, \DgDb{}, and \dsC{} form 
the basis of a \CsAF{} implementation --- that 
is, an Application Framework centered on \ConceptsDB{} --- 
that prioritizes deploying and 
accessing scientific data sets.  The information and 
assets included within a data set, and modeled 
via \CsAF{} components, can span several different 
requirements, including:

\begin{enumerate}

\item{} Data sets, which may include raw data files 
and/or code for accessing this data as well 
as demonstrations/implementations of algorithms 
for analyzing or otherwise processing the data 
(or data having a similar format/structure 
or scientific background)  

\item{} Full-text publications, potentially including 
both machine-readable and human-readable (e.g. \PDF{}) 
formats, annotates so as to link parts of the text 
document with corresponding parts/elements within its 
associated data set(s);

\item{} Systematic descriptions of research 
methods, protocols, and (if applicable) 
equipment, such as may be formally 
expressed by standards like 
\MIBBI{} (Minimum Information for Biological and Biomedical Investigations), 
\BioCoder{}, or \Pandore{} (a bioimaging 
library that contains an \q{Image Processing Objectives} 
Ontology);

\item{} Applications for viewing data sets, which may be 
pre-existing software or custom-built for an individual data set.

\end{enumerate}

The \MdsX{} libraries include code for interoperating 
with data structures documenting each of these 
aspects of scientific resourcing, which may involve 
querying data sets themselves or querying 
\APIs{} of scientific repositories where 
data sets and publications are hosted.}

\p{In order to interoperate seamlessly with a 
diverse range of scientific resources, \MdsX{} 
is divided into distinct modules organized 
around the \API{}s and data profiles of 
individual scientific portals/repositories.  
Therefore, LTS seeks to collaborate with 
organizations who maintain scientific portals 
so that open-access \MdsX{} modules 
may be made available to the general public 
targeted those specific scientific resources.}

\p{The following sections with discuss \MdsX{}, 
\dsC{}, and \DgDb{} in greater detail 
as well as provide more information about 
\MdsX{} modules.}

\p{}

