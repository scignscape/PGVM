\vspace{-1.5em}
\section{Overview}
\p{Linguistic Technology Systems (LTS) is developing a new 
database engine called \lConceptsDB{}, which is based on hypergraph data
modeling.  This new database engine features 
a multi-paradigm data representation strategy, which 
is informed by contemporary research into \AI{} and 
information semantics, allowing theoretical frameworks 
such as Conceptual Spaces, Conceptual Role Semantics, 
and Petri Nets to be applied toward the computational 
modeling of information spaces.  In addition, 
\ConceptsDB{} prioritizes application 
development --- particularly native/desktop-style applications. 
Data about application/session/user state can be 
directly stored in the database, so that \ConceptsDB{} 
can provide a convenient scaffolding for implementing 
desktop software.  Hypergraph-based data modeling 
ensures that information needed by the application 
can be readily marshaled between runtime formats 
and whichever persistent/serial/\GUI{} representations 
are necessary for application-level capabilites.}

\p{LTS is currently working on a prototype \ConceptsDB{} 
which is part of our \q{\lConceptsDB{} Application Framework} 
(\CsAF{}).  This framework is oriented to scientific computing 
and scientific software.  That is, \CsAF{} database instances 
can be employed at several sites within a 
scientific-computing and/or data-sharing platform: 
as an embedded component of published 
data sets; as a cloud service managing data and/or 
publication repositories; or as a tool for 
individual or institutional users to 
track data sets and publications.  
Our \q{\MOSAIC{}} code libraries 
(\q{Multi-Paradigm 
Ontologies for Scientific and Technical Publications}) 
encapsulate functionality applicable to 
using \ConceptsDB{} in publishing contexts.  
A \MOSAIC{} Portal is a data/publication 
repository where users can query and download 
resources following a protocol associated 
with \ConceptsDB{}.  The \q{\lMOSAIC{} Dataset Explorer} 
(\MdsX{}) libraries are designed for applications 
which interoperate with data/publication 
archives (including but not limited to \MOSAIC{} Portals).  
Individual data sets may also embed a version of 
\ConceptsDB{} --- specifically, \q{DigammaDB} 
(\DgDb{}), which is a light-weight, self-contained 
\ConceptsDB{}-compatible engine suitable to be distributed 
as source code within a data set/code repository.}

\p{To help researchers deploy data sets 
with customized desktop software, 
LTS has developed a \q{Dataset Creator} (\dsC{}) 
designed to be used with \DgDb{}.  This 
database engine and data-modeling technology 
(\MdsX{}, \DgDb{}, and \dsC{}) form a trio of libraries, 
which are helpful for the implementation of published 
data sets with special-purpose applications that are 
customized for viewing and examining information 
specific to the research methods and paradigms 
from which a data set originates; and 
for the implementation of tools to 
acquire and keep track of data sets and their 
associated publications.}

\p{Collectively, this trio of libraries form 
the basis of a \CsAF{} implementation --- i.e., an Application Framework centered 
on \ConceptsDB{} --- which prioritizes deploying and 
accessing scientific data sets.  The information and 
assets included within a data set, and modeled 
via \CsAF{} components, can span several different 
requirements, including:

\begin{enumerate}[leftmargin=12pt]

\item{} Data sets --- these may include raw data files 
and/or code for accessing this data as well 
as demonstrations/implementations of algorithms 
for analyzing or otherwise processing the data 
(or data having a similar format/structure 
or scientific background);  

\item{} Full-text publications --- potentially including 
both machine-readable and human-readable (e.g. \PDF{}) 
formats, annotated so as to link parts of a text 
document with corresponding parts/elements within its 
associated data set(s);

\item{} Systematic descriptions of research 
methods, protocols, and (if applicable) 
equipment, such as may be formally 
expressed by standards like 
\MIBBI{} (Minimum Information for Biological and Biomedical Investigations), 
\BioCoder{} (see Footnote~\hyperref[fnt:bioc]{\ref{fnt:bioc}}), or \Pandore{} (a bioimaging 
library that contains an \q{Image Processing Objectives} 
Ontology);

\item{} Applications for viewing data sets --- applications which 
may be in the form of pre-existing software, 
or custom-built for an individual data set.
\end{enumerate}}

\p{The \MOSAIC{} Dataset Explorer (\MdsX{}) libraries include 
code for interoperating 
with data structures documenting each of these 
aspects of scientific resources, which may involve 
querying data sets themselves or querying 
\API{}s of scientific corpora where 
data sets and publications are hosted.  
\lMdsX{} is built around what we call a 
\q{Scientific Data Repository Model} 
(\SDRM{}) which is concretized 
separately for each corpus or repository 
connected with a given \MdsX{} application.  
\lSDRM{} is divided into distinct concretizations, 
or \q{modules,} organized 
around the \API{}s and data profiles of 
individual scientific portals/repositories.  
For this reason, LTS seeks to collaborate with 
organizations who maintain scientific portals 
so that we can make open-access \SDRM{} modules 
available to the general public, 
targeting those specific scientific resources.}

\p{The following sections will describe \MdsX{}, 
\dsC{}, and \DgDb{} in greater detail 
as well as provide more information about 
\SDRM{} modules.}

%\p{}

